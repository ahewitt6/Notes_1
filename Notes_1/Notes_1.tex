\documentclass[12pt]{amsart}

\usepackage{enumerate,amsmath,amssymb,amsthm,mathtools,comment}

\usepackage{arydshln}
\usepackage{dashrule}
\usepackage{slashed}

%for Griffiths curly r
\usepackage{calligra}
\DeclareMathAlphabet{\mathcalligra}{T1}{calligra}{m}{n}
\DeclareFontShape{T1}{calligra}{m}{n}{<->s*[2.2]callig15}{}
\newcommand{\scripty}[1]{\ensuremath{\mathcalligra{#1}}}

\newcommand{\capk}{\frac{1}{4 \pi \epsilon_0}}

\begin{document}
\title{}
\author{Alec Hewitt}
\maketitle

\setlength{\parindent}{0mm}

\hdashrule[0.5ex][c]{\linewidth}{0.5pt}{1.5mm}
\begin{center}
These are the derivations that I have been transferring to a Latex document and I plan to add as many as possible  throughout my gap year and review them along the way.\\
Things that we need to add or improve:\\
- the dark matter section needs to be added to, especially more of the elementary calculations\\
- sommerfeld expansion needs to be understood more\\
- discrepancy between translation operators in QM between griffiths and sakurai
- discrepancy between position operators in QM griffiths and sakurai
\end{center}
\hdashrule[0.5ex][c]{\linewidth}{0.5pt}{1.5mm}


\section*{\underline{Classical Dynamics Notes}}
\section*{Chapter 1}
\begin{enumerate}

\item \underline{$x_i'=\sum_{j=1}^3 \lambda_{ij} x_j;\,\, x_i = \sum_{j=1}^3 \lambda_{ji} x_j'$}\\
$x_1'=\cos \theta x_1 + \sin \theta x_2 = \cos \theta x_1 + \cos( \theta - \frac{\pi}{2}) x_2 = \cos \theta x_1 + \cos ( \frac{\pi}{2}- \theta) x_2\\
x_2' = - \sin \theta x_1 + \cos \theta x_2 = \cos (\frac{\pi}{2} + \theta) x_1 + x_2 \cos \theta\\
\implies
\begin{cases}
	x_1' = \cos ( x_1', x_1) x_1 + \cos (x_1', x_2) x_2\\
	x_2' = \cos (x_2',x_1) x_1 + \cos (x_2',x_2) x_2
\end{cases}\\
\implies
\begin{cases}
	x_1' = \lambda_{11} x_1 + \lambda_{12} x_2
	x_2' = \lambda_{21} x_1 + \lambda_{22} x_2
\end{cases}\\
\implies
\begin{cases}
	x_1' = \lambda_{11} x_1 + \lambda_{12} x_2 + \lambda_{13} x_3\\
	x_2' = \lambda_{21} x_1 + \lambda_{22} x_2 + \lambda_{23} x_3\\
	x_3' = \lambda_{31} x_1 + \lambda_{32} x_2 + \lambda_{33} x_3
\end{cases}\\
\therefore x_i' = \lambda_{i1} x_1 + \lambda_{i2} x_2 + \lambda_{i3} x_3 = \sum_{j=1}^3 \lambda_{ij} x_j\\$
\underline{Inverse transformation}\\
$x_1 = x_1' \cos \theta - x_2' \sin \theta\\
=x_1' \cos(x_1', x_1) + x_2' \cos (\frac{\pi}{2} + \theta)\\
=x_1' \cos(x_1',x_1) + x_2' \cos(x_2',x_1) = \lambda_{11} x_1' + \lambda_{21} x_2'\\
\implies x_1 = \lambda_{11} x_1' + \lambda_{21} x_2' + \lambda_{31} x_3'= \sum_{j=1}^3 \lambda_{j1} x_j'\\
\therefore x_i = \sum_{j=1}^3 \lambda_{ji} x_j'$\\


\hdashrule[0.5ex][c]{\linewidth}{0.5pt}{1.5mm}


\item \underline{$\sum_i \lambda_{ij} \lambda_{ik} = \delta_{jk}$ or $\lambda^t \lambda=I$}\\
$\underline{i \neq k}\\
\sum_j \lambda_{1j} \lambda_{2j} = \lambda_{11} \lambda_{21} + \lambda_{12} \lambda_{22} \\
=\cos (x_1',x_1) \cos(x_2',x_1) + \cos(x_1',x_2) \cos(x_2',x_2)\\
=\cos \theta \cos(\theta + \pi/2) + \cos(\pi/2 - \theta) \cos (\theta)\\
= - \cos \theta \sin \theta + \cos \theta \sin \theta=0\\
\underline{i = k}\\
\sum_j \lambda_{1j} \lambda_{1j} = \lambda_{11}^2 + \lambda_{12}^2\\
= \cos(x_1', x_1)^2 + \cos(x_1',x_2)^2 = \cos \theta^2 + \cos(\pi/2 - \theta)^2\\
= \cos^2 \theta + \sin^2 \theta =1$
\\


\hdashrule[0.5ex][c]{\linewidth}{0.5pt}{1.5mm}


\underline{Note:} $\sum_i \lambda_{ij} \lambda_{ik} = \delta_{jk}\\$
\underline{Note:} $\lambda_{ij}^t = \lambda_{ji}$ (transpose)\\


\hdashrule[0.5ex][c]{\linewidth}{0.5pt}{1.5mm}


\item \underline{$\lambda^t = \lambda^{-1}$} (orthogonal)\\
Let $A = \lambda^t\\
(\lambda \lambda^t)_{ij} = ( \lambda A)_{ij} = \sum_k \lambda_{ik} A_{kj} = \sum_k \lambda_{ik} \lambda_{kj}^t\\
=\sum_k \lambda_{ik} \lambda_{jk}\\
\underline{recall:} \sum_k \lambda_{ik} \lambda_{jk} = \delta_{ij}\\
\implies \lambda \lambda^t = I \implies \lambda^t = \lambda^{-1}\\$


\hdashrule[0.5ex][c]{\linewidth}{0.5pt}{1.5mm}


\item \underline{$(\tilde{\mu} \tilde{\lambda})^t = ( \tilde{\mu} \tilde{\lambda})^{-1}$}\\
Spose $x_i' = \sum_j \lambda_{ij} x_j,\,\, x_k'' = \sum_i \mu_{ki} x_i'\\
\implies x_k''= \sum_j ( \sum_i \mu_{ki} \lambda_{ij}) x_j = \sum_j [ \tilde{\mu} \tilde{\lambda}]_{kj} x_j\\$
\underline{recall:} $( \tilde{\mu} \tilde{\lambda})^t = \tilde{\lambda}^t \tilde{\mu}^t;\,\, \lambda^t = \lambda^{-1};\,\, \mu^t = \mu^{-1}\\
\implies ( \mu \lambda)^t \mu \lambda = \lambda^t \mu^t \mu \lambda = \lambda^{-1} \mu^{-1} \mu \lambda = \lambda^{-1} \lambda = I\\
\therefore ( \tilde{\mu} \tilde{\lambda})^t = ( \tilde{\mu} \tilde{\lambda})^{-1}\\$


\hdashrule[0.5ex][c]{\linewidth}{0.5pt}{1.5mm}


In general, if $x_i' = \sum_j \lambda_{ij} x_j\\
\implies A_i' = \sum_j \lambda_{ij} A_j\\$


\hdashrule[0.5ex][c]{\linewidth}{0.5pt}{1.5mm}


\item \underline{$\vec{A} \cdot \vec{B} = A B \cos(\vec{A},\vec{B})$}\\
$\vec{A} \cdot \vec{B} = \sum_i A_i B_i\\
\frac{\vec{A} \cdot \vec{B}}{AB} = \sum_i \frac{A_i B_i}{AB};\,\, \frac{A_i}{A} = \Lambda_i^A;\,\, \frac{B_i}{B} = \Lambda_i^B\\$
(direction cosines)\\
$\implies \sum_i \Lambda_i^A \Lambda^B_i = \cos (\vec{A},\vec{B})\\
\therefore \vec{A} \cdot \vec{B} = A B \cos( \vec{A}, \vec{B})\\$
or just do $|u-v|^2$ distribute one side and then use law of cosines on the other and solve for dot product.


\hdashrule[0.5ex][c]{\linewidth}{0.5pt}{1.5mm}


\underline{Note:} $\vec{A}' \cdot \vec{B}' = \vec{A} \cdot \vec{B},\,\, i.e. \vec{A} \cdot \vec{B}$ is a scalar.\\


\hdashrule[0.5ex][c]{\linewidth}{0.5pt}{1.5mm}


$\vec{C}_i = ( \vec{A} \times \vec{B})_i = \sum_{jk} \epsilon_{ijk} A_j B_k\\
\epsilon_{ijk} = 
\begin{cases}\\
	0 $ any two indices match$\\
	1 $ even permutation frome (1,2,3)$\\
	-1 $ odd permutation$
\end{cases}\\$


\hdashrule[0.5ex][c]{\linewidth}{0.5pt}{1.5mm}


\item \underline{$|\vec{A} \times \vec{B}| = A B \sin \theta$};\,\, $\sin \theta = \sin(\vec{A},\vec{B})\\
A^2 B^2 \sin^2 \theta = A^2 B^2 - A^2 B^2 \cos ^2 \theta = A^2 B^2 - ( \vec{A} \cdot \vec{B})^2\\
=(\sum_i A_i^2)(\sum_i B_i^2) - (\sum_i A_i B_i)^2\\
= (A_2 B_3 - A_3 B_2)^2 + (A_3 B_1 - A_1 B_3)^2 + (A_1 B_2 - A_2 B_1)^2\\
= \sum_i |\vec{A} \times \vec{B}|^2_i = |\vec{A} \times \vec{B}|^2\\$


\hdashrule[0.5ex][c]{\linewidth}{0.5pt}{1.5mm}


\underline{Note:} $\vec{A} \times (\vec{B} \times \vec{C}) \neq (\vec{A} \times \vec{B}) \times \vec{C}\\
\vec{A} \times (\vec{B} \times \vec{C}) = (\vec{A} \times \vec{C}) \vec{B} - (\vec{A} \cdot \vec{B}) \vec{C}\\
\sum_k \epsilon_{ijk} \epsilon_{\ell m k} = \delta_{i \ell} \delta_{jm} - \delta_{im} \delta_{j \ell}\\
\vec{C} = \sum_{i,j,k} \epsilon_{ijk} \hat{e}_i A_j B_k$ (cross product in index notation)\\
\underline{Identities}\\
$\vec{A} \cdot(\vec{B} \times \vec{C}) = \vec{B} \cdot(\vec{C} \times \vec{A}) = \vec{C} \cdot(\vec{A} \rtimes \vec{B} = \vec{A} \vec{B} \vec{C}\\
\vec{A} \times (\vec{B} \times \vec{C}) = (\vec{A} \cdot \vec{C}) \vec{B} - (\vec{A} \cdot \vec{B}) \vec{C}\\
(\vec{A} \times \vec{B}) \cdot(\vec{C} \times \vec{D}) = \vec{A} \cdot[\vec{B} \times (\vec{C} \times \vec{D})]\\
= \vec{A} \cdot[(\vec{B} \cdot \vec{D}) \vec{C} - (\vec{B} \cdot \vec{C}) \vec{D}]\\
= ( \vec{A} \cdot \vec{C}) \vec{B} \cdot \vec{D}) - ( \vec{A} \cdot \vec{D}) \vec{B} \cdot \vec{C})\\
(\vec{A} \times \vec{B}) \times (\vec{C} \times \vec{D}) = [( \vec{A} \times \vec{B}) \cdot \vec{D}] \vec{C} - [( \vec{A} \times \vec{B}) \cdot \vec{C}] \vec{D}\\
= ( \vec{A} \vec{B} \vec{D}) = \vec{C} - ( \vec{A} \vec{B} \vec{C}) \vec{D} = ( \vec{A} \vec{C} \vec{D}) \vec{B} - ( \vec{B} \vec{C} \vec{D}) \vec{A}$\\


\hdashrule[0.5ex][c]{\linewidth}{0.5pt}{1.5mm}


\item \underline{$\dot{\hat{e}}_r = \dot{\theta} \hat{e}_{\theta};\,\, \dot{\hat{e}}_{\theta} = - \dot{\theta} \hat{e}_r$}\\
$\hat{e}_r^{(2)} - \hat{e}_r^{(1)} = d \hat{e}_r;\,\, \hat{e}^{(2)}_{\theta} - \hat{e}_{\theta}^{(1)} = d \hat{e}_{\theta}\\$
Analogous to $ds = \theta dr \implies d \hat{e}_r = d \theta \hat{e}_{\theta}\\$
$d \hat{e}_{\theta} = - d \theta \hat{e}_r$ (draw a picture, you can also draw a triangle involving er1 er2 and der with sides being their magnitudes)\\
$\implies \frac{d \hat{e}_r}{dt} = \dot{\hat{e}}_r = \dot{\theta} \hat{e}_{\theta};\,\, \frac{d \hat{e}_{\theta}}{dt} = \dot{\hat{e}}_{\theta} = - \dot{\theta} \hat{e}_r\\$
or just represent $\hat{e}_r=\cos \theta \hat{x} + \sin \theta \hat{y}$ and $\hat{e}_{\theta}$ as the same except $\theta \rightarrow \theta + \pi/2$


\hdashrule[0.5ex][c]{\linewidth}{0.5pt}{1.5mm}


\item \underline{$\vec{v} = \dot{\vec{r}} = \dot{r} \hat{e}_r + r \dot{\theta} \hat{e}_{\theta}$}\\
$\vec{v} = \frac{d \vec{r}}{dt} = \frac{d}{dt} (r \hat{e}_r) = \dot{r} \hat{e}_r + r \dot{\hat{e}}_r = \dot{r} \hat{e}_r + \dot{\theta} \hat{e}_{\theta}\\$





\hdashrule[0.5ex][c]{\linewidth}{0.5pt}{1.5mm}


\item \underline{$\vec{a} = ( \ddot{r} - r \dot{\theta}^2) \hat{e}_r + ( r \ddot{\theta} + 2 \dot{r} \dot{\theta}) \hat{e}_{\theta}$}\\
$\vec{a} = \frac{d}{dt} ( \dot{r} \hat{e}_r + r \dot{\theta} \hat{e}_{\theta}) = \ddot{r} \hat{e}_r + \dot{r} \dot{\hat{e}}_r + \dot{r} \dot{\theta} \hat{e}_{\theta} + r \ddot{\theta} \hat{e}_{\theta} + r \dot{\theta} \dot{\hat{e}}_{\theta}\\
= \ddot{r} \hat{e}_r + \dot{r} \dot{\theta} \hat{e}_{\theta} + \dot{r} \dot{\theta} \hat{e}_{\theta} + r \ddot{\theta} \hat{e}_{\theta} + r \dot{\theta}(- \dot{\theta} \hat{e}_r)\\
=(\ddot{r} - r \dot{\theta}^2) \hat{e}_r + ( r \ddot{\theta} + 2 \dot{r} \dot{\theta}) \hat{e}_{\theta}\\$


\hdashrule[0.5ex][c]{\linewidth}{0.5pt}{1.5mm}


\underline{rectangular $(x,y,z)$}\\
$d \vec{s} = dx_1 \hat{e}_1 + d x_2 \hat{e}_2 + d x_3 \hat{e}_3 \\
d s^2 = dx_1^2 + d x_2^2 + dx_3^2\\
v^2 = \sum_i \dot{x}_i^2\\
\vec{v} = \sum_i \dot{x}_i \hat{e}_i\\$


\underline{spherical $(r,\theta,\phi)$}\\
$d \vec{s} = dr \hat{e}_r + r d \theta \hat{e}_{\theta} + r \sin \theta d \phi \hat{e}_{\phi}\\
ds^2 = dr^2 + r^2 d \theta^2 + r^2 \sin^2 \theta d \phi^2\\
v^2 = \dot{r}^2 + r^2 \dot{\theta}^2 + r^2 \sin^2 \theta \dot{\phi}^2 = \frac{ds^2}{dt^2}\\
\vec{v} = \dot{r} \hat{e}_r + r \dot{\theta} \hat{e}_{\theta} + r \sin \theta \dot{\phi} \hat{e}_{\phi} = \frac{d \vec{s}}{dt}\\$


\underline{Cylindrical $(r, \phi,z)$}\\
$d \vec{s} = dr \hat{e}_r + r d \phi \hat{e}_{\phi} + dz \hat{e}_z\\
ds^2 = dr^2 + r^2 d \phi^2 + dz^2\\
v^2 = \dot{r}^2 + r^2 \dot{\phi}^2 + \dot{z}^2\\
\vec{v} = \dot{r} \hat{e}_r + r \dot{\phi} \hat{e}_{\phi} + \dot{z} \hat{e}_z\\$


\hdashrule[0.5ex][c]{\linewidth}{0.5pt}{1.5mm}


\item \underline{$\vec{v} = \vec{\omega} \times \vec{r}$} (tangential velocity)\\
$v = R \frac{d \theta}{dt} = R \dot{\theta} = r \sin \alpha \dot{\theta} = r \omega \sin \alpha = \vec{\omega} \times \vec{r}\\
\vec{\omega} \times \vec{r}$ since $\vec{v}$ counterclockwise is positive\\


\hdashrule[0.5ex][c]{\linewidth}{0.5pt}{1.5mm}


skipped 1.105 - 1.108\\


\hdashrule[0.5ex][c]{\linewidth}{0.5pt}{1.5mm}


\item \underline{$\frac{\partial \phi'}{\partial x_i'} = \sum_j \lambda_{ij} \nabla_j \phi;\,\, \lambda_{ij} = \frac{\partial x_j}{\partial x_i'}$}\\
$\phi'(x_1',x_2',x_3') = \phi(x_1,x_2,x_3)\\
\frac{\partial \phi'}{\partial x_i'} = \sum_j \frac{\partial \phi'}{\partial x_j} \frac{\partial x_j}{\partial x_i'} = \sum_j \frac{\partial \phi}{\partial x_j} \frac{\partial x_j}{\partial x_i'}\\$
\underline{recall:} $x_j = \sum_k \lambda_{kj} x_k'\\
\implies \frac{\partial x_j}{\partial x_i'} = \frac{\partial}{\partial x_i'} \sum_k \lambda_{kj} x_k' = \sum_k \lambda_{kj} \delta_{ki} = \lambda_{ij}\\
\implies \frac{\partial \phi'}{\partial x_i'} = \sum_j \lambda_{ij} \frac{\partial \phi}{\partial x_j} = \sum_j \lambda_{ij} \nabla_j \phi$\\


\hdashrule[0.5ex][c]{\linewidth}{0.5pt}{1.5mm}


$\nabla \phi = \sum_i \hat{e}_i \frac{\partial \phi}{\partial x_i}\\
\nabla \cdot \vec{A} = \sum_i \frac{\partial A_i}{\partial x_i}\\
\nabla \times \vec{A} = \sum_{i,j,k} \epsilon_{ijk} \frac{\partial A_j}{\partial x_i} \hat{e}_k\\
d \phi = \sum_i \frac{\partial \phi}{\partial x_i} d x_i = \sum_i (\nabla_i \phi) dx_i = \nabla \phi \cdot d \vec{s}\\
(d \phi)_{max} = \nabla \phi \cdot d \vec{s} = \nabla \phi ds \cos (0) = \nabla \phi ds\\
\hat{n} \cdot \nabla \phi = \frac{\partial \phi}{\partial n}\\
\nabla \cdot \nabla = \nabla^2 = \sum_i \frac{\partial^2}{\partial x_i^2}\\
\oint \vec{A} \cdot d \vec{a} = \int \nabla \cdot \vec{A} dv$ (Gauss's Law)\\
$\oint \vec{A} \cdot d \vec{s} = \int \nabla \times \vec{A} \cdot d \vec{a}$ (Stoke's theorem)\\


\hdashrule[0.5ex][c]{\linewidth}{0.5pt}{1.5mm}


\section*{Chapter 2}
\underline{Newton's Laws}\\
I. A body remains at rest or in uniform motion unless acted upon by a force.\\
II. A body acted upon by a force moves in such a manner that the time rate of change of momentum equals the force.\\
III. If two bodies exert forces on eachother, these forces are equal in magnitude and opposite in direction.\\
III'. If two bodies constitute an ideal, isolated system, then the accelerations of these bodies are always in opposite directions, and the ratio of the magnitudes of the accelerations is constant. This constant ratio is the inverse ratio of the masses of the bodies.


\hdashrule[0.5ex][c]{\linewidth}{0.5pt}{1.5mm}


\item \underline{$\sum_i \vec{p}_i =$ const.}\\
$\vec{F}_1 = - \vec{F}_2 \rm{ (Newton's\,\,third\,\,law)}\\
\implies \frac{d \vec{p}_1}{dt} + \frac{d \vec{p}_2}{dt} = \frac{d}{dt} \sum_i \vec{p}_i = 0\\
\implies \sum_i \vec{p}_i =$ const.\\


\hdashrule[0.5ex][c]{\linewidth}{0.5pt}{1.5mm}


A frame is inertial if newton's Laws are valid in that frame.\\


\hdashrule[0.5ex][c]{\linewidth}{0.5pt}{1.5mm}


$\vec{L} = \vec{r} \times \vec{p}$ (angular momentum)\\
$\vec{N} = \vec{r} \times \vec{F}$ (torque)\\


\hdashrule[0.5ex][c]{\linewidth}{0.5pt}{1.5mm}


I. total $\vec{p}$ is conserved when total force on a prarticle is zero.\\


\hdashrule[0.5ex][c]{\linewidth}{0.5pt}{1.5mm}


\item \underline{$\dot{\vec{L}} = \vec{r} \times \dot{\vec{p}} = \vec{N}$}\\
$\vec{N} \equiv \vec{r} \times \vec{F} = \vec{r} \times m \dot{\vec{v}} = \vec{r} \times \dot{\vec{p}}\\
\dot{\vec{L}} = \dot{\vec{r}} \times \vec{p} + \vec{r} \times \dot{\vec{p}} = m ( \vec{v} \times \vec{v}) + \vec{r} \times \vec{F} = \vec{r} \times \vec{F}\\
\therefore \dot{\vec{L}} = \vec{r} \times \dot{\vec{p}}\\$


\hdashrule[0.5ex][c]{\linewidth}{0.5pt}{1.5mm}


II. angular momentum of a particle subject to no torque is conserved\\


\hdashrule[0.5ex][c]{\linewidth}{0.5pt}{1.5mm}


\item \underline{$\vec{F} = - \nabla U$}\\
$\int_1^2 \vec{F} \cdot d \vec{r} = W_{12} = - \Delta U = U_1 - U_2\\
= - \int_1^2 \nabla U \cdot d \vec{r} \implies \vec{F} = - \nabla U\\$


\hdashrule[0.5ex][c]{\linewidth}{0.5pt}{1.5mm}


\item \underline{$\frac{d E}{dt} = \frac{\partial U}{\partial t}$},\,\, $U=U(\vec{r}(t),t)\\
E = T + U\\
\frac{d E}{dt} = \frac{dT}{dt} + \frac{dU}{dt}\\$
\underline{Note:} $\vec{F} \cdot d \vec{r} = m \frac{d \vec{v}}{dt} \cdot \frac{d \vec{r}}{dt} dt = m \frac{d \vec{v}}{dt} \cdot \vec{v} dt\\
= \frac{m}{2} \frac{d}{dt} (\vec{v} \cdot \vec{v}) dt = \frac{m}{2} \frac{d}{dt} v^2 dt = d (\frac{1}{2} m v^2) = d T\\
\implies \vec{F} \cdot \vec{v} = \frac{d T}{dt}\\
\frac{d U}{dt} = \sum_i \frac{\partial U}{\partial x_i} \frac{d x_i}{dt} + \frac{\partial U}{\partial t} = \nabla U \cdot \dot{\vec{r}} + \frac{\partial U}{\partial t}\\
\implies \frac{d E}{dt} = \vec{F} \cdot \vec{v} + \nabla U \cdot \vec{r} + \frac{\partial U}{\partial t} = \frac{\partial U}{\partial t}\\
\implies \frac{d E}{dt} = \vec{F} \cdot \vec{v} + \nabla U \cdot \vec{r} + \frac{\partial U}{\partial t} = \frac{\partial U}{\partial t}\\
\therefore \frac{dE}{dt} = \frac{\partial U}{\partial t},\,\, \frac{\partial U}{\partial t} \implies$ ( conservative $\vec{F}$)\\


\hdashrule[0.5ex][c]{\linewidth}{0.5pt}{1.5mm}


III. total Energy of a particle in a conservative vector field is a constant in time.


\section*{Classical Dynamics}
\section*{Chapter 6}

\item \underline{$\frac{\partial J}{\partial \alpha} |_{\alpha = 0} = 0$}\\
want to modify $y(x)$ between $x_1, x_2$ so that $J = \int_{x_1}^{x_2} f\{ y(x), y'(x); x \} d x\\$
has an extremum \\
$\implies y( \alpha, x) = y(0,x) + \alpha \eta(x)\\$
where $\eta(x_1) = \eta(x_2) = 0\\
\implies J(\alpha) = \int_{x_1}^{x_2} f \{ y(\alpha, x ), y'(\alpha,x); x \} d\\
\implies$ extremum occurs when $\frac{\partial J}{\partial \alpha} |_{\alpha = 0} = 0\\$
don't understand $\alpha = 0\\$


\hdashrule[0.5ex][c]{\linewidth}{0.5pt}{1.5mm}


\item \underline{$\frac{\partial f}{\partial y} - \frac{d}{dx} \frac{\partial f}{\partial y'} = 0$} (Euler's equation)\\
\underline{recall:} $J(\alpha) = \int_{x_1}^{x_2} f \{ y(\alpha ,x), y'(\alpha,x); x \} dx;\,\, y(\alpha,x) = y(0,x) + \alpha \eta(x)\\
\implies \frac{\partial J}{\partial \alpha} = \frac{\partial}{\partial \alpha} \int_{x_1}^{x_2} f\{ y, y' ;x \} dx\\
\implies \frac{\partial J}{\partial \alpha} = \int_{x_1}^{x_2} [ \frac{\partial f}{\partial y} \frac{\partial y}{\partial \alpha} + \frac{\partial f}{\partial y'} \frac{\partial y'}{\partial \alpha}] dx\\
\frac{\partial y}{\partial \alpha} = \eta(x);\,\, y' = y'(0,x) + \alpha \eta'(x) \implies \frac{\partial y'}{\partial \alpha} = \eta'(x)\\
\implies \frac{\partial J}{\partial \alpha} = \int_{x_1}^{x_2} (\frac{\partial f}{\partial y} \eta(x) + \frac{\partial f}{\partial y'} \eta'(x))dx\\
u = \frac{\partial f}{\partial y'},\,\, du = \frac{d}{dx} \frac{\partial f}{\partial y'} dx,\,\, dv = \eta'(x) ex, v= \eta(x)\\$
\underline{2nd term:} $(\eta(x) \frac{\partial f}{\partial y'} |_{x_1}^{x_2} - \int_{x_1}^{x_2} \eta(x) \frac{d}{dx} \frac{\partial f}{\partial y'} dx\\
\eta(x_2) = \eta(x_1) = 0\\
\implies \frac{\partial J}{\partial \alpha} = \int_{x_1}^{x_2} (\frac{\partial f}{\partial y} - \frac{d}{dx} ( \frac{\partial f}{\partial y'})) \eta(x) dx = 0\\
\therefore \frac{\partial f}{\partial y} = \frac{d}{dx} \frac{\partial f}{\partial y'}$


\hdashrule[0.5ex][c]{\linewidth}{0.5pt}{1.5mm}


\item \underline{$\frac{\partial f}{\partial x} - \frac{d}{dx} (f- y' \frac{\partial f}{\partial y'}) = 0$} (second form of Euler equation)\\
$\frac{df}{dx} = \frac{d}{dx} f \{ y, y' ; x \} = \frac{\partial f}{\partial y} \frac{dy}{dx} + \frac{\partial f}{\partial y'} \frac{d y'}{dx} + \frac{\partial f}{\partial x}\\
= y' \frac{\partial f}{\partial y} + y'' \frac{\partial f}{\partial y'} + \frac{\partial f}{\partial x}\\
\frac{d}{dx}(y' \frac{\partial f}{\partial y'}) = y'' \frac{\partial f}{\partial y'} + y' \frac{d}{dx} \frac{\partial f}{\partial y'}\\
\implies y'' \frac{\partial f}{\partial y'} = \frac{d}{dx} (y' \frac{\partial f}{\partial y'})-y' \frac{d}{dx} \frac{\partial f}{\partial y'}\\$
plug in\\
$\implies \frac{df}{dx} = y' \frac{\partial f}{\partial y} + \frac{d}{dx} (y' \frac{\partial f}{\partial y'}) - y' \frac{d}{dx} \frac{\partial f}{\partial y'} + \frac{\partial f}{\partial x}\\
\implies \frac{d}{dx} (y' \frac{\partial f}{\partial y'}) = \frac{df}{dx} - \frac{\partial f}{\partial x} - y' \frac{\partial f}{\partial y} + y' \frac{d}{dx} \frac{\partial f}{\partial y'}\\
= \frac{df}{dx} - \frac{\partial f}{\partial x} - y' (\frac{\partial f}{\partial y} - \frac{d}{dx} \frac{\partial f}{\partial y'})\\$
but $\frac{\partial f}{\partial y} - \frac{d}{dx} \frac{\partial f}{\partial y'}=0\\
\therefore \frac{\partial f}{\partial x} - \frac{d}{dx} (f- y' \frac{\partial f}{\partial y'})=0\\$
use this form when $\frac{\partial f}{\partial x} = 0\\
\implies f - y' \frac{\partial f}{\partial y'} = \rm{const}$,\,\, (if $\frac{\partial f}{\partial x} = 0)$\\


\hdashrule[0.5ex][c]{\linewidth}{0.5pt}{1.5mm}


\item \underline{$\frac{\partial f}{\partial y_i} - \frac{d}{dx} \frac{\partial f}{\partial y_i'} = 0$}$,\,\, i=1,2,\dots ,n\\
f= f\{ y_1 (x), y_1'(x), y_2(x),y_2'(x), \dots; x \}\\
or f= f \{ y_i(x), y_i'(x); x \} i = 1,2,\dots ,n\\
y_i(\alpha, x) = y_i(0,x) + \alpha \eta_i(x)\\
J= \int f\{ y_i(x), y_i'(x); x \} dx\\
\implies \frac{\partial J}{\partial \alpha} = \int_{x_1}^{x_2} \sum_i (\frac{\partial f}{\partial y_i} \frac{\partial y_i}{\partial \alpha} + \frac{\partial f}{\partial y_i'} \frac{\partial y_i'}{\partial \alpha}) dx\\
= \sum_i \int_{x_1}^{x_2}(\frac{\partial f}{\partial y_i} \eta_i(x) + \frac{\partial f}{\partial y_i'} \eta_i'(x))dx\\
u = \frac{\partial f}{\partial y_i'} du= \frac{d}{dx} \frac{\partial f}{\partial y_i'},\,\, dv = \eta_i'(x) dx,\,\, v= \eta_i(x)\\
\implies \sum_i[\int_{x_1}^{x_2} (\frac{\partial f}{\partial y_i} \eta_i(x))dx + (\frac{\partial f}{\partial_i/'} \eta_i(x)|_{x_1}^{x_2} - \int_{x_1}^{x_2} \eta_i(x) \frac{d}{dx} \frac{\partial f}{\partial y_i'} dx]\\
= \sum_i \int_{x_1}^{x_2} (\frac{\partial f}{\partial y_i} - \frac{d}{dx} \frac{\partial f}{\partial y_i'})\eta_i(x) dx\\
\therefore \frac{\partial f}{\partial y_i} - \frac{d}{dx} \frac{\partial f}{\partial y_i'} = 0\\$


\hdashrule[0.5ex][c]{\linewidth}{0.5pt}{1.5mm}


\item \underline{$
\begin{cases}
	\frac{\partial f}{\partial y} - \frac{d}{dx} \frac{\partial f}{\partial y'} + \lambda(x) \frac{\partial g}{\partial y} =0\\
	\frac{\partial f}{\partial z} - \frac{d}{dx} \frac{\partial f}{\partial z'} + \lambda(x) \frac{\partial g}{\partial z} =0
\end{cases}$} (I believe this is set up using lagrange multipliers set up a different way)\\
$g$ is constraint that $y_i, x$ must satisfy\\


\hdashrule[0.5ex][c]{\linewidth}{0.5pt}{1.5mm}


\item \underline{$\delta J = \int_{x_1}^{x_2} ( \frac{\partial f}{\partial y} - \frac{d}{dx} \frac{\partial f}{\partial y'} ) \delta y dx;$} $\delta J \equiv \frac{\partial J}{\partial \alpha} d \alpha\\
J = \int_{x_1}^{x_2} f \{ y, y' ; x \} dx\\
\delta J = \int_{x_1}^{x_2} ( \frac{\partial f}{\partial y} \delta y + \frac{\partial f}{\partial y'} \delta y' ) d\\
= \int_{x_1}^{x_2} ( \frac{\partial f}{\partial y} \delta y + \frac{d}{dx} ( \frac{\partial f}{\partial y'} \delta y) - \delta y \frac{d}{dx} \frac{\partial f}{\partial y'}) dx\\
= \int_{x_1}^{x_2} ( \frac{\partial f}{\partial y} - \frac{d}{dx} \frac{\partial f}{\partial y'}) \delta y dx$


\hdashrule[0.5ex][c]{\linewidth}{0.5pt}{1.5mm}


\section*{Chapter 7}


\item \underline{$\frac{\partial L}{\partial x_i} - \frac{d}{dt} \frac{\partial L}{\partial \dot{x}_i} = 0,$},\,\, $i=1,2,3\\$
$S = \int_{t_1}^{t_2} L dt \\
S = \int_{t_1}^{t_2} \delta (L(x_i, \dot{x}_i) dt\\
= \int_{t_1}^{t_2} ( \frac{\partial L}{\partial x_i} \delta x_i + \frac{\partial L}{\partial \dot{x}_i} \delta \dot{x}_i) dt\\
= \int_{t_1}^{t_2} [ \frac{\partial L}{\partial x_i} - \frac{d}{dt} \frac{\partial L}{\partial \dot{x}_i} ] \delta x_i dt = 0\\
\implies \frac{\partial L}{\partial x_i} - \frac{d}{dt} \frac{\partial L}{\partial \dot{x}_i} = 0\\$


\hdashrule[0.5ex][c]{\linewidth}{0.5pt}{1.5mm}


\underline{Generalized coordinates}\\
Suppose we have n particles $\implies n$ radius vectors to specify conditions $\implies 3n$ coordinates\\
If we have constraints then the amount of independent coordinates would be $s=3n - m$\\
for example, if $2$ particles were connected by rods say $1$, and $2$, then $\vec{r}_2 = \vec{r}_1 + \vec{a}$, this is $3$ constraints so $s=3m-3 = 3(m-1)$


\hdashrule[0.5ex][c]{\linewidth}{0.5pt}{1.5mm}


$\frac{\partial L}{\partial q_j} - \frac{d}{dt} \frac{\partial L}{\partial \dot{q}_j} = 0,\,\, j = 1, 2, \dots , s\\$
Lagranges EOM fro generalized coordinates\\


\hdashrule[0.5ex][c]{\linewidth}{0.5pt}{1.5mm}


\item \underline{$\frac{\partial L}{\partial q_j} - \frac{d}{dt} \frac{\partial L}{\partial \dot{q}_j} + \sum_k \lambda_k (t)
\frac{\partial f_k}{\partial q_j} = 0$}\\
Const. $f(x_{\alpha, i} , \dot{x}_{\alpha, i}; t) = c$ is non holonomic in general (non-integrable)\\
\underline{ex.} $\sum_i A_i \dot{x}_i + B = 0,\,\, i = 1,2, 3\\$
non-integrable unless $A_i = \frac{\partial f}{\partial x_i},\,\, B= \frac{\partial f}{\partial t}\\
\implies \sum_i \frac{\partial f}{\partial x_i} \dot{x}_i + \frac{\partial f}{\partial t} = \frac{df}{dt} = 0\\
\implies f(x-i, t) - const = 0\\$
i.e. this constraint can be put in the form of $f(x+_i,t) = 0\\$
in general\\
$\sum_j \frac{\partial f_k}{\partial q_j} d q_j + \frac{\partial f_k}{\partial t} dt = 0\\
\underline{recall:} \frac{\partial f}{\partial y_i} - \frac{d}{dx} \frac{\partial f}{\partial y_i'} + \sum_j \lambda_j(x) \frac{\partial g_j}{\partial y_i} = 0\\$
here, $g_j = f_j - const = 0 \implies \frac{\partial g_j}{\partial q_i} = \frac{\partial f_j}{\partial q_i}\\
\implies \frac{\partial L}{\partial q_j} - \frac{d}{dt} \frac{\partial L}{\partial \dot{q}_j} + \sum_k \lambda_k(t) \frac{\partial f_k}{\partial q_j} = 0\\$


\hdashrule[0.5ex][c]{\linewidth}{0.5pt}{1.5mm}


$Q_j = \sum_k \lambda_k \frac{\partial f_k}{\partial q_j}$ (generalized forces)


\hdashrule[0.5ex][c]{\linewidth}{0.5pt}{1.5mm}


\item \underline{$F_i = \dot{p}_i \iff \frac{d}{dt} ( \frac{ \partial L}{\partial \dot{q}_j}) - \frac{\partial L}{\partial q_j} = 0$}\\
($\impliedby$)\\
$\frac{\partial L}{\partial x_i} - \frac{d}{dt} \frac{\partial L}{\partial \dot{x}_i}=0\\
\frac{\partial (T- U)}{\partial x_i} - \frac{d}{dt} \frac{\partial ( T-U)}{\partial \dot{x}_i} = 0\\
T(\dot{x}_i) = T,\,\, U= U(x_i)\\
\implies - \frac{\partial U}{\partial x_i} - \frac{d}{dt} \frac{\partial T}{\partial \dot{x}_i} = 0\\$
but $- \frac{\partial U}{\partial x_i} = F_i$ (conservative)\\
$\frac{d}{dt} \frac{\partial T}{\partial \dot{x}_i} = \frac{d}{dt} \frac{\partial}{\partial \dot{x}_i} ( \sum_j \frac{1}{2} m \dot{x}_j^2) = \frac{d}{dt} \frac{1}{2} m \sum_j \delta_{ij} 2 \dot{x}-j\\
= \frac{d}{dt} m \dot{x}_i = \frac{dp_i}{dt}\\
\implies F_i = \dot{p}_i$\\
($\implies$)\\
$x_i - x_i(q_j,t) \implies \dot{x}_i = \sum_j \frac{\partial x_i}{\\partial q_j} \dot{q}_j + \frac{\partial x_i}{\partial t}\\
\frac{\partial \dot{x}_i}{\partial \dot{q}_j} = \frac{\partial x_i}{\partial q_j};\,\, p_j = \frac{\partial T}{\partial \dot{q}_j}\\
\delta W = \sum_i F_i \delta x_i = \sum_i F_i ( \sum_j \frac{\partial x_i}{\partial q_j} \delta q_j)\\
= \sum_{i,j} F_i \frac{\partial x_i}{\partial q_j} \delta q_j \equiv \sum_j Q_j \delta q_j\\
Q_j = \sum_i F_i \frac{\partial x_i}{\partial q_j}$ (generalized force)\\
$Q_j = - \frac{\partial U}{\partial q_j}\\$
using these facts lets prove theorem\\
$p_j = \frac{\partial T}{\partial \dot{q}_j} = \frac{\partial}{\partial \dot{q}_j}( \sum_i \frac{1}{2} m \dot{x}_I^2)\\
= \frac{1}{2} m \sum_i \frac{\partial \dot{x}_i}{\partial \dot{q}_j} 2 \dot{x}_i = m \sum_i \dot{x}_i \frac{\partial x_i}{\partial q_j}\\
\implies \dot{p}_j = m \sum_i ( \ddot{x}_i \frac{\partial x_i}{\partial q_j} + \dot{x}_i \frac{d}{dt} \frac{\partial x_i}{\partial q_j})\\
\frac{d}{dt} ( \frac{\partial x_i}{\partial q_j}) = \sum_k \frac{\partial^2 x_i}{\partial q_k \partial q_j} \dot{q}_k + \frac{\partial^2 x_i}{\partial t \partial q_j}\\
\implies \dot{p}_j = m \sum_i \ddot{x}_i \frac{\partial x_i}{\partial q_j} + m \sum_i \dot{x}_i(\sum_k \frac{\partial^2 x_i}{\partial q_k \partial q_j} \dot{q}_k + \frac{\partial^2 x_i}{\partial t \partial q_j})\\
= m \sum_i \ddot{x}_i \frac{\partial x_i}{\partial q_J} + \sum_{i,k} m \dot{x}_i \frac{\partial^2 x_i}{\partial q_k \partial q_j} \dot{q}_k + \sum_i m \dot{x}_i \frac{\partial^2 x_i}{\partial t \partial q_j}\\$
$\frac{\partial T}{\partial q_j} = \frac{\partial}{\partial q_j} \sum_i \frac{1}{2} m \dot{x}_i^2 = \sum_i m \frac{\partial \dot{x}_i}{\partial q_j} \dot{x}_i\\
= \sum_i m \dot{x}_i \frac{\partial}{\partial q_j} ( \sum_k \frac{\partial x_i}{\partial q_k} \dot{q}_k + \frac{\partial x_i}{\partial t}\\
\implies \dot{p}_j = Q_j + \frac{\partial T}{\partial q_j}\\
\underline{recall:} Q_j = - \frac{\partial U}{\partial q_j} ;\,\, p_j = \frac{\partial T}{\partial \dot{q}_j}\\
\implies \frac{d}{dt} ( \frac{\partial T}{\partial \dot{q}_j}) - \frac{\partial T}{\partial q_j} = - \frac{\partial U}{\partial q_j}\\
\implies \frac{d}{dt}( \frac{\partial ( T- U)}{\partial \dot{q}_j} ) - \frac{\partial (T- U)}{\partial q_j} = 0\\
\therefore \frac{d}{dt} ( \frac{\partial L}{\partial \dot{q}_j}) - \frac{\partial L}|{\partial q_j} = 0\\$


\hdashrule[0.5ex][c]{\linewidth}{0.5pt}{1.5mm}


\item \underline{$T = \sum_{j,k} a_{jk} \dot{q}_j \dot{q}_k;\,\, \sum_{\ell} \dot{q}_{\ell} \frac{\partial T}{\partial \dot{q}_\ell} = 2 T$}\\
$T = \frac{1}{2} \sum_{\alpha = 1}^n \sum_{\i=1}^3 m_{\alpha} \dot{x}_{\alpha, i}^2\\
x_{\alpha, i} = x_{\alpha, i} ( q_j, t),\,\, j= 1,2, \dots, s\\
\dot{x}_{\alpha, i} = \sum_{j=1}^s \frac{\partial x_{\alpha, i}}{\partial q_j} \dot{q}_j + \frac{\partial x_{\alpha, i}}{\partial t}\\
\dot{x}_{\alpha, i}^2 = ( \sum_j \frac{\partial x_{\alpha, i}}{\partial q_j} \dot{q}_j + \frac{\partial x_{\alpha,i}}{\partial t}) (\sum_k \frac{\partial x-{\alpha, i}}{\partial q_k} \dot{q}_k + \frac{\partial x_{\alpha, i}}{\partial t})\\
= \sum_{j,k} \frac{\partial x_{\alpha, i}}{\partial q_j} \frac{\partial x_{\alpha, j}}{\partial q_k} \dot{q}_j \dot{q}_k + 1 \sum_j \frac{\partial x_{\alpha, i}}{\partial q_j} \frac{\partial x_{\alpha, i}}{\partial t} \dot{q}_j + ( \frac{\partial x_{\alpha, i}}{{\partial t})^2\\
T = \sum_{\alpha} \sum_{i,j,k} \frac{1}{2} m_{\alpha} \frac{\partial x_{\alpha, i}}{\partial q_j} \frac{\partial x_{\alpha, i}}{\partial q_k} \dot{q}_j \dot{q}_k\\
+ \sum_[\alpha} \sum_{i,j} m_{\alpha} \frac{\partial x_{\alpha, i}}{\partial q_j} \frac{\partial x_{\alpha, i}}{\partial t} \dot{q}_j + \sum_{\alpha} \sum_i \frac{1}{2} m_{\alpha} ( \frac{\partial x_{\alpha, i}}{\partial t})^2\\
\implies T = \sum_{j, k} a_{jk} \dot{q}_j \dot{q}_k + \sum_j b_j \dot{q}_j + c\\
scleronomic \implies no explicit t dependence\\
\implies \frac{\partial x_{\alpha, i}}{\partial t} = 0,\,\, b_j,\,\, c=0\\
\therefore T = \sum_{j,k} a_{jk} \dot{q}_j \dot{q}_k Notice similarity with T = \sum_i \frac{1}{2} m \dot{x}_i^2\\
\implies \frac{\partial T}{\partial \dot{q}_{\ell}} = \sum_{j,k} a_{jk} \delta_{j \ell} \dot{q}_k + \sum_{j, k} a_{jk} \dot{q}_j \delta_{k \ell} = \sum_k a_{\ell k} \dot{q}_k + \sum_j a_{j \ell} \dot{q}_j\\ \\
\implies \sum_{\ell} \dot{q}_{\ell} \frac{\partial T}{\partial \dot{q}_{\ell}} = \sum_{k, \ell} a_{\ell k} \dot{q}_k \dot{q}_{\ell} + \sum_j a_{j \ell} \dot{q}_j \dot{q}_{\ell}\\
= 2 \sum_{j, k} a_{jk} \dot{q}_j \dot{q}_k = 2 T\\$


\hdashrule[0.5ex][c]{\linewidth}{0.5pt}{1.5mm}


\item \underline{$H = T + U = constant$}\\
isolated system $\implies \frac{\partial L}{\partial t} = 0,\,\, L(q_j, \dot{q}_j)\\
\implies \frac{d L}{d t} = \sum_j \frac{\partial L}{\partial q_j} \dot{q}_j + \sum_j \frac{\partial L}{\partial \dot{q}_j} \ddot{q}_j\\$
\underline{recall:} $\frac{\partial L}{\partial q_j} = \frac{d}{dt} \frac{\partial L}{\partial \dot{q}_j}\\
\implies \frac{dL}{dt} = \sum_j \frac{d}{dt} ( \frac{\partial L}{\partial \dot{q}_j}) \dot{q}_j + \sum_j \frac{\partial L}{\partial \dot{q}_j} \ddot{q_j}= \sum_j \frac{d}{dt} ( \frac{\partial L}{\partial \dot{q}_j} \dot{q}_j)\\
\implies \frac{d}{dt} ( L - \sum_j \dot{q}_j \frac{\partial L}{\partial \dot{q}_j})=0\\
\implies L - \sum_j \dot{q}_j \frac{\partial L}{\partial \dot{q}_j} = - H =$ constant ( definition of hamiltonian)\\
$U= U(q_j) \implies \frac{\partial U}{\partial \dot{q}_j} = 0\\
\implies \frac{\partial L}{\partial \dot{q}_j} = \frac{\partial T}{\partial \dot{q}_j}\\
\implies ( T- U) - \sum_j \dot{q}_j. \frac{\partial T}{\partial \dot{q}_j} = - H\\$
\underline{recall:} $2 T = \sum_j \dot{q}_j \frac{\partial T}{\partial \dot{q}_j}\\
\therefore ( T-U) - 2 T =- H \implies H = T + U = const.\\$
This holds if $U = U(q_j) and x = x(q_j)$ ( not time dependent.)\\


\hdashrule[0.5ex][c]{\linewidth}{0.5pt}{1.5mm}


\item \underline{$p_i =$ linear momentum = const.}\\
$L = L(x_i, \dot{x}_i),\,\, \delta \vec{r} = \sum_i \delta x_i \hat{e}_i\\
\implies \delta L = \sum_i \frac{\partial L}{\partial x_i} \delta x_i + \sum_i \frac{\partial L}{\partial \dot{x}_i} \delta \dot{x}_i = 0\\$
varied displacement $\implies \delta x_i$ independent of time\\
$\implies \delta \dot{x}_i = \frac{d}{dt} \delta x_i = 0\\
\implies \delta L = \sum_i \frac{\partial L}{\partial x_i} \delta x_i = 0\\
\implies \frac{\partial L}{\partial x_i} = 0 \implies \frac{d}{dt} \frac{\partial L}{\partial \dot{x}_i} = 0\\
\implies \frac{\partial L}{\partial \dot{x}_i} =$ constant.\\
$\implies \frac{\partial L}{\partial \dot{x}_j} = \frac{\partial T}{\partial \dot{x}_i} = p_i = m \dot{x}_ i =$ constant.\\


\hdashrule[0.5ex][c]{\linewidth}{0.5pt}{1.5mm}


\item \underline{$\vec{r} \times \vec{p} =$ constant} (conservation of angular momentum)\\
\underline{recall:} $\vec{v} = \vec{\omega} \times \vec{r} \implies \delta \vec{r} = \delta \vec{\theta} \times \vec{r}\\
\delta \theta$ is our varied displacement i.e. $\frac{d}{dt} \delta \theta = 0\\
\implies \delta \dot{\vec{r}} = \delta \vec{\theta} \times \dot{\vec{r}}\\
\delta L = \sum_i \frac{\partial L}{\partial x_i} \delta x_i + \sum_i \frac{\partial L}{\partial \dot{x}_i } \delta \dot{x}_i = 0\\
\frac{\partial L}{\partial x_i} = - \frac{\partial U}{\partial x_i} = \dot{p}_i;\,\, \frac{|partial L}{\partial \dot{x}_i} = \frac{\partial T}{\partial \dot{x}_i} = p_i\\
\implies \sum_i \dot{p}_i \delta x_i + \sum_i p_i \delta \dot{x}_i = 0\\
\implies \dot{\vec{p}} \cdot \delta \vec{r} + \vec{p} \cdot \delta \dot{\vec{r}} = 0\\
\implies \dot{\vec{p}} \cdot ( \delta \vec{\theta} \times \vec{r}) + \vec{p} \cdot \delta \vec{\theta} \times \dot{\vec{r}}\\$
\underline{recall:} $\vec{A} \cdot ( \vec{B} \times \vec{C}) = \vec{B} \cdot ( \vec{A} \times \vec{C})\\
\implies \delta \vec{\theta} \cdot ( \dot{\vec{p}} \times \vec{r}) + \delta \theta \cdot ( \vec{p} \times \dot{\vec{r}})\\
= \delta \vec{\theta} \cdot ( \dot{\vec{p}} \times \vec{r} + \vec{p} \times \dot{\vec{r}}) = 0 \implies \frac{d}{dt} ( \vec{r} \times \vec{p}) =0 \implies \vec{r} \times \vec{p} =$ const.



\hdashrule[0.5ex][c]{\linewidth}{0.5pt}{1.5mm}


$p_i = \frac{\partial L}{\partial \dot{x}_i} \implies p_i = \frac{\partial L}{\partial \dot{q}_i}$ (generalized momenta)\\
$\implies \frac{d}{dt} \frac{\partial L}{\partial \dot{q}_i} = \dot{p}_i = \frac{\partial L}{\partial q_i}$


\hdashrule[0.5ex][c]{\linewidth}{0.5pt}{1.5mm}


\item \underline{$H(q_k, p_k, t) = \sum_j p_j \dot{q}_j - L(q_k, \dot{q}_k, t)$}\\
may solve $p_j = \frac{\partial L}{\partial \dot{q}_j} for \dot{q}_j = \dot{q}_j ( q_k, p_k,t)\\
L(q_i, \dot{q}_i, t) \implies \int p_j d \dot{q}_j = L(q_i , \dot{q}_i, t)\\$
which can be solved for $\dot{q}_j$\\
$\implies \dot{q}_i = \dot{q}_i( q_k, p_k, t)\\$


\hdashrule[0.5ex][c]{\linewidth}{0.5pt}{1.5mm}


\item \underline{$\dot{q}_k = \frac{\partial H}{\partial p_k};\,\, - \dot{p}_k = \frac{\partial H}{\partial q_k}$}\\
$H = H(q_k, p_k, t);\,\, L= L(q_k, \dot{q}_k,t)\\
dH = \sum_i \frac{\partial H}{\partial q_i} d q_i + \sum_i \frac{\partial H}{\partial p_i} d p_i + \frac{\partial H}{\partial t} dt\\
H = \sum_j p_j \dot{q}_j - L( q_k, \dot{q}_k, t)\\
\implies \frac{\partial H}{\partial q_i} = - \frac{\partial L}{\partial q_i} = - \frac{d}{dt} p_i = - \dot{p}_i\\
\frac{\partial H}{\partial p_i} = \sum_j \delta_{ij} \dot{q}_j - L = \dot{q}_i\\
\implies dH = \sum_i( - \dot{p}_i dq_i + \dot{q}_i dp_i) - \frac{\partial L}{\partial t} dt\\
= \sum_i( \frac{\partial H}{\partial q_i} dq_i + \frac{\partial H}{\partial p_i} dp_i) - \frac{\partial L}{\partial t}\\
\implies \begin{cases} - \dot{p}_i = \frac{\partial H}{\partial q_k}\\
\dot{q}_k = \frac{\partial H}{\partial p_k} \end{cases}\\
- \frac{\partial L}{\partial t} = \frac{\partial H}{\partial t}$


\hdashrule[0.5ex][c]{\linewidth}{0.5pt}{1.5mm}


\underline{ $\int_{t_1}^{t_2} ( \frac{\partial L}{\partial q_j} - \frac{d}{dt} \frac{\partial L}{\partial \dot{q}_i}) \delta q_j dt = 0$}\\
$\delta \int_{t_1}^{t_2} L(q_j, \dot{q}_j, t) dt = 0\\
\implies \int_{t_1}^{t_2} ( \frac{\partial L}{\partial q_j} \delta q_j + \frac{\partial L}{\partial \dot{q}_j} \delta \dot{q}_j) dt = 0\\
\delta \dot{q}_j = \frac{d}{dt} \delta q_j\\
\implies \int_{t_1}^{t_2} ( \frac{\partial L}{\partial q_j} \delta q_j + \frac{\partial L}{\partial \dot{q}_j} \frac{d}{dt} \delta q_j) dt = 0\\
\implies \int_{t_1}^{t_2} ( \frac{\partial L}{\partial q_j} \delta q_j + \frac{d}{dt} ( \frac{\partial L}{\partial \dot{q}_j} \delta q_j) - \frac{d}{dt} ( \frac{\partial L}{\partial \dot{q}_j}) \delta q_j) dt\\
= \int_{t_1}^{t_2} ( \frac{\partial L}{\partial q_j} - \frac{d}{dt} \frac{\partial L}{\partial \dot{q}_j} ) \delta q_j = 0\\$


\hdashrule[0.5ex][c]{\linewidth}{0.5pt}{1.5mm}


\item \underline{$\int_{t_1}^{t_2} \sum_j \{ ( \dot{q}_j - \frac{\partial H}{\partial p_j} ) \delta p_j - ( \dot{p}_j + \frac{\partial H}{\partial q_j}) \delta q_j \} dt = 0$}\\
$L = \sum_j p_j \dot{q}_j - H( q_j, p_j, t)\\
\delta \int_{t_1}^{t_2} L d t = \delta \int_{t_1}^{t_2}( \sum_j p_j \dot{q}_j - H) dt = 0\\
\implies \int_{t_1}^{t_2} \sum_j ( p_j \delta \dot{q}_j + \dot{q}_j \delta p_j - \frac{\partial H}{\partial q_j} \delta q_j - \frac {\partial H}{\partial p_j} \delta p_j) dt = 0\\
\implies \int_{t_1}^{t_2} \sum_j ( - \dot{p}_j \delta q_j + \dot{q}_j \delta p_j - \frac{\partial H}{\partial q_j} \delta q_j - \frac{\partial H}{\partial p_j} \delta p_j) dt = 0\\
\implies \int_{t_1}^{t_2} \sum_j( - ( \dot{p}_j + \frac{\partial H}{\partial q_j}) \delta q_j + ( \dot{q}_j - \frac{\partial H}{\partial p_j}) \delta p_j) dt = 0$\\


\hdashrule[0.5ex][c]{\linewidth}{0.5pt}{1.5mm}


\item \underline{$\frac{d \rho}{dt} = 0$}\\
$N = \rho dV,\,\, \rho \sim$ density in phase space\\
$dV = d q_1 dq_2 \dots d q_s dp_1 dp_2 \dots dp_s\\$
this is in the $q_k-p_k \rightarrow x-y$ plane\\
consider a rectangle\\
number crossing left side $\implies \rho dq_k dp_k\\$
$\implies$ number crossing left side/dt $\implies \rho \frac{dq_k}{dt} dp_k = \rho \dot{q}_k dp_k\\$
Lower edge $\implies \rho \frac{dp_k}{dt} dq_k = \rho \dot{p}_k dq_k\\$
\# moving into rectangle/unit time\\
$= \rho( \dot{q}_k dp_k + \dot{p}_k dq_k)\\$
\# moving out is the same( taylor exp. $\rho \dot{q}_k$ and $\rho \dot{p}_k)\\$
$=\rho[q_k+dq_k]( \dot{q}_k dp_k ) + \rho[p_k+dp_k](\dot{p}_k dq_k)= [ \rho \dot{q}_k + \frac{\partial}{\partial q_k} ( \rho \dot{q}_k) d q_k] dp_k + [ \rho \dot{p}_k + \frac{\partial}{\partial p_k}( \rho \dot{p}_k) d p_k ] d q_k\\$
why $dq_k$ and $dp_k?\\$
total increase is the difference\\
$\implies \rho( \dot{q}_k d p_k + \dot{p}_k d q_k) - [ \rho \dot{q}_k + \frac{\partial}{\partial q_k} ( \rho \dot{q}_k) d q_k] dp_k\\
- [ \rho \dot{p}_+k + \frac{\partial}|{\partial p_k} ( \rho \dot{p}_k) dp_k] dq_k,\,\,$ total increase in density\\
$\implies -[ \frac{\partial}{\partial q_k} ( \rho \dot{q}_k) + \frac{\partial}{\partial p_k}( \rho \dot{p}_k)] dq_k dp_k = \frac{\partial \rho}{\partial t} dq_k dp_k\\
\implies \frac{\partial \rho}{\partial t} + \sum_{k=1}^s ( \frac{\partial \rho}{\partial q_k} \dot{q}_k + \rho \frac{\partial \dot{q}_k}{\partial q_k} + \frac{\partial \rho}{\partial p_k} \dot{p}_k + \rho \frac{\partial \dot{p}_k}{\partial p_k}) = 0\\
\underline{recall:} \dot{q}_k = \frac{\partial H}{\partial p_k};\,\, - \dot{p}_k = \frac{\partial H}{\partial q_k} \implies \frac{\partial \dot{q}_k}{\partial q_k} + \frac{\partial \dot{p}_k}{\partial p_k} = \frac{\partial^2 H}{\partial q_k \partial p_k} - \frac{\partial^2 H}{\partial q_k \partial p_k} = 0\\
\implies \frac{\partial \rho}{\partial t} + \sum_{k=1}^s ( \frac{\partial \rho}{\partial q_k} \dot{q}_k + \frac{\partial \rho}{\partial p_k} \dot{p}_k) = 0\\
\therefore \frac{d \rho}{dt} = 0$


\hdashrule[0.5ex][c]{\linewidth}{0.5pt}{1.5mm}


\item \underline{$\langle T \rangle = - \frac{1}{2} \langle \sum_{\alpha} \vec{F}_{\alpha} \cdot \vec{r}_{\alpha} \rangle =$ Virial} (Virial Theorem)\\
$S \equiv \sum_{\alpha} \vec{p}_{\alpha} \cdot \vec{r}_{\alpha}\\
\implies \frac{d S}{dt} = \sum_{\alpha} ( \dot{\vec{p}}_{\alpha} \cdot \vec{r}_{\alpha} + \vec{p}_{\alpha} \cdot \dot{\vec{r}}_{\alpha})\\
\langle \frac{d S}{dt} \rangle = \frac{1}{\tau} \int_0^{\tau} \frac{d S}{dt} dt = \frac{S(\tau) - S(0)}{\tau}\\
S$ periodic $\implies \langle \dot{S} \rangle = 0\\$
If not $\implies S$ bounded $\implies \frac{S(\tau) - S(0)}{\tau},\,\, \tau \rightarrow \infty \implies \langle dot{S} \rangle \rightarrow 0\\
\implies \langle \sum_{\alpha} \vec{p}_{\alpha} \cdot \dot{\vec{r}}_{\alpha} \rangle = 0 \langle \sum_{\alpha} \dot{\vec{p}}_{\alpha} \cdot \vec{r}_{\alpha} \rangle\\
T_{\alpha} = \frac{1}{2} m_{\alpha} v_{\alpha}^2 = \frac{1}{2} \vec{p}_{\alpha} \cdot \dot{\vec{r}}_{\alpha} \implies \vec{p}_{\alpha} \cdot \dot{\vec{r}}_{\alpha} = 2 T_{\alpha}\\
\implies \langle 2 \sum_{\alpha} T_{\alpha} \rangle = - \langle \sum_{\alpha} \vec{F}_{\alpha} \cdot \dot{\vec{r}}_{\alpha} \rangle\\
\therefore \langle T \rangle = - \frac{1}{2} \langle \sum_{\alpha} \vec{F}_{\alpha} \cdot \vec{r}_{\alpha} \rangle\\
$

\hdashrule[0.5ex][c]{\linewidth}{0.5pt}{1.5mm}


\item \underline{$\langle T \rangle = \frac{(n+1)}{2} \langle U \rangle$}\\
$\langle T \rangle = \frac{1}{2} \langle \sum_{\alpha} \vec{r}_{\alpha} \cdot \nabla U_{\alpha} \rangle;\,\, \vec{F}_{\alpha} = - \nabla U_{\alpha};\,\, F \propto r^n\\
U = k r^{n+1}\\
\implies \vec{r} \cdot \nabla U = \frac{dU}{dr} = k(n+1) r^{n+1} = ( n+1)U\\
\therefore \langle T \rangle = \frac{n+1}{2} \langle U \rangle$


\hdashrule[0.5ex][c]{\linewidth}{0.5pt}{1.5mm}


\section*{Chapter 8?}


\item \underline{$L = \frac{1}{2 \mu} | \dot{r}|^2 - U(r);\,\, \mu \equiv \frac{m_1 m_2}{m_1 + m_2}$}\\
$r= | \vec{r}_1 - \vec{r}_2| \implies U( \vec{r}_1, \vec{r}_2) = U(r)\\
\implies L = \frac{1}{2} m_1 | \dot{\vec{r}}_1|^2 + \frac{1}{2} m_2 | \dot{\vec{r}}_2|^2 - U(r)\\
choose \vec{R} = 0 \implies m_1 \vec{r}_1 + m_2 \vec{r}_2 =0\\
\vec{r} = \vec{r}_1 - \vec{r}_2 \implies m_1 ( \vec{r} +. \vec{r}_2) + m_2 \vec{r}_2 = 0\\
\implies \vec{r}_2 ( m_1 + m_2 ) = - m_1 \vec{r} \implies \vec{r}_2 = - \frac{m_1 \vec{r}}{m_1 + m_2}\\
likewise \vec{r}_1 = \frac{m_2}{m_1 + m_2} \vec{r}\\
\implies L = \frac{1}{2} m_1 ( \frac{m_2}{m_1 + m_2})^2 | \dot \vec{r}|^2 + \frac{1}{2} m_2 ( \frac{m_1}{m_1 + m_2})^2 | \dot{\vec{r}}|^2 - U(r)\\
= \frac{1}{2} \frac{m_1 m_2^2 + m_2 m_1^2}{( m_1 + m_2)^2} | \dot{\vec{r}}|^2 - U(r)\\
= \frac{1}{2} \frac{m_1 m_2}{m_1 + m_2} | \dot{\vec{r}}|^2 - U(r) = \frac{1}{2} \mu | \dot{\vec{r}}|^2 - U(r)\\
| \dot{\vec{r}}|^2 = \dot{r}^2 + r^2 \dot{\theta}^2\\
\implies L = \frac{1}{2 \mu} ( \dot{r}^2 + r^2 \dot{\theta}^2) - U(r)\\$


\hdashrule[0.5ex][c]{\linewidth}{0.5pt}{1.5mm}


\item \underline{$ \ell = \mu r^2 \dot{\theta} = const.$}\\
angular symmetry,$\,\, \theta \rightarrow \theta + \delta \theta\\
\implies \vec{L} = \vec{r} \times \vec{p} = const\\
\implies \dot{p}_{\theta} = \frac{\partial L}{\partial \theta} = 0 = \frac{d}{dt} \frac{\partial L}{\partial \dot{\theta}} = \frac{d}{dt} ( \mu r^2 \dot{t\theta}) = \mu r^2 \ddot{\theta} = 0\\
\implies \mu r^2 \dot{\theta} = const. = \ell\\$


\hdashrule[0.5ex][c]{\linewidth}{0.5pt}{1.5mm}


\item \underline{$\tilde{\vec{B}}_0 = \frac{k}{\omega} ( \hat{z} \times \tilde{\vec{E}}_0)$}\\
$\tilde{\vec{E}}(z,t) = \tilde{\vec{E}}_0 e^{i(kz- \omega t)};\,\, \tilde{\vec{B}}(z,t) = \tilde{\vec{B}}_0 e^{i( kz- \omega t)} (monochromatic plane waves)\\
\nabla \cdot \vec{E} = 0 \implies ( \tilde{\vec{E}}_0)_z i k e^{i ( kz- \omega t)} = 0 \implies \tilde{\vec{E}}_0)_z = 0\\
\nabla \cdot \vec{B} = 0 \implies ( \tilde{B}_0)_z=0\\
\implies$ electromagnetic waves are transverse\\
$\nabla \times \vec{E} = - \frac{\partial \vec{B}}{\partial t}\\
\implies \begin{vmatrix} \hat{x} & \hat{y} & \hat{z} \\ \partial_x & \partial_y & \partial_z \\ \tilde{E}_{0x} e^{\sim} & \tilde{E}_{0y} e^{\sim} & \tilde{E}_{0z} e^{\sim} \end{vmatrix} = - \tilde{E}_{0y} i k e^{\sim} \hat{x} + i k \tilde{E}_{0x} e^{\sim} \hat{y} + 0 \hat{z} = i \omega \tilde{\vec{B}} e^{\sim}\\
\implies - \tilde{E}_{0y} k = \omega \tilde{B}_{0x};\,\, k \tilde{E}_{0x} = \omega \tilde{B}_{-y}\\
\implies \omega \tilde{\vec{B}}_0 = - \tilde{\vec{E}}_0 \times \hat{z} k \implies \tilde{\vec{B}}_0 = \frac{k}{\omega} \hat{z} \times \tilde{\vec{E}}_0$


\section*{Chapter 9}
\underline{Newton's Third Law}
1. $\vec{f}_{\alpha \beta} = - \vec{f}_{\beta \alpha} ( \vec{f}_{\alpha \beta}$ the force on $\alpha$ due to $\beta$)\\
2. The forces must lie on a straight line joining the two particles\\


\hdashrule[0.5ex][c]{\linewidth}{0.5pt}{1.5mm}


$M = \sum_{\alpha} m_{\alpha}\\
\vec{R} = \frac{1}{M} \sum_{\alpha} m_{\alpha} \vec{r}_{\alpha}$ (center of mass)\\
$\vec{R} = \frac{1}{M} \int \vec{r} d m$


\hdashrule[0.5ex][c]{\linewidth}{0.5pt}{1.5mm}


$\vec{F}_{\alpha}^{(e)} \sim$ resultant force on $\alpha$ external to system\\
$\vec{f}_{\alpha} = \sum_{\beta} \vec{f}_{\alpha \beta} \sim$ resultant of internal forces 


\hdashrule[0.5ex][c]{\linewidth}{0.5pt}{1.5mm}


\item \underline{$M \ddot{\vec{R}} = \vec{F}$}\\
$\vec{F}_{\alpha} = \vec{F}_{\alpha}^{(e)} + \vec{f}_{\alpha}\\
\vec{f}_{\alpha \beta} = - \vec{f}_{\beta \alpha}\\
\vec{F}_{\alpha} = \dot{\vec{p}}_{\alpha} = m_{\alpha} \dddot{\vec{r}}_{\alpha} = \vec{F}_{\alpha}^{(e)} + \vec{f}_{\alpha}\\$
or\\
$\frac{d^2}{dt^2} (m_{\alpha} \vec{r}_{\alpha}) = \vec{F}_{\alpha}^{(e)} + \sum_{\beta} \vec{f}_{\alpha \beta}\\
\implies \frac{d^2}{dt^2} \sum_{\alpha} m_{\alpha} \vec{r}_{\alpha} = \sum_{\alpha} \vec{F}_{\alpha}^{(e)} + \sum_{\alpha} \sum_{\beta \neq \alpha} \vec{f}_{\alpha \beta};\,\,$ since $\vec{f}_{\alpha \alpha} = 0\\
\sum_{\alpha} \vec{F}_{\alpha}^{(e)} \equiv \vec{F};\,\, \sum_{\alpha} m_{\alpha} \vec{r}_{\alpha} = M \vec{R}\\
\sum_{\alpha} \sum_{\beta \neq \alpha} \vec{f}_{\alpha \beta} = \sum_{\alpha, \beta \neq \alpha} \vec{f}_{\alpha \beta} = \sum_{\beta, \alpha \neq \beta} \vec{f}_{\beta \alpha} = \sum_{\alpha , \beta \neq \alpha} \vec{f}_{\beta \alpha} = - \sum_{\alpha, \beta \neq \alpha} \vec{f}_{\alpha \beta}\\
\implies \sum_{\alpha, \beta \neq \alpha} \vec{f}_{\alpha \beta} = 0\\
\therefore M \ddot{\vec{R}} = \vec{F}$


\hdashrule[0.5ex][c]{\linewidth}{0.5pt}{1.5mm}


\item \underline{$\dot{\vec{P}} = M \ddot{\vec{R}} = \vec{F}$}\\
$\vec{P} = \sum_{\alpha} m_{\alpha} \dot{\vec{r}}_{\alpha} = \frac{d}{dt} \sum_{\alpha} \vec{r}_{\alpha} m_{\alpha} = M \dot{\vec{R}}\\
\implies \dot{\vec{P}}= M \ddot{\vec{R}}\\$


\hdashrule[0.5ex][c]{\linewidth}{0.5pt}{1.5mm}


\item \underline{$\vec{L} = \vec{R} \times \vec{P} + \sum_{\alpha} \vec{r}_{\alpha}' \times \vec{p}_{\alpha}'$}\\
$\vec{R} \sim$ center of mass, $\vec{r}_{\alpha}' \sim $ location of alpha particle from $\vec{R}$\\
$\vec{r}_{\alpha} \sim $ location of $\alpha$ from coordinate system\\
$\implies \vec{r}_{\alpha} = \vec{R} + \vec{r}_{\alpha}'\\
\vec{L}_{\alpha} = \vec{r}_{\alpha} \times \vec{p}_{\alpha}\\
\implies \vec{L} = \sum_{\alpha} \vec{L}_{\alpha} = \sum_{\alpha} \vec{r}_{\alpha} \times \vec{p}_{\alpha} = \sum_{\alpha} ( \vec{r}_{\alpha} \times m_{\alpha} \dot{\vec{r}}_{\alpha})\\
= \sum_{\alpha} ( \vec{R} + \vec{r}_{\alpha}') \times m_{\alpha} ( \dot{\vec{R}} + \dot{\vec{r}}_{\alpha}')\\
= \sum_{\alpha} m_{\alpha} [( \vec{R} \times \dot{\vec{R}}) + ( \vec{R} \times \dot{\vec{r}}_{\alpha}') + ( \vec{r}_{\alpha}' \times \dot{\vec{R}}) + ( \vec{r}_{\alpha}' \times \dot{\vec{r}}_{\alpha}')]\\
\sum_{\alpha} m_{\alpha} ( \vec{r}_{\alpha}' \times \dot{\vec{r}}) + \sum_{\alpha} m_{\alpha} ( \vec{R} \times \dot{\vec{r}}_{\alpha}')\\
= ( \sum_{\alpha} m_{\alpha} \vec{r}_{\alpha}') \times \dot{\vec{R}} + \vec{R} \times ( \sum_{\alpha} m_{\alpha} \dot{\vec{r}}_{\alpha}')$\\
but $\vec{r}_{\alpha}' = \vec{r}_{\alpha} - \vec{R}\\
\implies \sum_{\alpha} m_{\alpha}( \vec{r}_{\alpha} - \vec{R}) = \sum_{\alpha} m_{\alpha} \vec{r}_{\alpha} - \sum_{\alpha} m_{\alpha} \vec{R} = M \vec{R} - M \vec{R} = 0\\
\implies \vec{L} = M \vec{R} \times \dot{\vec{R}} + \sum_{\alpha} \vec{r}_{\alpha}' \times \vec{p}_{\alpha}' = \vec{R} \times \vec{P} + \sum_{\alpha} \vec{r}_{\alpha}' \times \vec{p}_{\alpha}'\\$
total angular momentum about an origin is angular momentum of CM about origin and angular momentum about CM.


\hdashrule[0.5ex][c]{\linewidth}{0.5pt}{1.5mm}


\item \underline{$\dot{\vec{L}} = \vec{N}^{(e)}$}\\
$\vec{L}_{\alpha} = \vec{r}_{\alpha} \times \vec{p}_{\alpha} \implies \dot{\vec{L}}_{\alpha} = \dot{\vec{r}}_{\alpha} \times \vec{p}_{\alpha} + \vec{r}_{\alpha} \times \dot{\vec{p}}_{\alpha} = \vec{r}_{\alpha} \times \dot{\vec{p}}_{\alpha}\\$
\underline{recall:} $\dot{\vec{p}}_{\alpha} = \vec{F}_{\alpha}^{(e)} + \vec{f}_{\alpha};\,\, \vec{f}_{\alpha} = \sum_{\beta} \vec{f}_{\alpha \beta}\\
\dot{\vec{L}}_{\alpha} = \vec{r}_{\alpha} \times ( \vec{F}_{\alpha}^{(e)} + \sum_{\beta} \vec{f}_{\alpha \beta})\\
\implies \dot{\vec{L}} = \sum_{\alpha} \dot{\vec{L}}_{\alpha} = \sum_{\alpha} \vec{r}_{\alpha} \times \vec{F}_{\alpha}^{(e)} + \sum_{\alpha, \beta \neq \alpha} \vec{r}_{\alpha} \times \vec{f}_{\alpha \beta}\\
\sum_{\alpha, \beta \neq \alpha} \vec{f}_{\alpha \beta} = \sum_{\alpha < \beta} \vec{r}_{\alpha} \times \vec{f}_{\alpha \beta} + \sum_{\alpha> \beta} \vec{r}_{\alpha} \times \vec{f}_{\alpha \beta}\\
= \sum_{\alpha< \beta} ( \vec{r}_{\alpha} \times \vec{f}_{\alpha \beta} + \vec{r}_{\beta} \times \vec{f}_{\beta \alpha} );\,\, \vec{r}_{\alpha \beta} \equiv \vec{r}_{\alpha} - \vec{r}_{\beta}\\
\implies \sum_{\alpha, \beta \neq \alpha} ( \vec{r}_{\alpha} \times \vec{f}_{\alpha \beta}) = \sum_{\alpha < \beta} [ ( \vec{r}_{\alpha} \times \vec{f}_{\alpha \beta}) - ( \vec{r}_{\beta} \times \vec{f}_{\alpha \beta})]\\
= \sum_{\alpha< \beta} ( \vec{r}_{\alpha} - \vec{r}_{\beta}) \times \vec{f}_{\alpha \beta} = \sum_{\alpha < \beta} ( \vec{r}_{\alpha \beta} \times \vec{f}_{\alpha \beta})\\$
only consider central internal forces\\
$\implies \vec{f}_{\alpha \beta}$ is in the same direction as  $\pm \vec{r}_{\alpha \beta}\\
\implies \vec{r}_{\alpha \beta} \times \vec{f}_{\alpha \beta} = 0\\
\implies \dot{\vec{L}} = \sum_{\alpha} [ \vec{r}_{\alpha} \times \vec{F}_{\alpha}^{(e)}] = \sum_{\alpha} \vec{N}_{\alpha}^{(e)} = \vec{N}^{(e)}\\$
V. If net resultant external torques about a given axis vanish $\implies$ angular momentum is conserved\\


\hdashrule[0.5ex][c]{\linewidth}{0.5pt}{1.5mm}



\underline{Note:} $\sum_{\beta} \vec{r}_{\alpha} \times \vec{f}_{\alpha \beta} = \sum_{\alpha, \beta \neq \alpha} ( \vec{r}_{\alpha} \times \vec{f}_{\alpha \beta}) = \sum_{\alpha < \beta} ( \vec{r}_{\alpha \beta} \times \vec{f}_{\alpha \beta}) = 0\\$
VI. total internal torque vanishes if internal forces are central i.e. $\vec{f}_{\alpha \beta} = - \vec{f}_{\beta \alpha}\\$


\hdashrule[0.5ex][c]{\linewidth}{0.5pt}{1.5mm}


\item \underline{$T = \sum_{\alpha} \frac{1}{2} m_{\alpha} v_{\alpha}'^2 + \frac{1}{2} M V^2$}\\
1 $\sim$ configuration of particles (initial position for $\vec{r}_{\alpha}$)\\
2 $\sim$ configuration of particles (final position of $\vec{r}_{\alpha}$)\\
1,2 can depend on $\alpha\\
W_{12} = \sum_{\alpha} \int_1^2 \vec{F}_{\alpha} \cdot d \vec{r}_{\alpha}\\$
\underline{recall:} $W = \int \vec{F} \cdot d \vec{r} = \int m \dot{v} v dt = \frac{1}{2} \int \frac{d m v^2}{dt} dt = \int T\\
\implies W_{12} = \sum_{\alpha} \int_1^2 d( \frac{1}{2} m_{\alpha} v_{\alpha}^2) = \sum_{\alpha} T_{\alpha 2} - \sum_{\alpha} T_{\alpha 1} = T_2 - T_1\\
T = \sum_{\alpha} T_{\alpha} = \sum_{\alpha} \frac{1}{2} m_{\alpha} v_{\alpha}^2\\
\dot{\vec{r}}_{\alpha} = \dot{\vec{r}'}_{\alpha} + \vec{R}\\
\dot{\vec{r}}_{\alpha} \cdot \dot{\vec{r}}_{\alpha} = v_{\alpha}^2 = ( \dot{\vec{r}}_{\alpha}' + \dot{\vec{R}}) \cdot ( \dot{\vec{r}}_{\alpha} ' + \dot{\vec{R}})\\
= v_{\alpha}'^2 + 2 \dot{\vec{r}}_{\alpha}' \cdot \dot{\vec{R}} + V^2\\
\implies T = \sum_{\alpha} \frac{1}{2} m_{\alpha} v_{\alpha}^2 = \sum_{\alpha} \frac{1}{2} m_{\alpha} v_{\alpha}'^2 + ( \sum_{\alpha} m_{\alpha} \vec{v}_{\alpha}') \cdot \dot{\vec{R}} + M V^2\\
but \sum_{\alpha} m_{\alpha} \vec{v}_{\alpha}' = \frac{d}{dt} ( \sum_{\alpha} m_{\alpha} \vec{r}_{\alpha}') = 0=M \vec{R}'$ (in cm system R' is the origin)\\
$\therefore T = \sum_{\alpha} \frac{1}{2} m_{\alpha} v_{\alpha}'^2 + M V^2\\$
VII. total $T$ is the $T$ of the particles relative to the center of mass and $T$ of center of mass.\\


\hdashrule[0.5ex][c]{\linewidth}{0.5pt}{1.5mm}


\item \underline{$E_1 = E_2$}\\
\underline{recall:} $W_{12} = \sum_{\alpha} \int_1^2 \vec{F}_{\alpha} \cdot d \vec{r}_{\alpha};\,\, \vec{F}_{\alpha} = \vec{F}_{\alpha}^{(e)} + \sum_{\beta} f_{\alpha \beta}\\
\implies W_{12} = \sum_{\alpha} \int_1^2 \vec{F}_{\alpha}^{(e)} \cdot d \vec{r}_{\alpha} + \sum_{\alpha, \beta \neq \alpha} \int_1^2 \vec{f}_{\alpha \beta} \cdot d \vec{r}_{\alpha}\\$
assume $\vec{F}_{\alpha}^{(e)},\,\, \vec{f}_{\alpha \beta}$ conservative\\
$\implies \vec{F}_{\alpha}^{(e)} = - \nabla_{\alpha} U_{\alpha};\,\, \vec{f}_{\alpha \beta} = - \nabla_{\alpha} \vec{U}_{\alpha \beta}\\$
\underline{Note:} $U_{\alpha} \neq \bar{U}_{\alpha \beta} \nabla_{\alpha}$ gradient performed with respect to coordinates of $\alpha$ th particle\\
$\implies \sum_{\alpha} \int_1^2 \vec{F}_{\alpha}^{(e)} \cdot d \vec{r}_{\alpha} = - \sum_{\alpha} \int_1^2 (\nabla_{\alpha} U_{\alpha}) \cdot d \vec{r}_{\alpha} = - \sum_{\alpha} U_{\alpha}|_1^2\\
\sum_{\alpha, \beta \neq \alpha} \int_1^2 \vec{f}_{\alpha \beta} \cdot d \vec{r}_{\alpha} = \sum_{\alpha< \beta} \int_1^2 \vec{f}_{|alpha \beta} \cdot d \vec{r}_{\alpha} + \sum_{\alpha > \beta} \int_1^2 \vec{f}_{\alpha \beta} \cdot d \vec{r}_{\alpha}\\
= \sum_{\alpha < \beta} ( \int_1^2 \vec{f}_{\alpha \beta} \cdot d \vec{r}_{\alpha} + \int_1^2 \vec{f}_{\beta \alpha} \cdot d \vec{r}_{\beta})\\
= \sum_{\alpha < \beta} \int_1^2 ( \vec{f}_{\alpha \beta} \cdot d \vec{r}_{\alpha} + \vec{f}_{\beta \alpha} \cdot d \vec{r}_{\beta})\\
= \sum_{\alpha < \beta} \int_1^2 \vec{f}_{\alpha \beta} \cdot ( d \vec{r}_{\alpha} - d \vec{r}_{\beta}) = \sum_{\alpha < \beta} \int_1^2 \vec{f}_{\alpha beta} \cdot d \vec{r}_{\alpha \beta}\\
\bar{U}_{\alpha \beta}$ function of distance between $\alpha$ and $\beta\\
\implies x_{\beta, i},\,\, x_{\alpha, i}\\
\implies d \bar{U}_{\alpha \beta} = \sum_i ( \frac{\partial \bar{U}_{\alpha \beta}}{\partial x_{\alpha, i}} d x_{\alpha, i} + \frac{\partial \bar{U}_{\alpha \beta}}{\partial x_{\beta, i}} dx_{\beta, i})\\
= ( \nabla_{\alpha} \bar{U}_{\alpha \beta}) \cdot d \vec{r}_{\alpha} + ( \nabla_{\beta} \bar{U}_{\alpha \beta}) \cdot d \vec{r}_{\beta}\\
\nabla_{\alpha} \bar{U}_{\alpha \beta} = - \vec{f}_{\alpha \beta},\,\, \bar{U}_{\alpha \beta} = \bar{U}_{\beta \alpha}\\
\implies \nabla_{\beta} \bar{U}_{\alpha \beta} = \nabla_{\beta} \bar{U}_{\beta \alpha} = - \vec{f}_{\beta \alpha} = \vec{f}_{\alpha \beta}\\
\implies d \bar{U}_{\alpha \beta} = \sum_i( \frac{\partial \bar{U}_{\alpha \beta}}{\partial x_{\alpha,i}} d x_{\alpha, i} + \frac{\partial \bar{U}_{\alpha \beta}}{\partial x_{\beta, i}} d x_{\beta, i})\\
= \nabla_{\alpha} \bar{U}_{\alpha \beta} \cdot d \vec{r}_{\alpha} + \nabla_{\beta} \bar{U}_{\alpha \beta} \cdot d \vec{r}_{\beta}\\
= - \vec{f}_{\alpha \beta} \cdot d \vec{r}_{\alpha} + \vec{f}_{\alpha \beta} \cdot d \vec{r}_{\beta}\\
= - \vec{f}_{\alpha \beta} \cdot ( d \vec{r}_{\alpha} - d \vec{r}_{\beta}) = - \vec{f}_{\alpha \beta} \cdot d \vec{r}_{\alpha \beta}\\
\implies \sum_{\alpha, \beta \neq \alpha} \int_1^2 \vec{f}_{\alpha \beta} \cdot d \vec{r}_{\alpha} = \sum_{\alpha < \beta} \int_1^2 \vec{f}_{\alpha \beta} \cdot d \vec{r}_{\alpha \beta}\\
= - \sum_{\alpha, \beta} \int_1^2 d \bar{U}_{\alpha \beta} = - \sum_{\alpha < \beta} \bar{U}_{\alpha \beta} |_1^2\\
\implies W_{12} = - \sum_{\alpha} U_{\alpha}|_1^2 - \sum_{\alpha < \beta} \bar{U}_{\alpha \beta} |_1^2\\$
total potential (external/internal) $= U = \sum_{\alpha} U_{\alpha} + \sum_{\alpha < \beta} \bar{U}_{\alpha \beta}\\
\implies W_{12} = -U|_1^2 = U_1 - U_2\\
\underline{recall:} W_{12} = T_2 - T_1\\
\implies U_1 - U_2 = T_2 - T_1\\
\implies T_1 + U_1 = T_2 + U_2\\
\therefore E_1 = E_2\\$
V III. total energy for conservative system is constant


\section*{Chapter 10}
\item \underline{$(\frac{d \vec{r}}{dt})_{fixed} = ( \frac{d \vec{r}}{dt})_{rotating} + \vec{\omega} \times \vec{r}$}\\
$x_i' \sim$ fixed$,\,\, x_i \sim$ rotating\\
$\vec{r}' \sim$ point in $x_i'$ system\\
$\vec{r} \sim$ same point in $x_i$ system\\
$\vec{R} \sim$ origin of $x_i$ relative to $x_i'$\\
$\implies \vec{r}' = \vec{R} + \vec{r}\\$
we assume origins are aligned for this calculation so $\frac{d \vec{r}'}{dt} = \frac{d \vec{r}}{dt}$
\underline{recall:} $\vec{v}_{trans} = \vec{\omega} \times \vec{r} \\
\implies (\frac{d \vec{r}}{dt})_{fixed} = \vec{\omega} \times \vec{r}$ (fixed point in x system)\\
$\therefore (\frac{d \vec{r}}{dt})_{fixed} = ( \frac{d \vec{r}}{dt})_{rotating} + \vec{\omega} \times \vec{r}$
Pretty sure this equation works when the origins of the two systems are aligned, that is $\vec{R}=0$ and moreover $\frac{d \vec{R}}{dt}=0$

\hdashrule[0.5ex][c]{\linewidth}{0.5pt}{1.5mm}


In general, $(\frac{d \vec{Q}}{dt})_{fixed} = ( \frac{d \vec{Q}}{dt})_{rotating} + \vec{\omega} \times \vec{Q}$\\


\hdashrule[0.5ex][c]{\linewidth}{0.5pt}{1.5mm}


\item \underline{$\vec{v}_{trans} = \vec{\omega} \times \vec{r}$}\\
$v_{tan} = \omega R = \omega r \sin \alpha\\
\implies \vec{v}_{tan} = \vec{\omega} \times \vec{r}$


\hdashrule[0.5ex][c]{\linewidth}{0.5pt}{1.5mm}


ended on 6.6 skipped rest of 6


\hdashrule[0.5ex][c]{\linewidth}{0.5pt}{1.5mm}

\underline{Note:} $(\frac{d \vec{\omega}}{dt})_{fixed} = (\frac{d \vec{\omega}}{dt})_{fixed} + \vec{\omega} \times \vec{\omega} = \dot{\vec{\omega}}$\\

\item \underline{$\vec{v}_f = \vec{V} + \vec{v}_r + \vec{\omega} \times \vec{r}$}\\
\underline{recall:} $\vec{r}' = \vec{R} + \vec{r}\\
\implies (\frac{d \vec{r}'}{dt})_{fixed} = ( \frac{d \vec{R}}{dt})_{fixed} + ( \frac{d \vec{r}}{dt})_{fixed}\\
\implies (\frac{d \vec{r}'}{dt})_{fixed} = ( \frac{d \vec{R}}{dt})_{fixed} + ( \frac{d \vec{r}}{dt})_{rotating} + \vec{\omega} \times \vec{r}\\
\vec{v}_f = \dot{\vec{r}'}_f \equiv (\frac{d \vec{r}'}{dt})_{fixed};\,\, \vec{V} \equiv \dot{\vec{R}}_f \equiv (\frac{d \vec{R}}{dt})_{fixed};\,\, \vec{v}_r \equiv \dot{\vec{r}}_r \equiv (\frac{d \vec{r}}{dt})_{rotating}\\
\therefore \vec{v}_f = \vec{V} + \vec{v}_r + \vec{\omega} \times \vec{r}\\$
$\vec{v}_f =$ Velocity relative to the fixed axes\\
$\vec{V} =$ Linear velocity of the moving origin\\
$\vec{v}_r =$ Velocity relative to rotating axes\\
$\vec{\omega} \times \vec{r} =$ Velocity due to the rotation of the moving axes


\hdashrule[0.5ex][c]{\linewidth}{0.5pt}{1.5mm}

$\star$
\item \underline{$\vec{F} = m \vec{a}_f = m \ddot{R}_f + m \vec{a}_r + m \dot{\vec{\omega}} \times \vec{r} + m \vec{\omega} \times(\vec{\omega} \times \vec{r}) + 2 m \vec{\omega} \times \vec{v}_r$;}\\
\item \underline{$\vec{F}_{eff} = m \vec{a}_r = \vec{F} - m \ddot{\vec{R}}_f - m \dot{\vec{\omega}} \times \vec{r} - m \vec{\omega} \times ( \vec{\omega} \times \vec{r}) - 1 m \vec{\omega} \times \vec{v}_r$}\\
$\vec{F} = m \vec{a}_f = m ( \frac{d \vec{v}_f}{dt})_{fixed}\\$
\underline{recall:}$(\frac{d \vec{Q}}{dt})_{fixed} = ( \frac{d \vec{Q}}{dt})_{rot} + \vec{\omega} \times \vec{Q};\,\, \vec{v}_f = \vec{V} + \vec{v}_r + \vec{\omega} \times \vec{r}$\\
$\implies (\frac{d \vec{v}_f}{dt})_{fixed} = ( \frac{d \vec{V}}{dt})_{fixed} + ( \frac{d \vec{v}_r}{dt})_{fixed} + \dot \vec{\omega} \times \vec{r} + \vec{\omega} \times ( \frac{d \vec{r}}{dt})_{fixed}\\$
\underline{Note:} $(\frac{d \vec{r}}{dt})_{fixed} (body) \neq v_f=(\frac{d \vec{r'}}{dt})_{fixed} (r'$ in fixed frame)
$\ddot{\vec{R}}_f \equiv ( \frac{d \vec{V}}{dt})_{fixed}\\
( \frac{d \vec{v}_r}{dt})_{fixed} = ( \frac{d \vec{v}_r}{dt})_{rotating} + \vec{\omega} \times \vec{v}_r = \vec{a}_r + \vec{\omega} \times \vec{v}_r\\
\vec{\omega} \times ( \frac{d \vec{r}}{dt})_{fixed} = \vec{\omega} \times (\frac{d \vec{r}}{dt})_{rotating} + \vec{\omega} \times (\vec{\omega} \times \vec{r})\\
= \vec{\omega} \times \vec{v}_r + \vec{\omega} \times ( \vec{\omega} \times \vec{r})\\
\therefore \vec{F} = m \vec{a}_f = m \ddot{\vec{R}}_f + m \vec{a}_r + m \vec{\omega} \times \vec{v}_r + \dot \vec{\omega} \times \vec{r} + m \vec{\omega} \times \vec{v}_r + m \vec{\omega} \times (\vec{\omega} \times \vec{r})\\
= m \ddot{\vec{R}}_f + m \vec{a}_r + m \dot \vec{\omega} \times \vec{r} + m \vec{\omega} \times (\vec{\omega} \times \vec{r}) + 2 m \vec{\omega} \times \vec{v}_r\\
\therefore \vec{F}_{eff} = m \vec{a}_r$\\ 
$- m \vec{\omega} \times (\vec{\omega} \times \vec{r})$ (centrifugal term)\\
$- 2m \vec{\omega} \times \vec{v}_r$ (coriolis force)


\hdashrule[0.5ex][c]{\linewidth}{0.5pt}{1.5mm}


\item \underline{$\vec{F}_{eff} = \vec{S} + m \vec{g} - 2 m \vec{\omega} \times \vec{v}_r$}\\
$\vec{F} = \vec{S} + m \vec{g}_0 (\vec{s} \sim$ sum of external forces)\\
$\vec{g}_0 = - G \frac{M \epsilon}{R^2} \vec{e}_R\\$
\underline{recall:} $\vec{F}_{eff} = \vec{F} - m \ddot{\vec{R}}_f - m \dot{\vec{\omega}} \times \vec{r} - m \vec{\omega} \times (\vec{\omega} \times \vec{r}) - 2 m \vec{\omega} \times \vec{v}_r\\
\implies \vec{F}_{eff} = \vec{S} + m \vec{g}_0 - m \ddot{\vec{R}}_f - m \dot{\vec{\omega}} \times \vec{r} - m \vec{\omega} \times (\vec{\omega} \times \vec{r}) - 2 m \vec{\omega} \times \vec{v}_r\\$
\underline{recall:} $(\frac{d \vec{Q}}{dt})_{fixed} = (\frac{d \vec{Q}}{dt})_{rot} + \vec{\omega} \times \vec{Q}\\
\implies (\frac{d \dot{\vec{R}}}{dt})_f = \dddot{\vec{R}}_f = (\frac{d \dot{\vec{F}}_f}{dt})_{rot} + \vec{\omega} \times \dot{\vec{R}}_f\\
\dot{\vec{R}}_f = (\frac{d \vec{R}}{dt})_{rot} + \vec{\omega} \times \vec{R}\\
\implies (\frac{d \dot{\vec{R}}_f}{dt})_{rot} = ( \ddot{\vec{R}}_{rot} + \dot{\vec{\omega}} \times \vec{R} + \vec{\omega} \times (\frac{d \vec{R}}{dt})_{rot}\\
\vec{R}$ is the origin in rotating frame so $(\frac{d \vec{R}}{dt})_{rot} = ( \ddot{\vec{R}}_{rot}=0\\
\vec{\omega} \sim$ rotation of the earth $\approx$ const $\implies \dot{\vec{\omega}} = 0\\
\implies (\ddot{\vec{R}})_{rot} =0\\
\implies \ddot{\vec{R}}_f = \vec{\omega} \times \dot{\vec{R}}_f\\
\dot{\vec{R}}_f = ( \frac{d \vec{R}}{dt})_{rot} + \vec{\omega} \times \vec{R} = \vec{\omega} \times \vec{R}\\
\implies \ddot{\vec{R}}_f = \vec{\omega} \times (\vec{\omega} \times \vec{R})\\
\implies \vec{F}_{eff} = \vec{S} + m \vec{g}_0 - m \vec{\omega} \times (\vec{\omega} \times \vec{R}) - m \vec{\omega} \times (\vec{\omega} \times \vec{r}) - 2 m \vec{\omega} \times \vec{v}_r\\
= \vec{S} + m \vec{g}_0 - m \vec{\omega} \times [ \vec{\omega} \times ( \vec{r} + \vec{R})] - 2 m \vec{\omega} \times \vec{v}_r - \vec{\omega} \times [ \vec{\omega} \times (\vec{r} + \vec{R})] \sim$ centrifugal term\\
$\vec{g} = \vec{g}_0 - \vec{\omega} \times [ \vec{\omega} \times (\vec{r} + \vec{R})]$ (what we experience)\\
$\therefore \vec{F}_{eff} = \vec{S} + m \vec{g} - 2 m \vec{\omega} \times \vec{v}_r\\$


\hdashrule[0.5ex][c]{\linewidth}{0.5pt}{1.5mm}





\section*{Chapter 11}


\item \underline{$\vec{v}_{\alpha} = \vec{V} + \vec{\omega} \times \vec{r}_{\alpha}$}\\
\underline{recall:} $(\frac{d \vec{r}}{dt})_{fixed} = (\frac{d \vec{r}}{dt})_{rotating} + \vec{\omega} \times \vec{r}\\$
rigid body $\implies (\frac{d \vec{r}}{dt})_{rot} = 0\\$
suppose particle also has translational velocity in fixed frame $\vec{V}\\$
$\implies (\frac{d \vec{r}_{\alpha}}{dt})_{fixed} = \vec{v}_{\alpha} = \vec{V} + \vec{\omega} \times \vec{r}_{\alpha}$
\\


\hdashrule[0.5ex][c]{\linewidth}{0.5pt}{1.5mm}


\item \underline{$T_{trans} = \frac{1}{2} \sum_{\alpha} m_{\alpha} V^2 = \frac{1}{2} M V^2;\,\, T_{rot} = \frac{1}{2} \sum_{\alpha} m_{\alpha} (\vec{\omega} \times \vec{r}_{\alpha})^2$}\\
$T_{\alpha} = \frac{1}{2} m_{\alpha} v_{\alpha}^2;\,\, \\$
\underline{recall:} $\vec{v}_{\alpha} = \vec{V} + \vec{\omega} \times \vec{r}_{\alpha}\\
\implies T= \frac{1}{2} \sum_{\alpha} m_{\alpha} ( \vec{V} + \vec{\omega} \times \vec{r}_{\alpha})^2\\
= \frac{1}{2} \sum_{\alpha} m_{\alpha} V^2 + \frac{1}{2} \sum_{\alpha} m_{\alpha} 2 \vec{V} \cdot ( \vec{\omega} \times \vec{r}_{\alpha}) + \frac{1}{2} \sum_{\alpha} m_{\alpha} ( \vec{\omega} \times \vec{r}_{\alpha})^2\\$
but $\sum_{\alpha} m_{\alpha} \vec{V} \cdot ( \vec{\omega} \times \vec{r}_{\alpha} ) = \vec{V} \cdot \vec{\omega} \times (\sum_{\alpha} m_{\alpha} \vec{r}_{\alpha})\\
= \vec{V} \cdot \vec{\omega} \times M \vec{R}$ Choose origins to coincide so that $\vec{R} = 0, \vec{r}_{\alpha}$ measured from the center of mass
$\therefore T= T_{trans} + T_{rot}\\$
w/ $T_{trans} = \frac{1}{2} M V^2;\,\, T_{rot} = \frac{1}{2} \sum_{\alpha} m_{\alpha} (\vec{\omega} \times \vec{r}_{\alpha})^2\\$


\hdashrule[0.5ex][c]{\linewidth}{0.5pt}{1.5mm}


\item \underline{$T_{rot} = \frac{1}{2} \sum_{i,j} I_{ij} \omega_i \omega_j;\,\, I_{ij} = \sum_{\alpha} m_{\alpha} (\delta_{ij} \sum_k x_{\alpha, k}^2 - x_{\alpha, i} x_{\alpha, j})$}\\
\underline{Note:} $(\vec{A} \times \vec{B})^2 = ( \vec{A} \times \vec{B}) \cdot ( \vec{A} \times \vec{B}) = A^2 B^2 - ( \vec{A} \cdot \vec{B})^2\\
\implies T_{rot} = \frac{1}{2} \sum_{\alpha} m_{\alpha}[ \omega^2 r_{\alpha}^2 - ( \vec{\omega} \cdot \vec{r}_{\alpha})^2]\\
= \frac{1}{2} \sum_{\alpha} m_{\alpha} [ ( \sum_i \omega_i^2)(\sum_k x_{\alpha, k}^2) - (\sum_i \omega_i x_{\alpha, i})(\sum_j \omega_j x_{\alpha, j} )]\\
= \frac{1}{2} \sum_{\alpha} m_{\alpha}[ ( \sum_{i,j} \omega_i \omega_j \delta_{ij})(\sum_{k} x_{\alpha, k}^2) - \sum_{i,j} \omega_i \omega_j x_{\alpha, i} x_{\alpha, j}]\\
= \frac{1}{2} \sum_{\alpha} \sum_{i,j} \omega_i \omega_j m_{\alpha} [ \delta_{ij} \sum_k x_{\alpha,k}^2 - x_{\alpha, i} x_{\alpha, j}]\\
= \frac{1}{2} \sum_{i, j} \omega_i \omega_j \sum_{\alpha} m_{\alpha} [ \delta_{ij} \sum_k x_{\alpha, k}^2 - x_{\alpha, i} x_{\alpha, j}]\\
= \frac{1}{2} \sum_{i, j} I_{ij} \omega_i \omega_j\\
I_{ij} = \sum_{\alpha} m_{\alpha} [ \delta_{ij} \sum_k x_{\alpha, k}^2 - x_{\alpha, i} x_{\alpha, j} ]\\$
\underline{Note:} $If I_{ij} = I \delta_{ij} \implies T_{rot} = \frac{1}{2} \sum_{i,j} I \delta_{ij} \omega_i \omega_j\\
= \frac{1}{2} I \sum_i \omega_i^2 = \frac{1}{2} I \omega^2\\
\sum_{\alpha} m_{\alpha} \rightarrow \int \rho d V\\
\implies I_{ij} = \sum_V \rho(\vec{r})( \delta_{ij} \sum_k x_k^2 - x_i x_j) dV\\$


\hdashrule[0.5ex][c]{\linewidth}{0.5pt}{1.5mm}


\item \underline{$\vec{L} = \sum_{\alpha} m_{\alpha} [ r_{\alpha}^2 \vec{\omega} - \vec{r}_{\alpha} ( \vec{r}_{\alpha} \cdot \vec{\omega})];\,\, L_i = \sum_j I_{ij} \omega_j$}\\
$\vec{L} = \sum_{\alpha} \vec{r}_{\alpha} \times \vec{p}_{\alpha}\\
\vec{p}_{\alpha} = m \vec{v}_{\alpha} = m_{\alpha} \vec{\omega} \times \vec{r}_{\alpha}$ ( body system: fixed)\\
$\implies \vec{L} = \sum_{\alpha} \vec{r}_{\alpha} \times m_{\alpha} \vec{\omega} \times \vec{r}_{\alpha} = \sum_{\alpha} m_{\alpha} \vec{r}_{\alpha} \times ( \vec{\omega} \times \vec{r}_{\alpha})\\
\vec{A} \times ( \vec{B} \times \vec{A}) = A^2 \vec{B} - \vec{A} ( \vec{A} \cdot \vec{B})\\
\implies \vec{L} = \sum_{\alpha} m_{\alpha} [ r_{\alpha}^2 \vec{\omega} - \vec{r}_{\alpha}( \vec{r}_{\alpha} \cdot \vec{\omega})]\\
L_i = \sum_{\alpha} m_{\alpha} [ r_{\alpha}^2 \omega_i - x_{\alpha, i} ( \sum_j x_{\alpha, j} \omega_j)]\\
= \sum_{\alpha} m_{\alpha} [ r_{\alpha}^2 \sum_j \delta_{ij} \omega_j - x_{\alpha, i} ( \sum_j x_{\alpha, j} \omega_j ) ]\\
= \sum_j \omega_j \sum_{\alpha} m_{\alpha}[ r_{\alpha}^2 \delta_{ij} - x_{\alpha, i} x_{\alpha, j}]\\
= \sum_j I_{ij} \omega_j \implies \vec{L} = \{ I \} \cdot \vec{\omega}$\\


\hdashrule[0.5ex][c]{\linewidth}{0.5pt}{1.5mm}


\item \underline{$\vec{L} = I \vec{\omega}$}(Principal axis)\\
Principal axes are axes such that $I_{ij} = \delta_{ij} I_i\\$
suppose that we have such a coordinates system, then \\
$L_i = \sum_j I_{ij} \omega_j = \sum_j \delta_{ij} I_i \omega_j = I_i \omega_i\\$
i.e. $\omega_i$ is the angular velocity oriented on the ith principal axis\\
or in an arbitrary coordinate system ($\vec{\omega}$ still oriented on principal axis) $\implies \vec{L} = I \vec{\omega}\\$
also $T_{rot} = \frac{1}{2} \sum_{i,j} I_i \delta_{j} \omega_i \omega_j = \frac{1}{2} \sum_i I_i \omega_i^2\\$


\hdashrule[0.5ex][c]{\linewidth}{0.5pt}{1.5mm}


\item \underline{$I_{ij} = J_{ij} - M (a^2 \delta_{ij} - a_i a_j)$}\\
Spose we want to find $I_{ij}$ (center of mass ($x_i$ has origin $O$)) given arbitrary $T_{ij}$ system $X_i$ origin $Q$\\
$\vec{a}$ points from arbitrary origin to center of mass origin
$\implies J_{ij} = \sum_{\alpha} m_{\alpha} (\delta_{ij} \sum_k X_{\alpha, k}^2 - X_{\alpha, i} X_{\alpha, j})\\
Q$ to $O$ is $\vec{a}\\
\implies \vec{R} = \vec{r} + \vec{a} \implies X_i = a_i + x_i\\
J_{ij} = \sum_{\alpha} m_{\alpha} (\delta_{ij} \sum_k ( a_k + x_{\alpha, k})^2 - ( a_i + x_{\alpha, i})(a_j + x_{\alpha,j})\\
= \sum_{\alpha} m_{\alpha}(\delta_{ij} \sum_k a_k^2 + 2 \delta_{ij} \sum_k a_k x_{\alpha, k} + \delta_{ij} \sum_k x_{\alpha, k}^2 - a_i a_j - a_i x_{\alpha,j}\\
- x_{\alpha, i} a_j - x_{\alpha, i} x_{\alpha, j}) = \sum_{\alpha,} m_{\alpha} (\delta_{ij} \sum_k a_k^2 + 2 \delta_{ij} \sum_k a_k x_{\alpha, k} - a_i a_j - a_i x_{\alpha, j} - x_{\alpha, i} a_j) + \sum_{\alpha} m_{\alpha} (\delta_{ij} \sum_k x_{\alpha, k}^2 - x_{\alpha, i} x_{\alpha, j}\\$
\underline{recall:} $O$ at center of mass $\implies \sum_{\alpha} m_{\alpha} x_{\alpha, k} = 0\\
\implies j_{ij} = \sum_{alpha} m_{\alpha} \delta_{ij} a^2 - \sum_{\alpha} m_{\alpha} a_i a_j + I_{ij}\\
= I_{ij} + M \delta_{ij} a^2 - M a_i a_j\\
\therefore I_{ij} = J_{ij} - M (\delta_{ij} a^2 - a_i a_j)\\$



\section*{Chapter 11}
\item \underline{$\vec{\omega}_m \cdot \vec{\omega}_n = 0$} (redo)\\
\underline{recall:} $L_i = I_i \omega_i ,\,\, I_i \sim$ principal moment$,\,\, \omega_i \sim$ angular velocity about this axis after coordinate change in any basis $\implies \vec{L} = I \vec{\omega}$ where $\vec{\omega}$ is about the principal axis.\\
for nth principal moment $\vec{\omega}_m$ points in direction of principal axis\\
$\implies L_{im} = I_m \omega_{im}\\$
\underline{recall:} $L_i = \sum_j I_{ij} \omega_j\\
\implies L_{im} = \sum_k I_{ik} \omega_{km}\\
\implies \sum_k I_{ik} \omega_{km} = I_m \omega_{im}\\
k \leftrightarrow{} i n \rightarrow m\\$
$\implies \sum_i I_{ki} \omega_{in} = I_n \omega_{kn}\\$
mult $\omega_{in}$ sum $i$ and $\omega_{km}$ sum $k$\\
$\implies 
\begin{cases} \sum_i \sum_k I_{ik} \omega_{km} \omega_{in} = \sum_i I_m \omega_{im} \omega_{in}\\
\sum_k \sum_i I_{ki} \omega_{in} \omega_{km} = \sum_k I_n \omega_{kn} \omega_{km}\\
\end{cases}$
subtract $(use I_{ik} = I_{ki})\\
\implies (I_m - I_n) \sum_{\ell} \omega_{\ell m} \omega_{\ell n} = 0;\,\, I_m \neq I_n\\
\implies \sum_{\ell} \omega_{\ell m} \omega_{\ell n} = 0 \implies \vec{\omega}_m \cdot \vec{\omega}_n=0$


\hdashrule[0.5ex][c]{\linewidth}{0.5pt}{1.5mm}


\item \underline{$I_{ij}'=\sum_{k,\ell} \lambda_{ik} I_{k \ell} \lambda_{\ell j}^t;\,\, \tilde{I}'=\tilde{\lambda} \tilde{I} \tilde{\lambda}^{-1}$}\\
\underline{recall:} $L_k = \sum_{\ell} I_{k \ell} \omega_{\ell} ;\,\, L_i'=\sum_j I_{ij}' \omega_j'\\
x_i=\sum_j \lambda_{ij}^t x_j'=\sum_{j} \lambda_{ji} x_j'$ (dont understand)\\
$\implies L_k = \sum_m \lambda_{mk} L_m'$ and $\omega_{\ell} = \sum_j \lambda_{j \ell} \omega_j'\\$
plug into $L_k = \sum_{\ell} I_{k \ell} \omega_{\ell}\\
\implies \sum_m \lambda_{mk} L_m'=\sum_{\ell,j} I_{k \ell} \lambda_{j \ell} \omega_j'$ mult by $ \lambda_{ik}$ sum k\\
$\implies \sum_m(\sum_k \lambda_{ik} \lambda_{mk}) L_m'=\sum_j ( \sum_{k,\ell} \lambda_{ik} \lambda_{j \ell} I_{k \ell} ) \omega_j'\\$
\underline{recall:} $\sum_k \lambda_{ik} \lambda_{mk} = \delta_{im}\\
\implies \sum_m \delta_{im} L_m'=L_i'=\sum_j(\sum_{k, \ell} \lambda_{ik} \lambda_{j \ell} I_{k \ell}) \omega_j'\\$
but $L_i'=\sum_j I_{ij}' \omega_j'\\$
$\implies \sum_j ( I_{ij}') \omega_j'= \sum_j( \sum_{k, \ell} \lambda_{ik} \lambda_{j \ell} I_{k \ell}) \omega_j'\\
\implies I_{ij}'=\sum_{k, \ell} \lambda_{ik} \lambda_{j \ell} I_{k \ell} \\
\therefore I_{ij}'=\sum_{k, \ell} \lambda_{ik} I_{k \ell} \lambda_{\ell j}^t \implies \tilde{I}'=\tilde{\lambda} \tilde{I} \tilde{\lambda}^t\\$
but $\tilde{\lambda}^t = \tilde{\lambda}^{-1}$ ( orthogonal)\\
$\therefore \tilde{I}'=\tilde{\lambda} \tilde{I} \tilde{\lambda}^{-1}$\\

\underline{Purpose:} Given the transformations of vectors, $\vec{L}$ and $\vec{\omega}$, which transform according to $\lambda$ we need to figure out how to transform the tensor $I_{ij}$

\underline{Shortened method: } plug $L_k = \sum_m \lambda_{mk} L_m'$ and $\omega_{\ell} = \sum_j \lambda_{j \ell} \omega_j'$ into $L_k = \sum_{\ell} I_{k \ell} \omega_{\ell}$ and get into the form $\implies \sum_j ( I_{ij}') \omega_j'= \sum_j( \sum_{k, \ell} \lambda_{ik} \lambda_{j \ell} I_{k \ell}) \omega_j'$ which is the same as $L'= \{\tilde{I}'\} \vec{\omega}'$


\hdashrule[0.5ex][c]{\linewidth}{0.5pt}{1.5mm}


\item \underline{$| I_{m \ell} - I_j \delta_{m \ell} |=0$} (j denotes jth eigenvalue, book doesn't elucidate this)\\
Want to find condition that must be satisfied for a coordinate transformation that diagonolizes $I_{ij}$, in this new system:\\
$I'_{ij}=I_i \delta_{ij}$\\
\underline{recall:}  $I'_{ij}=\sum_{k, \ell} \lambda_{ik} \lambda_{j \ell} I_{k \ell}$\\
$\implies I_i \delta_{ij}=\sum_{k, \ell} \lambda_{ik} \lambda_{j \ell} I_{k \ell}$\\
mult by $\lambda_{im}$ sum on $i$\\
$\implies \sum_i \lambda_{im} I_i \delta_{ij}=\sum_{i,k,\ell} \lambda_{ik} \lambda_{im} \lambda_{j \ell} I_{k \ell}\\
\implies \lambda_{jm} I_j = \sum _{k, \ell} \delta_{km} \lambda_{j \ell} I_{k\ell} = \sum_{\ell} \lambda_{j \ell} I_{m \ell}$\\
\underline{Note:}$ \lambda_{jm} I_j= \sum_{\ell} \delta_{m \ell} \lambda_{j \ell} I_j\\
\implies \sum_{\ell}( I_{m \ell} \lambda_{j \ell} - \delta_{m \ell} \lambda_{j \ell} I_j)= \sum_{\ell} ( I_{m \ell} - \delta_{m \ell} I_j) \lambda_{j \ell}\\
\implies \sum_{\ell} I_{m \ell} \lambda_{j \ell} - (\sum_{\ell} \delta_{m \ell} \lambda_{j \ell}) I_j= \tilde{I} \tilde{\lambda}-\lambda_{jm} I_j\\
\implies \tilde{I} \tilde{\lambda}=I_j \tilde{\lambda}$ (don't fully understand this one)\\
or we could write $\lambda_{j \ell} \rightarrow \vec{\lambda}_j$ so that we instead have\\
$\tilde{I} \vec{\lambda}_j = I_j \vec{\lambda}_j$ 
This makes more sense since $I_j$ corresponds to the jth eigenvector\\
\underline{Shortened version:}\\  $I'_{ij} = I_i \delta_{ij},\,\, I'_{ij}=\sum_{k, \ell} \lambda_{ik} \lambda_{j \ell} I_{k \ell} \rightarrow \tilde{I} \vec{\lambda}_j = I_j \vec{\lambda}_j \rightarrow |I_{m \ell} - I_j \delta_{m \ell}| = 0$\\
$I_i \delta_{ij}=\sum_{k, \ell} \lambda_{ik} \lambda_{j \ell} I_{k \ell}$ multiply by $\lambda_{im}$ and sum on i \\
$\implies \sum_i(I_i \delta_{ij} \lambda_{im} - \lambda_{ji} I_{mi})= I_j \vec{\lambda}_j - \tilde{I} \vec{\lambda}_j=0\\
\implies |I_j \vec{\lambda}_j - \tilde{I} \vec{I}_j|=0$


\hdashrule[0.5ex][c]{\linewidth}{0.5pt}{1.5mm}


\item \underline{$\vec{\omega}_m \cdot \vec{\omega}_n=0$} (don't understand well)\\
Let $\vec{\omega}_j$ be oriented along $I_j$ principal axis w/ components $\omega_{1 j}, \omega_{2 j} , \omega_{3 j}\\$
\underline{recall:} $L_k= \sum_{\ell} I_{k \ell} \omega_{\ell}\\$
mth principal moment $\implies L_{im} = \sum_j (I_{ij})_m \omega_{jm} ;\,\, (I_{ij})_m = I_m \delta_{ij}\\
\implies L_{im} = \sum_j I_m \delta_{ij} \omega_{jm} = I_m \omega_{im}\\$
alternatively $L_{im} = \sum_k I_{ik} \omega_{im}\\$
set equal $\implies \sum_k I_{ik} \omega_{im} = I_m \omega_{im};\,\, m \rightarrow n;\,\, k \leftrightarrow i\\$
$\implies \sum_i I_{ki}  \omega_{kn} = I_n \omega_{kn}\\$
mult first by $\omega_{in}$ sum i mult 2nd by $\omega_{km}$ sum k\\
$\implies \sum_{i,k} I_{ik} \omega_{km} \omega_{in} = \sum_i I_m \omega_{im} \omega_{in}\\
\implies \sum_{i,k} I_{ki} \omega_{in} \omega_{km} = \sum_k I_n \omega_{kn} \omega_{km}\\$
subtract\\
$\implies I_m \sum_i \omega_{im} \omega_{in} - I_n \sum_k \omega_{km} \omega_{kn} = 0\\
i,k \rightarrow \ell\\
\implies I_m \sum_{\ell} \omega_{\ell m} \omega_{\ell n} - I_n \sum_{\ell} \omega_{\ell m} \omega_{\ell n} = 0\\
\implies (I_m - I_n) \sum_{\ell} \omega_{\ell m} \omega_{\ell n} = 0,\,\, I_m \neq I_n\\
\implies \sum_{\ell} \omega_{\ell m} \omega_{\ell n} = 0 \implies if n \neq m\\
\implies \vec{\omega}_m \cdot \vec{\omega}_n=0\\$
\\
\underline{Shortened:} Start with $\sum_k I_{ik} \omega_{im} = I_m \omega_{im}$ then get it to this step\\
$\implies \sum_{i,k} I_{ik} \omega_{km} \omega_{in} = \sum_i I_m \omega_{im} \omega_{in}\\
\implies \sum_{i,k} I_{ki} \omega_{in} \omega_{km} = \sum_k I_n \omega_{kn} \omega_{km}\\$
and subtract


\hdashrule[0.5ex][c]{\linewidth}{0.5pt}{1.5mm}


\item \underline{ $\vec{\omega}_m \cdot \vec{\omega}_n = 0$} (redo)\\
spose $\vec{L}_m = \tilde{I} \vec{\omega}_m = I_m \vec{\omega}_m,\,\, I_m \sim$ eigenvalue, $\vec{\omega}_m \sim$ eigenvector\\
$\implies L_{im} = \sum_k I_{ik} \omega_{k m} = I_m \omega_{im}\\$
similarly $\sum_i I_{ki} \omega_{i n} = I_n \omega_{k n}\\
\implies \begin{cases} \sum_{k,i} I_{ik} \omega_{km} \omega_{in} = \sum_i I_m \omega_{im} \omega_{in}\\
\sum_{i,k} I_{ki} \omega_{in} \omega_{km} = \sum_k I_n \omega_{kn} \omega_{km} \end{cases}\\
\implies \sum_i I_m \omega_{im} \omega_{in} = \sum_k I_n \omega_{kn} \omega_{km}\\
\implies ( I_m - I_n ) \sum_i \omega_{im} \omega_{in} = 0\\$
assume $n \neq m\\$
$\implies \sum_i \omega_{im} \omega_{in} = \vec{\omega}_m \cdot \vec{\omega}_n = 0$


                                                                                                                                                                                                                                                                                                                                                                                                                                                                                                                                                                                                                                                                                                                                                                                                                                                                 
                                                                                                                                                                                                                                                                                                                                                                                                                                                                                                                                                                                                                                                                                                                                                                                                                                                                 \underline{Note:} If $I_2 = I_3 \implies \vec{\omega}_1 \perp \vec{\omega}_2,\,\, \vec{\omega}_1 \perp \vec{\omega}_3\\$


\hdashrule[0.5ex][c]{\linewidth}{0.5pt}{1.5mm}

\item \underline{$\vec{\omega} \sim$ real;$\,\,$ it is assumed $\{ \tilde{I} \}$ is hermitian $\implies$ real}\\
\underline{recall:} $\sum_{k} I_{ik} \omega_{km} = I_m \omega_{im}\\
k \leftrightarrow i;\,\, m \rightarrow n \implies \sum_i I_{ki} \omega_{in} = I_n \omega_{kn}\\
\implies \sum_i I_{ki}^* \omega_{in}^* = I_n^* \omega_{kn}^*\\
\implies \sum_k I_{ik} \omega_{km} = I_m \omega_{im} ;\,\, \sum_i I_{ki}^* \omega_{in}^* = I_n^* \omega_{kn}^*\\$
mult first by $\omega_{in}^*$ sum i;$\,\,$ mult second $\omega_{km} sum k;\,\, I$ is symmetric and real\\
$\sum_{k,i} I_{ik} \omega_{km} \omega_{in}^* = \sum_i I_m \omega_{im} \omega_{in}^*;\,\, \sum_{i,k} I_{ki}^* \omega_{in}^* \omega_{km} = \sum_k I_n^* \omega_{kn}^* \omega_{km}\\
I_{ki}^* = I_{ik}$ (Hermitian)\\
$\implies \sum_i I_m \omega_{im} \omega_{in}^* = \sum_k I_n^* \omega_{kn}^* \omega_{km}\\
\implies \sum_{\ell} I_m \omega_{\ell m} \omega_{\ell n}^* = \sum_{\ell} I_n^* \omega_{\ell n}^* \omega_{\ell m}\\
\implies (I_m - I_n^*) \sum_{\ell} \omega_{\ell m} \omega_{\ell n}^* = 0\\$
if $m=n \implies \sum_{\ell} \omega_{\ell m} \omega_{\ell m}^* = \vec{\omega}_m \cdot \vec{\omega}_m^* = | \vec{\omega}_m|^2 \geq 0$\\
$\implies (I_m - I_m^*)=0 \implies I_m=I_m^*$, i.e., $I_m$ is real.
$\{ \tilde{I} \}$ real $\implies \vec{\omega}_m$ is real


\hdashrule[0.5ex][c]{\linewidth}{0.5pt}{1.5mm}

$\star$ classical\\
Any real, symmetric tensor has the following properties:
1. diagonalization may be achieved by an appropriate rotation of axes, a similarity transformation
2. eigenvalues are obtained by the secular determinant and are real
3. eigenvectors are real and orthogonal


\hdashrule[0.5ex][c]{\linewidth}{0.5pt}{1.5mm}


Transformation of one coordinate system to another, represented by\\
$\vec{x} = \{ \lambda \} \vec{x}'$


\hdashrule[0.5ex][c]{\linewidth}{0.5pt}{1.5mm}


\item \underline{$\hat{R}_z(\theta) = \begin{pmatrix} \cos \theta & \sin \theta  & 0 \\ \sin \theta& \cos \theta & 0 \\ 0 & 0 & 1 \end{pmatrix}$}\\
$\hat{e}'_1 = \cos \theta \hat{e}_1 + \sin \theta \hat{e}_2\\
\hat{e}'_2 = - \sin \theta \hat{e}_1 + \cos \theta \hat{e}_2\\
\hat{e}'_3 = \hat{e}_3\\
\implies \begin{pmatrix} \hat{e}'_1 \\ \hat{e}'_2 \\ \hat{e}'_3 \end{pmatrix} = \begin{pmatrix} \cos \theta & \sin \theta & 0 \\ - \sin \theta & \cos \theta & 0 \\ 0 & 0 & 1 \end{pmatrix} \begin{pmatrix} \hat{e}_1 \\ \hat{e}_2 \\ \hat{e}_3 \end{pmatrix}\\$
so for example, if we wanted to rotate $\vec{A} \rightarrow \vec{A}'\\
\implies \vec{A}' = \hat{R}_z(\theta) \vec{A} = A_1' \hat{e}_1' + A'_2 \hat{e}_2' + A'_3 \hat{e}_3'\\
= (A_1'\,\, A_2'\,\, A_3') \begin{pmatrix} \hat{e}_1' \\ \hat{e}_2' \\ \hat{e}_3' \end{pmatrix} = (A_1' \,\, A_2' \,\, A_3' ) \hat{R}_z(\theta) \begin{pmatrix} \hat{e}_1 \\ \hat{e}_2 \\ \hat{e}_3 \end{pmatrix} = ( A_1 \,\, A_2 \,\, A_3) \begin{pmatrix} \hat{e}_1 \\ \hat{e}_2 \\ \hat{e}_3 \end{pmatrix}$


\hdashrule[0.5ex][c]{\linewidth}{0.5pt}{1.5mm}


\item \underline{$\vec{x}=\lambda_{\psi} \lambda_{\theta} \lambda_{\phi} \vec{x}'$}\\
Suppose we want to get from fixed system $\vec{x}'$ to $\vec{x}''''=\vec{x}$ (body system)\\
first rotate about $x_3'$ by $\phi$\\
$\implies \vec{x}''=\lambda_{\phi} \vec{x}'\\$
then take this system and rotate about $x_1''$ by $\theta \implies \vec{x}''' = \lambda_{\theta} \vec{x}''\\$
rotate about $x_3'''$ by $\psi$\\
$\implies \vec{x}'''' \equiv \vec{x} = \lambda_{\psi} \vec{x}''' = \lambda_{\psi} \lambda_{\theta} \vec{x}'' = \lambda_{\psi} \lambda_{\theta} \lambda_{\phi} \vec{x}'\\
\implies \lambda = \lambda_{\psi} \lambda_{\theta} \lambda_{\phi}\\
\lambda_{\phi} = \begin{pmatrix} \cos \phi & \sin \phi & 0 \\ - \sin \phi & \cos \phi & 0 \\ 0 & 0 & 1 \end{pmatrix} \sim x_3' \sim z - rotation\\
\lambda_{\theta} = \begin{pmatrix} 1 & 0 & 0 \\ 0 & \cos \theta & \sin \theta \\ 0 & - \sin \theta & \cos \theta \end{pmatrix} \sim x_1 '' \sim x-rotation\\
\lambda_{\psi} = \begin{pmatrix} \cos \psi & \sin \psi & 0 \\ - \sin \psi & \cos \psi & 0 \\ 0 & 0 & 1 \end{pmatrix} \sim x_3''' \sim $ z-rotation\\
\underline{Shortened:} rotate about $x_3'$ by $\phi$ then about $x_1''$ by $\theta$ then $x_3'''$ by $\psi$, i.e., $z\sim \phi,\,\, x \sim \theta,\,\, z \sim \psi$


\hdashrule[0.5ex][c]{\linewidth}{0.5pt}{1.5mm}


\underline{Spherical Coordinates:}\\
physics angles $\theta$ and $\phi$\\
$x=r \cos \phi \sin \theta\\
y = r \sin \theta \sin \phi\\
z=r \cos \theta$



\hdashrule[0.5ex][c]{\linewidth}{0.5pt}{1.5mm}



\item \underline{$\omega_1 = \dot{\phi} \sin \theta \sin \psi + \dot{\theta} \cos \psi,\,\, \omega_2 = \dot{\phi} \sin \theta \cos \psi - \dot{\theta} \sin \psi,\,\, \omega_3 = \dot{\phi} \cos \theta + \dot{\psi}$}(pg. 441)\\
$\omega$'s are calculated in the body system
for this derivation, the notation $\dot{\phi} \sim r$ means (current coordinates) $\sim$ (spherical coordinates) and is intended to show the analogy between the coordinates in this derivation and spherical coordinates. refer to 11-9 fig (c) for this derivation\\
$\theta \sim \theta,\,\, \psi \sim (90 - \phi),\,\ \dot{\phi} \sim r$ in $x_1,\,\, x_2,\,\, x_3$ system\\
$\dot{\phi}_1=R \sin \psi = \dot{\phi} \sin \theta \sin \psi \sim$ along $x_1\\$
$\dot{\phi}_2=R \cos \psi = \dot{\phi} \sin \theta \cos \psi \sim$ along $x_2\\$
$\dot{\phi}_3= \dot{\phi} \cos \theta \sim$ along $x_3\\$
\\
$\dot{\theta} \sim r,\,\, - \psi \sim \phi,\,\, 90 \sim \theta (i.e. \dot{\theta}$ is on the $x_1 x_2$ plane)\\
so $\dot{\theta}_1 = \dot{\theta} \cos \psi \\
\dot{\theta}_2 = - \dot{\theta} \sin \psi\\
\dot{\theta}_3=0\\
\\
\dot{\psi} \sim r,\,\, 0 \sim \theta,\,\, 0 \sim \phi\\
\dot{\psi}_1 = 0\\
\dot{\psi}_2 = 0\\
\dot{\psi}_3=\dot{\psi}$\\


\hdashrule[0.5ex][c]{\linewidth}{0.5pt}{1.5mm}


\item \underline{$
\begin{cases}
	(I_2-I_3) \omega_2 \omega_3 - I_1 \dot{\omega}_1 =0\\
	(I_3-I_1) \omega_3 \omega_1 - I_2 \dot{\omega}_2=0\\
	(I_1-I_2) \omega_1 \omega_2 - I_3 \dot{\omega}_3=0
\end{cases}
$}\\
$U=0 \implies L=T_{rot} + T_{trans},\,\,$ can always transform to the body system so that $T_{trans}=0\\$\\
\underline{recall:} $\{ T_{rot} = \frac{1}{2} \sum_{i,j} I_{ij} \omega_i \omega_j,\,\, I_{ij} = \delta_{ij} I_j \}$
$\implies L=T_{rot};\,\, T_{rot} = T = \frac{1}{2} \sum_i I_i \omega_i^2\\$
\underline{Note:} we rotated coordinates into principal axes\\
generalized coordinates = Euler angles in this derivation\\
\underline{recall:} $ \frac{\partial L}{\partial q_j} - \frac{d}{dt} \frac{\partial L}{\partial \dot{q}_j}=0\\
\implies \frac{\partial T}{\partial \psi} - \frac{d}{dt} \frac{\partial T}{\partial \dot{\psi}}=0$\\
$\implies \sum_i \frac{\partial T}{\partial \omega_i} \frac{\partial \omega_i}{\partial \psi} - \frac{d}{dt} \sum_i \frac{\partial T}{\partial \omega_i} \frac{\partial \omega_i}{\partial \dot{\psi}}=0\\
\begin{cases}
	\frac{\partial \omega_1}{\partial \psi} = \dot{\phi} \sin \theta \cos \psi - \dot{\theta} \sin \psi = \omega_2\\
	\frac{\partial \omega_2}{\partial \psi} = - \dot{\phi} \sin \theta \sin \psi - \dot{\theta} \cos \psi = -\omega_1\\
	\frac{\partial \omega_3}{\partial \psi} = 0
\end{cases}$
\\
and
\\
$\begin{cases} 
	\frac{\partial \omega_1}{\partial \dot{\psi}} = \frac{\partial \omega_2}{\partial \dot{\psi}}=0\\
	\frac{\partial \omega_3}{\partial \dot{\psi}}=1
\end{cases}$
and\\
$\frac{\partial T}{\partial \omega_i} = \frac{\partial}{\partial \omega_i} \frac{1}{2} \sum_j I_j \omega_j^2 = \frac{1}{2} \sum_j I_j \frac{\partial \omega_j}{\partial \omega_i} 2 \omega_j = I_i \omega_i\\
\implies \sum_i \frac{\partial T}{\partial \omega_i}\frac{\partial \omega_i}{\partial \psi} - \frac{d}{dt} \sum_i \frac{\partial T}{\partial \omega_i} \frac{\partial \omega_i}{\partial \dot{\psi}}\\
=\sum_i I_i \omega_i \frac{\partial \omega_i}{\partial \psi} - \frac{d}{dt} \sum_i I_i \omega_i \frac{\partial \omega_i}{\partial \dot{\psi}}\\
=I_1 \omega_1 \omega_2 + I_2 \omega_2(-\omega_1) - \frac{d}{dt}(I_3 \omega_3)\\$
(draw coord system; refer to page 447) we permute the axes to obtain two different equations (permutations)\\
$\implies 1 \rightarrow 2,\,\, 2 \rightarrow 3,\,\, 3 \rightarrow 1\\
\implies (I_2-I_3) \omega_2 \omega_3 - I_1 \dot{\omega}_1 = 0\\$
or\\
 $1 \rightarrow 3,\,\, 3 \rightarrow 2,\,\, 2 \rightarrow 1\\
\implies (I_3 - I_1) \omega_3 \omega_1 - I_2 \dot{\omega}_2=0\\
\therefore
\begin{cases}
	(I_2-I_3) \omega_2 \omega_3 - I_1 \dot{\omega}_1 =0\\
	(I_3 - I_1) \omega_3 \omega_1 - I_2 \dot{\omega}_2=0\\
	(I_1 - I_2) \omega_1 \omega_2 - I_3 \dot{\omega}_3=0
\end{cases}
$\\
(I wonder what would happen if we used $\theta$ or $\phi$ as generalized coordinates, would we just get the permutations?)


\hdashrule[0.5ex][c]{\linewidth}{0.5pt}{1.5mm}


\underline{
$\begin{cases}
	I_1 \dot{\omega}_1 -(I_2 - I_3) \omega_2 \omega_3 = N_1\\
	I_2 \dot{\omega}_2 - (I_3 - I_1) \omega_3 \omega_1 = N_2\\
	I_3 \dot{\omega}_3 - (I_1 - I_2) \omega_1 \omega_2 = N_3
\end{cases}$ (Eulers equations in a force field)}\\
\underline{recall:} $(\frac{d \vec{L}}{dt})_{fixed}=\vec{N}$ (this was derived in an in an inertial reference frame, hence "fixed")\\
\underline{recall:} $( \frac{d \vec{Q}}{dt})_{fixed} = (\frac{d \vec{Q}}{dt})_{rotating} + \vec{\omega} \times \vec{Q}\\
\implies (\frac{d \vec{L}}{dt})_{fixed} = ( \frac{d \vec{L}}{dt})_{body} + \vec{\omega} \times \vec{L} = \vec{N}\\
\implies (( \frac{d \vec{L}}{dt})_{body} + \vec{\omega} \times \vec{L})_3 = ( \vec{N})_3 = N_3\\
\implies \dot{L}_3 + ( \vec{\omega} \times \vec{L})_3 = \dot{L}_3 + \omega_1 L_2 - \omega_2 L_1 = N_3,\,\, (x_3 \sim$ body axis)\\
\underline{recall:} $L_i = I_i \omega_i (x_i$ aligned with principle axes)\\
$\implies I_3 \dot{\omega}_3 - (I_1 - I_2) \omega_1 \omega_2 = N_3\\
x_1 \rightarrow x_2,\,\, x_2 \rightarrow x_3,\,\, x_3 \rightarrow x_1\\
\implies I_1 \dot{\omega}_1 - (I_2 - I_3)\omega_2 \omega_3 = N_1\\
x_1 \rightarrow x_3,\,\, x_3 \rightarrow x_2,\,\, x_2 \rightarrow x_1\\
I_2 \dot{\omega}_2 - (I_3 - I_1) \omega_3 \omega_1 = N_2\\
\therefore
\begin{cases}
	I_1 \dot{\omega}_1 - (I_2 - I_3) \omega_2 \omega_3 = N_1\\
	I_2 \dot{\omega}_2 - (I_3 - I_1) \omega_3 \omega_1 = N_2\\
	I_3 \dot{\omega}_3 - (I_1 - I_2) \omega_1 \omega_2 = N_3
\end{cases}$


\hdashrule[0.5ex][c]{\linewidth}{0.5pt}{1.5mm}

$\star \star$
\item \underline{$(I_i - I_j) \omega_i \omega_j - \sum_k ( I_k \dot{\omega}_k - N_k) \epsilon_{ijk} = 0$}\\
\underline{recall:} 
$\begin{cases}
	(I_2 - I_3) \omega_2 \omega_3 - (I_1 \dot{\omega}_1 - N_1)=0\\
	(I_3-I_1) \omega_3 \omega_1 - ( I_2 \dot{\omega}_2 - N_2)=0\\
	(I_1-I_2) \omega_1 \omega_2 - ( I_3 \dot{\omega}_3 - N_3)=0\\
\end{cases}\\
(I_2-I_3) \omega_2 \omega_3 - (I_1 \dot{\omega}_1 -N_1)=0\\$
\underline{recall:} $\epsilon_{ijk} = $
$\begin{cases} $
	$1$ even permutation of $(1,2,3)\\$
	$-1$ odd permutation of $(1,2,3)\\$
	$0\,\, i=j$ or $j=k$ or $i =k\\
\end{cases}\\$
$\implies (I_2-I_3) \omega_2 \omega_3 - (I_1 \dot{\omega}_1 - N_1) \epsilon_{231} =(I_2 - I_3) \omega_2 \omega_3 - \sum_k ( I_k \dot{\omega}_k - N_k ) \epsilon_{23k} =0\\
\implies (I_i - I_j) \omega_i \omega_j - \sum_k(I_k \dot{\omega}_k - N_k) \epsilon_{ijk}=0\\$


\hdashrule[0.5ex][c]{\linewidth}{0.5pt}{1.5mm}

$\star$
\item \underline{$\begin{cases}
	\omega_1(t)=A \cos \Omega t\\
	\omega_2(t)=A \sin \Omega t\\
\end{cases}\,\, I_1=I_2 \neq I_3$ (symmetric top)}\\
$I_1=I_2 \neq I_3$ (symmetric top)\\
\underline{recall:} $(I_i - I_j) \omega_i \omega_j - \sum_{k} (I_k \dot{\omega}_k - N_k) \epsilon_{ijk}=0\\
\implies 
\begin{cases}
	(I_1 - I_3) \omega_2 \omega_3 - I_1 \dot{\omega}_1=0\\
	(I_3 - I_1) \omega_3 \omega_2- I_1 \dot{\omega}_2=0\\
	- I_3 \dot{\omega}_3=0
\end{cases}\\
I_3 \neq 0 ;\,\, I_3 \dot{\omega}_3=0 \implies \omega_3(t) = const.\\$
other two\\
$\implies 
\begin{cases}
	\dot{\omega}_1=-(\frac{I_3- I_1}{I_1}\omega_3) \omega_2\\
	\dot{\omega}_2 = (\frac{I_3-I_1}{I_1} \omega_3) \omega_1
\end{cases}\\
\Omega \equiv \frac{I_3 - I_1}{I_1} \omega_3 \\
\implies
\begin{cases}
	\dot{\omega}_1 + \Omega \omega_2=0\\
	\dot{\omega}_2 - \Omega \omega_1 = 0\\
\end{cases}
\implies
\begin{cases}
	\dot{\omega}_1 + \Omega \omega_2=0\\
	i \dot{\omega}_2-i \Omega \omega_1 = 0
\end{cases}
$
(smart)\\
$\implies ( \dot{\omega}_1 + i \dot{\omega}_2) + \Omega(\omega_2 - i \omega_1)\\
= (\dot{\omega}_1 + i \dot{\omega}_2) + i \Omega (-i \omega_2) - \omega_1)\\
=( \dot{\omega}_1+i \dot{\omega}_2) - i \Omega(\omega_1 + i \omega_2) = 0\\
\eta \equiv \omega_1 + i \omega_2\\
\implies \dot{\eta} - i \Omega \eta =0 \implies \eta(t) = A e^{i \Omega t} \implies \eta = \omega_1 + i \omega_2 = A \cos \Omega t + i A \sin \Omega t\\
\implies \begin{cases}
	\omega_1(t)=A \cos \Omega t\\
	\omega_2(t)=A \sin \Omega t\\
\end{cases}\\$
$\star \star$ need to figure out a more intuitive way to solve that differential equation, using basic methods from ODE's


\hdashrule[0.5ex][c]{\linewidth}{0.5pt}{1.5mm}


\underline{Note:} $|\vec{\omega}| = \sqrt{ \omega_1^2 + \omega_2^2 + \omega_3^2} = \sqrt{A^2 + const} = const\\$


\hdashrule[0.5ex][c]{\linewidth}{0.5pt}{1.5mm}


\underline{Note:} $\vec{\omega}$ precesses about $x_3$ with frequency $\Omega;\\$
force free $\implies \vec{L} \sim const \implies T_{rot} = \frac{1}{2} \vec{\omega} \cdot \vec{L} = const.\\$


\hdashrule[0.5ex][c]{\linewidth}{0.5pt}{1.5mm}


\underline{claim:} $\vec{L}, \vec{\omega}, \vec{e}_3$ (body) lie in the same plane, i.e., $\vec{L} \cdot (\vec{\omega} \times \vec{e}_3)=0\\$
\underline{Proof:}\\
$\vec{\omega} \times \hat{e}_3 = 
\begin{vmatrix}
	\hat{e}_1 & \hat{e}_2 & \hat{e}_3 \\ \omega_1 & \omega_2 & \omega_3 \\ 0 & 0 & 1 
\end{vmatrix}
=\omega_2 \hat{e}_1 - \omega_1 \hat{e}_2\\
\implies \vec{L} \cdot( \vec{\omega} \times \hat{e}_3) = I_1 \omega_1 \omega_2 - I_2 \omega_1 \omega_2 = I_1 \omega_1 \omega_2 - I_1 \omega_1 \omega_2 = 0$\\
\\


\hdashrule[0.5ex][c]{\linewidth}{0.5pt}{1.5mm}


Skipped 9.6-9.11


\hdashrule[0.5ex][c]{\linewidth}{0.5pt}{1.5mm}

















\section*{\underline{Electrodynamics Notes}}

\section*{Chapter 2}

\hdashrule[0.5ex][c]{\linewidth}{0.5pt}{1.5mm}

$\vec{F}=\frac{1}{4 \pi \epsilon_{0}} \frac{q Q}{\scripty{r}^2} \hat{\scripty{r}}$;		$\vec{\scripty{r}}:=\vec{r}-\vec{r'}$;	$\vec{r'} \sim$ source;		$\vec{r} \sim$ field point

\hdashrule[0.5ex][c]{\linewidth}{0.5pt}{1.5mm}

\item \underline{$\vec{F}=Q\vec{E}$}\\
$\vec{F}=\sum_i \vec{F_i}=\frac{1}{4 \pi \epsilon_0} \sum_i \frac{q_i Q}{\scripty{r}_i^2}\hat{\scripty{r}_i}=
Q \sum_i \vec{E_i}=Q \vec{E}$


\hdashrule[0.5ex][c]{\linewidth}{0.5pt}{1.5mm}


\item \underline{$\vec{E}(\vec{r})=\frac{1}{4 \pi \epsilon_0} \int \frac{\rho(\vec{r}')}{\scripty{r}^2} \hat{\scripty{r}}\, d \tau'$}\\

$\vec{E}=\frac{1}{4 \pi \epsilon_0} \sum_i \frac{q_i}{\scripty{r}_i^2}\hat{\scripty{r}}_i \rightarrow \frac{1}{4 \pi \epsilon_0} \int \frac{1}{\scripty{r}^2} \hat{\scripty{r}}\, dq,\,\, dq=\rho(\vec{r}') d \tau '$\\
$\implies \vec{E}(\vec{r})=\frac{1}{4 \pi \epsilon_0} \int \frac{\rho(\vec{r}')}{\scripty(r)^2} \hat{\scripty{r}} d \tau'$


\hdashrule[0.5ex][c]{\linewidth}{0.5pt}{1.5mm}


\item \underline{$\oint \vec{E} \cdot d \vec{a} = \frac{q}{\epsilon_0};\,\,$ (point charge placed at origin)}\\
\underline{recall:} $\vec{E} = \capk \frac{q}{r^2} \hat{r};\,\, d \vec{a} = r^2 \sin \theta d \theta d \phi \hat{r}\\
\oint \vec{E} \cdot d \vec{a} = \frac{q}{4 \pi \epsilon_0} \int_0^{2 \pi} \int_0^{\pi} \frac{1}{r^2} r^2 \sin \theta d \theta d \phi= \frac{q}{\epsilon_0}$


\hdashrule[0.5ex][c]{\linewidth}{0.5pt}{1.5mm}



\underline{ $\oint \vec{E} \cdot d\vec{a}=\frac{Q_{enc}}{\epsilon_0}$ (discrete charge distribution)}\\
$\oint \vec{E} \cdot d \vec{a} = \sum_i ^n (\oint \vec{E_i} \cdot d \vec{a}) = \sum_{i=1}^n (\frac{1}{\epsilon_0} q_i)$\\
$\therefore \oint \vec{E} \cdot d \vec{a} = \frac{Q_{enc}}{\epsilon_0}$


\hdashrule[0.5ex][c]{\linewidth}{0.5pt}{1.5mm}


\item \underline{$\nabla \cdot \vec{E} = \frac{\rho}{\epsilon_0}$}\\
$\oint \vec{E} \cdot d \vec{a} = \frac{1}{\epsilon_0} \int \rho d \tau' \implies \int \nabla \cdot \vec{E} d \tau'=\frac{1}{\epsilon_0} \int \rho d \tau'$\\
$\therefore \nabla \cdot \vec{E} = \frac{\rho}{\epsilon_0}$


\hdashrule[0.5ex][c]{\linewidth}{0.5pt}{1.5mm}


\item \underline{$\nabla \cdot (\frac{\hat{r}}{r^2}) = 4 \pi \delta^3(\vec{r}) \implies \nabla \cdot ( \frac{\hat{\scripty{r}}}{\scripty{r}^2})=4 \pi \delta^3(\vec{\scripty{r}})$}\\
$\nabla \cdot (\frac{\hat{r}}{r^2})=\frac{1}{r^2} \frac{\partial}{\partial r}(r^2 \frac{1}{r^2})=0$ if $r \neq 0$\\
$\int \nabla \cdot (\frac{\hat{r}}{r^2}) d \tau ' = \int (\frac{\hat{r}}{r^2})(r^2 \sin \theta d \phi d \theta \hat{r}) = 2 \pi \int_0^\pi \sin \theta d \theta=4 \pi$\\
Integral ~ constant and zero everywhere but origin\\
$\therefore \nabla(\frac{r}{r^2})=4 \pi \delta^3(\vec{r})$


\hdashrule[0.5ex][c]{\linewidth}{0.5pt}{1.5mm}


\item \underline{$\nabla \cdot \vec{E} = \frac{1}{\epsilon_0} \rho(\vec{r}')$ (continuous charge distribution)}\\

$\vec{E}(\vec{r})=\frac{1}{4 \pi \epsilon_0} \int \frac{\hat{\scripty{r}}}{\scripty{r}^2} \rho(\vec{r}') d \tau'$\\
$\nabla \cdot \vec{E}=\frac{1}{4 \pi \epsilon_0} \int \rho(\vec{r}') \nabla \cdot (\frac{\hat{\scripty{r}}}{\scripty{r}^2} )d \tau'\\
=\frac{1}{4 \pi \epsilon_0} \int 4 \pi \delta^3 (\vec{\scripty{r}})\rho(\vec{r}') d \tau'\\
\therefore \nabla \cdot \vec{E}=\frac{1}{\epsilon_0} \rho(\vec{r})$


\hdashrule[0.5ex][c]{\linewidth}{0.5pt}{1.5mm}


\item \underline{$\oint \vec{E} \cdot d \vec{\ell}=0;\,\, \nabla \times \vec{E}=0$ (point charge)}\\
$\vec{E}=\capk \frac{q}{r^2}\hat{r} \implies \int_{\vec{a}}^{\vec{b}} \vec{E} \cdot d \vec{\ell}; \,\, d \ell = dr \hat{r} + r d\theta \hat{\theta} + r \sin \theta d \phi \hat{\phi}\\
\implies \int_{\vec{a}}^{\vec{b}} \vec{E} \cdot d \vec{\ell}=\capk q \int_a^b r^{-2} dr =\capk q ( \frac{1}{r_a}-\frac{1}{r_b})$\\
if $r_a=r_b \implies \oint \vec{E} \cdot d \vec{\ell}=0 \implies \nabla \times \vec{E}=0$


\hdashrule[0.5ex][c]{\linewidth}{0.5pt}{1.5mm}


$\nabla \times \vec{E} = \nabla \times \sum_i \vec{E_i} = \sum_i \nabla \times \vec{E_i}=0$ (discrete)


\hdashrule[0.5ex][c]{\linewidth}{0.5pt}{1.5mm}


Since $ \oint \vec{E} \cdot d \vec{\ell}=0 \implies$ independent of path $\implies V(\vec{r}):=-\int_O^{\vec{r}} \vec{E} \cdot d \vec{\ell}$ with $O$ being the reference point.


\hdashrule[0.5ex][c]{\linewidth}{0.5pt}{1.5mm}


\item \underline{$V(\vec{b})-V(\vec{a})=-\int_{\vec{a}}^{\vec{b}} \vec{E} \cdot d \vec{\ell}$}\\

$V(\vec{b})-V(\vec{a})=-\int_O^{\vec{b}} \vec{E} \cdot d \vec{\ell} + \int_O^{\vec{a}} \vec{E} \cdot d \vec{\ell}\\
=-(\int_O^{\vec{b}} \vec{E} \cdot d \vec{\ell} + \int_{\vec{a}}^{O} \vec{E} \cdot d \vec{\ell})=-\int_{\vec{a}}^{\vec{b}} \vec{E} \cdot d \vec{l}$



\hdashrule[0.5ex][c]{\linewidth}{0.5pt}{1.5mm}


\item \underline{$\vec{E}=-\nabla V$}\\
$V(\vec{b})-V(\vec{a})=\int_{\vec{a}}^{\vec{b}} \nabla V \cdot d \vec{\ell} = - \int_{\vec{a}}^{\vec{b}} \vec{E} \cdot d \vec{\ell}\\
\implies \vec{E}=-\nabla V$


\hdashrule[0.5ex][c]{\linewidth}{0.5pt}{1.5mm}


$V'(\vec{r})=-\int_{O'}^{\vec{r}} \vec{E} \cdot d \vec{\ell} = - \int_{O'}^O \vec{E} \cdot d \vec{\ell} -  \int_{O}^{\vec{r}}\vec{E} \cdot d \vec{\ell} =K + V(\vec{r})$


\hdashrule[0.5ex][c]{\linewidth}{0.5pt}{1.5mm}


$V'(\vec{b})-V'(\vec{a})=V(\vec{b})-V(\vec{a}); \,\, \nabla V'=\nabla V$


\hdashrule[0.5ex][c]{\linewidth}{0.5pt}{1.5mm}


\item \underline{$V=\sum_i V_i$}


$\vec{F}=\sum_i \vec{F_i}=Q \sum_i \vec{E_i}=Q \vec{E} \implies \vec{E} = \sum_i \vec{E_i}\\
\implies V=-\int \vec{E} \cdot d \vec{\ell}= \sum_i (-\int \vec{E_i} \cdot d \vec{\ell})=\sum_i V_i$


\hdashrule[0.5ex][c]{\linewidth}{0.5pt}{1.5mm}

\item \underline{$\nabla^2  V= -\frac{\rho}{\epsilon_0}$}\\

$\vec{E}=-\nabla V,\,\, \nabla \cdot \vec{E}=\nabla \cdot (-\nabla V)=-\nabla^2 V= \frac{\rho}{\epsilon_0}$


\hdashrule[0.5ex][c]{\linewidth}{0.5pt}{1.5mm}


 \item \underline{$V(\vec{r})=\capk \int \frac{\rho(\vec{r}')}{\scripty{r}} d \tau '$} (reference point at $\infty$)

$\vec{E}=\capk \frac{q}{r^2} \hat{r},\,\, d \vec{\ell}=dr \hat{r} + r d \theta \hat{\theta} + r \sin \theta d \phi \hat{\phi}\\
V(\vec{r})=-\int_O^{\vec{r}} \vec{E} \cdot d \vec{r}'=-\capk \int_{\infty}^{r} \frac{q}{r'^2} d r'=\capk (\frac{q}{r'}|_\infty^r=\capk \frac{q}{r}$\\
in general $V(\vec{r})=\capk \frac{q}{\scripty{r}}$ for a point charge\\
$\implies V(\vec{r})=\sum_i V_i = \sum_i^n \capk \frac{q_i}{\scripty{r}_i};\,\, dq=\lambda(\vec{r}');\,\, dq=\sigma(\vec{r}') d a'$\\
$\therefore V(\vec{r})=\capk \int \frac{1}{\scripty{r}} dq= \capk \int \frac{\rho(\vec{r}')}{\scripty{r}} d \tau'$


\hdashrule[0.5ex][c]{\linewidth}{0.5pt}{1.5mm}


\item \underline{$\vec{E}_{above}-\vec{E}_{below}=\frac{\sigma}{\epsilon_0} \hat{n}$}

$\oint \vec{E} \cdot d \vec{A}=\int \vec{E_a} \cdot d \vec{A_a} + \int \vec{E_b} \cdot d \vec{A_b}= \int \vec{E_a} \cdot (\hat{n}_a dA) + \int \vec{E_b} \cdot (\hat{n}_b dA)\\
\hat{n}_a=-\hat{n}_b=\hat{n} \implies \oint \vec{E} \cdot d \vec{A}= \int (\vec{E_a} \cdot \hat{n}-\vec{E_b} \cdot \hat{n}) dA=\frac{Q}{\epsilon_0}\\
=\frac{1}{\epsilon_0} \int \sigma dA \implies \vec{E_a}\cdot \hat{n}-\vec{E_b} \cdot \hat{n}= E_a^{\perp} - E_b^{\perp}=\frac{\sigma}{\epsilon_0}\\
\oint \vec{E} \cdot d \vec{\ell}=\int \vec{E_a} \cdot (\hat{\ell} d \ell)-\int \vec{E_b} \cdot (\hat{\ell} d \ell)\\
\implies \vec{E_a} \cdot \vec{l}-\vec{E_b} \cdot \vec{l}=E_a^{\parallel}-E_b^{\parallel}=0\\
\therefore \vec{E_a}-\vec{E_b}=\frac{\sigma}{\epsilon_0} \hat{n}$


\hdashrule[0.5ex][c]{\linewidth}{0.5pt}{1.5mm}


\underline{Note}: $\vec{E_a} \cdot \hat{n}-\vec{E_b} \cdot \hat{n}=\frac{\sigma}{\epsilon_0} \hat{n} \cdot \hat{n}=E_a^{\perp} - E_b^{\perp}=\frac{\sigma}{\epsilon}$ and $\vec{E_a} \cdot \hat{\ell}-\vec{E_b} \cdot \hat{\ell}=\frac{1}{\epsilon_0}\hat{n} \cdot \hat{l}=E_b^{\parallel}-E_a^{\parallel}=0$


\hdashrule[0.5ex][c]{\linewidth}{0.5pt}{1.5mm}


\item \underline{$\frac{\partial V_a}{\partial n} - \frac{\partial V_b}{\partial n} = - \frac{\sigma}{\epsilon_0};\,\, V_{above}=V_{below}$}\\
$V_{above}-V_{below}=-\int_{\vec{a}}^{\vec{b}} \vec{E} \cdot d \vec{l} = 0$ since $\vec{b}-\vec{a}=\epsilon$\\
$\implies V_a=V_b$\\
$\underline{recall}: \int(\vec{E_a} \cdot \hat{n}- \vec{E_b} \cdot \hat{n}) da=- \int (\nabla V_a \cdot \hat{n}- \nabla V_b \cdot \hat{n} ) da= - \int ( \frac{\partial V_a}{\partial n}- \frac{\partial V_b}{\partial n}) da = \int \frac{\sigma}{\epsilon_0} da\\
\implies \frac{\partial V_a}{\partial n} - \frac{\partial V_b}{\partial n}=-\frac{\sigma}{\epsilon_0}$ why $\frac{\partial V}{\partial n}=\nabla V \cdot \hat{n}?$ (right side is directional derivative)


\hdashrule[0.5ex][c]{\linewidth}{0.5pt}{1.5mm}


\item \underline{$V(\vec{b})-V(\vec{a})=W/Q$} or $W=QV(\vec{r})$ if ref is at $\infty$
the work you must do to move a charge from a to b

$W=\int_{\vec{a}}^{\vec{b}} \vec{F}_{ex} \cdot d \vec{\ell}=-\int_{\vec{a}}^{\vec{b}} \vec{F}_{electric} \cdot d \vec{\ell}=-Q \int_{\vec{a}}^{\vec{b}} \vec{E} \cdot d \vec{\ell} = Q(V(\vec{b})-V(\vec{a}))$\\
or $W=Q(V(\vec{r})-V(\infty))=QV(\vec{r})$


\hdashrule[0.5ex][c]{\linewidth}{0.5pt}{1.5mm}


\underline{Note:} for this next part it is helpful to visualize $\frac{q_i q_j}{\scripty{r}_{ij}}$ as a matrix, this matrix is symmetric, and note that $\sum_i \sum_{j>i}$ is just the sum of all components above the main diagonal. Viewing it this way makes it obvious that $\sum_i \sum_{j>i} = 2 \sum_i \sum{j \neq i}$\\


\hdashrule[0.5ex][c]{\linewidth}{0.5pt}{1.5mm}


\item \underline{$W=\capk \sum_{i=1}^n \sum_{j>i}^n \frac{q_i q_j}{\scripty{r}_{ij}}$}\\
bring $q_1$ in from $\infty\\
 W_1=0$\\
Bring $q_2$ to position $\vec{r}_2$\\
$W_2=q_2 V_1(\vec{r}_2)=\capk q_2 \frac{q_1}{\scripty{r}_{12}}\\
W_3=\capk q_3 (\frac{q_1}{\scripty{r}_{13}} + \frac{q_2}{\scripty{r}_{23}})\\
W_4=\capk q_4(\frac{q_1}{\scripty{r}_{14}}+ \frac{q_2}{\scripty{r}_{24}}+ \frac{q_3}{\scripty{r}_{34}})\\
W_{tot}=\capk (\frac{q_1 q_2}{\scripty{r}_{12} }+\frac{q_1 q_3}{\scripty{r}_{13}}+\frac{q_1 q_4}{\scripty{r}_{14}}+\frac{q_2 q_3}{\scripty{r}_{23}}+\frac{q_2 q_4}{\scripty{r}_{24}}+\frac{q_3 q_4}{\scripty{r}_{34}})\\
=\capk(q_1 \sum_{j>1}^4 \frac{q_j}{\scripty{r}_{1j} }+ q_2 \sum_{j>2}^4 \frac{q_j}{\scripty{r}_{2j}} + q_3 \sum_{j>3}^4 \frac{q_j}{\scripty{r}_{3j}}=\capk \sum_{i=1}^4 \sum_{j>i}^4 \frac{q_i q_j}{\scripty{r}_{ij}}\\
\therefore W=\capk \sum_{i=1}^n \sum_{j>i}^n \frac{q_i q_j}{\scripty{r}_{ij}}$


\hdashrule[0.5ex][c]{\linewidth}{0.5pt}{1.5mm}


\item \underline{$W=\frac{1}{2} \sum_{i=1}^n q_i V(\vec{r}_i)$}\\
$\underline{recall:} W=\capk \sum_{i=1}^n \sum_{j>i}^n \frac{q_i q_j}{\scripty{r}_{ij}}\\
\underline{Note:} W=\capk \, [ \, \sum_{j>1}^n \frac{q_1 q_j}{\scripty{r}_{ij}}+ \sum_{j>2}^n \frac{q_2 q_j}{\scripty{r}_{2j}}+ \sum_{j>3}^n \frac{q_3 q_j}{\scripty{r}_{3 j}} +\cdot \cdot \cdot ]\\
=\capk [(\frac{q_1 q_2}{\scripty{r}_{12}} + \frac{q_1 q_3}{\scripty{r}_{13}}+ \cdot \cdot \cdot) + (\frac{q_2 q_3}{\scripty{r}_{23}} + \cdot \cdot \cdot) + ( \cdot \cdot \cdot)]\\
$\underline{Also Note:}$ \capk \sum_{i=1}^n \sum_{j \neq i}^n \frac{q_i q_j}{\scripty{r}_{ij}}= \capk [ \sum_{j \neq 1}^n \frac{q_1 q_j}{\scripty{r}_{1j}} + \sum_{j \neq 2}^n \frac{q_2 q_j}{\scripty{r}_{2j}} + \sum_{j \neq 3}^n \frac{q_3 q_j}{\scripty{r}_{3 j}} + \cdot \cdot \cdot] =\capk [ (\frac{q_1 q_2}{\scripty{r}_{12}} + \frac{q_1 q_3}{\scripty{r}_{13}}+ \cdot \cdot \cdot) + ( \frac{q_2 q_1}{\scripty{r}_{21}}+ \frac{q_2 q_3}{\scripty{r}_{23}}+ \cdot \cdot \cdot) + ( \frac{q_3 q_1}{\scripty{r}_{31}} + \frac{q_3 q_2}{\scripty{r}_{32} }+ \cdot \cdot \cdot)=\capk [ (2 \frac{q_1 q_2}{\scripty{r}_{12}}+ 2 \frac{q_1 q_3}{\scripty{r}_{13}} + 2 \frac{q_2 q_3}{\scripty{r}_{23}} ] = 2W\\
\implies W= \capk \sum_{i=1}^n \sum_{j \neq i}^n \frac{q_i q_j}{\scripty{r}_{ij}}=\frac{1}{2} \sum_{i=1}^n q_i ( \sum_{j \neq i}^n \frac{q_j}{\scripty{r}_{ij}})=\frac{1}{2} \sum_{i=1}^n q_i V(\vec{r}_i)$\\
$V(\vec{r}_i)$ is the potential of all other charges besides $q_i$ at the position $\vec{r}_i$


\hdashrule[0.5ex][c]{\linewidth}{0.5pt}{1.5mm}


\item \underline{$W=\frac{1}{2} \int \rho V d \tau$}\\
$W=\frac{1}{2} \sum_{i=1}^n q_i V(\vec{r}_i) \implies W=\frac{1}{2} \int V(\vec{r}) d q = \frac{1}{2} \int \rho V d \tau'$\\


\hdashrule[0.5ex][c]{\linewidth}{0.5pt}{1.5mm}


\item \underline{$W=\frac{\epsilon_0}{2} \int_{all space} E^2 d \tau$}\\
$W= \frac{1}{2} \int \rho V d \tau';$\,\, get in terms of fields\\
$\implies \rho = \epsilon_0 \nabla \cdot \vec{E} \implies W= \frac{\epsilon_0}{2} \int (\nabla \cdot \vec{E}) V d \tau'\\
\underline{recall:} \nabla \cdot ( \vec{E} V)= V( \nabla \cdot \vec{E} + \vec{E} \cdot \nabla V\\
\implies W=\frac{\epsilon_0}{2} [ \int ( \nabla \cdot (\vec{E} V) - \vec{E} \cdot \nabla V) d \tau' ]\\
= \frac{\epsilon_0}{2} [ \int \nabla \cdot ( \vec{E} V) d \tau' + \int E^2 d \tau' ]\\
=\frac{\epsilon_0}{2} [\int E^2 d \tau' + \oint \vec{E} V \dot d \vec{a} ] da \approx r^2, \vec{E} V \approx \frac{1}{r^3}\\
\implies \vec{E} V \cdot d \vec{a} \approx \frac{1}{r} \rightarrow 0$ as $r \rightarrow \infty\\
\therefore W= \frac{\epsilon_0}{2} \int_{all space} E^2 d \tau'$\\


\hdashrule[0.5ex][c]{\linewidth}{0.5pt}{1.5mm}


\underline{Note:} for $\vec{E} = \vec{E}_1 + \vec{E}_2 W\neq W_1 + W_2\,\,$ since\\
$W=\frac{\epsilon_0}{2} \int ( \vec{E}_1 + \vec{E}_2 )^2 d \tau'$


\hdashrule[0.5ex][c]{\linewidth}{0.5pt}{1.5mm}


\underline{Conductor Properties}\\
(i) $\vec{E}=0$ Inside a conductor\\
(ii) $\rho = 0$ since $\nabla \cdot \vec{E} = 0$\\
(iii) net charge lies on the surface of conductor\\
(iv) conductor is an equipotential, if $\vec{a}$, $\vec{b}$ in conductor then\\
 $V(\vec{b}) -V(\vec{a})=-\int_{\vec{a}}^{\vec{b}} \vec{E} \cdot d \vec{\ell} = 0 \implies V(\vec{b})=V(\vec{a})\\$
(v) $\vec{E}$ perp to surface outside conductor\\


\hdashrule[0.5ex][c]{\linewidth}{0.5pt}{1.5mm}


\underline{ $\sigma= - \epsilon_0 \frac{\partial V}{\partial n}$ (on a conductor)}\\
\underline{recall:} $\vec{E}_a-\vec{E}_b = \frac{\sigma}{\epsilon_{0}} \hat{n} \implies \vec{E}=\frac{\sigma}{\epsilon_0} \hat{n}$ since $\vec{E}_b=0$ (inside conductor)\\
$\implies E^{\perp} = -\frac{\partial V}{\partial n} = \frac{\sigma}{\epsilon_0} \implies \sigma= - \epsilon_0 \frac{\partial V}{\partial n}\\$


\hdashrule[0.5ex][c]{\linewidth}{0.5pt}{1.5mm}


\item \underline{$\vec{E}_{other} = \frac{1}{2} ( \vec{E}_a+\vec{E}_b)=\vec{E}_{avg} \implies \vec{f}=\sigma \vec{E}_{avg} = \frac{1}{2} \sigma ( \vec{E}_a+ \vec{E}_b)$}\\
\underline{Note:} $\vec{E}_a,\vec{E}_b$ also include $\vec{E}_{other}$
$\vec{E} = \vec{E}_{patch} + \vec{E}_{other} \\$
for a surface charge\\
$\oint \vec{E} \cdot d \vec{a} = 2 E A = \frac{\sigma}{\epsilon_0} A \implies E = \frac{\sigma}{2 \epsilon_0}\\
\implies ( \vec{E}_{patch})_{above} = - (\vec{E}_{patch})_{below} = \frac{ \sigma}{2 \epsilon_0}\\
\implies \vec{E}_{above}= \vec{E}_{other} + \frac{\sigma}{2 \epsilon_0} \hat{n}\\
\vec{E}_{below} = \vec{E}_{other} - \frac{\sigma}{2 \epsilon_0} \hat{n}\\
\implies \vec{E}_{other} = \frac{1}{2} ( \vec{E}_{above} + \vec{E}_{below}) = \vec{E}_{avg}$


\hdashrule[0.5ex][c]{\linewidth}{0.5pt}{1.5mm}


\item \underline{$\vec{f} = \frac{1}{2 \epsilon_0} \sigma^2 \hat{n};\,\, \vec{p} = \frac{\epsilon_0}{2} E^2$} (conductor) ( electrostatic pressure)\\
$\vec{E}_a - \vec{E}_b = \frac{\sigma}{\epsilon_0} \hat{n} \implies \vec{E}_a = \frac{\sigma}{\epsilon_0} \hat{n};\,\, \vec{f} = \frac{\sigma}{2} \vec{E}_a = \frac{1}{2 \epsilon_0} \sigma^2\\
\sigma= \epsilon_0 E \implies |\vec{f}| = P = \frac{\epsilon_0}{2} E^2$

\hdashrule[0.5ex][c]{\linewidth}{0.5pt}{1.5mm}




$V=V_+-V_- = - \int_{(-)}^{(+)} \vec{E} \cdot d \vec{l}$ (conductors)\\
\underline{Theorem:} If you double $Q$ you double $\rho$ for a conductor\\
\underline{proof:}\\
Suppose I have a charge $Q$ on a conductor with electric field given by $\vec{E}_0= \capk \int \frac{\rho}{\scripty{r}^2} \hat{\scripty{r}} d \tau'$. Now suppose I double $Q$, a way to obtain the corresponding electric field is to double $\vec{E}_0$ which is to say $\vec{E} = 2 \vec{E}_0 = \capk \int \frac{2 \rho}{\scripty{r}^2} \hat{\scripty{r}} d \tau'$. In other words, to obtain the new electric field simply double $\rho$. The second uniqueness theorem guarantees that this electric field uniquely corresponds to the field corresponding to $2Q$.\\


\hdashrule[0.5ex][c]{\linewidth}{0.5pt}{1.5mm}


\item \underline{$C \equiv \frac{Q}{V}$}\\
$\vec{E}= \frac{1}{4 \pi \epsilon_0} \int \frac{\rho}{\scripty{r}} \hat{\scripty{r}} d \tau$\\
double $Q \implies$ double $\rho \implies$ double $\vec{E}$\\
$\implies$ double $V \implies V \propto Q \implies Q=CV\\
\therefore C \equiv \frac{Q}{V}$\\


\hdashrule[0.5ex][c]{\linewidth}{0.5pt}{1.5mm}


\item \underline{$C=\frac{A \epsilon_0}{d}$}\\
$C=\frac{Q}{V} \implies \oint \vec{E} \cdot d \vec{A} = EA = \frac{Q}{\epsilon_0} \implies E = \frac{Q}{A \epsilon_0} \implies V= Ed = \frac{Qd}{A \epsilon}\\
\therefore C= \frac{A \epsilon_0}{d}$


\hdashrule[0.5ex][c]{\linewidth}{0.5pt}{1.5mm}


\item \underline{$W=\frac{1}{2} C V^2$}\\
$W= \int_a^b \vec{F}_{ex} \cdot d \vec{\ell} = - \int_a^b \vec{F}_{el} \cdot d \vec{\ell} = - \int_a ^b dq \vec{E} \cdot d \vec{\ell}\\
=V dq;\,\, Q= CV \implies \int_{0}^{q} \frac{Q}{C} d Q = \frac{1}{2} \frac{Q^2}{C} = \frac{1}{2} C V^2\\$


 \hdashrule[0.5ex][c]{\linewidth}{0.5pt}{1.5mm}
 
 
 \section*{\underline{EM: Chapter 3}}
 
 \underline{Theorem}
 1. The value of $V$ at point $\vec{r}$ is the average value of $V$ over a spherical surface of radius $R$ centered at $\vec{r}$:\\
 $V(\vec{r}) = \frac{1}{4 \pi R^2} \oint_{sphere} V da.\\$
 
 2. As a consequence, $V$ can have no local maxima or minima; the extreme values of $V$ must occur at the boundaries. (For if $V$ had a local maximum at $\vec{r}$, then by the very nature of maximum I could draw a sphere around $\vec{r}$ over which all values of $V$ -- and a fortiori the average-- would be less than at $\vec{r}$.)\\
 
 \underline{Proof:}\\
 calculate avg potential of sphere of radius $R$ due to charge out of sphere\\
$ \implies V = \frac{1}{4 \pi \epsilon_0} \frac{q}{\scripty{r}}$ on surface\\
 $z \sim$ distance from charge to center of sphere;$\,\, R \sim$ sphere radius,$\,\, \scripty{r} \sim$ dist from surface to charge.\\
$ \vec{\scripty{r}} \cdot \vec{\scripty{r}} = \scripty{r}^2 = ( \vec{z} - \vec{R}) \cdot ( \vec{z} - \vec{R}) = z^2 + R^2 -2 z R \cos \theta \\
 da = R^2 \sin \theta d \theta d \phi\\
 V_{ave}= \frac{1}{4 \pi R^2} \frac{q}{4 \pi \epsilon_0} \int ( z^2 + R^2 - 2 z R \cos \theta)^{-1/2} R^2 \sin \theta d \theta d \phi\\
 U=z^2 + R^2 - 2 z R \cos \theta;\,\, dU=2 z R \sin \theta d \theta\\
 \implies V_{ave} = \frac{q}{4 \pi \epsilon_0} \frac{1}{2 z R} ( \sqrt{z^2 + R^2 - 2 z R \cos \theta})|_0^{\pi}\\
 =\frac{q}{4 \pi \epsilon} \frac{1}{2 z R} [ ( z+ R) - (z-R)]= \frac{1}{4 \pi \epsilon_0} \frac{q}{z}\\$
 =potential caused by q at center of sphere.\\
 
 
  \hdashrule[0.5ex][c]{\linewidth}{0.5pt}{1.5mm}


\underline{First uniqueness theorem (Laplace equation): } The solution to Laplace's equation on some volume V is uniquely determined if V is specified on the boundary surface S.

\underline{Proof:}\\
Given B on boundary assume there are two solutions inside\\
$\nabla^2 V_1=0 and \nabla^2 V_2=0\\$
$V_3 \equiv V_1 - V_2\\
\implies \nabla^2 V_3 = \nabla^2 V_1 - \nabla^2 V_2=0\\$
$V_3$ takes value 0 on boundaries since $V_1 = V_2$ there. Since all extrema occur on the boundaries $V_3 = 0\\
\therefore V_2 = V_1\\$


\underline{Corollary (Poisson's equation): } The potential in a volume V is uniquely determined if (a) the charge density throughout the region, and (b) the value of V on all boundaries, are specified.

\underline{Proof: }\\
Assume not. $\nabla^2 V_1 = - \frac{1}{\epsilon_0} \rho\\
\nabla^2 V_2 = -\frac{1}{\epsilon_0} \rho \implies \nabla^2 V_3 = \nabla^2 V_1 - \nabla^2 V_2 = 0$\\
and $V_3$ is zero on boundaries $\implies V_3 =0 \implies V_1 = V_2$


\underline{Second uniqueness theorem: } in a volume V surrounded by conductors and containing a specified charge density $\rho$, the electric field is uniquely determined if the total charge on each conductor is given. (The region as a whole can be bounded by another conductor, or else unbounded.) 
 
 \underline{Proof:}\\
 Spose $\nabla \cdot \vec{E}_1 = \frac{\rho}{\epsilon_0};\,\, \nabla \cdot \vec{E}_2 = \frac{\rho}{\epsilon_0}\\
 \oint_{ith conducting surface} \vec{E}_1 \cdot d \vec{a} = \frac{Q_i}{\epsilon_0};\,\, \oint_{ith conducting surface} \vec{E}_2 \cdot d \vec{a} = \frac{Q_i}{\epsilon_0}\\
 \oint_{outer boundary} \vec{E}_1 \cdot d \vec{a} = \frac{Q_{tot}}{\epsilon_0};\,\, \oint_{outer boundary} \vec{E}_2 \cdot d \vec{a} = \frac{Q_{tot}}{\epsilon_0}\\
 \vec{E}_3=\vec{E}_1-\vec{E}_2\\
 \nabla \cdot \vec{E}_3 = \nabla \cdot \vec{E}_1 - \nabla \cdot \vec{E}_2 = \frac{\rho}{\epsilon_0} - \frac{\rho}{\epsilon_0} = 0 \implies \oint \vec{E}_3 \cdot d \vec{a} = 0\\$
 each conductor is an equipotential $\implies V_3 \sim$ constant over each conducting surface (not necessarily the same constant)\\
$ \nabla \cdot (V_3 \vec{E}_3) = V_3 (\nabla \cdot \vec{E}_3) + \vec{E}_3 \cdot (\nabla V_3) = - (E_3)^2$\\
$\vec{E}_3=-\nabla V_3\\
\int_V \nabla \cdot (V_3 \vec{E}_3) d \tau = \oint_S V_3 \vec{E}_3 \cdot d \vec{a} = V_3 \oint_S  \vec{E}_3 \cdot d \vec{a} = =\int_V (E_3)^2 d \tau,$ since V is a constant on each conducting surface.\\
$\int_V (E_3)^2 d \tau = 0 \implies$ since $(E_3)^2$ cannot be negative so $E_3=0 \implies \vec{E}_3=0 \implies \vec{E}_2=\vec{E}_1.$\\

 
 \hdashrule[0.5ex][c]{\linewidth}{0.5pt}{1.5mm}


\item \underline{$V(\vec{r}) = \capk \sum_{n=0}^{\infty} \frac{1}{r^{n+1}} \int (r')^n P_n(\cos \alpha ) \rho ( \vec{r}') d \tau'$}\\
\underline{recall:} $V(\vec{r}) - \capk \int \frac{1}{\scripty{r}} \rho(\vec{r}') d \tau'\\
\scripty{r}^2 = (\vec{r} - \vec{r}')^2 = r^2 + (r')^2 - 2 r r' \cos \alpha\\
=r^2[1 + (\frac{r'}{r})^2 - 2 ( \frac{r'}{r}) \cos \alpha ]\\
\implies \scripty{r} = r \sqrt{1+ \epsilon};\,\, \epsilon \equiv (\frac{r'}{r}) ( \frac{r'}{r} - 2 \cos \alpha)\\
\underline{recall:} (1 + \epsilon)^{\alpha} = \sum_{n=0}^{\infty} \begin{pmatrix} \alpha \\ n \end{pmatrix} \epsilon^n;\,\, \begin{pmatrix} \alpha \\ n \end{pmatrix} = \frac{\alpha (\alpha - 1) ( \alpha -2) \dots (\alpha - n+1)}{n!}\\
\implies \begin{pmatrix} -\frac{1}{2} \\ 1 \end{pmatrix} = \frac{-\frac{1}{2}}{1!} = - \frac{1}{2}$ since $ -\frac{1}{2} = ( \alpha - n + 1)$ we stop at first term\\
$\begin{pmatrix} - \frac{1}{2} \\ 2 \end{pmatrix}= \frac{-\frac{1}{2}(-\frac{1}{2} - 1)}{2 !} = \frac{3}{8}\\
\begin{pmatrix} -\frac{1}{2} \\ 3 \end{pmatrix} = \frac{-\frac{1}{2}(-\frac{1}{2} -1)(-\frac{1}{2} - 2)}{3!} = -\frac{5}{16}\\
\implies \frac{1}{\scripty{r}} = \frac{1}{r} ( 1 + \epsilon)^{-1/2}=\frac{1}{r}(1-\frac{1}{2} \epsilon + \frac{3}{8} \epsilon^2 - \frac{5}{16} \epsilon^3 + \dots)$
$\implies \frac{1}{\scripty{r}} = \frac{1}{r}[1-\frac{1}{2}(\frac{r'}{r})(\frac{r'}{r}-2 \cos \alpha ) + \frac{3}{8}(\frac{r'}{r}-2 \cos \alpha )^2\\
-\frac{5}{16}(\frac{1}{r'^3}(\frac{r'}{r} - 2 \cos \alpha)^3 + \dots ]\\$
combine like orders (think about how you would do this)\\
$= \frac{1}{r} [ 1+(\frac{r'}{r})(\cos \alpha) + (\frac{r'}{r})^2(\frac{3 \cos^2 \alpha -1}{2}) + (\frac{r'}{r})^3(\frac{5 \cos^3 \alpha - 3 \cos \alpha }{2})+ \dots ]$
$\implies \frac{1}{\scripty{r}} = \frac{1}{r} \sum_{n=0}^{\infty} (\frac{r'}{r})^n P_n(\cos \alpha)\\
\therefore V(\vec{r})= \capk \int \rho(\vec{r}') [ \frac{1}{r} \sum_{n=0}^{\infty} (\frac{r'}{r})^n P_n(\cos \alpha) ] d \tau'\\
=\capk \sum_{n=0}^{\infty} \frac{1}{r^{n+1}} \int (r')^n P_n (\cos \alpha) \rho(\vec{r}') d \tau'\\$
or\\
$V(\vec{r}) = \capk [ \frac{1}{r} \int \rho (\vec{r}')  d \tau' + \frac{1}{r^2} \int r' \cos \alpha \rho (\vec{r}') d \tau'+ \frac{1}{r^3} \int (r')^2(\frac{3}{2} \cos^2 \alpha - \frac{1}{2}) \rho(\vec{r}') d \tau' + \dots ]\\$
\underline{purpose: } The purpose of this derivation is to separate the charge distribution from the evaluation point.


 \hdashrule[0.5ex][c]{\linewidth}{0.5pt}{1.5mm}
 
 
 \underline{Note:} at large $r,\,\, V_{mon}(\vec{r}) = \capk \frac{1}{r} \int \rho(\vec{r}') d \tau= \capk \frac{Q}{r}$


   
   
 \hdashrule[0.5ex][c]{\linewidth}{0.5pt}{1.5mm}
 
 
 \item \underline{$V_{dip} (\vec{r}) = \capk \frac{\vec{p} \cdot \hat{r}}{r^2};\,\, \vec{p} \equiv \int \vec{r}' \rho(\vec{r}') d \tau'$}\\
 for dipole note $\int \rho(\vec{r}') d \tau' = 0\\$
 so $V(\vec{r}) \approx \capk \frac{1}{r^2} \int r' \cos \alpha \rho(\vec{r}') d \tau'\\
 \capk \frac{1}{r^2} \int ( \hat{r} \cdot \vec{r}') \rho(\vec{r}') d \tau' = \capk \frac{1}{r^2} \hat{r} \cdot ( \int \vec{r}' \rho (\vec{r}') d \tau ')\\
 p \equiv \int \vec{r}' \rho (\vec{r}') d \tau'\\
 \therefore V_{dip} (\vec{r}) = \capk \frac{\vec{p} \cdot \hat{r}}{r^2}$
  
 
  \hdashrule[0.5ex][c]{\linewidth}{0.5pt}{1.5mm}


\underline{Note:} $\{ \vec{p} = \int \vec{r}' \rho(\vec{r}') d \tau ' \rightarrow \vec{p} = \sum_i \vec{r}_i' q_i (discrete);\,\,
\vec{p} = q \vec{r}_+'-q \vec{r}_-'=q(\vec{r}_+'-\vec{r}_-') = q \vec{d}\\$
a pure monopole has $\vec{p}=0$\\
The dipole moment of a point charge is not invariant under translation, for example, shifting the origin by $\vec{a}$ results in a dipole moment of:\\
$\vec{p} = \int \bar{\vec{r}}' \rho(\vec{r}') d \tau' = \int ( \vec{r}' - \vec{a}) \rho(\vec{r}') d \tau'\\
= \int \vec{r}' \rho (\vec{r}') d \tau' - \vec{a} \int \rho(\vec{r}') d \tau' = \vec{p} - Q \vec{a}$
 
 \hdashrule[0.5ex][c]{\linewidth}{0.5pt}{1.5mm}
 
 
 \item \underline{$\vec{E}_{dip} (r, \theta) = \frac{p}{4 \pi \epsilon_0 r^3}(2 \cos \theta \hat{r} + \sin \theta \hat{\theta})$}\\
 $\vec{E}_{dip}(r, \theta) = E_r \hat{r} + E_{\theta} \hat{\theta} + E_{\phi} \hat{\phi}\\
 \vec{E} = - \nabla V\\
  \implies 
  \begin{cases}
 	E_r = -(\nabla V)_r = - \frac{\partial V}{\partial r}\\
 	E_{\theta} = - (\nabla V)_{\theta} = - \frac{1}{r} \frac{\partial V}{\partial \theta}\\
 	E_{\phi} = - ( \nabla V) _{\phi} = - \frac{1}{r \sin \theta} \frac{\partial V}{\partial \phi}
 \end{cases}$\\
 \underline{recall:} $ V_{dip}(r, \theta) = \frac{\hat{r} \cdot \vec{p}}{4 \pi \epsilon_0 r^2} =\frac{p \cos \theta}{4 \pi \epsilon_0 r^2}\\$
 take derivs\\
$ \implies E_r = \frac{2 p \cos \theta}{4 \pi \epsilon_0 r^3};\,\, E_{\theta} = \frac{p \sin \theta}{4 \pi \epsilon_0 r^3};\,\, E_{\phi} = 0\\
 \therefore \vec{E}_{dip} (r, \theta) = \frac{p}{4 \pi \epsilon_0 r^3} ( 2 \cos \theta \hat{r} + \sin \theta \hat{\theta})$
 
 
 \hdashrule[0.5ex][c]{\linewidth}{0.5pt}{1.5mm}


\underline{Note:} \\
$ \hat{x} = \sin \theta \cos \phi \hat{r} + \cos \theta \cos \phi \hat{\theta} - \sin \phi \hat{\phi}\\
\hat{y} = \sin \theta \sin \phi \hat{r} + \cos \theta \sin \phi \hat{\theta} + \cos \phi \hat{\phi}\\
\hat{z} = \cos \theta \hat{r} - \sin \theta \hat{\theta}$


\hdashrule[0.5ex][c]{\linewidth}{0.5pt}{1.5mm}

\item \underline{$\vec{E}_{dip} (r, \theta) = \frac{1}{4 \pi \epsilon_0 r^3}(3 (\vec{p} \cdot \hat{r}) \hat{r} - \vec{p})$}\\
\underline{recall:} $ \vec{E}_{dip} (r,\theta) = \frac{p}{4 \pi \epsilon_0 r^3}(2 \cos \theta \hat{r} + \sin \theta \hat{\theta})\\
p \cos \theta = \vec{p} \cdot \hat{r},\,\, \vec{p}$ points in z direction\\
$\implies \vec{E}_{dip}(r, \theta) = \frac{1}{4 \pi \epsilon_- r^3}(2 ( \vec{p} \cdot \hat{r}) \hat{r} + p \sin \theta \hat{\theta})\\
\vec{p} = p \hat{z} = p (\cos \theta \hat{r} - \sin \theta \hat{\theta}) = ( \vec{p} \cdot \hat{r}) \hat{r} - p \sin \theta \hat{\theta}\\
\therefore \vec{E}_{dip}(r, \theta) = \frac{1}{4 \pi \epsilon_0} \frac{1}{r^3}(2 ( \vec{p} \cdot \hat{r}) \hat{r} + ( \vec{p} \cdot \hat{r}) \hat{r} - \vec{p})= \capk \frac{1}{r^3}(3 (\vec{p} \cdot \hat{r}) \hat{r} - \vec{p})$\\


\hdashrule[0.5ex][c]{\linewidth}{0.5pt}{1.5mm}


\section*{\underline{Electrodynamics: Chapter 4}}


\underline{Note:} $\vec{p}=\alpha \vec{E}$ but more generally $\vec{p}=\tilde{\alpha} \vec{E}\\$
(polarization constant/tensor for dipole moment $\vec{p}$)\\


\hdashrule[0.5ex][c]{\linewidth}{0.5pt}{1.5mm}


\item \underline{$\vec{N} = \vec{p} \times \vec{E}$} (torque of dipole in uniform electric field)\\
$\vec{N} = \sum_i \vec{N}_i = \vec{r}_+ \times \vec{F}_+ + \vec{r}_- \times \vec{F}_- = \frac{\vec{d}}{2} \times ( q \vec{E}) + (-\frac{\vec{d}}{2}) \times (-q \vec{E})\\
=(q \vec{d}) \times \vec{E} = \vec{p} \times \vec{E}\\$


\hdashrule[0.5ex][c]{\linewidth}{0.5pt}{1.5mm}


\item \underline{$\vec{F} = ( \vec{p} \cdot \nabla ) \vec{E}$} (force on dipole in nonuniform field)\\
$\vec{F} = \vec{F}_+ + \vec{F}_- = q \vec{E}_+ - q \vec{E}_- = q (\Delta \vec{E})\\
\Delta E_i = (\nabla E_i) \cdot d \vec{x} \approx \nabla E_i \cdot \vec{d} = ( \vec{d} \cdot \nabla) E_i\\
\implies \vec{F} = q (\vec{d} \cdot \nabla ) \vec{E} = (\vec{p} \cdot \nabla) \vec{E}\\$


\hdashrule[0.5ex][c]{\linewidth}{0.5pt}{1.5mm}


$\vec{P} \equiv \frac{\sum \vec{p}_i}{V} =$ dipole moment per unit volume\\


\hdashrule[0.5ex][c]{\linewidth}{0.5pt}{1.5mm}

\item \underline{$\sigma_b \equiv \vec{P} \cdot \hat{n};\,\, \rho_b \equiv - \nabla \cdot \vec{P}$}\\
this seems to be pretty general, but they start from a dipole potential which is not general, why?
\underline{recall:} $V(\vec{r}) = \capk \frac{\vec{p} \cdot \scripty{r}}{\scripty{r}^2} (single dipole)\\
\implies V(\vec{r}) = \capk \int_V \frac{\vec{P}(\vec{r}') \cdot \scripty{r}}{\scripty{r}^2} d \tau'\\$
\underline{Note:} $\nabla'(\frac{1}{\scripty{r}}) = \frac{\hat{\scripty{r}}}{\scripty{r}^2}\\
\implies V(\vec{r} ) = \capk \int_V \vec{P}(\vec{r}') \cdot \nabla'(\frac{1}{\scripty{r}}) d \tau'\\
\vec{P} \cdot \nabla '(\frac{1}{\scripty{r}}) = \nabla' \cdot (\frac{\vec{P}}{\scripty{r}}) - \frac{1}{\scripty{r}} \nabla' \cdot \vec{P}\\
\implies V(\vec{r})=\capk [ \int_V \nabla' \cdot (\frac{\vec{P}}{\scripty{r}}) d \tau' - \int_V \frac{1}{\scripty{r}} \nabla' \cdot \vec{P} d \tau']\\
=\capk [ \oint_{S} \frac{1}{\scripty{r}} \vec{P} \cdot d \vec{a}' - \int_V \frac{1}{\scripty{r}}(\nabla' \cdot \vec{P}) d \tau']\\
\therefore \sigma_b=\vec{P} \cdot \hat{n};\,\, \rho_b = - \nabla \cdot \vec{P}\\
V(\vec{r}) = \capk \oint_S \frac{\sigma_b}{\scripty{r}} d a' + \capk \int_V \frac{\rho_b}{\scripty{r}} d \tau'$\\


\hdashrule[0.5ex][c]{\linewidth}{0.5pt}{1.5mm}


\item \underline{$\nabla \cdot \vec{D} = \rho_f;\,\, \vec{D} \equiv \epsilon_0 \vec{E} + \vec{P}$}\\
\underline{recall:} $\nabla \cdot \vec{E} = \frac{\rho}{\epsilon_0}\\
\implies \epsilon_0 \nabla \cdot \vec{E} = \rho = \rho_b + \rho_f = - \nabla \cdot \vec{P} + \rho_f\\
\implies \nabla \cdot (\epsilon_0 \vec{E} + \vec{P}) = \rho_f \implies \nabla \cdot \vec{D} = \rho_f \implies \oint \vec{D} \cdot d \vec{a} = Q_{f\,enc}\\$


\hdashrule[0.5ex][c]{\linewidth}{0.5pt}{1.5mm}


\underline{note:} $\nabla \times \vec{D} = \epsilon_0 \nabla \times \vec{E} + \nabla \times \vec{P} = \nabla \times \vec{P} \neq 0\\$
Don't understand 4.2.3 and 4.2.2\\


\hdashrule[0.5ex][c]{\linewidth}{0.5pt}{1.5mm}


\underline{Boundary conditions}\\
\item \underline{$D_a^{\perp} - D_b^{\perp} = \sigma_f;\,\, \vec{D}_a^{\parallel} - \vec{D}_b^{\parallel} = \vec{P}_a^{\parallel} - \vec{P}_b^{\parallel}$}\\
\underline{recall:} $\oint \vec{D} \cdot d \vec{a} = Q_f\\
\implies \int \vec{D} \cdot \hat{n}_a da + \int \vec{D} \cdot \hat{n}_b da = \int \sigma_f da\\$
\underline{Note:} $Q=\int \sigma_f$ da and not $\oint \sigma_f da$\\
$\hat{n}_a=- \hat{n}_b = \hat{n}\\
\implies D^{\perp}_a - D^{\perp}_b=\sigma_f\\$
\underline{recall:} $ \nabla \times \vec{D} = \nabla \times \vec{P}\\
\implies \int \nabla \times \vec{D} \cdot d \vec{a} = \int \nabla \times \vec{P} \cdot d \vec{a}\\
\implies \oint \vec{D} \cdot d \vec{\ell} = \oint \vec{P} \cdot d \vec{\ell}\\
\vec{D}_a \cdot \vec{\ell} - \vec{D}_b \cdot \vec{\ell}= \vec{P}_a \cdot \vec{\ell} - \vec{P}_b \cdot \vec{\ell}\\
\implies \vec{D}_a^{\parallel} - \vec{D}_b^{\parallel} = \vec{P}_a^{\parallel} - \vec{P}_b^{\parallel};\,\,$ of course $\vec{\ell}$ is parallel to $\vec{D}_a^{\parallel}|$ and this is why $\vec{D}_a^{\parallel}$ is a vector.\\


\hdashrule[0.5ex][c]{\linewidth}{0.5pt}{1.5mm}


$\vec{P}=\epsilon_0 \chi_e \vec{E}\\$


\hdashrule[0.5ex][c]{\linewidth}{0.5pt}{1.5mm}


\item \underline{$\vec{D} = \epsilon \vec{E}$} (Linear Dielectrics)\\
$\vec{D} = \epsilon_0 \vec{E} + \vec{P} = \epsilon_0 \vec{E} + \epsilon_0 \chi_e \vec{E} = \epsilon_0 ( 1 + \chi_e) \vec{E} = \epsilon \vec{E}\\
\epsilon \equiv \epsilon_0 ( 1 + \chi_e) = \epsilon_0 \epsilon_r,\,\, \epsilon_r \equiv ( 1+ \chi_e)\\
\epsilon \sim$ permitivity,$\,\, \epsilon_r \sim$ relative permitivity\\


\hdashrule[0.5ex][c]{\linewidth}{0.5pt}{1.5mm}

\item \underline{$\vec{D} = \epsilon_0 \vec{E}_{vac};$}$\,\, \vec{D}$ in a region of homogeneous Linear Dielectric\\
$\nabla \cdot \vec{E} = \frac{\rho_f}{\epsilon_0};\,\, \nabla \times \vec{E} = 0\\
\implies \vec{E} = \vec{E}_{vac}\\$
$\vec{E}_{vac}$ is the field caused by free charge distribution in absence of dielectric\\
$\implies \nabla \cdot \vec{D} = \rho_f;\,\, \nabla \times \vec{D} = \epsilon_0 \nabla \times \vec{E} + \epsilon_0 \chi_e \nabla \times \vec{E} = 0\\
\implies \vec{D} = \vec{D}_{vac} = \epsilon_0 \vec{E}_{vac}$\\


\hdashrule[0.5ex][c]{\linewidth}{0.5pt}{1.5mm}


\item \underline{$C=\epsilon_r C_{vac}$}\\
$C=\frac{Q}{V},\,\, Q=Q_0,\,\, \epsilon_r = \frac{E_0}{E} \implies C= \frac{Q_0}{Ed} = \frac{\epsilon_r Q_0}{E_0} = \epsilon_r C_{vac}$;
$Q_0=Q$ because imagine the capacitor is taken off of the wire, the charge has nowhere to go if a dielectric is placed in the middle


\hdashrule[0.5ex][c]{\linewidth}{0.5pt}{1.5mm}


\underline{Note:} for a linear dielectric $\vec{D} = \epsilon \vec{E} = \epsilon_0 \epsilon_r \vec{E} = \epsilon_0 \vec{E}_0\\
\implies \epsilon_r = \frac{E_0}{E}$


\item \underline{$\rho_b = - ( \frac{\chi_e}{1+ \chi_e}) \rho_f$}\\
$\vec{P} = \epsilon_0 \chi_e \vec{E},\,\, \vec{D} = \epsilon \vec{E} \implies \vec{P} = \frac{\epsilon_0 \chi_e}{\epsilon} \vec{D}\\
\implies \rho_b = - \nabla \cdot \vec{P} = - \nabla \cdot (\epsilon_0 \frac{\chi_e}{\epsilon} \vec{D}) = - \epsilon_0 \frac{\chi_e}{\epsilon_0(1+ \chi_e)} \rho_f\\
\therefore \rho_b = - \frac{\chi_e}{1 + \chi_e} \rho_f\\$


\hdashrule[0.5ex][c]{\linewidth}{0.5pt}{1.5mm}


\underline{Boundary conditions for Linear Dielectrics}\\
\underline{recall:} $D^{\perp}_{above} - D_{below}^{\perp} = \sigma_f\\
\implies \epsilon_a E^{\perp}_a - \epsilon_b E^{\perp}_b = \sigma_f\\
\implies \epsilon_a \frac{\partial V_a}{\partial n} - \epsilon_b \frac{\partial V_b}{\partial n} = - \sigma_f\\
- \int_{-\epsilon}^{\epsilon} \vec{E} \cdot d \vec{\ell} = V \implies V_a = V_b\\$


\hdashrule[0.5ex][c]{\linewidth}{0.5pt}{1.5mm}


\item \underline{$W= \frac{1}{2} \int \vec{D} \cdot \vec{E} d \tau$}\\
\underline{recall:} $W= \frac{\epsilon_0}{2} \int E^2 d \tau \implies W = \frac{\epsilon}{2} \int E^2 d \tau$ guess\\
\underline{recall:} $W = \int \rho V d \tau \implies \Delta W = \int ( \Delta \rho_f) V d \tau\\$
\underline{Note:} $\Delta \rho_f d \tau$ is almost like an effective charge, when you bring in the charge, the density changes due to interactions\\
$\nabla \cdot \vec{D} = \rho_f \implies \Delta \rho_f = \nabla \cdot ( \Delta \vec{D})\\
\implies \Delta W = \int [ \nabla \cdot ( \Delta \vec{D}) ] V d \tau\\
\nabla \cdot [ ( \Delta \vec{D}) V] = [ \nabla \cdot ( \Delta \vec{D})]V + \Delta \vec{D} \cdot ( \nabla V)\\
\implies \Delta W = \int \nabla \cdot [ (\Delta \vec{D}) V] d \tau + \int ( \Delta \vec{D}) \cdot \vec{E} d \tau\\
but \int \nabla \cdot [ ( \Delta \vec{D}) V] d \tau = \oint \Delta \vec{D} V \cdot d \vec{a} \sim \frac{1}{r^2} \frac{1}{r} r^2 = \frac{1}{r} \rightarrow 0\\
\Delta W = \int (\Delta \vec{D}) \cdot \vec{E} d \tau$ ( any material)\\
assume linear dielectric $\implies \vec{D} = \epsilon \vec{E}\\
\frac{1}{2} \Delta ( \vec{D} \cdot \vec{E}) = \frac{1}{2} \Delta \vec{D} \cdot \vec{E} + \frac{1}{2} \vec{D} \cdot \Delta \vec{E} = \epsilon \vec{E} \cdot \Delta \vec{E} = \Delta \vec{D} \cdot \vec{E}\\
\implies \Delta W = \frac{1}{2} \int \Delta ( \vec{D} \cdot \vec{E}) d \tau = \frac{1}{2} \Delta ( \int \vec{D} \cdot \vec{E} d \tau\\
\therefore W= \frac{1}{2} \int \vec{D} \cdot \vec{E} d \tau\\$


\hdashrule[0.5ex][c]{\linewidth}{0.5pt}{1.5mm}


\underline{note:} $\frac{\epsilon_0}{2} \int E^2 d \tau$ bring in all charges ( freee and bound) and glue them into place.\\
$\frac{1}{2} \int \vec{D} \cdot \vec{E} d \tau$ bring free charges and allow dielectric toorient itself, since we control free charges rather than bound charges this makes more sense for Delectrics.\\


\hdashrule[0.5ex][c]{\linewidth}{0.5pt}{1.5mm}\\


\item \underline{$ C= \frac{\epsilon_0 w}{d} ( \epsilon_r \ell - x \chi_e)$}\\
\underline{recall:} $C= \frac{q}{V} = \frac{q}{Ed};\,\, \oint \vec{D} \cdot d \vec{a} = q\\
\oint \vec{D} \cdot d \vec{a} = D_w A^{cap}_w + D_{wo} A^{cap}_{wo} = q\\
A^{cap}_w$ is the area of the capacitor with dielectric\\
$A^{cap}_w = ( \ell - x) w;\,\, A^{cap}_{wo} = x w\\
\implies \oint \vec{D} \cdot d \vec{a} = \epsilon E ( \ell - x ) w + \epsilon_0 E x w = q\\
\epsilon= \epsilon_0 ( 1 + \chi_e ) + \epsilon_0 \epsilon_r\\
\implies \frac{q}{E} = \epsilon_0 w ( \epsilon_r \ell - x \chi_e)\\
\therefore C = \frac{\epsilon_0 w}{d}(\epsilon_r \ell - x \chi_e)\\$


\hdashrule[0.5ex][c]{\linewidth}{0.5pt}{1.5mm}\\


\item \underline{$ F = - \frac{\epsilon_0 w \chi_e}{2d} V^2$} ( electrical force caused by pulling dielectric out of a capacitor)\\
$dW = f_{me} dx\\
f_{me} = - F \implies F = - \frac{d W}{dx}\\$
\underline{recall:} $C = \frac{\epsilon_0 w}{d} ( \epsilon_r \ell - \chi_e x)\\
W = \frac{1}{2} \frac{Q^2}{C} \\
\implies F = - \frac{dW}{dx} = - \frac{\partial W}{\partial C} \frac{dC}{dx} = \frac{1}{2} \frac{Q^2}{C^2} \frac{d C}{dx} = \frac{1}{2} V^2 ( - \frac{\epsilon_0 w \chi_e}{d})\\
\therefore F = - \frac{\epsilon_0 w \chi_e}{2d} V^2$\\


\hdashrule[0.5ex][c]{\linewidth}{0.5pt}{1.5mm}\\



\section*{Electrodynamics}


\underline{Chapter 5}\\
$\vec{F}_{mag} = Q(\vec{v} \times \vec{B}) \implies \vec{F} = Q[\vec{E} + ( \vec{v} \times \vec{B})]$ (Lorentz force law)


\hdashrule[0.5ex][c]{\linewidth}{0.5pt}{1.5mm}


\underline{Magnetic forces do no work}\\
$d W_{mag} = \vec{F}_{mag} \cdot d \vec{\ell} = Q(\vec{v} \times \vec{B}) \cdot \vec{v} dt = 0\\$


\hdashrule[0.5ex][c]{\linewidth}{0.5pt}{1.5mm}


\underline{Note:} $\vec{I} = \lambda \vec{v} \lambda \sim$ (moving charges)\\
$\vec{F}_{mag} = \int (\vec{v} \times \vec{B}) dq = \int (\vec{v} \times \vec{B}) \lambda d \ell = \int (\vec{I} \times \vec{B}) d \ell\\
\implies \vec{F}_{mag} = \int I ( d \vec{\ell} \times \vec{B}) = I \int d \vec{\ell} \times \vec{B}$ (if $I \sim$ const along wire)


\hdashrule[0.5ex][c]{\linewidth}{0.5pt}{1.5mm}


$\vec{K} = \frac{d \vec{I}}{d \ell_{\perp}} ;\,\, \vec{K} = \sigma \vec{v};\,\, \vec{J} = \frac{d I}{d a_{\perp}};\,\, \vec{J} = \rho \vec{v}\\$


\hdashrule[0.5ex][c]{\linewidth}{0.5pt}{1.5mm}


$\vec{F}_{mag} = \int (\vec{K} \times \vec{B}) d a;\,\, \vec{F}_{mag} = \int(\vec{J} \times \vec{B}) d \tau$


\hdashrule[0.5ex][c]{\linewidth}{0.5pt}{1.5mm}


\item \underline{$\nabla \cdot \vec{J} = - \frac{\partial \rho}{\partial t}$}\\
\underline{recall:} $J \equiv \frac{d I}{da_{\perp}} \implies \int J d a_{\perp} = \int \vec{J} \cdot d \vec{a} = I\\
\implies \oint_s \vec{J} \cdot d \vec{a} = \int_V \nabla \cdot \vec{J} d \tau\\
\implies \int_V \nabla \cdot \vec{J} d \tau =. - \frac{d}{dt} \int \rho d \tau$ (negative because outward flow of current decreases charge)\\
$\implies \int_V \nabla \cdot \vec{J} d \tau = - \int \frac{\partial \rho}{\partial t} d \tau\\
\therefore \nabla \cdot \vec{J} = - \frac{\partial \rho}{\partial t}\\$


\hdashrule[0.5ex][c]{\linewidth}{0.5pt}{1.5mm}


Stationary charges $\implies$ constant electric fields; electrostatics\\
steady currents $\implies$ const magnetic fields; magnetostatics\\
or\\
$\frac{\partial \rho}{\partial t} = 0$ vs $\frac{\partial \vec{J}}{\partial t}=0\\$


\hdashrule[0.5ex][c]{\linewidth}{0.5pt}{1.5mm}


$\vec{B}(\vec{r}) = \frac{\mu_0}{4 \pi} \int \frac{\vec{I} \times \hat{\scripty{r}}}{\scripty{r}^2} d \ell' = \frac{\mu_0}{4 \pi} I \int \frac{d \vec{\ell}' \times \hat{\scripty{r}}}{\scripty{r}^2}\\$


\hdashrule[0.5ex][c]{\linewidth}{0.5pt}{1.5mm}


\item \underline{$\oint \vec{B} \cdot d \vec{\ell} = \mu_0 I_{enc}$}\\
$\oint \vec{B} \cdot  d \vec{\ell} = \oint \frac{\mu_0 I}{2 \pi s} d \ell = \frac{\mu_0 I}{2 \pi s} \oint d \ell = \mu_0 I\\$
in general\\
$\oint \vec{B} \cdot d \vec{\ell} = \mu_0 I_{enc};\,\, d \vec{\ell} = ds \hat{s} + s d \phi \hat{\phi} + ds \hat{z}\\$


\hdashrule[0.5ex][c]{\linewidth}{0.5pt}{1.5mm}


\item \underline{$\nabla \times \vec{B} = \mu_0 \vec{J}$}\\
$I_{enc} = \int \vec{J} \cdot d \vec{a}\\
\implies \oint \vec{B} \cdot d \vec{\ell} =\mu_0 \int \vec{J} \cdot d \vec{a}\\
\implies \oint \vec{B} \cdot d \vec{\ell} = \int \nabla \times \vec{B}  \cdot d \vec{a} = \mu_0 \int \vec{J} \cdot d \vec{a}\\
\therefore \nabla \times \vec{B} = \mu_0 \vec{J}$\\


\hdashrule[0.5ex][c]{\linewidth}{0.5pt}{1.5mm}


\item \underline{$\nabla \cdot \vec{B} = 0$}\\
\underline{recall:} $\vec{B} (\vec{r}) = \frac{\mu_0}{4 \pi} \int \frac{\vec{J}(\vec{r}') \times \scripty{r}}{\scripty{r}^2} d \tau'\\
\implies \nabla \cdot \vec{B}.= \frac{\mu_0}{4 \pi} \int \nabla \cdot (\vec{J}(\vec{r}') \times \frac{\hat{\scripty{r}}}{\scripty{r}^2}) d \tau'\\
\vec{\scripty{r}} = (x-x') \hat{x} + (y-y') \hat{y} + (z- z') \hat{z} = \vec{r} - \vec{r}'\\
\nabla \cdot (\vec{J} ( \vec{r}') \times \frac{\hat{\scripty{r}}}{\scripty{r}^2}) = \frac{\hat{\scripty{r}}}{\scripty{r}^2} \cdot (\nabla \times \vec{J}) - \vec{J} \cdot (\nabla \times \frac{\hat{\scripty{r}}}{\scripty{r}^2})\\
\nabla \times \vec{J}(\vec{r}')=0\\
\implies \nabla \cdot (\vec{J}(\vec{r}') \times \frac{\hat{\scripty{r}}}{\scripty{r}^2}) = - \vec{J} \cdot (\nabla \times \frac{\hat{\scripty{r}}}{\scripty{r}^2})\\$
\underline{Note:} $\nabla \times \frac{\hat{\scripty{r}}}{\scripty{r}^2} = 0\\
\therefore \nabla \cdot \vec{B} =0$\\


\hdashrule[0.5ex][c]{\linewidth}{0.5pt}{1.5mm}


\item \underline{$\nabla \times \vec{B} = \mu_0 \vec{J}$}\\
$\nabla \times \vec{B} = \frac{\mu_0}{4 \pi} \int \nabla \times (\vec{J} \times \frac{\hat{\scripty{r}}}{\scripty{r}^2}) d \tau'\\
\nabla \times (\vec{J} \times \frac{\hat{\scripty{r}}}{\scripty{r}^2}) = \vec{J}( \nabla \cdot \frac{\hat{\scripty{r}}}{\scripty{r}^2}) - ( \vec{J}  \cdot \nabla ) \frac{\hat{\scripty{r}}}{\scripty{r}^2}\\
\underline{recall:} \nabla \cdot ( \frac{\hat{\scripty{r}}}{\scripty{r}^2}) = 4 \pi \delta^3(\vec{\scripty{r}})\\
\implies \nabla \times \vec{B} = \frac{\mu_0}{4 \pi} \int \vec{J}(\vec{r}') 4 \pi \delta^3(\vec{r} - \vec{r}') d \tau' = \mu_0 \vec{J}( \vec{r})\\$
\underline{Note:} $- ( \vec{J} \cdot \nabla) \frac{\hat{\scripty{r}}}{\scripty{r}^2} = ( \vec{J} \cdot \nabla') \frac{\hat{\scripty{r}}}{\scripty{r}^2}\\$
similar to $\frac{\partial}{\partial x} f(x- x') = - \frac{\partial}{\partial x'} f(x-x')\\
\frac{\partial}{\partial x} f(x-x') = f' = - \frac{\partial}{\partial x'} f(x-x') = f'\\
{[(\vec{J} \cdot \nabla') \frac{\hat{\scripty{r}}}{\scripty{r}^2}]}_{x} = ( \vec{J} \cdot \nabla') \frac{x-x'}{\scripty{r}^3} = \nabla' \cdot [ \frac{(x-x')}{\scripty{r}^3} \vec{J}] - ( \frac{x-x'}{\scripty{r}^3})(\nabla' \cdot \vec{J})\\$
(product rule 5)\\
\underline{recall:} $\nabla \cdot \vec{J} = - \frac{\partial \rho}{\partial t} \implies \nabla \cdot \vec{J} = 0$ (steady currents)\\
$\implies [-( \vec{J} \cdot \nabla) \frac{\hat{\scripty{r}}}{\scripty{r}^2}]_x = \nabla' \cdot [ \frac{(x-x')}{\scripty{r}^3} \vec{J}]\\
\implies \nabla \times \vec{B} = \mu_0 \vec{J} ( \vec{r}) + \int \nabla ' \cdot [ \frac{(x-x')}{\scripty{r^3}} \vec{J}] d \tau'\\
= \mu_0 \vec{J}(\vec{r}) + \oint_S \frac{(x-x')}{\scripty{r}^3} \vec{J} \cdot d \vec{a} \vec{J} \rightarrow 0$ or $r \rightarrow \infty\\
\therefore \nabla \times \vec{B} = \mu_0 \vec{J}(\vec{r})$


\hdashrule[0.5ex][c]{\linewidth}{0.5pt}{1.5mm}


\item \underline{$\oint \vec{B} \cdot d \vec{\ell} = \mu_0 I_{enc}$}\\
\underline{recall:} $\nabla \times \vec{B} = \mu_0 \vec{J}\\
\implies \int (\nabla \times \vec{B}) \cdot d \vec{a} = \mu_0 \int \vec{J} \cdot d \vec{a} = \mu_0 I_{enc}\\
\therefore \oint \vec{B} \cdot d \vec{\ell} = \mu_0 I_{enc}\\$


\hdashrule[0.5ex][c]{\linewidth}{0.5pt}{1.5mm}


Just as $\nabla \times \vec{E} = 0 \implies \vec{E} = - \nabla V\\
\nabla \cdot \vec{B} = 0 \implies \vec{B} = \nabla \times \vec{A}$


\hdashrule[0.5ex][c]{\linewidth}{0.5pt}{1.5mm}


\item \underline{$\nabla^2 \vec{A} = - \mu_0 \vec{J};\,\, \vec{A} (\vec{r}) = \frac{\mu_0}{4 \pi} \int \frac{\vec{J}(\vec{r}')}{\scripty{r}} d \tau'$}\\
$\nabla \times \vec{B} = \nabla \times (\nabla \times \vec{A}) = \nabla (\nabla \cdot \vec{A}) - \nabla ^2 \vec{A} = \mu_0 \vec{J}\\$
Set $\nabla \cdot \vec{A} = 0$ (Coulomb gauge) Lets prove we can do this:\\
Suppose $\vec{A}_0$ not divergenceless\\
$\implies \vec{A} = \vec{A}_0 + \nabla \lambda \implies \nabla \cdot \vec{A} = \nabla \cdot \vec{A}_0 + \nabla^2 \lambda = 0\\
\implies \nabla^2 \lambda = - \nabla \cdot \vec{A}_0 \implies \lambda = \frac{1}{4 \pi} \int \frac{\nabla \cdot \vec{A}_0}{\scripty{r}} d \tau'\\$
i.e. if $\vec{A}_0$ is not divergenceless then we can always add $\nabla \lambda$ to make it divergenceless.\\
$\therefore \nabla^2 \vec{A} =- \mu_0 \vec{J}\\
\therefore \vec{A}(\vec{r}) = \frac{\mu_0}{4 \pi} \int \frac{\vec{J} (\vec{r}')}{\scripty{r}} d \tau'\\$


\hdashrule[0.5ex][c]{\linewidth}{0.5pt}{1.5mm}


\item \underline{$B_{above}^{\perp} = B_{below}^{\perp}$}\\
\underline{recall:} $\nabla \cdot \vec{B} = 0 \implies \oint \vec{B} \cdot d \vec{a} = 0\\
\implies \vec{B}_a \cdot \hat{n} - \vec{B}_b \cdot \hat{n} = B_a^{\perp} - B_b^{\perp} = 0\\
\therefore B_a^{\perp} = B_b^{\perp}\\$


\hdashrule[0.5ex][c]{\linewidth}{0.5pt}{1.5mm}


\item \underline{$B_a^{\parallel} - B_b^{\parallel} = \mu_0 K$}\\
\underline{recall:} $\nabla \times \vec{B} = \mu_0 \vec{J} \implies \oint \vec{B} \cdot d \vec{\ell} = \mu_0 I_{enc}\\
\implies \vec{B}_a \vec{\ell} - \vec{B}_b \cdot \vec{\ell} = (B_a^{\parallel} - B_b^{\parallel}) \ell = \mu_0 K \ell\\
\therefore B_a^{\parallel} - B_b^{\parallel} = \mu_0 K\\$


\hdashrule[0.5ex][c]{\linewidth}{0.5pt}{1.5mm}

$
\begin{cases}
B_a^{\perp} = B_b^{\perp}\\
B_a^{\parallel} - B_b^{\parallel} = \mu_0 K
\end{cases}\\
\implies \vec{B}_a - \vec{B}_b = \mu_0 (\vec{K} \times \hat{n})$\\


\hdashrule[0.5ex][c]{\linewidth}{0.5pt}{1.5mm}


\item \underline{$\vec{A}_a = \vec{A}_b$}\\
\underline{recall:} $\nabla \cdot \vec{A} = 0$ (Coulomb gauge)\\
$\implies \int \nabla \cdot \vec{A} d \tau = \oint \vec{A} \cdot d \vec{a} = 0\\
\implies \vec{A}_a \cdot \hat{n} - \vec{A}_b \cdot \hat{n} = 0\\
\implies A^{\perp}_a = A_b^{\perp}\\$
\underline{recall:} $\nabla \times \vec{A} = \vec{B}\\
\implies \in (\nabla \times \vec{A}) \cdot d \vec{a} = \oint \vec{A} \cdot d \vec{\ell} = \int \vec{B} \cdot d \vec{a}\\
\oint \vec{A} \cdot d \vec{\ell}$ is an amperian loop and z of the sides have thickness $\epsilon \rightarrow 0 so \int \vec{B} \cdot d \vec{a} \approx B \epsilon \ell \rightarrow\\
\implies \oint \vec{A} \cdot d \vec{\ell} = 0\\
\implies A^{\parallel}_a \ell - A^{\parallel}_b \ell = 0\\
\implies A^{\parallel}_a = A^{\parallel}_b\\
\therefore \vec{A}_a = \vec{A}_b\\$


\hdashrule[0.5ex][c]{\linewidth}{0.5pt}{1.5mm}


\item \underline{$\frac{\partial \vec{A}_a}{\partial n} - \frac{\partial \vec{A}_b}{\partial n} = - \mu_0 \vec{K}$}\\
Let $z$ be perpendicular to the surface and $\vec{K} = k \hat{x}\\$
\underline{recall:} $\vec{B}_{above} - \vec{B}_{below} = \mu_0 (\vec{K} \times \hat{n});\,\,, \nabla \times \vec{A} = \vec{B};\,\, \nabla \cdot \vec{A} = 0\\
\implies \nabla \times \vec{A}_a - \nabla \times \vec{A}_b = \mu_0 \vec{K} \times \hat{n} = \mu_0 K \hat{x} \times \hat{z} = \mu_0 K \hat{y}\\
\implies ( \partial_y A_{az} - \partial_z A_{ay}) \hat{x} - ( \partial_x A_{az} - \partial_z A_{ax}) \hat{y} + ( \partial_x A_{ay} - \partial_y A_{ax}) \hat{z}\\
- [ ( \partial_y A_{bz} - \partial_z A_{by}) \hat{x} - ( \partial_x A_{bz} - \partial_z A_{bx}) \hat{y} + ( \partial_x A_{by} - \partial_y A_{bx}) \hat{z}]\\
\implies ( \partial_x A_{bz} - \partial_z A_{bx} ) - ( \partial_x A_{az} - \partial_z A_{ax}) = \mu_0 K\\
\nabla \cdot \vec{A} = 0 \implies \oint \vec{A} \cdot d \vec{A} = 0 \implies \vec{A}_a \cdot \hat{n} - \vec{A}_b \cdot \hat{n}=0\\
\implies A_{az} = A_{bz}\\
\implies \partial_z A_{ax} - \partial_z A_{bx} = \mu_0 K\\
\implies \frac{\partial A_{ax}}{\partial n} - \frac{\partial A_{bx}}{\partial n} = \mu_0 K\\
\implies \frac{\partial}{\partial n} ( A_{ax}, A_{ay}, A_{az}) = \frac{\partial }{\partial n}(A_{bx}, A_{by}, A_{bz}) = (\mu_0 K, 0,0)\\
\therefore \frac{\partial \vec{A}_a}{\partial n} - \frac{\partial \vec{A}_b}{\partial n} = - \mu_0 \vec{K}$\\


\hdashrule[0.5ex][c]{\linewidth}{0.5pt}{1.5mm}


\underline{redo:}\\
$\nabla \times \vec{A}_a - \nabla \times \vec{A}_b = \epsilon_{ijk} \partial_i A_j^a - \epsilon_{ijk} \partial_i A^b_j\\
= \mu_0 (\vec{k} \times \hat{n})_k = \mu_0 \epsilon_{ijk} k_i n_j\\$
choose $\hat{n} = \hat{z}$ and $\vec{K} = K \hat{x}\\
\implies \mu_0 \epsilon_{ijk} K_i n_j = \mu_0 \epsilon_{x j k} k_x n_j = \mu_0 \epsilon_{xzk} K_x\\
\implies$ only nonzero component is $k=y\\
\implies \epsilon_{ijy} \partial_i A_j^a - \epsilon_{ijy} \partial_i A_j^b\\
= \epsilon_{ijy} \partial_i ( A_j^a - A_j^b) = \epsilon_{xzy} \partial_x (A_z^a- A_z^b) + \epsilon_{zxy} \partial_z (A_x^a - A_x^b) = \mu_0 \epsilon_{xzy} K_x\\
\oint \vec{A} \cdot d \vec{a} = 0 \implies A_z^a = A^b_z\\
\implies \partial_z ( A_x^a - A_x^b) = - \mu_0 K_x\\
\therefore \frac{\partial \vec{A}^a}{\partial n} - \frac{\partial \vec{A}^b}{\partial n} = - \mu_0 \vec{K}$


\hdashrule[0.5ex][c]{\linewidth}{0.5pt}{1.5mm}


\item \underline{$\oint T d \vec{\ell} = - \int \nabla T \times d \vec{a}$}\\
\underline{recall:} $\int (\nabla \times \vec{v}) \cdot d \vec{a} = \oint \vec{v} \cdot d \vec{\ell}\\
\vec{v} = \vec{c} T\\
\implies \nabla \times ( \vec{c} T) = T ( \nabla \times \vec{c}) - \vec{c} \times \nabla T = - \vec{c} \times \nabla T\\
\implies \int \nabla \times \vec{v} \cdot d \vec{a} = - \int \vec{c} \times \nabla T \cdot d \vec{a} = \vec{c} \cdot \oint T d \vec{\ell}\\$
\underline{recall:} $\vec{A} \cdot ( \vec{B} \times \vec{C} = \vec{B} \cdot ( \vec{C} \times \vec{A}) = \vec{C} \cdot ( \vec{A} \times \vec{B})\\
\implies d \vec{a} \cdot ( \vec{c} \times \nabla T) = \vec{c} \cdot ( \nabla T \times d \vec{a})\\
\implies - \vec{c} \cdot \int \nabla T \times d \vec{a} = \vec{c} \oint T d \vec{\ell}\\
\therefore \oint T d \vec{\ell} = - \int \nabla T \times d \vec{a}\\$


\hdashrule[0.5ex][c]{\linewidth}{0.5pt}{1.5mm}


\item \underline{ $\oint \hat{r} \cdot \vec{r}' d \vec{\ell}' = - \hat{r} \times \int d \vec{a}'$}\\
\underline{recall:} $\oint T d\vec{\ell} = - \int \nabla T \times d \vec{a}\\
\implies \oint \hat{r} \cdot \vec{r}' d \vec{\ell}' = - \int \nabla ' ( \hat{r} \cdot \vec{r}') \times d \vec{a}'\\
\nabla' ( \hat{r} \cdot \vec{r}') = \hat{r} \times ( \nabla' \times \vec{r} ') + \vec{r}' \times ( \nabla '  \times \hat{r}) + ( \hat{r} \cdot \nabla ' ) \vec{r}' + ( \vec{r}' \cdot \nabla ') \hat{r}\\
= ( \hat{r} \cdot \nabla') \vec{r}' = ( \hat{r})_i \partial_i' x_j = ( \hat{r})_i \delta_{ij} = \hat{r}_j\\
\implies \hat{r} \cdot \oint \vec{r}' d \vec{\ell}' = - \int \hat{r} \times d \vec{a}' = - \hat{r} \times \int d \vec{a}'\\
\therefore \oint \hat{r} \cdot \vec{r}' d \vec{\ell}' = - \hat{r} \times \int d \vec{a}'$


\hdashrule[0.5ex][c]{\linewidth}{0.5pt}{1.5mm}


\item \underline{$\vec{A}_{dip} (\vec{r}) = \frac{\mu_0}{4 \pi} \frac{m \times \hat{r}}{r^2};\,\, m \equiv I \int d \vec{a} = I \vec{a}$}\\
\underline{recall:} $\frac{1}{\scripty{r}} = \frac{1}{r} \sum_{n=0}^{\infty} (\frac{r'}{r})^n P_n (\cos \alpha) ;\,\, \vec{A} =\frac{\mu_0 I}{4 \pi} \oint \frac{1}{\scripty{r}} d \vec{\ell}'\\
= \frac{\mu_0 I}{4 \pi} \sum_{n=0}^{\infty} \frac{1}{r^{n+1}} \oint ( r')^n P_n (\cos \alpha) d \vec{\ell}'\\
\implies \vec{A} (\vec{r}) = \frac{\mu_0 I}{4 \pi} [ \frac{1}{r} \oint d \vec{\ell}' + \frac{1}{r^2} \oint r' \cos \alpha d \vec{\ell}'\\
+ \frac{1}{r^3} \oint (r')^2 (\frac{3}{23} \cos^2 \alpha - \frac{1}{2}) d \vec{\ell}' + \dots ]\\
d \vec{\ell}' = dx \hat{} + d y \hat{y} + d xz \hat{z}$ carry integration out\\
$\implies \oint d \vec{\ell}' = 0\\
\vec{A}_{dip} ( \vec{r}) = \frac{\mu_0 I}{4 \pi r^2} \oint r' \cos \alpha d \vec{\ell}' = \frac{\mu_0 I}{4 \pi r^2} \oint \hat{r} \cdot \vec{r} d \vec{\ell}'\\$
but $\oint \hat{r} \cdot \vec{r} d \vec{\ell}' = - \hat{r} \times \int d \vec{a}'$ \\
$\implies \frac{\mu_0 I}{4 \pi r^2} \oint (\hat{r} \cdot \vec{r}') d \vec{\ell}' = \frac{\mu_0 I}{4 \pi r^2} \int d \vec{a}' \times \hat{r}\\
\therefore \vec{A}_{dip}(\vec{r}) = \frac{\mu_0}{4 \pi} \frac{\vec{m} \times \hat{r}}{r^2}$


\hdashrule[0.5ex][c]{\linewidth}{0.5pt}{1.5mm}


\item \underline{$\vec{B}_{dip} ( \vec{r}) = \frac{\mu_0 m}{4 \pi r^3}(2 \cos \theta \hat{r} + \sin \theta \hat{\theta})$}\\
$\vec{A}_{dip}(\vec{r}) = \frac{\mu_0}{4 \pi} \frac{m \sin \theta}{r^2} \hat{\phi}\\
\implies \nabla \times \vec{A} = \frac{\mu_0 m}{4 \pi r^3}(2 \cos \theta \hat{r} + \sin \theta \hat{\theta})\\$


\hdashrule[0.5ex][c]{\linewidth}{0.5pt}{1.5mm}


\item \underline{$\vec{B}_{dip}(\vec{r}) = \frac{\mu_0}{4 \pi} \frac{1}{r^3} [3 ( \vec{m} \cdot \hat{r}) \hat{r} - \vec{m}]$}\\
\underline{recall:} $\vec{B}_{dip}(\vec{r}) = \frac{\mu_0 m}{4 \pi r^3}(2 \cos \theta \hat{r} + \sin \theta \hat{\theta})\\
\vec{B}_{dip}(\vec{r}) = \frac{\mu_0}{4 \pi r^3}(2 ( \vec{m} \cdot \hat{r}) \hat{r} + m \sin \theta \hat{\theta})\\$
\underline{recall:} $\hat{z} = \cos \theta \hat{r} - \sin \theta \hat{\theta}\\
\implies \sin \theta \hat{\theta} = \cos \theta \hat{r} - \hat{z}\\
\implies \vec{B}_{dip}(\vec{r}) = \frac{\mu_0}{4 \pi r^3}(2 ( \vec{m} \cdot \hat{r}) \hat{r} + m \cos \theta \hat{r} - m \hat{z})\\
Let \vec{m} = m \hat{z}\\
\therefore \vec{B}_{dip}(\vec{r}) =\frac{\mu_0}{4 \pi r^3} (3 (\vec{m} \cdot \hat{r}) \hat{r} - \vec{m})$


\hdashrule[0.5ex][c]{\linewidth}{0.5pt}{1.5mm}


\section*{Chapter 6}

\item \underline{$\vec{N} = \vec{m} \times \vec{B}$}\\
$\vec{N} = \frac{\vec{a}}{2} \times \vec{F} + ( - \frac{\vec{a}}{3} \times (- \vec{F})) = \vec{a} \times \vec{F}\\
\implies \vec{N} = a F \sin \theta \hat{x},\,\, | \vec{F}| = | q \vec{v}  \times \vec{B}| = I \ell B = I bB\\
\implies \vec{N} = ab I B \sin \theta \hat{x} = m B \sin \theta \hat{x} = \vec{m} \times \vec{B}\\$


\hdashrule[0.5ex][c]{\linewidth}{0.5pt}{1.5mm}


\item \underline{$\vec{F} = \nabla ( \vec{m} \cdot \vec{B})$}\\ (for a square magnetic dipole oriented on y,z plane, $\vec{B}$ is oriented in some direction)
$\vec{B}( 0 , \epsilon, z) = \vec{B}(0,0,z) + \epsilon \frac{\partial \vec{B}}{\partial y}|_{(0,0,z)}\\
\vec{B}(0,y,\epsilon) = \vec{B}(0,y,0) + \epsilon \frac{\partial \vec{B}}{\partial z} |_{(0,y,0)}\\
\vec{F}_1 = - I \int d \vec{z} \times \vec{B}(0,0,z)\\
\vec{F}_2 = I \int d \vec{y} \times \vec{B}(0,y,0)\\
\vec{F}_3 = I \int d \vec{z} \times \vec{B}(0, \epsilon, z)\\
= I \int d \vec{z} \times \vec{B}(0,0,z) + \epsilon I \int d \vec{z} \times \frac{\partial \vec{B}}{\partial y}|_{(0,0,z)}\\
\vec{F}_4 = - I \int d \vec{y} \times \vec{B}(0,y,\epsilon)\\
= - I \int d \vec{y} \times \vec{B}(0,y,0) - \epsilon I \int d \vec{y} \times \frac{\partial \vec{B}}{\partial t}|_{(0,y,0)}\\
\implies \vec{F}_{net} = \epsilon \int I d \vec{z} \times \frac{\partial \vec{B}}{\partial y} - \epsilon I \int d \vec{y} \times \frac{\partial \vec{ B}}{\partial z}\\
= \epsilon^2 I \hat{z} \times \frac{\partial \vec{B}}{\partial y} - \epsilon^2 I \hat{y} \times \frac{\partial \vec{B}}{\partial z}\\
m= \epsilon^2 I\\
\implies m ( \hat{z} \times \hat{x} \frac{\partial B_x}{\partial y} + \hat{z} \times \hat{y} \frac{\partial B_z}{\partial y}\\
- \hat{y} \times \hat{x} \frac{\partial B_x}{\partial z} - \hat{y} \times \hat{z} \frac{\partial B_z}{\partial z})\\
= m ( - ( \frac{\partial B_y}{\partial y} + \frac{\partial B_z}{\partial z}) \hat{x} + \frac{\partial B_x}{\partial} \hat{y} + \frac{\partial B_x}{\partial z} \hat{z})\\$
use $\nabla \cdot \vec{B} = 0\\
= m ( \frac{\partial B_x}{\partial x} \hat{x} + \frac{\partial B_x}{\partial y} \hat{y} + \frac{\partial B_x}{\partial z} \hat{z})\\
= m \nabla B_x but \vec{m} = \epsilon^2 I \hat{x} = m \hat{x}\\$
So $\vec{m} \cdot \vec{B} = m B_x\\
\therefore \vec{F}_{net} = \nabla ( m B_x) = \nabla ( \vec{m} \cdot \vec{B})$


\hdashrule[0.5ex][c]{\linewidth}{0.5pt}{1.5mm}


skipped 6.1.3\\


\hdashrule[0.5ex][c]{\linewidth}{0.5pt}{1.5mm}


\item \underline{$\int ( \nabla \times \vec{v}) d \tau' = - \int \vec{v} \times d \vec{a}$}\\
\underline{recall:} $\int ( \nabla \cdot \vec{E}) d \tau' = \oint \vec{E} \cdot d \vec{a}\\$
Let $\vec{E} = \vec{v} \times \vec{c}\\$
\underline{recall:} $\nabla \cdot ( \vec{v} \times \vec{c}) = \vec{c} \cdot ( \nabla \times \vec{v}) - \vec{v} \cdot \nabla \times \vec{c} = \vec{c} \cdot ( \nabla \times \vec{v})\\
\implies \vec{c} \cdot \int ( \nabla \times \vec{v}) d \tau' = \int ( \vec{v} \times \vec{c}) \cdot d \vec{a} = - \int d \vec{a} \cdot ( \vec{c} \times \vec{v})\\
= - \int \vec{c} \cdot ( \vec{v} \times d \vec{a}') = - \vec{c} \cdot \int \vec{v} \times d \vec{a}'\\
\therefore \int ( \nabla \times \vec{v}) d \tau' = - \int \vec{v} \times d \vec{a}'$


\hdashrule[0.5ex][c]{\linewidth}{0.5pt}{1.5mm}


\item \underline{$\vec{A}(\vec{r}) = \frac{\mu_0}{4 \pi} \int_V \frac{\vec{J}_b(\vec{r}')}{\scripty{r}} d \tau' + \frac{\mu_0}{4 \pi} \oint_S \frac{\vec{K}_b(\vec{r}')}{\scripty{r}} d a'\\
\vec{J}_b = \nabla \times \vec{M};\,\, \vec{K}_b = \vec{M} \times \hat{n}$}\\
\underline{recall:} $\vec{A}( \vec{r}) = \frac{\mu_0}{4 \pi} \frac{\vec{m} \times \hat{r}}{\scripty{r}^2}\\
\implies \vec{A}( \vec{r}) = \frac{\mu_0}{4 \pi} \int \frac{\vec{M} \times \hat{\scripty{r}}}{\scripty{r}^2} d \tau'\\
\nabla' \frac{1}{\scripty{r}} = \frac{\hat{\scripty{r}}}{\scripty{r}^2}\\
\implies \vec{A} ( \vec{r}) = \frac{\mu_0}{4 \pi} \int \vec{M} \times ( \nabla ' \frac{1}{\scripty{r}}) d \tau'\\
\implies \vec{A}(\vec{r}) = \frac{\mu_0}{4 \pi} \{ \int \frac{1}{\scripty{r}}[ \nabla' \times \vec{M}( \vec{r}')] d \tau - \int \nabla ' \times [ \frac{M(\vec{r}')}{\scripty{r}}] d \tau' \}\\$
\underline{recall:} $\int_V( \nabla \times \vec{v}) d\tau = - \oint \vec{v} \times d \vec{a}'\\
\implies - \int \nabla ' \times ( \frac{M}{\scripty{r}}) d \tau' = \oint \frac{\vec{M}}{\scripty{r}} \times d \vec{a}' = \oint \frac{\vec{M} \times \hat{n}}{\scripty{r}} d a'\\
\therefore \vec{A} = \frac{\mu_0}{4 \pi} \{ \int_V \frac{vec{J}_b(\vec{r}')}{\scripty{r}} d \tau' + \frac{\mu_0}{4 \pi} \oint_S \frac{\vec{K}_b(\vec{r}')}{\scripty{r}} d a'\\$


\hdashrule[0.5ex][c]{\linewidth}{0.5pt}{1.5mm}


skipped 6.2.2


\hdashrule[0.5ex][c]{\linewidth}{0.5pt}{1.5mm}


\item \underline{$\vec{H} \equiv \frac{1}{\mu_0} \vec{B} - \vec{M},\,\, \nabla \times \vec{H}= \vec{J}_f$}\\
\underline{recall:} $\nabla \times \vec{B} = \mu_0 \vec{J}\\
\implies \frac{1}{\mu_0} \nabla \times \vec{B} = \vec{J}_f + \vec{J}_b = \vec{J}_f + \nabla \times \vec{M}\\
\therefore \nabla \times ( \frac{\vec{B}}{\mu_0} - \vec{M}) = \vec{J}_f = \nabla \times \vec{H}\\$
or $\oint \vec{H} \cdot d \vec{\ell} = I_{f_{enc}}$


\hdashrule[0.5ex][c]{\linewidth}{0.5pt}{1.5mm}


\underline{Note:} $\nabla \cdot \vec{H} = - \nabla \cdot \vec{M}$





\hdashrule[0.5ex][c]{\linewidth}{0.5pt}{1.5mm}


\item \underline{$H_{above}^{\perp} - H_{below}^{\perp} = - ( M_{above}^{\perp} - M_{below}^{\perp})$}\\
\underline{recall:} $\nabla \cdot \vec{H} = - \nabla \cdot \vec{M}_a\\
\implies \oint \vec{H} \cdot d \vec{a}\\
\therefore H_{above}^{\perp} - H_{below}^{\perp} = - ( M_{above}^{\perp} - M_{below}^{\perp})\\$


\hdashrule[0.5ex][c]{\linewidth}{0.5pt}{1.5mm}


\item \underline{$\vec{H}_a^{\parallel} - \vec{H}_b^{\parallel} = \vec{K}_f \times \hat{n}$}\\
\underline{recall:} $\nabla \times \vec{H} = \vec{J}_f\\
\implies \oint \vec{H} \cdot d \vec{\ell} = \int \vec{J}_f \cdot d \vec{a}\\
\implies \vec{H}_a^{\parallel} \cdot \vec{\ell} - \vec{H}_a^{\parallel} \cdot \vec{\ell} = \int(\vec{K}_f \times \hat{n}) \cdot d \vec{\ell} = ( \vec{K}_f \times \hat{n}) \cdot \vec{\ell}\\
\implies \vec{H}_a^{\parallel} - \vec{H}_b^{\parallel} = \vec{K}_f \times \hat{n}\\$


\hdashrule[0.5ex][c]{\linewidth}{0.5pt}{1.5mm}


$\vec{M} = \chi_m \vec{H}$ (linear media)\\
$\implies \vec{B} = \mu_0 ( \vec{H} + \vec{M}) = \mu_0(1 + \chi_m) \vec{H} = \mu \vec{H}\\
\mu \equiv \mu_0 ( 1 + \chi_m)\\$


\hdashrule[0.5ex][c]{\linewidth}{0.5pt}{1.5mm}


\item \underline{$U = - \vec{m} \cdot \vec{B}$}\\
$U = - \int \vec{F} \cdot d \vec{x} = - \int \nabla ( \vec{m} \cdot \vec{B}) \cdot ( r d \phi \hat{\phi})\\
= - \int_0^{\phi} \nabla ( \vec{m} \cdot \vec{B})_{\phi} r d \phi\\
= - \int_0^{\phi} \frac{1}{r} \frac{\partial ( \vec{m} \cdot \vec{B})}{\partial \phi} r d \phi = - \vec{m} \cdot \vec{B}\\
\phi$ is angle between $\vec{m}$ and $\vec{B}\\$
should we start at $\pi/2$ instead?\\
or we could do\\
$U = \int_{\infty}^{\vec{r}} \vec{F} \cdot d \vec{\ell} = - \int_{\infty}^{\vec{r}} \nabla( \vec{m} \cdot \vec{B} ) \cdot d \vec{\ell}\\
= - \vec{m} \cdot \vec{B} (\vec{r}) + \vec{m} \cdot \vec{B}(\infty) = - \vec{m} \cdot \vec{B}$ (brings dipole in from infinity and aligns it in the magnetic field\\


\hdashrule[0.5ex][c]{\linewidth}{0.5pt}{1.5mm}

\section*{Chapter 7}

\item \underline{$\vec{J} = \sigma \vec{E}$}\\
$\vec{J} = \sigma \vec{f};\,\, \vec{f} = \vec{E} + \vec{v} \times \vec{B}\\
\vec{J} = \sigma \vec{E}$ small $v$\\
$\vec{J} = \sigma \vec{f},\,\, \vec{f} \sim$ force per unit charge\\
$\sigma = \infty \sim$ conductor,\,\, $\sigma = 0 \sim$ insulator\\
$\vec{f} = ( \vec{E} + \vec{v} \times \vec{B})$ (if electromagnetic force is pushing charges)\\
$\implies \vec{J} = \sigma \vec{E}\\$


\hdashrule[0.5ex][c]{\linewidth}{0.5pt}{1.5mm}


\item \underline{$V = \frac{\rho \ell}{A} I$}\\
$J = \frac{I}{A} = \frac{e i_e}{A} = \frac{e N v_d}{A \Delta x} = n e v_d\\
v_f = v_0 + a \Delta t,\,\, \vec{F} = m a = q E \implies a = \frac{qE}{m}\\
\implies v_f = v_d = \frac{qE \tau}{m}\\
\implies J = \frac{n e^2 \tau}{m} E = \sigma E\\
J \ell = \sigma E \ell = \sigma V\\
\implies \frac{I \ell}{A} = \sigma V\\
\implies V = \frac{\ell}{\sigma A} I = \frac{\rho \ell}{A} I\\$


\hdashrule[0.5ex][c]{\linewidth}{0.5pt}{1.5mm}


It would make sense to define $V = IR, here R = \frac{\rho \ell}{A}$


\hdashrule[0.5ex][c]{\linewidth}{0.5pt}{1.5mm}


\item \underline{$\varepsilon = \oint \vec{f} \cdot d \vec{\ell} = \oint \vec{f}_s \cdot d \vec{\ell}$}
two forces drive current\\
$\vec{f}_s \sim$ source force ( confined to a single portion)\\
$\vec{E} \sim$ communicates $\vec{f}_s$\\
$\implies \vec{f} = \vec{f}_s + \vec{E}\\
\implies \varepsilon \equiv \oint \vec{f} \cdot d \vec{\ell} = \oint \vec{f}_p \cdot d \vec{\ell}\\$


\hdashrule[0.5ex][c]{\linewidth}{0.5pt}{1.5mm}


Ideally $\vec{f} = 0 \implies \vec{f}_s = - \vec{E}\\
\implies V = - \int_a^b \vec{E} \cdot d \vec{\ell} = \int_a^b \vec{f}_s \cdot d \vec{\ell} = \oint \vec{f}_s \cdot d \vec{\ell} = 0\\$


\hdashrule[0.5ex][c]{\linewidth}{0.5pt}{1.5mm}


\item \underline{$\varepsilon = - \frac{d \Phi}{dt}$}\\
Pulling a square loop through a magnetic field pointing into page, sides with varying length x are perpendicular to the force so do not contribute
$\varepsilon = \oint \vec{f}_{mag} \cdot d \vec{\ell} = v B \int d \ell = v B h\\
\Phi = B h x \implies \frac{d \Phi}{dt} = B h \frac{dx}{dt} = - B h v = \varepsilon\\
\therefore \varepsilon = - \frac{d \Phi}{dt}\\$


\hdashrule[0.5ex][c]{\linewidth}{0.5pt}{1.5mm}


\item \underline{$\varepsilon = - \frac{d \Phi}{dt}$}\\
\underline{proof:}\\
$d \Phi = \Phi( t + dt) - \Phi (t) = \Phi_{ribbon} = \int_{ribbon} \vec{B} \cdot d \vec{a}\\
d \vec{a} = ( \vec{v} \times d \vec{\ell}) dt\\
\implies \Phi = \int_{ribbon} \vec{B} \cdot ( \vec{v} \times d \vec{\ell}) dt\\
\implies \frac{d \Phi}{dt} = \int_{ribbon} \vec{B} \cdot ( \vec{v} \times d \vec{\ell})\\
\vec{v} \sim$ velocity of wire;\,\, $\vec{u} \sim$ velocity of charges\\
$\implies \vec{w} = \vec{v} + \vec{u}$ is the resultant velocity\\
$\vec{u} \propto d \vec{\ell}\\
\implies \frac{d \Phi}{dt} = \oint \vec{B} \cdot (( \vec{w} - \vec{u}) \times d \vec{\ell}) = \int_{ribbon} \vec{B} \cdot ( \vec{w} \times d \vec{\ell})\\
\vec{B} ( \vec{w} \times d \vec{\ell}) = - ( \vec{w} \times \vec{B}) \cdot d \vec{\ell}\\
\implies \frac{d \Phi}{dt} = - \oint ( \vec{w} \times \vec{B}) \cdot d \vec{\ell}\\
\implies \frac{d\Phi}{dt} = - \oint \vec{f}_{mag} \cdot d \vec{\ell} = - \varepsilon\\$
I think the last step is justified since $\vec{E}= -\vec{w} \times \vec{B}$ since the net force on a charge is 0

\hdashrule[0.5ex][c]{\linewidth}{0.5pt}{1.5mm}


\item \underline{$\nabla \times \vec{E} = - \frac{\partial \vec{B}}{\partial t}$}\\
$\varepsilon = \oint \vec{E} \cdot d \vec{\ell} = - \frac{d \Phi}{dt} = - \int \frac{\partial \vec{B}}{\partial t} \cdot d \vec{a}\\
\implies \int \nabla \times \vec{E} \cdot d \vec{a} = - \int \frac{\partial \vec{B}}{\partial t} \cdot d \vec{a}\\
\therefore \nabla \times \vec{E} = - \frac{\partial \vec{B}}{\partial t}\\$


\hdashrule[0.5ex][c]{\linewidth}{0.5pt}{1.5mm}


Lenz's law: Nature abhors a change in flux.\\


\hdashrule[0.5ex][c]{\linewidth}{0.5pt}{1.5mm}


\item \underline{$\vec{E} = - \frac{1}{4 \pi} \frac{\partial }{\partial t} \int \frac{\vec{B} \times \hat{\scripty{r}}}{\scripty{r}^2} d \tau$}\\
$\rho = 0\\
\implies \begin{cases} \nabla \cdot \vec{E} = 0;\,\, \nabla \times \vec{E} = - \frac{\partial \vec{B}}{\partial t}\\
\nabla \cdot \vec{B} = 0;\,\, \nabla \times \vec{B} = \mu_0 \vec{J} \end{cases}\\$
analogous to\\
$\vec{B} = \frac{\mu_0}{4 \pi} \int \frac{\vec{I} \times \hat{\scripty{r}}}{\scripty{r}^2} d \ell = \frac{\mu_0}{4 \pi} \int \frac{\vec{J} \times \hat{\scripty{r}}}{\scripty{r}^2} d \tau\\
\implies \vec{E} = - \frac{1}{4 \pi} \int \frac{(\frac{\partial \vec{B}}{\partial t}) \times \hat{\scripty{r}}}{\scripty{r}^2} d \tau = - \frac{1}{4 \pi} \frac{\partial }{\partial t} \int \frac{\vec{B} \times \hat{\scripty{r}}}{\scripty{r}^2} d \tau\\$ dont understand why $\scripty{r}$ is independent of time


\hdashrule[0.5ex][c]{\linewidth}{0.5pt}{1.5mm}


\item \underline{$M_{21} = \frac{\mu_0}{4 \pi} \oint \oint \frac{d \vec{\ell}_1 \cdot d \vec{\ell}_2}{ \scripty{r}}$}\\
two loops of wire\\
loop 1 has current $I_1$\\
$\implies \vec{B}_1 = \frac{\mu_0}{4 \pi} I_1 \oint \frac{d \vec{\ell}_1 \times \hat{\scripty{r}}}{\scripty{r}^2} \implies \vec{B}_1 \propto I_1\\
\Phi_2 = \iint \vec{B}_1 \cdot d \vec{a}_2 \implies \Phi_2 \propto I_1 \implies \Phi_2 = M_{21} I_1\\
M_{21} \sim$ mutual inductance\\
$\Phi_2 = \int \vec{B}_1 \cdot d \vec{a}_2 = \int ( \nabla \times \vec{A}) \cdot d \vec{a}_2 = \oint \vec{A}_1 \cdot d \vec{\ell}_2\\
\vec{A}_1 = \frac{\mu_0 I_1}{4 \pi} \oint \frac{ d \vec{\ell}_1}{\scripty{r}}\\
\implies \Phi_2 = \frac{\mu_0 I_1}{4 \pi} \oint( \oint \frac{d \vec{\ell}_1}{\scripty{r}}) \cdot d \vec{\ell}_2\\
\therefore M_{21} = \frac{\mu_0}{4 \pi} \oint \oint \frac{d \vec{\ell}_1 \cdot d \vec{\ell}_2}{\scripty{r}}\\$


\hdashrule[0.5ex][c]{\linewidth}{0.5pt}{1.5mm}


1. $M_{21}$ is a perfectly geometric property\\
2. $M_{21} = M_{12} \equiv M$\\


\hdashrule[0.5ex][c]{\linewidth}{0.5pt}{1.5mm}


\item \underline{$\Phi = LI$}\\
if you have a current in loop 1, EMF is induced in loop $2 \implies \varepsilon_2 = - \frac{d \Phi_2}{dt} = - M \frac{d I_1}{dt}\\$
and it will also induce an EMF in itself\\
$\implies \Phi= LI L \sim$ self inductance\\

\hdashrule[0.5ex][c]{\linewidth}{0.5pt}{1.5mm}


\item \underline{$\varepsilon = - L \frac{d I}{dt}$}\\
\underline{recall:} $\varepsilon = - \frac{d \Phi}{dt} = - L \frac{d I}{dt}\\$


\hdashrule[0.5ex][c]{\linewidth}{0.5pt}{1.5mm}


\item \underline{$W= \frac{1}{2} L I^2$} (work to get a current going with inductor)\\
$\frac{d W}{dt} = - \varepsilon I = L I \frac{d I}{dt}\\
\implies W = \frac{1}{2} L I^2\\$
\underline{Note:} $\frac{d q \varepsilon}{dt} = I \varepsilon$ since  $\frac{d^2 I}{dt^2}=0$ from kirchoffs loop law.


\hdashrule[0.5ex][c]{\linewidth}{0.5pt}{1.5mm}


\item \underline{$W = - Q \varepsilon$}\\
$V = - \int_{\vec{a}}^{\vec{b}} \vec{E} \cdot d \vec{\ell};\,\, \varepsilon = \oint \vec{E} \cdot d \vec{\ell}\\
\implies W = \oint \vec{F}_{ex} \cdot d \vec{\ell};\,\, \vec{F}_{ex} = - \vec{F}_{el}\\
\implies W = - \int \vec{F}_{\ell} \cdot d \vec{\ell} = - Q \oint \vec{E} \cdot d \vec{\ell} = - Q \varepsilon\\
\implies \frac{dW}{dt} = - \varepsilon I\\$


\hdashrule[0.5ex][c]{\linewidth}{0.5pt}{1.5mm}


\item \underline{$W = \frac{1}{2 \mu_0} \int_{all-space} B^2 d \tau$}\\
$\Phi = \int \vec{B} \cdot d \vec{a} = \int ( \nabla \times \vec{A}) \cdot d \vec{a} = \oint \vec{A} \cdot d \vec{\ell}\\
\implies LI = \oint \vec{A} \cdot d \vec{\ell}\\
\implies W = \frac{1}{2} L I^2 = \frac{1}{2} I \oint \vec{A} \cdot d \vec{\ell} = \frac{1}{2} \oint ( \vec{A} \cdot \vec{I}) d \ell\\
\implies W = \frac{1}{2} \int_V \vec{A} \cdot \vec{J} d \tau\\$
\underline{recall:} $\nabla \times \vec{B} = \mu_0 \vec{J}\\
\implies W = \frac{1}{2 \mu_0} \int \vec{A} \cdot ( \nabla \times \vec{B}) d \tau\\$
but $\nabla \cdot ( \vec{A} \times \vec{B}) = \vec{B} \cdot ( \nabla \times \vec{A}) - \vec{A} \cdot ( \nabla \times \vec{B})\\
\implies \vec{a} \cdot ( \nabla \times \vec{B}) = \vec{B} \cdot \vec{B} - \nabla \cdot ( \vec{A} \times \vec{B})\\
\implies W = \frac{1}{2 \mu_0} \int B^2 d \tau - \frac{1}{2 \mu_0} \oint( \vec{A} \times \vec{B}) \cdot d \vec{a}\\
d \vec{a} \propto r^2\,\, A \propto \frac{1}{r};\,\, B \propto \frac{1}{r^2}\\$
$\rightarrow$ all-space $\implies \frac{1}{2 \mu_0} \oint \oint \vec{A} \times \vec{B} c\dot d \vec{a} = 0\\
\therefore W = \frac{1}{2 \mu_0} \int_{all-space} B^2 d \tau\\$


\hdashrule[0.5ex][c]{\linewidth}{0.5pt}{1.5mm}


\item \underline{$\nabla \times \vec{B} = \mu_0 \vec{J} + \mu_0 \epsilon_0 \frac{\partial \vec{E}}{\partial t}$}\\
\underline{recall:} $\nabla \cdot \vec{J} = - \frac{\partial \rho}{\partial t} = - \epsilon_0 \frac{\partial \nabla \cdot \vec{E}}{\partial t} = - \nabla \cdot ( \epsilon_0 \frac{\partial \vec{E}}{\partial t})\\
\implies \vec{J}_{disp} = \epsilon_0 \frac{\partial \vec{E}}{\partial t}\\
\implies \nabla \times \vec{B} = \mu_0 \vec{J} + \mu_0 \vec{J}_{disp}$


\hdashrule[0.5ex][c]{\linewidth}{0.5pt}{1.5mm}



\underline{Maxxwell's Equations}\\
$\nabla \dot \vec{E} = \frac{1}{ \epsilon_0} \rho$ (Gauss's Law)\\
$\nabla \cdot \vec{B} = 0$ ( no name)\\
$\nabla \times \vec{E} = - \frac{\partial \vec{B}}{\partial t}$ (Faraday's Law)\\
$\nabla \times \vec{B} = \mu_0 \vec{J} + \mu_0 \epsilon_0 \frac{\partial \vec{E}}{\partial t}$ (Ampere/Maxwell Law)\\


\hdashrule[0.5ex][c]{\linewidth}{0.5pt}{1.5mm}


\item \underline{$\vec{J}_p = \frac{\partial \vec{P}}{\partial t}$} (Polarization current)\\
Polarization induces a charge density of $\sigma_b = \vec{P} \cdot \hat{n} = P$ on one end and $- \sigma_b$ on the other. If $P$ increases\\
$\implies d I = \frac{\partial \sigma_b}{\partial t} d a_{\perp} = \frac{\partial P}{\partial t} d a_{\perp}\\
\implies \frac{d \vec{I}}{\partial a_{\perp}} = \vec{J} = \frac{\partial \vec{P}}{\partial t}\\$


\hdashrule[0.5ex][c]{\linewidth}{0.5pt}{1.5mm}


\underline{Check:} $\nabla \cdot \vec{J}_P = \nabla \cdot \frac{\partial \vec{P}}{\partial t} = \frac{\partial}{\partial t} \nabla \cdot \vec{P} = - \frac{\partial \rho_b}{\partial t}\\$


\hdashrule[0.5ex][c]{\linewidth}{0.5pt}{1.5mm}


\item \underline{$\begin{cases} \nabla \cdot \vec{D} = \rho_f,\,\, \nabla \times \vec{E} = - \frac{\partial \vec{B}}{\partial t} \\ \nabla \cdot \vec{B} = 0,\,\, \nabla \times \vec{H} = \vec{J}_f + \frac{\partial \vec{D}}{\partial t} \end{cases}$}\\
$\nabla \cdot \vec{E} = \frac{1}{\epsilon_0} \rho = \frac{1}{\epsilon_0} ( \rho_f - \nabla \cdot \vec{P})\\
\implies \nabla \cdot ( \epsilon_0 \vec{E} + \vec{P}) = \nabla \cdot \vec{D} = \rho_f\\
\nabla \times \vec{B} = \mu_0 \vec{J} + \mu_0 \epsilon_0 \frac{\partial \vec{E}}{\partial t}\\
\vec{J} = \vec{J}_f + \nabla \times \vec{M} + \frac{\partial \vec{P}}{\partial t}\\
\implies \nabla \times \vec{B} = \mu_0 \vec{J}_f + \mu_0 ( \frac{\partial}{\partial t}( \epsilon_0 \vec{E} + \vec{P})) + \mu_0 \nabla \times \vec{M}\\
\implies \nabla \times ( \frac{\vec{B}}{\mu_0} - \vec{M}) = \nabla \times \vec{H} = \vec{J}_f + \frac{\partial \vec{D}}{\partial t}\\$


\hdashrule[0.5ex][c]{\linewidth}{0.5pt}{1.5mm}


\underline{Linear Media}\\
$\begin{cases} \vec{P} = \epsilon_0 \chi_e \vec{E};\,\, \vec{M} = \chi_m \vec{H}\\
\vec{D} = \epsilon \vec{E};\,\l, \vec{H} = \frac{1}{ \mu} \vec{B} \end{cases}\\
\vec{J}_d \equiv \frac{\partial \vec{D}}{\partial t}$ (displaceement current)\\


\hdashrule[0.5ex][c]{\linewidth}{0.5pt}{1.5mm}


\underline{Maxwell's equations Integral form}\\
over any closed surfaces $\begin{cases} \oint_S \vec{D} \cdot d \vec{a} = Q_{f_{enc}} \\ \oint_S \vec{B} \cdot d \vec{a} = 0 \end{cases}\\$
for any surfaces bounded by the closed loop $\mathcal{P}. \begin{cases} \oint_{\mathcal{P}} \vec{E} \cdot d \vec{\ell} = - \frac{d}{dt} \int_S \vec{B} \cdot d \vec{a}\\ \oint_{\mathcal{P}} \vec{H} \cdot d \vec{\ell} = I_{f_{enc}} + \frac{d}{dt} \int_S \vec{D} \cdot d \vec{a} \end{cases}\\$


\hdashrule[0.5ex][c]{\linewidth}{0.5pt}{1.5mm}


$1 \sim$ above\\
$2 \sim$ below\\


\hdashrule[0.5ex][c]{\linewidth}{0.5pt}{1.5mm}


\item \underline{$D_1^{\perp} - D_2^{\perp} = \sigma_f$}\\
$\oint_S \vec{D} \cdot d \vec{a} = \vec{D}_1 \cdot \vec{a} - \vec{D}_2 \cdot \vec{a} = \sigma_f a\\
\implies D_1^{\perp} - D_2^{\perp} = \sigma_f\\$


\hdashrule[0.5ex][c]{\linewidth}{0.5pt}{1.5mm}


likewise $B_1^{\perp} - B_2^{\perp} = 0\\$


\hdashrule[0.5ex][c]{\linewidth}{0.5pt}{1.5mm}


\item \underline{$E_1^{\parallel} - E_2^{\parallel} = 0$}\\
$\oint_{\mathcal{P}} \vec{E} \cdot d \vec{\ell} = \vec{E}_1 \cdot \vec{\ell} - \vec{E}_2 \cdot \vec{\ell} = - \int \frac{\partial \vec{B}}{\partial t} \cdot d \vec{a} = - ( \frac{\partial \vec{B}_1}{\partial t} \cdot \vec{a} - \frac{\partial \vec{B}_2}{\partial t} \cdot \vec{a})\\
a \rightarrow 0 \implies E_1^{\parallel} - E_2^{\parallel} = 0\\$


\hdashrule[0.5ex][c]{\linewidth}{0.5pt}{1.5mm}


\item \underline{$\vec{H}_1^{\parallel} - \vec{H}_2^{\parallel} = \vec{K}_f \times \hat{n}$}\\
\underline{recall:} $\oint_{\mathcal{P}} \vec{H} \cdot d \vec{\ell} = I_{f_{enc}} + \frac{d}{dt} \int_S \vec{D} \cdot d \vec{a}\\
\implies \vec{H}_1 \cdot \vec{\ell} - \vec{H}_2 \cdot \vec{\ell} = I_{f_{enc}} = \vec{K}_f \cdot \hat{n} \times \vec{|ell}\\
= - \vec{K}_f \cdot ( \vec{\ell} \times \hat{n}) = - ( \hat{n} \times \vec{k}_f) \cdot \vec{\ell} = ( \vec{K}_f \times \hat{n}) \cdot \vec{\ell}\\
\therefore \vec{H}_1^{\parallel} - \vec{H}_2^{\parallel} = \vec{K}_f \times \hat{n}\\$


\hdashrule[0.5ex][c]{\linewidth}{0.5pt}{1.5mm}

\section*{Chapter 8}

\item \underline{$\frac{\partial \rho}{\partial t} = - \nabla \cdot \vec{J} $}(local conservation of charge)\\
$Q(t) = \int_V \rho(\vec{r},t) d \tau\\
\frac{dQ}{dt} = - \oint_S \vec{J} \cdot d \vec{a}\\
\implies \int_V \frac{\partial \rho}{\partial t} d \tau = - \int_V \nabla \cdot \vec{J} d \tau\\
\therefore \frac{\partial \rho}{\partial t} = - \nabla \cdot \vec{J}\\$


\hdashrule[0.5ex][c]{\linewidth}{0.5pt}{1.5mm}


\item \underline{$u = \frac{1}{2}( \epsilon_0 E^2 + \frac{1}{\mu_0} B^2)$}\\
\underline{recall:} $W_e = \frac{\epsilon_0}{2} \int E^2 d \tau;\,\, W_m = \frac{1}{2 \mu_0} \int B^2 d \tau\\
\implies W_{tot} = \frac{1}{2} \int ( \epsilon_0 E^2 + \frac{1}{2 \mu_0} B^2) d \tau\\
\therefore u = \frac{1}{2}( \epsilon_0 E^2 + \frac{1}{\mu_0} B^2)$


\hdashrule[0.5ex][c]{\linewidth}{0.5pt}{1.5mm}


\item \underline{$\frac{dW}{dt} = - \frac{d}{dt} \int_V \frac{1}{2} ( \epsilon_0 E^2 + \frac{1}{\mu_0} B^2) d \tau - \frac{1}{\mu_0} \oint_S ( \vec{E} \times \vec{B}) \cdot d \vec{a};\,\, S \equiv \frac{1}{\mu_0} ( \vec{E} \times \vec{B}$}\\
$d W = \vec{F} \cdot d \vec{\ell} = \vec{F} \cdot \vec{v} dt = q( \vec{E} + \vec{v} \times \vec{B}) \cdot \vec{v} dt = q \vec{E} \cdot \vec{v} dt = \rho \vec{E} \cdot \vec{v} d \tau dt\\
\implies \frac{dW}{dt} = \int_V \vec{E} \cdot ( \rho \vec{v}) d \tau = \int_V \vec{E} \cdot \vec{J} d \tau\\$
\underline{recall:} $\nabla \times \vec{B} = \mu_0 \vec{J} + \mu_0 \epsilon_0 \frac{\partial \vec{E}}{\partial t}\\
\implies \frac{1}{\mu_0} \nabla \times \vec{B} - \epsilon_0 \frac{\partial \vec{E}}{\partial t} = \vec{J}\\
\implies \vec{E} \cdot \vec{J} = \frac{1}{\mu_0} \vec{E} c\dot ( \nabla \times \vec{B}) - \epsilon_0 \vec{E} \cdot \frac{\partial \vec{E}}{\partial t}\\
\nabla \cdot ( \vec{E} \times \vec{B}) = \vec{B} \cdot ( \nabla \times \vec{E}) - \vec{E} \cdot ( \nabla \times \vec{B})\\
\implies \vec{E} \cdot 9 \nabla \times \vec{B}) = \vec{B} \cdot ( \nabla \times \vec{E}) - \nabla \cdot ( \vec{E} \times \vec{B})\\
\implies \int_V \vec{E} \cdot \vec{J} d \tau = \frac{1}{\mu_0} \int_V \vec{B} \cdot ( \nabla \times \vec{E}) d \tau - \frac{1}{\mu_0} \int_V \nabla \cdot( \vec{E} \times \vec{B}) d \tau - \frac{\epsilon_0}{2} \frac{d}{dt} \int E^2 d \tau\\$
\underline{recall:} $\nabla \times \vec{E} = - \frac{\partial \vec{B}}{\partial t}\\
\implies \int_V \vec{E} \cdot \vec{J} d \tau = - \frac{1}{ \mu_0} \int \vec{B} \cdot \frac{\partial \vec{B}}{\partial t} d \tau - \frac{\epsilon_0}{2} \frac{d}{dt} \int E^2 d \tau - \frac{1}{\mu_0} \oint_S \vec{E} \times \vec{B} \cdot d \vec{A}\\
= - \frac{d}{dt} \int_V \frac{1}{2} ( \epsilon_0 E^2 + \frac{1}{\mu_0} B^2) d \tau - \oint_S \vec{S} \cdot d \vec{a}\\$


\hdashrule[0.5ex][c]{\linewidth}{0.5pt}{1.5mm}


\item \underline{$\frac{\partial u}{\partial t} = - \nabla \cdot \vec{S}$} (continuity equation for energy)\\
Spose no work is done on charges $\implies \frac{dW}{dt} = 0\\
\implies \int \frac{\partial u}{\partial t} d \tau = - \oint \vec{S} \cdot d \vec{a} = - \int \nabla \cdot \vec{S} d \tau\\
\implies \frac{\partial u}{\partial t} = - \nabla \cdot \vec{S}$


\hdashrule[0.5ex][c]{\linewidth}{0.5pt}{1.5mm}


\item \underline{$\vec{f} = \nabla \cdot \stackrel{\leftrightarrow}{T} - \epsilon_0 \mu_0 \frac{\partial \vec{S}}{\partial t}$}\\
$\vec{F} = \int_V ( \vec{E} + \vec{v} \times \vec{B}) \rho d \tau = \int_V ( \rho \vec{E} + \vec{J} \times \vec{B}) d \tau\\
\vec{f} = \rho \vec{E} + \vec{J} \times \vec{B}\\$
\underline{recall:} $\rho = \epsilon_0 \nabla \cdot \vec{E};\,\, \nabla \times \vec{B} = j\mu_0 \vec{J} + \mu_0 \epsilon_0 \frac{\partial \vec{E}}{\partial t}\\
\vec{f} = \epsilon_0 ( \nabla \cdot \vec{E}) \vec{E} + \frac{1}{\mu_0} (\nabla \times \vec{B}) \times \vec{B} - \epsilon_0 \frac{\partial \vec{E}}{\partial t} \times \vec{B}\\
\frac{\partial \vec{E} \times \vec{B}}{\partial t} = \frac{\partial \vec{E}}{\partial t} \times \vec{B} + \vec{E} \times \frac{\partial \vec{B}}{\partial t}\\$
\underline{recall:} $\frac{\partial \vec{B}}{\partial t} = - \nabla \times \vec{E}\\
\implies \frac{\partial \vec{E}}{\partial t} \times \vec{B} = \frac{\partial ( \vec{E} \times \vec{B})}{\partial t} + \vec{E} \times \\( \nabla \times \vec{E})\\
\implies \vec{f} = \epsilon_0 ( \nabla \cdot \vec{E}) \vec{E} - \frac{1}{\mu_0} \vec{B} \times ( \nabla \times \vec{B})\\
- \epsilon_0 \frac{\partial}{\partial t} ( \vec{E} \times \vec{B}) - \epsilon_0 \vec{E} \times ( \nabla \times \vec{E})\\$
\underline{Note:} $( \nabla \cdot \vec{B}) \vec{B} = 0\\
\nabla( E^2) = 2 ( \vec{E} \cdot \nabla ) \vec{E} + 2 \vec{E} \times ( \nabla \times \vec{E})\\
\implies \vec{E} \times ( \nabla \times \vec{E}) = \frac{1}{2} \nabla ( E^2) - ( \vec{E} \cdot \nabla) \vec{E}\\
\implies \vec{f} = \epsilon_0 [ ( \nabla \cdot \vec{E}) \vec{E} + ( \vec{E} \cdot \nabla ) \vec{E} ] + \frac{1}{\mu_0} [ ( \nabla \cdot \vec{B}) \vec{B} + ( \vec{B} \cdot \nabla ) \vec{B}] - \frac{1}{2} \nabla ( \epsilon_0 E^2 + \frac{1}{\mu_0} B^2) - \epsilon_0 \frac{\partial}{\partial t} ( \vec{E} \times \vec{B})\\
\vec{f}_{\subset} = \epsilon_0  ( \nabla \cdot \vec{E}) \vec{E} + ( \vec{E} \cdot \nabla ) \vec{E} - \frac{1}{2} \nabla E^2]\\
\implies f_i = \epsilon_0 [ ( \sum_j \partial_j E_j) E_i + ( \sum_j E_j \partial_j) E_i - \frac{1}{2} \partial_i E^2]\\
= \epsilon_0 \sum_j ( ( \partial_j E_j ) E_i + E_j ( \partial_j E_i) - \frac{1}{2} \delta_{ij} \partial_j E^2)\\
=\epsilon_0 \sum_j \partial_j( E_i E_j - \frac{2}{2} \delta_{ij} E^2)\\$
do the same with $\vec{B}\\
\implies f_i = \sum_j \partial_j [ \epsilon_0 ( E_i E_j - \frac{1}{2} \delta_{ij} E^2) + \frac{1}{\mu_0 } ( B_i B_j - \frac{1}{2} \delta_{ij} B^2)]\\
= ( \nabla \cdot \stackrel{\leftrightarrow}{T} )_i\\
\therefore \vec{f} = \nabla \cdot \stackrel{\leftrightarrow}{T}  - \epsilon_0 \mu_0 \frac{\partial \vec{S}}{\partial t}\\$


\hdashrule[0.5ex][c]{\linewidth}{0.5pt}{1.5mm}


\item \underline{$\vec{P} = \mu_0 \epsilon_0 \int_V \vec{S} d \tau$}(momentum stored in fields)\\
$\vec{F} = \frac{d \vec{p}_{mech}}{dt},\,\, \vec{p}_{mech}$ (momentum of particles in V)\\
$\implies \vec{F} = \int_V \vec{f} d \tau = \int_V( \nabla \cdot \stackrel{\leftrightarrow}{T} - \epsilon_0 \mu_0 \frac{\partial \vec{S}}{\partial t}) d \tau\\
\implies \vec{F} = \int_V \nabla \cdot \stackrel{\leftrightarrow}{T} d \tau - \epsilon_0 \mu_0 \frac{d}{dt} \int_V \vec{S} d \tau\\
= \oint_S \stackrel{\leftrightarrow}{T} \cdot d \vec{a} - \epsilon_0 \mu_0 \frac{d}{dt} \int_V \vec{S} d \tau\\$
this tells us if particles gain momentum, fields lose momentum, also $-\stackrel{\leftrightarrow}{T}$ represents the flow of momentum of the fields, it is a stress tensor and I would think it acts externally hence it should be thought of as the negative in some way.
$\oint_S \stackrel{\leftrightarrow}{T} \cdot d \vec{a} \sim$( momentum flowing through surface)\\
$\vec{g} = \epsilon_0 \mu_0 \vec{S}$ (momentum density)\\
$\vec{P} = \mu_0 \epsilon_0 \int_V \vec{S} d \tau$ (momentum in fields)\\


\hdashrule[0.5ex][c]{\linewidth}{0.5pt}{1.5mm}


$\vec{\ell} = \vec{r} \times \vec{g} = \epsilon_0 [ \vec{r} \times ( \vec{E} \times \vec{B})]\\$


\hdashrule[0.5ex][c]{\linewidth}{0.5pt}{1.5mm}


\section*{Chapter 9}


\item \underline{$\frac{\partial^2 f}{\partial z^2} = \frac{1}{v^2} \frac{\partial^2 f}{\partial t^2}$}\\
$\Delta F = T \sin \theta' - T \sin \theta\\
\sin \theta \approx \tan \theta\\
\implies \Delta F = T ( \tan \theta' - \tan \theta)\\
= T ( \frac{\partial f}{\partial z}|_{z + \Delta z} - \frac{\partial f}{\partial z}|_z) \approx T \frac{\partial^2 f}{\partial z^2} \Delta z\\$
but $\Delta F = \mu \Delta z \frac{\partial^2 f}{\partial t^2}\\
\implies T \frac{\partial^2 f}{\partial z^2} = \mu \frac{\partial^2 f}{\partial t^2} \implies \frac{\partial^2 f}{\partial z^2} = \frac{\mu}{T} \frac{\partial^2 f}{\partial t^2} = \frac{1}{v^2} \frac{\partial^2 f}{\partial t^2}\\
$

\hdashrule[0.5ex][c]{\linewidth}{0.5pt}{1.5mm}


$f(z,t) = A \cos [ k ( z- v t) + \delta ]\\
f(z,t) = Re\{ A e^{i(kz- \omega t + \delta)} \} = Re \{ \tilde{A} e^{i(kz- \omega t)}\}\\
\tilde{A} = A e^{i \delta} \implies \tilde{f}(z,t) = \tilde{A} e^{i ( k z- \omega t)}\\$
any wave can be expressed $\tilde{f}(z,t) = \int_{- \infty}^{\infty} \tilde{A}(k) e^{i(kz- \omega t)} dk\\
\omega = \omega(k)$ ( dispersion relation)


\hdashrule[0.5ex][c]{\linewidth}{0.5pt}{1.5mm}


\underline{reflection/transmission}\\
$\tilde{f}_I (z,t) = \tilde{A}_I e^{i ( k_1 z - \omega t)} (z<0)\\
\tilde{f}_R (z,t) = \tilde{A}_I e^{i( k_1 z - \omega t)} (z<0)\\
k$ determined by medium, $\omega$ determined by source\\
$\tilde{f}_T(z,t) = \tilde{A}_T e^{i(k_2 z - \omega t)}\\$


\hdashrule[0.5ex][c]{\linewidth}{0.5pt}{1.5mm}


\item \underline{$A_R= ( \frac{v_2 - v_1}{vg_2 + v_1}) A_I,\,\, A_T = ( \frac{2 v_2}{v_1 + v_1}) A_I$}\\
$\tilde{f}(z,t) = \begin{cases} \tilde{A}_I e^{i(k_1 z- \omega t)} + \tilde{A}_R e^{i(-k_1 z - \omega t)} z<0\\
\tilde{A}_T e^{i(k_2 z - \omega t)} z>0 \end{cases}\\$
knot at $z=0$ (negligible mass)\\
$z>0$ different string\\
$f(0^-, t) = f( 0^+,t)\\$
must be cts at knot otherwise there would be a break\\
$\frac{\partial f}{\partial z} |_{0^-} = \frac{\partial f}{\partial z} |_{0^+}$ otherwise there would be a force on a mass of zero ( infinite acceleration)\\
\underline{Note:} $\frac{\partial f}{\partial z} |_{0^+} - \frac{\partial f}{\partial z}|_{0^-}{\Delta z} = \frac{\partial^2 f}{\partial z^2} \propto \frac{\partial^2 f}{\partial z^2}| \propto \vec{F}\\
\implies \tilde{f}(0^-, t) = \tilde{f}(0^+, t) ;\,\, \frac{\partial 
\tilde{f}}{\partial z} |_{0^-} = \frac{\partial \tilde{f}}{\partial z}|_{0^+}\\
\implies \tilde{A}_I + \tilde{A}_{R} = \tilde{A}_T;\,\, k_1 ( \tilde{A}_I - \tilde{A}_R) = k_2 \tilde{A}_T\\
\implies \tilde{A}_R = ( \frac{k_1 - k_2}|{k_1 + k_2}) \tilde{A}_I;\,\, \tilde{A}_T = \frac{2 k_1}{k_1 + k_2}\\$
\underline{recall:} $v= \frac{\lambda}{T} = \frac{2 \pi}{T} \frac{\lambda}{2 \pi} = \frac{\omega}{k} \implies k= \frac{\omega}{v}\\
\implies \tilde{A}_R = ( \frac{\frac{\omega}{v_1} - \frac{\omega}{v_2}}{\frac{\omega}{v_1} + \frac{\omega}{v_2}}) \tilde{A}_I = ( \frac{\frac{v_2 - v_1}{v_1 v_2}}{\frac{v_2 + v_1}{v_1 v_2}}) \tilde{A}_I = ( \frac{v_2 - v_1}{v_2 + v_1}) \tilde{A}_I\\
\tilde{A}_T = ( \frac{ \frac{2 \omega}{v_1}}{\frac{\omega}{v_1} + \frac{\omega}{v_2}}) = ( \frac{\frac{2 \omega}{v_1}}{\frac{v_2 + v_1}{v_1 v_2}}) = ( \frac{2 v_2}{v_1 + v_2}) \tilde{A}_I\\
\implies \tilde{A}_R = A_R e^{i \delta_R} = ( \frac{v_2 - v_1}{v_2 + v_1}) A_I e^{i \delta_I};\,\,\, A_T e^{i \delta_T} = ( \frac{2 v_2}{v_1 + v_2}) A_I e^{i \delta_I}\\$
$2^{nd}$ string lighter $\implies ( \mu_2 < \mu_1 \implies \sqrt{\frac{T}{\mu_2}} > \sqrt{\frac{T}{\mu_1}} \implies v_2> v_1)\\
\implies \delta_R = \delta_T= \delta_I$ (no phase shift)\\
$\implies A_R = ( \frac{v_2 - v_1}{v_2 + v_1}) A_I,\,\, A_T = ( \frac{2 v_2}{v_2 + v_1}) A_I\\
2^{nd}$ string heavier $\implies v_2 < v_1\\$
reflected $\pi$ shifted $\implies \delta_R + \pi = \delta_I = \delta_T\\
\implies$ reflected wave is upside down\\
$\implies A_R = ( \frac{v_1 - v_2}{v_2 + v_1|}) A_I,\,\, A_T = ( \frac{2 v_2}{v_2 + v_1}) A_I\\
2^{nd} \mu= \infty\\
\implies A_R = A_I,\,\, A_T = 0\\$


\hdashrule[0.5ex][c]{\linewidth}{0.5pt}{1.5mm}


$\tilde{f}(z,t) = \tilde{A} e^{i( kz- \omega t)} \hat{n}\\$
oscillates parallel to $\hat{n} and \hat{n} \cdot \hat{z} = 0\\
\theta \sim$ polarization angle (angle between $\hat{x}$ and $\hat{n}$)\\
$\implies \hat{n} = \cos \theta \hat{x} + \sin \theta \hat{y}\\
\implies \tilde{f}(z,t) = ( \hat{A} \cos \theta) eT^{i ( kz- \omega t)} \hat{x} + ( \tilde{A} \sin \theta) e^{i(kz- \omega t)} \hat{y}\\$


\hdashrule[0.5ex][c]{\linewidth}{0.5pt}{1.5mm}


\item \underline{$\nabla^2 \vec{E} = \mu_0 \epsilon_0 \frac{\partial^2 \vec{E}}{\partial t^2} ;\,\, \nabla^2 \vec{B} = \mu_0 \epsilon_0 \frac{\partial^2 \vec{B}}{\partial t^2}$}\\
$\nabla \times ( \nabla \times \vec{E}) = \nabla ( \nabla \cdot \vec{E}) - \nabla^2 \vec{E} = \nabla \times ( - \frac{\partial \vec{B}}{\partial t})\\
\implies \nabla^2 \vec{E} = \frac{\partial}{\partial t} \nabla \times \vec{B} = \mu_0 \epsilon_0 \frac{\partial^2 \vec{E}}{\partial t^2}\\
\nabla \times ( \nabla \times \vec{B}) = \nabla ( \nabla \cdot \vec{B}) - \nabla^2 \vec{B} = \nabla \times ( \mu_0 \epsilon_0 \frac{\partial \vec{E}}{\partial t} )\\
\implies - \nabla^2 \vec{B} = \mu_0 \epsilon_0 \frac{\partial}{\partial t} ( \nabla \times \vec{E}) = - \mu_0 \epsilon_0 \frac{\partial^2 \vec{B}}{\partial t^2}\\
\implies \nabla^2 \vec{B} = \mu_0 \epsilon_0 \frac{\partial^2 \vec{B}}{\partial t^2}\\
\implies c = \frac{1}{\sqrt{\mu_0 \epsilon_0}}\\$


\hdashrule[0.5ex][c]{\linewidth}{0.5pt}{1.5mm}









\section*{Chapter 10}
\section*{Chapter 10}
\item \underline{$V(\vec{r},t) = \frac{1}{4 \pi \epsilon_0} \frac{q c}{(\scripty{r} c- \vec{\scripty{r}} \cdot \vec{v})}$}\\
\underline{recall:} $V(\vec{r},t) = \frac{1}{4 \pi \epsilon_0} \int \frac{\rho(\vec{r}',t_r)}{\scripty{r}} d \tau';\,\, d \tau' = \frac{d \tau}{1- \hat{\scripty{r}} \cdot \vec{v}/c}\\
\rho(\vec{r}\,', t_r) = q \delta (\vec{r} - \vec{r}\,')\\
\implies \vec{V}(\vec{r},t) = \frac{1}{4 \pi \epsilon_0} \int \frac{q \delta(\vec{r} - \vec{r}')}{\scripty{r}(1- \hat{\scripty{r}}\cdot \vec{v}/c)} d \tau\\
= \frac{1}{4 \pi \epsilon_0} \frac{q}{\scripty{r} - \vec{\scripty{r}} \cdot \vec{v}/c} = \frac{1}{4 \pi \epsilon_0} \frac{qc}{\scripty{r} c - \vec{\scripty{r}} \cdot \vec{v}}\\$


\hdashrule[0.5ex][c]{\linewidth}{0.5pt}{1.5mm}


\item \underline{$\vec{A}(\vec{r},t) = \frac{\mu_0}{4 \pi} \frac{q c \vec{v}}{\scripty{r} - \vec{\scripty{r}} \cdot \vec{v}} = \frac{\vec{v}}{c^2} V(\vec{r},t)$} (don't understand)\\
\item \underline{$\vec{E}(\vec{r},t) = \frac{q}{4 \pi \epsilon} \frac{\scripty{r}}{(\vec{\scripty{r}} \cdot \vec{u})^3}[(c^2-v^2) \vec{u} + \vec{\scripty{r}} \times(\vec{u} \times \vec{a})]$}


\hdashrule[0.5ex][c]{\linewidth}{0.5pt}{1.5mm}


\item \underline{$\vec{E}(\vec{r},t) = \frac{q}{4 \pi \epsilon_0} \frac{\scripty{r}}{(\vec{\scripty{r}} \cdot \hat{u})^3}[(c^2- v^2) \vec{u} + \scripty{r} \times (\vec{u} \times \vec{a})]$} (point charge)\\
\underline{recall:} $V(\vec{r},t) = \frac{1}{4 \pi \epsilon} \frac{q c}{(\scripty{r} c - \vec{\scripty{r}} \cdot \vec{v})},\,\, \vec{A}(\vec{r},t) = \frac{\vec{v}}{c^2} V(\vec{r},t)\\
\vec{E} = - \nabla V - \frac{\partial \vec{A}}{\partial t};\,\, \vec{B} = \nabla \times \vec{A}\\
\underline{Note:} \vec{\scripty{r}} = \vec{r} - \vec{w}(t_r),\,\, \vec{v} = \dot{\vec{w}}(t_r),\,\, | \vec{r} - \vec{w}(t_r)| = c (t- t_r)\\
\nabla V = \frac{qc}{4 \pi \epsilon_0}(- \frac{1}{(\scripty{r} c - \vec{\scripty{r}} \cdot \vec{v})^2}) \nabla (\scripty{r} c - \vec{\scripty{r}} \cdot \vec{v})$ (think about $(\nabla V)_i = \partial_i V)\\$
$\scripty{r}= c(t-t_r) \implies \nabla \scripty{r} = - c \nabla t_r\\
\nabla (\vec{\scripty{r}} \cdot \vec{v}) = (\vec{\scripty{r}} \cdot \nabla ) \vec{v} + ( \vec{v} \cdot \nabla ) \vec{\scripty{r}} + \vec{\scripty{r}} \times ( \nabla \times \vec{v}) + \vec{v} \times (\nabla \times \vec{\scripty{r}})$ (product roule)\\


\item \underline{$1^{\rm{st}}$ term: $(\vec{\scripty{r}} \cdot \nabla )\vec{v}= \vec{a}(\vec{\scripty{r}} \cdot \nabla t_r)$}\\
$(\vec{\scripty{r}} \cdot \nabla )\vec{v} = ( \scripty{r}_x \frac{\partial}{\partial x} + \scripty{r}_y \frac{\partial}{\partial y} + \scripty{r}_z \frac{\partial}{\partial z} ) \vec{v}(t_r)\\
= \scripty{r}_x \frac{d \vec{v}}{dt_r} \frac{\partial t_r}{\partial x} + \scripty{r}_y \frac{d \vec{v}}{d t_r} \frac{\partial t_r}{\partial y} + \scripty{r}_z \frac{d \vec{v}}{dt_r} \frac{\partial t_r}{\partial z}\\
= \vec{a}(\vec{\scripty{r}} \cdot \nabla t_r)\\$
\\

\item \underline{$2^{nd}$ term: $\vec{v}(\vec{v} \cdot \nabla t_r) =(\vec{v} \cdot \nabla) \vec{\scripty{r}}$}\\
 $(\vec{v} \cdot \nabla) \vec{\scripty{r}} = ( \vec{v} \cdot \nabla) (\vec{r} - \vec{w}(t_r)) = (\vec{v} \cdot \nabla ) \vec{r} - ( \vec{v} \cdot \nabla ) \vec{w}\\
(\vec{v} \cdot \nabla) \vec{r} = ( v_x \frac{\partial}{\partial x} + v_y \frac{\partial}{\partial y} + v_z \frac{\partial}{\partial z})(x \hat{x} + y \hat{y} + z \hat{z})\\
= v_x \hat{x} + v_y \hat{y} + v_z \hat{z} = \vec{v}\\
(\vec{v} \cdot \nabla) \vec{w} = \sum_i v_i \frac{\partial}{\partial x_i} w_j = \sum_i v_i \frac{\partial t_r}{\partial x_i} \frac{\partial w_j}{\partial t_r}\\
= (\sum_i v_i (\nabla t_r)_i) \vec{v} = \vec{v}(\vec{v} \cdot \nabla t_r)$\\
\\

\item \underline{$3^{\rm{rd}}$ term: $\nabla \times \vec{v}= \nabla t_r \times \vec{a}$}\\
 $\nabla \times \vec{v} = ( \frac{\partial v_z}{\partial y} - \frac{\partial v_y}{\partial z}) \hat{x} + ( \frac{\partial v_x}{\partial z} - \frac{\partial v_z}{\partial x})\hat{y} +( \frac{\partial v_y}{\partial x} - \frac{\partial v_x}{\partial y}) \hat{z}\\
= (\frac{\partial t_r}{\partial y} \frac{\partial v_z}{\partial t_r} - \frac{\partial t_r}{\partial z} \frac{\partial v_y}{\partial t_r})\hat{x} + ( \frac{\partial t_r}{\partial z} \frac{\partial v_x}{\partial t_r} - \frac{\partial t_r}{\partial x} \frac{\partial v_z}{\partial t_r})\hat{y} + ( \frac{\partial t_r}{\partial z} \frac{\partial v_y}{\partial t_r} - \frac{\partial t_r}{\partial y} \frac{\partial v_x}{\partial t_r}) \hat{z}\\
= \nabla t_r \times \dot{\vec{v}} = - \vec{a} \times \nabla t_r\\$
or\\
$(\nabla \times \vec{v})_i = \epsilon_{ijk} \partial_j v_k = \epsilon_{ijk} \partial_j t_r a_k = \nabla t_r \times \vec{a}\\
\underline{recall:} \vec{A} \times (\vec{B} \times \vec{C}) = \vec{B} (\vec{A} \cdot \vec{C}) - \vec{C}(\vec{A} \cdot \vec{B})$ (BAC-CAB)\\
$\implies \vec{\scripty{r}} \times ( \nabla \times \vec{v}) = \scripty{r} \times ( \nabla t_r \times \vec{a}) = \nabla t_r ( \scripty{r} \cdot \vec{a}) - \vec{a} ( \vec{\scripty{r}} \cdot \nabla t_r)$
\\

\item \underline{$4^{\rm{th}}$ term: $\vec{v} \times ( \nabla \times \vec{\scripty{r}})= \vec{v} ( \vec{v} \cdot \nabla t_r) - \nabla t_r ( \vec{v} \cdot \vec{v})$}\\
 $\nabla \times \vec{\scripty{r}} = \nabla \times \vec{r} - \nabla \times \vec{w};\,\, \nabla \times \vec{r} = 0\\$
like third term $\nabla \times \vec{w} = - \vec{v} \times \nabla t_r\\
\underline{recall:} \vec{A} \times (\vec{B} \times \vec{C}) = \vec{B} (\vec{A} \cdot \vec{C}) - \vec{C}(\vec{A} \cdot \vec{B})$ (BAC-CAB)\\
$\implies \vec{v} \times ( \nabla \times \vec{\scripty{r}}) = \vec{v} \times ( \vec{v} \times \nabla t_r) = \vec{v} ( \vec{v} \cdot \nabla t_r) - \nabla t_r ( \vec{v} \cdot \vec{v})$\\
\\
$\nabla(\vec{\scripty{r}} \cdot \vec{v}) = (\scripty{r} \cdot \vec{v}) \vec{v} + ( \vec{v} \cdot \nabla ) \vec{\scripty{r}} + \vec{\scripty{r}} \times (\nabla \times \vec{v}) + \vec{v} \times ( \nabla \times \vec{\scripty{r}})\\
= [ \vec{a} ( \vec{\scripty{r}} \cdot \nabla t_r)] + [ \vec{v} - \vec{v}(\vec{v} \cdot \nabla t_r)] - [ \vec{\scripty{r}} \times (\vec{a} \times \nabla t_r)] + [ \vec{v} \times (\vec{v} \times \nabla t_r)]\\
\underline{recall:} \vec{A} \times (\vec{B} \times \vec{C}) = \vec{B} (\vec{A} \cdot \vec{C}) - \vec{C}(\vec{A} \cdot \vec{B})$ (BAC-CAB)\\
$\implies \nabla (\vec{\scripty{r}} \cdot \vec{v}) = \vec{a} (\vec{\scripty{r}} \cdot \nabla t_r) + \vec{v} - \vec{v}(\vec{v} \cdot \nabla t_r) - \vec{a} (\vec{\scripty{r}} \cdot \nabla t_r) + \nabla t_r (\vec{\scripty{r}} \cdot \vec{a})\\
+ \vec{v} (\vec{v} \cdot \nabla t_r) - \nabla t_r (\vec{v} \cdot \vec{v})\\$
\underline{Note:} $\vec{A} \times (\vec{B} \times \vec{C}) \neq (\vec{A} \times \vec{B}) \times \vec{C}\\
\implies \nabla (\vec{\scripty{r}} \cdot \vec{v}) = \vec{v} + ( \vec{\scripty{r}} \cdot \vec{a} - v^2) \nabla t_r$\\
$\implies \nabla V = \frac{qc}{4 \pi \epsilon_0} \frac{-1}{(\scripty{r} c - \vec{\scripty{r}} \cdot \vec{v})^2} \nabla (\scripty{r} c - \vec{\scripty{r}} \cdot \vec{v})\\
= \frac{qc}{4 \pi \epsilon_0} \frac{-1}{(\scripty{r} c - \vec{\scripty{r}} \cdot \vec{v})^2}(- c^2 \nabla t_r - \vec{v} - (\vec{\scripty{r}} \cdot \vec{a} - v^2) \nabla t_r)\\
= \frac{qc}{4 \pi \epsilon_0} \frac{1}{(\scripty{r} c - \vec{\scripty{r}} \cdot \vec{v})^2}(\vec{v} + ( c^2 - v^2 + \vec{\scripty{r}} \cdot \vec{a}) \nabla t_r)\\
- c \nabla t_r = \nabla \scripty{r} = \nabla \sqrt{\vec{\scripty{r}} \cdot \vec{\scripty{r}}} = \frac{1}{2 \sqrt{\vec{\scripty{r}} \cdot \vec{\scripty{r}}} \nabla (\vec{\scripty{r}} \cdot \vec{\scripty{r}})\\
= \frac{1}{2 \sqrt{\vec{\scripty{r}}}} \cdot \vec{\scripty{r}}}(\vec{\scripty{r}} \times ( \nabla \times \vec{\scripty{r}}) + \vec{\scripty{r}} \times (\nabla \times \vec{\scripty{r}}) + ( \vec{\scripty{r}} \cdot \nabla)\vec{\scripty{r}} + ( \vec{\scripty{r}} \times \nabla) \vec{\scripty{r}})\\
= \frac{1}{2 \scripty{r}}( 2 \vec{\scripty{r}} \times (\nabla \times \vec{\scripty{r}}) + 2 ( \vec{\scripty{r}} \cdot \nabla) \vec{\scripty{r}})\\
= \frac{1}{\scripty{r}}[( \vec{\scripty{r}} \cdot \nabla) \vec{\scripty{r}} + \vec{\scripty{r}} \times (\nabla \times \vec{\vec{r}})]\\
but ( \vec{\scripty{r}} \times \nabla) \vec{\scripty{r}} = ( \vec{\scripty{r}} \nabla ) \vec{r} - ( \vec{\scripty{r}} \cdot \nabla) \vec{w}\\$
\underline{ Aside:} $\{ ( \vec{\scripty{r}} \cdot \nabla ) \vec{w} = \scripty{r}^i \partial_i w^j = \scripty{r}^i \frac{\partial t_r}{\partial x_i} \frac{\partial w^j}{\partial t_r} = ( \vec{\scripty{r}} \cdot \nabla t_r) \vec{w}\\$
and \\
$( \vec{\scripty{r}} \cdot \nabla) \vec{r} = \scripty{r}^i \partial_i x^j = \scripty{r}^i \delta_i^j = \scripty{r}^j = \vec{\scripty{r}} \} \\
\implies ( \vec{\scripty{r}} \cdot \nabla ) \vec{\scripty{r}} = \vec{\scripty{r}} - ( \vec{\scripty{r}} \cdot \nabla t_r) \vec{w}\\$
\underline{recall:} $\nabla \times \vec{\scripty{r}} = ( \vec{v} \times \nabla t_r)\\
\implies \vec{\scripty{r}} \times ( \nabla \times \vec{\scripty{r}}) = \vec{\scripty{r}} \times (\vec{v} \times \nabla t_r)
\implies - c \Delta t_r = \frac{1}{\scripty{r}} [(\vec{\scripty{r}} \cdot \nabla) \vec{\scripty{r}} + \vec{\scripty{r}} \times ( \vec{v} \times \vec{\scripty{r}})] = \frac{1}{\scripty{r}}[ \vec{\scripty{r}} - \vec{v} (\vec{\scripty{r}} \cdot \nabla t_r) + \vec{\scripty{r}} \times (\vec{v} \times \nabla t_r)]\\$
but\\
$\vec{\scripty{r}} \times (\vec{v} \times \nabla t_r) = \vec{v} (\vec{\scripty{r}} \cdot \nabla t_r) - \nabla t_r(\vec{\scripty{r}} \cdot \vec{v})\\
\implies -c \nabla t_r = \frac{1}{\scripty{r}} [ \vec{\scripty{r}} - \vec{v} (\vec{\scripty{r}} \cdot \nabla t_r) + \vec{v}(\vec{\scripty{r}} \cdot \nabla t_r) - \nabla t_r (\vec{\scripty{r}} \cdot \vec{v})]\\
= \frac{1}{\scripty{r}}[ \vec{\scripty{r}} - \nabla t_r(\vec{\scripty{r}} \cdot \vec{v})]\\
\implies \nabla t_r = - \frac{\vec{\scripty{r}}}{\scripty{r} c - \vec{\scripty{r}} \cdot \vec{v}}\\
\implies \nabla V = \frac{1}{4 \pi \epsilon_0} \frac{qc}{(\scripty{r} c - \vec{\scripty{r}} \cdot \vec{v})^3} [( \scripty{r} c - \vec{\scripty{r}} \cdot \vec{v}) \vec{v} - ( c^2 - v^2 + \vec{\scripty{r}} \cdot \vec{a}) \vec{\scripty{r}}]\\$
Similarly,\\
$\frac{\partial \vec{A}}{\partial t} = \frac{1}{4 \pi \epsilon_0} \frac{qc}{(\scripty{r} c - \vec{\scripty{r}} \cdot \vec{v})^3}[(\scripty{r} c - \vec{\scripty{r}} \cdot \vec{v})(-\vec{v} + \scripty{r} \vec{a}/c)]\\
\vec{u} = c \hat{\scripty{r}} - \vec{v}\\
\vec{E}(\vec{r},t) = \frac{q}{4 \pi \epsilon_0} \frac{\scripty{r}}{(\vec{\scripty{r}} \cdot \vec{u})^3}[(c^2 - v^2) \vec{u} + \vec{\scripty{r}} \times (\vec{u} \times \vec{a})]\\$
ended on 10.72


\hdashrule[0.5ex][c]{\linewidth}{0.5pt}{1.5mm}

\item \underline{$\vec{E} = - \nabla V - \frac{\partial \vec{A}}{\partial t}$}\\
\underline{recall:} $\vec{B} = \nabla \times \vec{A};\,\, \nabla \times \vec{E} = - \frac{\partial \vec{B}}{\partial t}\\
\implies \nabla \times \vec{E}= - \frac{\partial}{\partial t} ( \nabla \times \vec{A})\\
\implies \nabla \times ( \vec{E} + \frac{\partial \vec{A}}{\partial t}) =0 \implies \vec{E} + \frac{\partial \vec{A}}{\partial t} = - \nabla V\\
\therefore \vec{E} = - \nabla V - \frac{\partial \vec{A}}{\partial t}$


\hdashrule[0.5ex][c]{\linewidth}{0.5pt}{1.5mm}\\


\item \underline{$\nabla^2 V + \frac{\partial}{\partial t} ( \nabla \cdot \vec{A}) = - \frac{1}{\epsilon_0} \rho;\,\, (\nabla^2 \vec{A} - \mu_0 \epsilon_0 \frac{\partial^2 \vec{A}}{\partial t^2}) - \nabla ( \nabla \cdot \vec{A} + \mu_0 \epsilon_0 \frac{\partial V}{\partial t}) = - \mu_0 \vec{J}$}\\
\underline{recall:} $\vec{E} = - \nabla V - \frac{\partial \vec{A}}{\partial t};\,\, \nabla \cdot \vec{E} = \frac{1}{\epsilon_0} \rho\\
\implies \nabla \cdot \vec{E} = - \nabla^2 V - \frac{\partial}{\partial t}(\nabla \cdot \vec{A}) =\frac{1}{\epsilon_0} \rho\\
\therefore \nabla^2 V + \frac{\partial }{\partial t} ( \nabla \cdot \vec{A}) = - \frac{1}{\epsilon_0} \rho\\$
\underline{recall:} $\vec{B} = \nabla \times \vec{A};\,\, \vec{E} = - \nabla V - \frac{\partial \vec{A}}{\partial t};\,\, \nabla \times \vec{B} = \mu_0 \vec{J} + \mu_0 \epsilon_0 \frac{\partial \vec{E}}{\partial t}\\
\implies \nabla \times \vec{B} = \nabla \times (\nabla \times \vec{A}) = \mu_0 \vec{J} - \mu_0 \epsilon_0 \nabla ( \frac{\partial V}{\partial t}) - \mu \epsilon_0 \frac{\partial^2 \vec{A}}{\partial t^2}\\
\nabla\times ( \nabla \times \vec{A}) = \nabla ( \nabla \cdot \vec{A}) - \nabla^2 \vec{A}\\
\implies \nabla ( \nabla \cdot \vec{A}) - \nabla^2 \vec{A} = \mu_0 \vec{J} -\mu_0 \epsilon_0 \nabla( \frac{\partial V}{\partial t}) - \epsilon_0 \mu_0 \frac{\partial^2 \vec{A}}{\partial t^2}\\
\therefore ( \nabla^2 \vec{A} - \mu_0 \epsilon_0 \frac{\partial^2 \vec{A}}{\partial t^2}) - \nabla ( \nabla \cdot \vec{A} + \mu_0 \epsilon_0 \frac{\partial V}{\partial t}) = - \mu_0 \vec{J}$\\


\hdashrule[0.5ex][c]{\linewidth}{0.5pt}{1.5mm}\\


\item \underline{$\vec{A}' = \vec{A} + \nabla \lambda;\,\, V' = V - \frac{\partial \lambda}{\partial t}$} (gauge transformation)\\
$\vec{A}' = \vec{A} + \vec{\alpha};\,\, V' = V + \beta\\
\vec{A}',V'$ have the same fields as $\vec{A}, V\\
\nabla \times \vec{A} = \vec{B} \implies \nabla \times \vec{A}' = \nabla \times \vec{A} + \nabla \times \vec{\alpha} = \vec{B} \implies \nabla \times \vec{\alpha} = 0\\
\implies \vec{\alpha} = \nabla \lambda\\
\vec{E} = - \nabla V - \frac{\partial \vec{A}}{\partial t} = - \nabla V' - \frac{\partial \vec{A}'}{\partial t}\\
= - \nabla V - \nabla \beta - \frac{\partial \vec{A}}{\partial t} - \frac{\partial \vec{\alpha}}{\partial t} \implies \nabla \beta + \frac{\partial \vec{\alpha}}{\partial t} = 0\\
\implies \nabla \beta + \nabla \frac{\partial \lambda}{\partial t} = \nabla ( \beta + \frac{\partial \lambda}{\partial t})=0\\
\implies \beta + \frac{\partial \lambda}{\partial t} = k(t) \implies \beta = - \frac{\partial \lambda}{\partial t} + k(t) = - \frac{\partial}{\partial t} \lambda + \frac{\partial}{\partial t} \int_0^t k(t') d t'\\
= - \frac{\partial}{\partial t}( \lambda + \int_0 ^t k(t') d t') = - \frac{\partial}{\partial t} \lambda',\,\, relabel \lambda' \rightarrow \lambda\\
\therefore
\begin{cases}\\
	\vec{A}' = \vec{A} + \nabla \lambda\\
	V' = V - \frac{\partial \lambda}{\partial t}
\end{cases}\\$


\hdashrule[0.5ex][c]{\linewidth}{0.5pt}{1.5mm}\\


\item \underline{$V(\vec{r},t) = \capk \int \frac{\rho(\vec{r}',t)}{\scripty{r}} d \tau'$} (Coulomb gauge)\\
\underline{recall:} $\nabla^2 V + \frac{\partial}{\partial t} ( \nabla \cdot \vec{A}) = - \frac{1}{\epsilon_0} \rho\\
\nabla \cdot \vec{A} = 0$ (Coulomb Gauge)\\
$\implies \nabla^2 = - \frac{1}{\epsilon_0} \rho\\
\therefore  V(\vec{r},t) = \frac{1}{4 \pi \epsilon_0} \int \frac{\rho(\vec{r}',t)}{\scripty{r}} d \tau'\\$


\hdashrule[0.5ex][c]{\linewidth}{0.5pt}{1.5mm}\\

\item \underline{$\Box^2 V = - \frac{1}{\epsilon_0} \rho;\,\, \Box^2 \vec{A} = - \mu_0 \vec{J}$}\\
\underline{recall:} $\nabla^2 V + \frac{\partial}{\partial t} ( \nabla \cdot \vec{A})= - \frac{1}{\epsilon_0} \rho;\,\, ( \nabla^2 \vec{A} - \mu_0 \epsilon_0 \frac{\partial^2 \vec{A}}{\partial t^2}) - \nabla ( \nabla \cdot \vec{A} + \mu \epsilon_0 \frac{\partial V}{\partial t}) = - \mu_0 \vec{J}\\$
Choose $\nabla \cdot \vec{A} + \mu_0 \epsilon_0 \frac{\partial V}{\partial t} = 0$ ( Lorentz gauge)\\
$\implies\\
\begin{cases}
	\nabla^2 \vec{A} - \mu_0 \epsilon_0 \frac{\partial^2 \vec{A}}{\partial t^2} = - \mu_0 \vec{J}\\
	\nabla^2 V - \mu \epsilon_0 \frac{\partial^2 B}{\partial t^2} = - \frac{1}{\epsilon_0} \rho
\end{cases}\\
\\
\nabla^2 -\mu \epsilon_0 \frac{\partial^2}{\partial t^2} \equiv \Box^2\\
\\
\therefore
\begin{cases}
	\Box^2 \vec{A} = - \mu_0 \vec{J}\\
	\Box^2 V = - \frac{1}{\epsilon_0} \rho
\end{cases}$\\


\hdashrule[0.5ex][c]{\linewidth}{0.5pt}{1.5mm}


\item \underline{$\vec{p}_{can} = \vec{p} + q \vec{A};\,\, U_{vel} = q (V- \vec{v} \cdot \vec{A});\,\, \frac{d \vec{p}_{can}}{dt} = - \nabla U_{vel}$}\\
\underline{recall:} $\vec{E} = - \nabla V - \frac{\partial \vec{A}}{\partial t},\,\, \vec{B} = \nabla \times \vec{A};\\
\vec{F} = q \vec{E} + q \vec{v} \times \vec{B}\\
\implies \vec{F} = q (\vec{E} + \vec{v} \times \vec{B}) = q (- \nabla V - \frac{\partial \vec{A}}{\partial t} + \vec{v} \times (\nabla \times \vec{A})) = \frac{d \vec{p}}{dt};\,\, \vec{p} = m \vec{v}\\
\vec{v} \times (\nabla \times \vec{A}) = \nabla(\vec{v} \cdot \vec{A}) - ( \vec{v} \cdot \nabla ) \vec{A}$ (product rule)\\
$\implies \frac{d \vec{p}}{dt} = q (- \nabla V - \frac{\partial \vec{A}}{\partial t} + \nabla (\vec{v} \cdot \vec{A}) - ( \vec{v} \cdot \nabla) \vec{A})\\
= - q[\frac{\partial \vec{A}}{\partial t} + ( \vec{v} \cdot \nabla) \vec{A} + \nabla(V- \vec{v} \cdot \vec{A})]\\
\underline{Note:} \frac{d \vec{A}}{dt} = \frac{dx}{dt} \frac{\partial \vec{A}}{\partial x} + \frac{dy}{dt} \frac{\partial \vec{A}}{\partial y} + \frac{dz}{dt} \frac{\partial \vec{A}}{\partial z} + \frac{\partial \vec{A}}{\partial t}\\
= ( \vec{v} \cdot \nabla) \vec{A} + \frac{\partial \vec{A}}{\partial t}$ (convective derivative)\\
$\implies \frac{d \vec{p}}{dt} = - q [ \frac{d \vec{A}}{dt} + \nabla (V - \vec{v} \cdot \vec{A})]\\
\implies \frac{d}{dt}(\vec{p} + q \vec{A}) = - \nabla [q(V - \vec{v} \cdot \vec{A})]\\
\therefore \vec{p}_{can} = \vec{p} + q \vec{A}\\
\therefore U_{vel} = q ( V - \vec{v} \cdot \vec{A})\\
\implies \frac{d \vec{p}_{can}}{dt} = - \nabla U_{vel}$\\


\hdashrule[0.5ex][c]{\linewidth}{0.5pt}{1.5mm}


Similarly $\frac{d}{dt} (T + q v ) = \frac{\partial}{\partial t} [q(V - \vec{v} \cdot \vec{A})]$


\hdashrule[0.5ex][c]{\linewidth}{0.5pt}{1.5mm}


\item \underline{$V(\vec{r}) = \capk \int \frac{\rho(\vec{r}')}{\scripty{r}} d \tau';\,\, \vec{A}(\vec{r}) = \frac{\mu_0}{4 \pi} \int \frac{\vec{J}(\vec{r}')}{\scripty{r}} d \tau'$}\\
\underline{recall:} $\Box^2 V = - \frac{1}{\epsilon_0} \rho;\,\, \Box^2 \vec{A} = - \mu_0 \vec{J}\\$
static $\implies \nabla^2 V = - \frac{1}{\epsilon_0} \rho;\,\, \nabla^2 \vec{A} = - \mu_0 \vec{J}\\
\therefore V(\vec{r}) = \capk \int \frac{\rho(\vec{r}')}{\scripty{r}} d \tau';\,\, \vec{A} (\vec{r}) = \frac{\mu_0}{4 \pi} \int \frac{\vec{J}(\vec{r}')}{\scripty{r}} d \tau'$\\


\hdashrule[0.5ex][c]{\linewidth}{0.5pt}{1.5mm}


We want $B(\vec{r},t)$, $\vec{A}(\vec{r},t)$ (non-static) \\
EM waves travel at the speed of light $B(\vec{r},t)$ gives us the potential at $\vec{r}$ "now" which is not what the source is doing right now there is a delay of $t- \frac{\scripty{r}}{c} = t_r$\\
$\implies V(\vec{r},t) = \capk \int \frac{\rho(\vec{r},t_r)}{\scripty{r}} d \tau';\,\, \vec{A} = \frac{\mu_0}{4 \pi} \int \frac{\rho(\vec{r},t_r)}{\scripty{r}} d \tau'\\$
(retarded potentials) this argument does not work for $\vec{E} and \vec{B}$\\


\hdashrule[0.5ex][c]{\linewidth}{0.5pt}{1.5mm}


\item \underline{$\Box^2 V = - \frac{1}{\epsilon_0} \rho;\,\, \Box^2 \vec{A} = - \mu_0 \vec{J}$} using $V,\,\, \vec{A}$ as above\\
\underline{proof:}\\
\underline{recall:} $V(\vec{r},t) = \frac{1}{4 \pi \epsilon} \int \frac{\rho(\vec{r}',t_r)}{\scripty{r}} d \tau';\,\, \vec{A}(\vec{r},t) = \frac{\mu_0}{4 \pi} \int \frac{\vec{J}(\vec{r}',t_r)}{\scripty{r}} d \tau'\\
\nabla V = \frac{1}{4 \pi \epsilon_0} \int [ (\nabla \rho) \frac{1}{\scripty{r}} + \rho \nabla (\frac{1}{\scripty{r}})] d \tau'\\
\nabla \rho = \sum_i \frac{\partial \rho}{\partial x_i} \hat{x}_i = \sum_i \frac{\partial t_r}{\partial x_i} \frac{\partial \rho}{\partial t_r} \hat{x}_i = \sum_i \frac{\partial t_r}{\partial x_i} \frac{\partial t}{\partial t_r} \frac{\partial \rho}{\partial t} \hat{x}_i\\
\frac{\partial t}{\partial t_r} = ( \frac{\partial t_r}{\partial t})^{-1} = ( \frac{\partial}{\partial t}(t- \frac{\scripty{r}}{c}))^{-1} = 1\\
\implies \nabla \rho = \dot{\rho} \sum_i \frac{\partial t_r}{\partial x_i} \hat{x}_i= \dot{\rho} \nabla t_r = \dot{\rho} \nabla (t- \frac{\scripty{r}}{c}) = - \frac{\dot{\rho}}{c} \nabla \scripty{r}\\
\nabla \scripty{r} = \hat{\scripty{r}};\,\, \nabla ( \frac{1}{\scripty{r}}) = - \frac{\hat{\scripty{r}}}{\scripty{r}^2}\\
\implies \nabla V = \frac{1}{4 \pi \epsilon_0} \int [ - \frac{\dot{\rho}}{c} \frac{\hat{\scripty{r}}}{\scripty{r}} - \rho \frac{\hat{\scripty{r}}}{\scripty{r}^2}] d \tau'\\
\nabla \cdot \nabla V = \nabla^2 V = \frac{1}{4 \pi \epsilon_0} \int [ - \frac{1}{c} \nabla \cdot ( \dot{\rho} \frac{\hat{\scripty{r}}}{\scripty{r}}) - \nabla \cdot ( \rho \frac{\hat{\scripty{r}}}{\scripty{r}^2})] d \tau' \\$
\underline{recall:} $\nabla \cdot (f \vec{A}) = f \nabla \cdot \vec{A} + \vec{A} \cdot ( \nabla f)\\
\implies \nabla^2 V = \frac{1}{4 \pi \epsilon_0} \int [ - \frac{1}{c}( \dot{\rho} \nabla \cdot ( \frac{\scripty{r}}{\scripty{r}}) + \nabla \dot{\rho} \cdot ( \frac{\scripty{r}}{\scripty{r}})) - ( \rho \nabla \cdot \frac{\scripty{r}}{\scripty{r}^2} + \nabla \rho \cdot \frac{\scripty{r}}{\scripty{r}^2})] d \tau'\\
\capk \int [ - \frac{1}{c} \dot{\rho} \nabla \cdot ( \frac{\scripty{r}}{\scripty{r}}) - \frac{1}{c} \nabla \dot{\rho} \cdot ( \frac{\scripty{r}}{\scripty{r}})- \rho \cdot \nabla \cdot ( \frac{\scripty{r}}{\scripty{r}^2}) - \nabla \rho \cdot ( \frac{\scripty{r}}{\scripty{r}^2})] d\tau'\\
\nabla \cdot ( \frac{\hat{\scripty{r}}}{\scripty{r}}) = \frac{1}{\scripty{r}^2};\,\, \nabla \dot{\rho} = \sum_i \frac{\partial \dot{\rho}}{\partial x_i} \hat{x}_i = \sum_i \frac{\partial t_r}{\partial x_i} \frac{\partial t}{\partial t_r} \frac{\partial \dot{\rho}}{\partial t} \hat{x}_i = \ddot{\rho} \nabla t_r = - \frac{1}{c} \ddot{\rho} \nabla \scripty{r}\\
=- \frac{1}{c} \ddot{\rho} \hat{\scripty{r}};\,\, \nabla \cdot ( \frac{\hat{r}}{\scripty{r}^2}) = 4 \pi \delta^3( \scripty{r})\\
\nabla^2 V = \capk \int [ - \frac{1}{c} \dot{\rho} \frac{1}{\scripty{r}^2} + \frac{1}{c^2} \ddot{\rho} \frac{1}{\scripty{r}} - 4 \pi \rho \delta^3 ( \vec{\scripty{r}}) + \frac{1}{c} \dot{\rho} \frac{1}{\scripty{r}^2}] d \tau\\
= \capk \int [ \frac{1}{c^2} \ddot{\rho} \frac{1}{\scripty{r}} - 4 \pi \rho \delta^3 ( \vec{\scripty{r}})] d \tau'\\
= \frac{1}{c^2} \frac{\partial^2}{\partial t^2} ( \capk \int \frac{\rho}{\scripty{r}} d \tau') - \frac{1}{\epsilon_0} \rho( \vec{r}) = \frac{1}{c^2} \frac{\partial^2 V}{\partial t^2} - \frac{1}{\epsilon_0} \rho(\vec{r}') $\\


\hdashrule[0.5ex][c]{\linewidth}{0.5pt}{1.5mm}

$\star$
This logic applies to advanced potentials\\
$V_a (\vec{r},t) = \capk \int \frac{\rho(\vec{r}',t_a)}{\scripty{r}} d \tau',\,\, \vec{A}_a (\vec{r},t) = \frac{\mu_0}{4 \pi} \int \frac{\vec{j}(\vec{r}',t_a)}{\scripty{r}} d \tau'\\
t_a \equiv t + \frac{\scripty{r}}{c}\\$


\hdashrule[0.5ex][c]{\linewidth}{0.5pt}{1.5mm}


\item \underline{$\vec{E}(\vec{r},t) = \capk \int [ \frac{\rho(\vec{r}',t_r}{\scripty{r}^2} \hat{\scripty{r}} + \frac{\dot{\rho}(\vec{r},t_r)}{c \scripty{r}} \hat{\scripty{r}} - \frac{\dot{\vec{J}}(\vec{r}',t_r)}{c^2 \scripty{r}}] d \tau'$}\\ \item \underline{$\vec{B}(\vec{r},t) = \frac{\mu_0}{4 \pi} \int [ \frac{\vec{J}(\vec{r}',t_r)}{\scripty{r}^2} + \frac{\dot{J}(\vec{r}',t_r)}{c \scripty{r}}] \times \hat{\scripty{r}} d \tau'$}\\
\underline{recall:} $\vec{E} = - \nabla V - \frac{\partial \vec{A}}{\partial t};\,\, \nabla V = \capk \int [ - \frac{\dot{\rho}}{c} \frac{\hat{\scripty{r}}}{\scripty{r}} - \rho \frac{\hat{\scripty{r}}}{\scripty{r}^2}] d \tau'\\
\vec{A} = \frac{\mu_0}{4 \pi} \int \frac{\vec{J}(\vec{r}',t_r)}{\scripty{r}} d \tau'\\
\implies \frac{\partial \vec{A}}{\partial t} = \frac{\mu_0}{4 \pi} \int \frac{\dot{\vec{J}}}{\scripty{r}} d \tau'\\
\therefore \vec{E}(\vec{r},t) = \capk \int [ \frac{\rho(\vec{r}',t_r)}{\scripty{r}^2} \hat{\scripty{r}} + \frac{\dot{\rho}(\vec{r}',t_r)}{c \scripty{r}} \hat{\scripty{r}} - \frac{\dot{\vec{J}}(\vec{r}',t_r)}{c^2 \scripty{r}}] d \tau'\\
\vec{B} = \nabla \times \vec{A}\\
\nabla \times \vec{A} = \frac{\mu_0}{4 \pi} \int \nabla \times ( \frac{\vec{J}}{\scripty{r}}) d \tau'\\
\nabla \times ( \frac{\vec{J}}{\scripty{r}}) = \frac{1}{\scripty{r}} \nabla \times \vec{J} - \vec{J} \times ( \nabla \frac{1}{\scripty{r}})\\
\implies \nabla \times \vec{A} = \frac{\mu_0}{4 \pi} \int [ \frac{1}{\scripty{r}} ( \nabla \times \vec{J}) - \vec{J} \times(\nabla \frac{1}{\scripty{r}})]\\
( \nabla \times \vec{J})_x = \frac{\partial J_z}{\partial y} - \frac{\partial J_y}{\partial z}\\
\frac{\partial J_z}{\partial y} = \frac{\partial t_r}{\partial y} \frac{\partial t}{\partial t_r} \frac{\partial J_z}{\partial t} = \dot{J}_z \frac{\partial t_r}{\partial y} = - \frac{1}{c} \frac{\partial \scripty{r}}{\partial y} \dot{J}_z\\
\frac{\partial J_y}{\partial z} = - \frac{1}{c} \frac{\partial \scripty{r}}{\partial z} \dot{J}_y\\
\implies (\nabla \times \vec{J})_x = - \frac{1}{c}[ \dot{J}_z \frac{\partial \scripty{r}}{\partial y} - \dot{J}_y \frac{\partial \scripty{r}}{\partial z}] = \frac{1}{c} [ \dot{\vec{J}} \times ( \nabla \scripty{r})]_x\\
\implies \nabla \times \vec{J} = \frac{1}{c}(\dot{\vec{J}} \times (\nabla \scripty{r})) = \frac{1}{c} \dot{\vec{J}} \times \hat{\scripty{r}}\\
\nabla (\frac{1}{\scripty{r}}) = - \frac{\hat{\scripty{r}}}{\scripty{r}^2}\\
\therefore \vec{B}(\vec{r},t) = \frac{\mu_0}{4 \pi} \int [ \frac{\vec{J}(\vec{r}',t_r)}{\scripty{r}^2} + \frac{\dot{vec{J}}(\vec{r}',t_r)}{c \scripty{r}}] \times \hat{\scripty{r}} d \tau'$\\


\hdashrule[0.5ex][c]{\linewidth}{0.5pt}{1.5mm}


\item \underline{$L' = \frac{L}{1- v \cos \theta/c}$}\\
$L' = L + \Delta L,\,\, \Delta L$ is the extra distance light must travel to reach the front of the train (since as soon as the photon leaves the caboose the train travels distance $\Delta L$ by the time it reaches the front)\\
in time $\Delta t$, the light from caboose travels from back to front of the train, that is,$\,\, \Delta t = \frac{L'}{c}\\$
in this time the train has traveled $\Delta t = \frac{\Delta L}{v} = \frac{L' - L}{v}\\$
$\implies \frac{L'}{c} = \frac{L'-L}{v} \implies L' = \frac{L}{1-v/c}\\$
If you are at some angle from the train then $\frac{L' \cos \theta}{c} = \frac{L'-L}{v} \implies L' = \frac{L}{1-v \cos \theta/c}\\$
$\therefore L' = \frac{L}{1- \hat{\scripty{r}} \cdot \vec{v}/c}\\$


\hdashrule[0.5ex][c]{\linewidth}{0.5pt}{1.5mm}


$\implies \tau' = \frac{\tau}{1- \hat{\scripty{r}} \cdot \vec{v}/c}\\$


\hdashrule[0.5ex][c]{\linewidth}{0.5pt}{1.5mm}


\underline{Note on retarded time:} light we see from stars left at the retarded time, this delay is $\frac{\scripty{r}}{c},\,\, \rho (\vec{r},t_r)$ is the density f the source that we see right now, which was its actual density at $t- \frac{\scripty{r}}{c}$\\


\hdashrule[0.5ex][c]{\linewidth}{0.5pt}{1.5mm}


\underline{$V( \vec{r},t) = \frac{1}{4 \pi \epsilon_0} \{ \frac{q_0 \cos[\omega( t- \scripty{r}_+/c)]}{\scripty{r}_+} - \frac{q_0 \cos [ \omega( t - \scripty{r}_-/c)]}{\scripty{r}_-}\}$}\\
$q(t)$ at $\frac{\vec{d}}{2} -q(t)$ at $- \frac{\vec{d}}{2},\,\, q(t) = q_0 \cos ( \omega t)\\
\implies \vec{p}(t) = p_0 \cos \omega t \hat{z};\,\, p_0 = q_0 d\\
\implies \rho(\vec{r}, t) = q_0 \cos ( \omega t) \delta^{(3)} ( \vec{r} - \frac{\vec{d}}{2}) - q_0 \cos ( \omega t) \delta^{(3)} ( \vec{r} + \frac{\vec{d}}{2})\\$
\underline{recall:} $V( \vec{r}, t) = \frac{1}{4 \pi \epsilon_0} \int \frac{\rho(\vec{r}', t_r)}{\scripty{r}} d \tau'\\
= \frac{1}{4 \pi \epsilon_0} \{ \int \frac{q_0 \cos ( \omega t_r) \delta^{(3)}( \vec{r}' - \frac{\vec{d}}{2})}{\scripty{r}} d \tau'- \int \frac{q_0 \cos( \omega t_r) \delta^{(3)} ( \vec{r}' + \frac{\vec{d}}{2})}{\scripty{r}} d \tau'\\
= \frac{1}{4 \pi \epsilon} \{ \frac{q_0 \cos[ \omega( t- \scripty{r}_+/c)]}{\scripty{r}_+} - \frac{q_0 \cos[ \omega( t- \scripty{r}_-/c)]}{\scripty{r}_-} \}\\$


\hdashrule[0.5ex][c]{\linewidth}{0.5pt}{1.5mm}


\underline{$\scripty{r}_{\pm} = \sqrt{ r^2 \mp r d \cos \theta + ( d/2)^2}$}\\
$\vec{\scripty{r}}_+ = \vec{r} - \frac{\vec{d}}{2}\\
\implies \scripty{r}_+^2 = ( \vec{r} - \frac{\vec{d}}{2}) \cdot ( \vec{r} - \frac{\vec{d}}{2}) = r^2 - \vec{r} \cdot \vec{d} + \frac{d^2}{4}\\
= r^2 - r d \cos \theta + ( d/2)^2\\
\vec{\scripty{r}}_- = \vec{r} + \frac{\vec{d}}{2}\\
\implies \scripty{r}_-^2 = r^2 + r d \cos \theta + ( \frac{d}{2})^2\\
\therefore \scripty{r}_{\pm} = \sqrt{ r^2 \mp r d \cos \theta + ( \frac{d}{2})^2}$


\hdashrule[0.5ex][c]{\linewidth}{0.5pt}{1.5mm}


\underline{$\cos[ \omega ( t- \scripty{r}_{\pm}/c)] \approx \cos [ \omega( t- r/c)] \cos ( \frac{\omega d}{2c} \cos \theta) \mp \sin[ \omega( t- r/c)] \sin ( \frac{\omega d}{2 c} \cos \theta)$} $(d<< r)$\\
\underline{recall:} $\scripty{r}_{\pm} = \sqrt{ r^2 \mp r d \cos \theta + ( \frac{d}{2})^2};\,\, \sqrt{1 + x } \approx 1 + \frac{x}{2}\\
= r \sqrt{1 \mp \frac{d}{r} \cos \theta + ( \frac{d}{2 r})^2}\\
\approx r ( 1 \mp \frac{d}{2 r} \cos \theta + \frac{1}{2} ( \frac{d}{2 r})^2) \approx r ( 1. mp \frac{d}{2r} \cos t\theta)\\
\implies \frac{1}{ \scripty{r}_{|pm}} = \frac{1}{ r( 1 \mp \frac{d}{2r} \cos \theta)} = \frac{1}{r} \frac{1}{1 \mp \frac{d }{2r} \cos \theta} \approx \frac{1}{r} ( 1 \pm \frac{d}{2r} \cos \theta)\\
\implies \cos [ \omega( t- \scripty{r}_{|pm}/c)] \approx \cos [ \omega( t- \frac{r}{c} ( 1 \mp \frac{d}{2r} \cos \theta))]\\
= \cos [ \omega ( r- \frac{r}{c}) \pm \frac{\omega d}{2 c} \cos \theta]\\$
\underline{recall:} $\cos ( \alpha \pm \beta) = \cos \alpha \cos \beta \mp \sin \alpha \sin \beta\\
\implies \cos [ \omega( t- \scripty{r}_{\pm}/c)]\ \approx \cos [ \omega ( t- \frac{r}{c})] \cos ( \frac{\omega d}{2 c} \cos \theta) \mp \sin [ \omega( t- \frac{r}{c})] \sin (\frac{\omega d}{2 c} \cos \theta)$



   
       
\section*{\underline{Quantum Mechanics}}
\section*{Chapter 2: QM}


\item \underline{$- \frac{\hbar^2}{2m} \frac{d^2 \psi}{dx^2} + V \psi = E \psi;\,\, \frac{d \varphi}{dt} = - \frac{i E}{\hbar} \varphi$}\\
\underline{recall:} $i \hbar \frac{\partial \Psi}{\partial t} = - \frac{\hbar^2}{2m} \frac{\partial^2 \Psi}{\partial x^2} + V \Psi\\
\Psi(x,t) = \psi(x) \varphi(t)\\
\implies \frac{\partial \Psi}{\partial t} = \psi \frac{d \varphi}{d t},\,\, \frac{\partial^2 \Psi}{\partial x^2} = \frac{d^2 \psi}{dx^2} \varphi\\
\implies i \hbar^2 \psi \frac{d \varphi}{dt} = - \frac{\hbar^2}{2m} \frac{d^2 \psi}{dx^2} \varphi + V \psi \varphi\\
\implies i \hbar \frac{1}{\varphi} \frac{d \varphi}{dt} = - \frac{\hbar^2}{2m} \frac{1}{\psi} \frac{d^2 \psi}{dx^2} + V = E\\
\therefore \begin{cases} - \frac{\hbar^2}{2m} \frac{d^2 \psi}{dx^2} + V \psi = E \psi\\ i \hbar \frac{d \phi}{dt} = E \varphi \end{cases}$


\hdashrule[0.5ex][c]{\linewidth}{0.5pt}{1.5mm}


$\varphi(t) = e^{-iEt/\hbar}$


\hdashrule[0.5ex][c]{\linewidth}{0.5pt}{1.5mm}


properties of $\psi(x,t) w/ V(x,t) = V(x)$\\
1. stationary: $\Psi(x,t) = \psi(x) e^{-iEt/\hbar}\\$
but $|\Psi|^2 = \Psi^* \Psi = | \psi(x)|^2$ (ind. of time)\\
2. they have definite energy\\
time independent Schrodinger equation\\
$\hat{H} \psi = ( - \frac{\hbar^2}{2m} \frac{\partial^2}{\partial x^2} + V(x)) \psi = E \psi\\
\langle H \rangle = \int \psi^* \hat{H} \psi dx = E \int | \psi|^2 dx = E \int | \Psi|^2 dx = E\\
\implies \langle H^2 \rangle = \int \psi^* \hat{H}^2 \psi dx = E^2\\
\implies \sigma_H^2 = \langle H^2 \rangle - \langle H \rangle^2 = E^2 - E^2 = 0\\$
3. General solution: $\Psi(x,t) = \sum_{n=1}^{\infty} c_n \psi_n (x) e^{- i E_n t/\hbar}\\$
properties 1,2 work for separable but not general solutions.\\


\hdashrule[0.5ex][c]{\linewidth}{0.5pt}{1.5mm}


\item \underline{ $\sum_{n=1}^{\infty} |c_n|^2 = 1;\,\, \langle H \rangle = \sum_{n=1}^{\infty} |c_n|^2 E_n$}\\
$1= \int | \Psi(x,0)|^2 dx = \int ( \sum_m c_m^* \psi_m^*(x))(\sum_n s_n \psi_n(x)) dx (easy to generalize to t=t)\\
= \sum_{m,n} c_m^* c_n \int \psi_m^* \psi_n dx = \sum_{m,n} c_m^* c_n \delta_{mn} = \sum_{n=1}^{\infty} |c_n|^2\\
\langle H \rangle = \int \Psi^* \hat{H} \Psi dx = \int (\sum_m c_m^* \psi_m^* \varphi_m^*) \hat{H} ( \sum_n c_n \psi_n \varphi) dx\\
= \sum_{m,n} c_m^* c_n e^{-i(E_n - E_m) t/\hbar} E_n \int \psi_m^* \psi_n dx\\
= \sum_{m,n} c_m^* c_n e^{-i(E_n - E_m)t/\hbar} E_n \delta_{mn}\\
= \sum_n |c_n|^2 E_n\\$
Probability of getting energy $E_n = |c_n|^2 = \langle \psi_n | \Psi \rangle$


\hdashrule[0.5ex][c]{\linewidth}{0.5pt}{1.5mm}


\item \underline{$\psi_n(x) = \sqrt{\frac{2}{a}} \sin( \frac{n \pi}{a} x);\,\, E_n = \frac{\hbar^2 k_n^2}{2m} = \frac{n^2 \pi^2 \hbar^2}{2 m a^2}$}\\
(infinite square well)\\
$V(x) = \begin{cases} 0,\,\, 0 \leq x \leq a\\
\infty,\,\, o.w. 
\end{cases}\\
- \frac{\hbar^2}{2m} \frac{d^2 \psi}{dx^2} = E \psi\\
\frac{d^2 \psi}{dx^2} = - k^2 \psi,\,\, k \equiv \frac{\sqrt{2 m E}}{\hbar}\\
\psi(x) = A \sin k x + B \cos kx\\
\psi(0) = \psi(a) = 0 \implies B = 0\\
\implies \psi(x) = A \sin kx\\
\psi(a) = A \sin ka = 0 \implies k_n = \frac{n \pi}{a},\,\, n \in \mathbb{N}\\
\implies E_n = \frac{\hbar^2 k_n^2}{2m} = \frac{n^2 \pi^2 \hbar^2}{a^2}\\
\int_0^{\infty} |\psi_n|^2 dx = 1 \implies |A|^2 = \frac{2}{a}\\
\therefore \psi_n(x) = \sqrt{ \frac{2}{a}} \sin( \frac{n \pi}{a} x)\\$


\hdashrule[0.5ex][c]{\linewidth}{0.5pt}{1.5mm}


\underline{properties}\\
1. alternate even/odd: $\psi_1$ even $\psi_2$ odd $\dots\\$
2. $\psi_1$ has no nodes ( except end pts, dont count),$\,\, \psi_2 \sim 1$ node,$\,\, \psi_2 \sim 2$ nodes\\
3. $\int \psi_m^* (x) \psi_n(x) dx = \delta_{mn}\\$
4. complete: any function $f(x) = \sum_{n=1}^{\infty} c_n \psi_n(x)\\$


\hdashrule[0.5ex][c]{\linewidth}{0.5pt}{1.5mm}


\item \underline{$\int_{- \infty}^{\infty} f^* ( \hat{a}_{\pm} g) dx = \int_{-\infty}^{\infty} ( \hat{a}_{\mp} f)^* g dx$}\\
\underline{proof:}\\
$\int_{- \infty}^{\infty} f^*( \hat{a}_{\pm} g) dx = \frac{1}{\sqrt{2 \hbar m \omega}} \int_{- \infty}^{\infty} f^*( \mp \hbar \frac{d}{dx} + m \omega x) g dx\\$
int. by parts $\int f^* ( \frac{d g}{dx}) dx = - \int ( \frac{df}{dx})^* g dx\\
\implies \int_{- \infty}^{\infty} f^* ( \hat{a}_{\pm} g) dx = \frac{1}{\sqrt{ 2 \hbar m \omega}} \int [ ( \pm \hbar \frac{d}{dx} + m \omega x) f]^* g dx\\
= \int_{- \infty}^{\infty} ( \hat{a}_{\pm} f)^* g dx\\$


\hdashrule[0.5ex][c]{\linewidth}{0.5pt}{1.5mm}


\item \underline{$\hat{a}_+ \psi_n = \sqrt{n+1} \psi_{n+1};\,\, \hat{a}_- \psi_n = \sqrt{n} \psi_{n-1}$}\\
\underline{recall:} $\psi_n ( x) = A_n ( a_+)^n \psi_0(x)\\
\psi_{n+1} = A_{n+1} ( \hat{a}_+)^{n+1} \psi_0(x);\,\, \psi_n = A_n ( \hat{a}_+)^n \psi_0\\
\implies \psi_{n+1} = ( \hat{a}_+) \frac{A_{n+1}}{A_n} ( A_n ( \hat{a}_+)^n \psi_0(x)) = \hat{a}_+ \frac{1}{c_n} \psi_n\\
\implies \hat{a}_+ \psi_n = c_n \psi_{n+1}$ similarly $\hat{a}_- \psi_n = d_n \psi_{n-1}\\$
what are $c_n$ and $d_n$?\\
$\int_{- \infty}^{\infty} ( \hat{a}_{\pm} \psi_n)^* (\hat{a}_{\pm} \psi_n) = \int_{- \infty}^{\infty} ( \hat{a}_{mp} \hat{a}_{\pm} \psi_n)^* \psi_n dx\\
\hat{a}_- \hat{a}_+ \psi_n = (n +1) \psi_n ;\,\, \hat{a}_+ \hat{a}_- \psi_n = n \psi_n\\$
\underline{proof:}\\
$\hbar \omega ( \hat{a}_- \hat{a}_+ + \frac{1}{2}) \psi_n = E_n \psi_n\\
\implies \hat{a}_- \hat{a}_+ \psi_n = \frac{E_n}{\hbar \omega} \psi_n + \frac{1}{2} \psi_n\\
E_n = \hbar \omega ( n + \frac{1}{2})\\
\implies \hat{a}_- \hat{a}_+ \psi_n = ( n+ \frac{1}{2}) \psi_n + \frac{1}{2} \psi_n ( n+1) \psi_n \\
\int_{- \infty}^{\infty} ( \hat{a}_+ \psi_n)^* ( \hat{a}_+ \psi_n) dx = | c_n |^2 \int |\psi_{n+1}|^2 dx\\
= \int_{-\infty}^{\infty} ( \hat{a}_- \hat{a}_+ \psi_n)^* \psi_n dx = ( n+1) \int | \psi_n|^2 d\\
\int_{- \infty}^{\infty} ( \hat{a}_- \psi_n)^*( \hat{a}_- \psi_n) dx = |d_n|^2 \int | \psi_{n-1}|^2 d = \int ( \hat{a}_+ \hat{a}_- \psi_n)^* \psi_n dx\\
= n \int | \psi_n |^2 dx \implies |c_n|^2 = ( n+1);\,\, |d_n|^2 = n\\
\therefore \hat{a}_+ \psi_n = \sqrt{n+1} \psi_{n+1};\,\, \hat{a}_- \psi_n = \sqrt{n} \psi_{n-1}$



\underline{recall:} $V= \frac{1}{2} k x^2;\,\, \omega = \sqrt{\frac{k}{m}} \implies V= \frac{1}{2} m \omega^2 x^2$

\hdashrule[0.5ex][c]{\linewidth}{0.5pt}{1.5mm}


\item \underline{$\hat{H} = \hbar \omega( \hat{a}_{\pm} \hat{a}_{\mp} \pm \frac{1}{2}) \psi = E \psi$}\\
\underline{recall:} $H \psi = E \psi;\,\, \hat{p} = - i \hbar \frac{d}{dx}\\
\implies - \frac{\hbar^2}{2m} \frac{d^2 \psi}{d x^2}  + \frac{1}{2} m \omega^2 x^2 \psi = E \psi\\
\implies \frac{1}{2m} [(-i \hbar \frac{d}{dx})^2 + m^2 \omega^2 x^2] \psi = E \psi\\
\implies \frac{1}{2m} [ \hat{p}^2 + ( m \omega x)^2] \psi = E \psi \\$
Lets factor this, if they were numbers\\
$u^2 + v^2 = ( i u + v)(-i u + v)\\$
so lets consider\\
$\frac{1}{2m} ( i \hat{p} + m \omega x)(-i \hat{p} + m \omega x)\\$
define $\hat{a}_{\pm} = \frac{1}{\sqrt{2 \hbar m \omega}}( \mp i \hat{p} + m \omega x)$ (factor is for convenience)\\
$\implies \hat{a}_- \hat{a}_+ = \frac{1}{2 \hbar m \omega}  (i \hat{p} + m \omega x)(-i \hat{p} + m \omega x)\\
= \frac{1}{2 \hbar m \omega}( \hat{p}^2 + i m \omega ( \hat{p} x - x \hat{p}) + ( m \omega x)^2)\\$
\underline{Note:} $\{( \hat{p} x - x \hat{p}) \psi = - i \hbar \frac{d}{dx} ( x \psi) + i \hbar x \frac{d}{dx} \psi\\
= - i \hbar \psi - i \hbar x \frac{d\psi}{dx} + i \hbar x \frac{d \psi}{dx} = - i \hbar \psi\\
\implies [ \hat{p}, \hat{x}] = - i \hbar \}\\
\implies \hat{a}_- \hat{a}_+ = \frac{1}{2 \hbar m \omega} ( \hat{p}^2 + \hbar m \omega + ( m \omega x)^2)\\
= \frac{1}{\hbar \omega} \frac{1}{2m} [ \hat{p}^2 + ( m \omega x)^2] + \frac{1}{2} = \frac{1}{\hbar \omega} \hat{H} + \frac{1}{2}\\$
similarly\\
$\hat{a}_+ \hat{a}_- = \frac{1}{\hbar \omega} \hat{H} - \frac{1}{2}\\
\implies [ \hat{a}_-, \hat{a}_+]= \hat{a}_- \hat{a}_+ - \hat{a}_+ \hat{a}_-\\
= \frac{1}{2} + \frac{1}{2} = 1\\
\implies \hat{H} = \hbar \omega ( \hat{a}_+ \hat{a}_- + \frac{1}{2})\\
\therefore \hat{H} \psi = \hbar \omega( \hat{a}_{\pm} \hat{a}_{\mp} \pm \frac{1}{2}) = E \psi$


\hdashrule[0.5ex][c]{\linewidth}{0.5pt}{1.5mm}


\item \underline{$\hat{H} \psi = E \psi \implies \hat{H} ( \hat{a}_+ \psi) = ( E + \hbar \omega) \hat{a}_+ \psi$ and $\hat{H} ( \hat{a}_- \psi) = ( E - \hbar \omega)( \hat{a}_- \psi)$}\\
\underline{recall:} $\hat{H} = \hbar \omega ( \hat{a}_+ \hat{a}_- + \frac{1}{2});\,\, [ \hat{a}_-, \hat{a}_+] = 1\\
\implies \hat{H} ( \hat{a}_+ \psi) = \hbar \omega ( \hat{a}_+ \hat{a}_- + \frac{1}{2})(\hat{a}_+ \psi)\\
= \hbar \omega ( \hat{a}_+ \hat{a}_- \hat{a}_+ + \frac{1}{2} \hat{a}_+) \psi = \hbar \omega ( \hat{a}_+(1+ \hat{a}_+ \hat{a}_-) + \frac{1}{2} \hat{a})_) \psi\\
= \hbar \omega( \hat{a}_+ \hat{a})_+ \hat{a}_- + \frac{3}{2} \hat{a}_+) \psi = [ \hbar \omega \hat{a}_+( \hat{a}_+ \hat{a}_- + \frac{1}{2}) + \hbar \omega \hat{a}_+] \psi\\
= \hat{a}_+ \hat{H} \psi + \hbar \omega \hat{a}_+ \psi = ( E + \hbar \omega) \hat{a}_+ \psi \Box\\$


\hdashrule[0.5ex][c]{\linewidth}{0.5pt}{1.5mm}


applying $\hat{a}_-$ repeatedly ends up giving us negative energy which cannot happen\\
$\implies \hat{a}_- \psi_0=0 ( \psi_0$ lowest rung)\\


\hdashrule[0.5ex][c]{\linewidth}{0.5pt}{1.5mm}


\item \underline{$\psi_0(x) = ( \frac{m \omega}{\pi \hbar})^{1/4} e^{- \frac{m \omega}{2 \hbar} x^2};\,\, E_0 = \frac{\hbar \omega}{2}$}\\
$\hat{a}_- \psi_0 = 0\\
\implies \frac{1}{\sqrt{2 \hbar m \omega}} ( - i \hat{p} + m \omega x) \psi_0=0\\
\implies ( \hbar \frac{d}{dx} + m \omega x) \psi_0=0\\
\implies \frac{d \psi_0}{dx} + \frac{m \omega x}{\hbar} \psi_0=0\\
\implies \psi'_0 =- \frac{m \omega}{\hbar} x \psi_0\\
\implies \int \frac{d \psi_0}{\psi_0} = - \frac{m \omega}{\hbar} \int x dx + C = - \frac{m \omega}{2 \hbar} x^2 + C\\
\implies \psi_0(x) = A e^{-\frac{m \omega}{2 \hbar} x^2}\\
\int_{- \infty}^{\infty} = |A|^2 \int_{- \infty}^{\infty} e^{-m \omega x^2/\hbar} dx = |A|^2 \sqrt{\frac{\pi \hbar}{m \omega}} = 1\\
\implies |A| = (\frac{m \omega}{\pi \hbar})^{1/4}\\
\therefore \psi_0(x) = ( \frac{m \omega}{\pi \hbar})^{1/4} e^{- \frac{m \omega}{2 \hbar} x^2}\\
\hat{H} \psi_0 = E \psi\\
\implies \hbar \omega(\hat{a}_+ \hat{a}_- + \frac{1}{2}) \psi_0 = \frac{\hbar \omega}{2} \psi_0 = E_0 \psi_0\\
\implies E_0 = \frac{\hbar \omega}{2}\\$


\hdashrule[0.5ex][c]{\linewidth}{0.5pt}{1.5mm}


$\psi_n$ obtained after applying ( $\hat{a}_+)^n$ and normalizing \\
$\implies \psi_n(x) = A_n ( \hat{a}_+)^n \psi_0(x)\\$


\hdashrule[0.5ex][c]{\linewidth}{0.5pt}{1.5mm}


\item \underline{$E_n = \hbar \omega ( n + \frac{1}{2})$}\\
\underline{Proof:} (Induction)\\
\underline{Base Case:} $\hat{H} = \hbar \omega ( \hat{a}_+ \hat{a}_- + \frac{1}{2}) \psi_0 = \frac{\hbar \omega}{2} \psi_0\\$
$\implies E_0 = \frac{\hbar \omega}{2}\\$
\underline{Induction Step:} Assume $E_n = \hbar \omega ( n + \frac{1}{2})\\$
want $E_{n+1} = \hbar \omega ( n + \frac{3}{2})\\$
$\psi_{n+1} = k \hat{a}_+ \psi_n;\,\, k \sim$ const.\\
Know $\hat{H} \psi_n = E_n \psi_n\\
\implies \hat{H} \hat{a}_+ \psi_n = E_n \hat{a}_+ \psi_n\\
\hbar \omega ( \hat{a}_+ \hat{a}_- + \frac{1}{2}) \hat{a}_+ \psi_n = \hbar \omega( \hat{a}_+ \hat{a}_- \hat{a}_+ + \frac{\hat{a}_+}{2}) \psi_n\\
= \hbar \omega ( \hat{a}_+(1+ \hat{a}_+ \hat{a}_-) + \frac{\hat{a}_+}{2}) \psi_n\\
= \hbar \omega \hat{a}_+ ( \hat{a}_+ \hat{a}_- + \frac{1}{2} + 1) = \hat{a}_+ ( \hat{H} + \hbar \omega) \psi_n\\
= (E_n + \hbar \omega) \hat{a})_+ \psi_n\\
\implies \hat{H} \psi_{n+1} = E_{n+1} \psi_{n+1} = \hbar \omega ( n + \frac{3}{2}) \psi_{n+1}\\
\therefore E_n = \hbar \omega(n + \frac{1}{2})$

\hdashrule[0.5ex][c]{\linewidth}{0.5pt}{1.5mm}


\item \underline{$\psi_n = \frac{1}{\sqrt{n!}} ( \hat{a}_+)^n \psi_0$}\\
$\psi_1 = \hat{a}_+ \psi_0;\,\, \psi_2 = \frac{1}{\sqrt{2}} \hat{a}_+ \psi_1 = \frac{1}{\sqrt{2}} ( \hat{a}_+)^2 \psi_0;\\
\psi_3 = \frac{1}{\sqrt{3}} \hat{a}_+ \psi_2 = \frac{1}{ \sqrt{3 \cdot 2 }} ( \hat{a}_+)^3 \psi_0;\,\, \psi_4 = \frac{1}{\sqrt{4 \cdot 3 \cdot 2}} ( \hat{a}_+)^4 \psi_0\\
\therefore \psi_n = \frac{1}{\sqrt{n!}} (\hat{a}_+)^2 \psi_0;\,\, A_n = \frac{1}{\sqrt{n!}}\\
$

\hdashrule[0.5ex][c]{\linewidth}{0.5pt}{1.5mm}


\item \underline{$\int_{- \infty}^{\infty} \psi_m^* \psi_n dx = \delta_{mn}$}\\
$\int_{-\infty}^{\infty} \psi_m^* ( \hat{a}_+ \hat{a}_-) \psi_n dx = \int_{- \infty}^{\infty} \psi_m^* \hat{a}_+ \sqrt{\psi_{n-1}} dx\\
= \int_{- \infty}^{\infty} \psi_m^* \sqrt{n} \sqrt{n} \psi_n dx = n \int_{\-\infty}^{\infty} \psi_m^* \psi_n dx\\
=\int_{- \infty}^{\infty} ( \hat{a}_- \psi_m)^* \hat{a}_- \psi_n dx = \int_{- \infty} ^{\infty} ( \hat{a}_+ \hat{a}_- \psi_m)^* \psi_n dx\\
= \int_{- \infty}^{\infty}( \hat{a})_+ \sqrt{m} \psi_{m-1})^* \psi_n dx\\
= m \int_{- \infty}^{\infty} \psi_m^* \psi_n dx \implies m=n or \int_{- \infty}^{\infty} \psi_m^* \psi_n dx = 0\\
\therefore \int_{- \infty}^{\infty} \psi_m^* \psi_n dx = \delta_{mn}\\$


\hdashrule[0.5ex][c]{\linewidth}{0.5pt}{1.5mm}


\item \underline{ $\psi_n(x) = ( \frac{m \omega}{\pi \hbar})^{1/4} \frac{1}{\sqrt{2^n n!}} H_n ( \xi) e^{- \xi^2/2};\,\, E_n = \hbar \omega ( n+ \frac{1}{2}),\,\,$ (analytic method)}\\
$- \frac{\hbar^2}{2m} \frac{d^2 \psi}{dx^2} + \frac{1}{2} m \omega^2 x^2 \psi = E \psi\\
\xi = \sqrt{ \frac{m \omega}{\hbar}} x\\$


\section*{Chapter 3}


\underline{Theorem 1: (discrete spectra)}\\ Hermitian operators have real eigenvalues\\
\underline{Proof}\\
Suppose $\hat{Q} f=q f$ and $\langle f | \hat{Q} f \rangle= \langle \hat{Q} f | f \rangle\\
\implies q \langle f | f \rangle = q^* \langle f| f \rangle \implies q=q^*$\\
\underline{Theorem 2:} Eigen functions corresponding to distinct eigenvalues are orthogonal\\
\underline{Proof:}\\
Suppose $\hat{Q} f = q f$ and $\hat{Q} g = q' g$ and $\langle f | \hat{Q} g \rangle = \langle \hat{Q} f| g \rangle$ \\
$\implies q' \langle f | g \rangle = q^* \langle f | g \rangle,\,\, q^* = q$ and $q' \neq q$\\
$\implies \langle f | g \rangle = 0$


\hdashrule[0.5ex][c]{\linewidth}{0.5pt}{1.5mm}


\underline{Axiom:} The eigenfunctions of an observable operator are complete: Any function ( in Hilbert space) can be expressed as a linear combination of them.


\hdashrule[0.5ex][c]{\linewidth}{0.5pt}{1.5mm}


\underline{Generalized Statistical Interpretation:} If you measure an observable $Q(x,p)$ on a particle in the state $\Psi(x,t)$, you are certain to get one of the eigenvalues of the hermitian operator $\hat{Q}(x,-i \hbar \frac{d}{dx})$. If the spectrum $\hat{Q}$ is discrete, the probability of getting the particular eigenvalue $q_n$ associated with the (orthonormalized) eigenfunction $f_n(x)$ is \\
$|c_n|^2$, \,\, where $c_n= \langle f_n | \psi \rangle$\\
If the spectrum is continuous, with real eigenvalues $q(z)$ and associated (Dirac-orthonormalized) eigenfunction $f_z(x)$ is\\
$|c(z)|^2 dz$ where $c(z) \langle f_z | \Psi \rangle$ \\
Upon measurement, the wave function "collapses" to the corresponding eigenstate.


\hdashrule[0.5ex][c]{\linewidth}{0.5pt}{1.5mm}


\underline{random facts:} $\Psi(x,t)=\sum_n c_n(t) f_n(x)$ (discrete)\\
 $c_{n}(t) = \langle f_n | \Psi \rangle = \int f_n^*(x) \Psi (x,t) dx\\
1=\langle \Psi | \Psi \rangle = \langle ( \sum_{n'} c_{n'} f_{n'}) | ( \sum_n c_n f_n) \rangle = \sum_{n'} \sum_n c_{n'}^* c_n \langle f_{n'} | f_n \rangle = \sum_{n,n'} c_n^* c_{n'}  \delta_{n n'}=\sum_n |c_n|^2$\\


\hdashrule[0.5ex][c]{\linewidth}{0.5pt}{1.5mm}


$\langle Q \rangle = \langle \Psi | \hat{Q} \Psi \rangle = \langle ( \sum_n c_n f_n ) | \hat{Q} ( \sum_{n'} c_{n'} f_{n'}) \rangle = \sum_{n,n'} c_n^* c_{n'} q_{n'} \langle f_n | f_{n'} \rangle = \sum_{n,n'} c_n^* c_{n'} q_{n'} \delta_{n n'} = \sum_n |c_n|^2 q_n$\\


\hdashrule[0.5ex][c]{\linewidth}{0.5pt}{1.5mm}


$g_y(x)= \delta(x-y)$ ( eigen functions of $\hat{x}$ ) \\
$c(y)= \langle g_y | \Psi \rangle = \int_{-\infty}^{\infty} \delta(x-y) \Psi (x,t) dx = \Psi(y,t) $


\hdashrule[0.5ex][c]{\linewidth}{0.5pt}{1.5mm}


$f_p(x) = \frac{1}{\sqrt{2 \pi \hbar}} \exp(ipx/\hbar)$ ( Dirac - ortho-normalized eigenfunctions of $\hat{p}$)\\
$c(p)=\langle f_p | \Psi \rangle = \frac{1}{\sqrt{2 \pi \hbar}} \int_{-\infty}^{\infty} e^{-ipx/\hbar} \Psi(x,t) dx\\$


\hdashrule[0.5ex][c]{\linewidth}{0.5pt}{1.5mm}





\hdashrule[0.5ex][c]{\linewidth}{0.5pt}{1.5mm}


$\Phi(p,t) = \frac{1}{\sqrt{2 \pi \hbar} }\int_{-\infty}^{\infty} e^{\frac{-ipx}{\hbar}} \Psi (x,t) dx = c(p)\\
\Psi(x,t) = \frac{1}{\sqrt{2 \pi \hbar}} \int_{-\infty}^{\infty} e^{\frac{ipx}{\hbar}} \Phi (p,t) dp\\$ fourier transform


\hdashrule[0.5ex][c]{\linewidth}{0.5pt}{1.5mm}


$| \Phi (p,t) |^2 dp$ (probability momentum is in range dp)


\hdashrule[0.5ex][c]{\linewidth}{0.5pt}{1.5mm}


\item \underline{$\sigma_{A}^2 \sigma_{B}^2 \geq (\frac{1}{2 i} \langle [ \hat{A}, \hat{B} ] \rangle )^2$}\\
\underline{recall:} $\sigma_A^2 = \langle ( \hat{A} - \langle A \rangle ) \Psi | ( \hat{A} - \langle A \rangle ) \Psi \rangle = \langle f | f\rangle \\
 \equiv (\hat{A}- \langle A \rangle ) \Psi ; \sigma_{B}^2 = \langle g | g \rangle,\,\, g \equiv ( \hat{B} - \langle B \rangle )\\
\sigma_A^2 \sigma_B^2 = \langle f | f \rangle \langle g | g \rangle$ \\
\underline{recall:} $ | u \cdot v | \leq ||u|| ||v||$ (Schwarz inequality) see 345 in linear algebra textbook\\
by analogy $| \langle f | g \rangle |^2 \leq \langle f | f \rangle \langle g | g\rangle \\
|z|^2 = [Re(z)]^2 + [Im(z)]^2 \geq [Im(z)]^2 = [\frac{1}{2i} (z-z^*)]^2$\\
set $z= \langle f | g \rangle$ and note $(\hat{A} - \langle A \rangle)$ is Hermitian\\
$\sigma_{A}^2 \sigma_B^2 = \langle f | f \rangle \langle g | g \rangle \geq | \langle f | g \rangle |^2 = |z|^2 \geq ( \frac{1}{2i} [ \langle f | g \rangle - \langle g|f \rangle])^2\\
\langle f | g \rangle = \langle ( \hat{A}- \langle A \rangle ) \Psi | \hat{B} - \langle B \rangle ) \Psi \rangle = \langle \Psi | \langle \Psi | ( \hat{A} - \langle A \rangle ) ( \hat{B} - \langle B \rangle )\Psi \rangle\\
= \langle \Psi |( \hat{A} \hat{B} - \hat{A} \langle B \rangle - \hat{B} \langle A \rangle + \langle A \rangle \langle B \rangle) \Psi \rangle \\
 =\langle \Psi | \hat{A} \hat {B} \Psi \rangle - \langle B \rangle \langle \Psi | \hat{A} \Psi \rangle - \langle A \rangle \langle \Psi | \hat{B} \Psi \rangle + \langle A \rangle \langle B \rangle \langle \Psi | \Psi \rangle \\
 \langle \hat{A} \hat{B} \rangle - \langle B \rangle \langle A \rangle - \langle A \rangle \langle B \rangle + \langle A \rangle \langle B \rangle \\
 =\langle \hat{A} \hat{B} \rangle- \langle A \rangle \langle B \rangle$\\
 by analogy $\langle g | f \rangle = \langle \hat{B} \hat{A} \rangle - \langle A \rangle \langle B \rangle\\
 \langle f | g \rangle - \langle g | f \rangle = \langle \hat{A} \hat{B} \rangle - \langle \hat{B} \hat{ A } \rangle = \langle [ \hat{A} , \hat{B} ] \rangle\\
 \therefore \sigma_A^2 \sigma_B^2 \geq (\frac{1}{2i} \langle [\hat{A}, \hat{B}] \rangle )^2$\\
 \underline{Shortened:}\\
 $\sigma_{A}^2 \sigma_B^2 = \langle f | f \rangle \langle g | g \rangle \geq | \langle f | g \rangle |^2 = |z|^2 \geq ( \frac{1}{2i} [ \langle f | g \rangle - \langle g|f \rangle])^2\\$
 plug in f and g and simplify
 
 
 \hdashrule[0.5ex][c]{\linewidth}{0.5pt}{1.5mm}


\item \underline{$\sigma_x \sigma_p \geq \frac{\hbar}{2}$}\\
Let $ \hat{A} = x$\,\,\,$ \hat{B} = \hat{p} = -i \hbar \frac{d}{dx}\\
\implies \sigma_x^2 \sigma_p^2 \geq ( \frac{1}{2i} \langle [\hat{x} , \hat{p}] \rangle)^2 = ( \frac{1}{2i} i \hbar)^2 = ( \frac{\hbar}{2})^2\\
\implies \sigma_x \sigma_p \geq \frac{\hbar}{2}$


\hdashrule[0.5ex][c]{\linewidth}{0.5pt}{1.5mm}


\underline{Note:} There is an uncertainty principle for every non-commuting set of observables (incompatible observables)


\hdashrule[0.5ex][c]{\linewidth}{0.5pt}{1.5mm}


\item \underline{$\Psi(x) = A e^{(x-\langle x \rangle )^2/(2\hbar)} e^{i \langle p \rangle x/\hbar}$}\\
\underline{recall:} $\langle f | f \rangle \langle g | g \rangle \geq | \langle f | g \rangle |^2 \geq | Im ( \langle f | g \rangle)|^2$ (Schwarz inequality)\\
\\
\underline{Note:} we are trying to figure out when the uncertainty principle becomes an equality and since this occurs when $\langle f | f \rangle \langle g | g \rangle= | Im ( \langle f | g \rangle)|^2$ then we must also have $\langle f | f \rangle \langle g | g \rangle = | \langle f | g \rangle |^2$\\
\\
Schwarz inequality becomes equality if $g(x) = c f(x);\,\, c \in \mathbb{C}$\\
\underline{recall:} $\sigma_A^2 \sigma_B^2 \geq [ Im ( \langle f | g \rangle ) ]^2 \implies$ equality occurs if $Re(\langle f | g \rangle  = Re ( c \langle f | f \rangle ) = Re (c)=0\\
\implies c = ia \implies g(x) = ia f(x) \\
\underline{recall:} g(x)=(\hat{A} - \langle A \rangle ) \Psi = (\hat{p}-\langle p \rangle p ) \Psi;\,\, f(x) = (x-\langle x \rangle ) \Psi;\,\, \hat{p}= - i \hbar \frac{d}{dx}\\
\implies ( - i \hbar \frac{d}{dx} - \langle p \rangle ) \Psi = i a (\hat{x}- \langle x \rangle ) \Psi$\\
unfinished\\
$\therefore \Psi(x) = A e^{- a(x- \langle x \rangle)^2/2 \hbar} e^{i \langle p \rangle x/\hbar}$\\
\underline{Note:} the constants $A, a , \langle x \rangle, and \langle p \rangle$ may all depend on time and may force the wave function to evolve away from the minimal packet uncertainty

 \hdashrule[0.5ex][c]{\linewidth}{0.5pt}{1.5mm}


\item \underline{$\frac{d}{dt} \langle Q \rangle = \frac{i}{\hbar} \langle [ \hat{H} , \hat{Q} ] \rangle + \langle \frac{\partial \hat{Q}}{\partial t} \rangle$}\\
$\frac{d}{dt} \langle Q \rangle = \frac{d}{dt} \langle \Psi | \hat{Q} \Psi \rangle = \langle \frac{\partial \Psi}{\partial t} | \hat{Q} \Psi \rangle + \langle \Psi | \frac{\partial \hat{Q}}{\partial t} \Psi \rangle + \langle \Psi | \hat{Q} \frac{\partial \Psi}{\partial t} \rangle\\$
\underline{recall:} $i \hbar \frac{\partial \Psi}{\partial t} = \hat{H} \Psi,\,\,$ Here $\hat{H} = \frac{\hat{p}^2}{2m} + V\\
\implies \frac{d}{dt} \langle Q \rangle = - \frac{1}{i \hbar} \langle \hat{H} \Psi |  \hat{Q} \Psi \rangle + \frac{1}{i \hbar} \langle \Psi | \hat{Q} \hat{H} \Psi \rangle + \langle \frac{\partial \hat{Q}}{ \partial t} \rangle \\
\therefore \frac{d}{dt} \langle Q \rangle = \frac{i}{\hbar} \langle [\hat{H}, \hat{Q} ] \rangle + \langle \frac{\partial \hat{Q}}{\partial t} \rangle\\$


 \hdashrule[0.5ex][c]{\linewidth}{0.5pt}{1.5mm}
\item\underline{$\Delta E \Delta t \geq \frac{\hbar}{2}$} (dont understand)\\
\underline{recall:} $\sigma_{A}^2 \sigma_B^2 \geq ( \frac{1}{2i} \langle [ \hat{A} , \hat{B} ] \rangle )^2;\,\,$ take $A=H$ and $B=Q\\
\implies \sigma_H^2 \sigma_Q^2 \geq ( \frac{1}{2i} \langle [ \hat{H}, \hat{Q}] \rangle )^2;\,\,$ assume $\langle \frac{\partial \hat{Q}}{\partial t} \rangle = 0\\
\implies \sigma_H^2 \sigma_Q^2 \geq ( \frac{1}{2i} \frac{\hbar}{i} \frac{d \langle Q \rangle }{d t})^2 \implies \sigma_H \sigma_Q \geq \frac{\hbar}{2} | \frac{d \langle Q \rangle }{ dt}|\\
\Delta t \equiv \frac{\sigma_Q}{|d \langle Q \rangle/dt|};\,\, \Delta E \equiv \sigma_H\\
\therefore \Delta E \Delta t \geq \frac{\hbar}{2}$\\


\hdashrule[0.5ex][c]{\linewidth}{0.5pt}{1.5mm}


$\Psi(x,t),\,\, \Phi(p,t),\,\, c_n(t)$ are all ''components'' of $|S(t) \rangle$. e.g. for $\vec{A};\,\,A_x = \hat{i} \cdot \vec{A}\\$
by analogy $\Psi(x,t)=\langle x | S(t) \rangle,\,\, \Phi(p,t)=\langle p | S(t) \rangle\\$
$c_n(t) = \langle n | S(t) \rangle,\,\,$ in position basis\\
$| x \rangle = g_x,\,\, | p \rangle = f_p\\
\Psi,\,\, \Phi,\,\, \{ c_n \}$ contain same information\\
$|S(t) \rangle \rightarrow \int \Psi (y,t) \delta(x-y) dy = \int \Phi (p,t) \frac{1}{\sqrt{2 \pi \hbar}} e^{ipx/\hbar} dp\\
=\sum_n c_n e^{-iE_nt/\hbar} \psi_n(x)\\$


\hdashrule[0.5ex][c]{\linewidth}{0.5pt}{1.5mm}


$| \beta \rangle = \hat{Q} | \alpha \rangle$ ( operators are linear transformations on Hilbert space)


\hdashrule[0.5ex][c]{\linewidth}{0.5pt}{1.5mm}


\item \underline{$b_m = \sum_n Q_{mn} a_n$ (discrete)}\\
$| \alpha \rangle= \sum_n a_n | e_n \rangle,\,\, | \beta \rangle = \sum_n b_n | e_n \rangle,\,\, a_n \langle e_n | \alpha \rangle,\,\, b_n = \langle e_n | \beta \rangle;\,\, \langle e_m | \hat{Q} | e_n \rangle \equiv Q_{mn}\\
| \beta \rangle = \hat{Q} | \alpha \rangle \implies \sum_n b_n | e_n \rangle = \sum_n a_n \hat{Q} | e_n \rangle\\
\implies \sum_n b_n \langle e_m | e_n \rangle = \sum_n \delta_{mn} b_n = b_m = \sum_n a_n \langle e_m | \hat{Q} | e_n \rangle = \sum_n Q_{mn} a_n$\\


\hdashrule[0.5ex][c]{\linewidth}{0.5pt}{1.5mm}

$\star$ need side by side comparison of these.\\
$\hat{x}$ ( position operator) $\rightarrow$ ($x$ ( in position space);$\,\, i \hbar \frac{\partial}{\partial p}$ ( in momentum space))\\
$\hat{p}$ ( momentum operator) $\rightarrow (-i \hbar \frac{\partial}{\partial x}$ ( in position space);\,\, p ( in momentum space))\\


\hdashrule[0.5ex][c]{\linewidth}{0.5pt}{1.5mm}


$\langle f| = \int f^* [\dots] dx$ (bra)\\
a bra spits out a complex number when it hits a vector $|g \rangle\\
| \alpha \rangle \rightarrow \begin{pmatrix} a_1\\ a_2 \\ \vdots \\ a_n \end{pmatrix}$ (finite dimensional space)\\
$\langle \beta | \rightarrow (b_1^*, b_2^*, \dots , b_n^*)\\
\langle \beta | \alpha \rangle = \sum_i b_i^* a_i\\
\hat{P} \equiv | \alpha \rangle \langle \alpha | \implies \hat{P} | \beta \rangle = ( \langle \alpha | \beta \rangle ) | \alpha \rangle$ (projection operator; picks out portion of $| \beta \rangle$ that lies along $| \alpha \rangle)$\\


\hdashrule[0.5ex][c]{\linewidth}{0.5pt}{1.5mm}


\item \underline{$\sum_n |e_n \rangle \langle e_n | = 1;\,\, \int |e_z \rangle \langle e_z | dz = 1$}\\
$| \alpha \rangle = \sum_n (\langle e_n | \alpha \rangle ) | e_n \rangle = \sum_n ( | e_n \rangle \langle e_n |)| \alpha \rangle\\
\implies \sum_n |e_n \rangle \langle e_n |$ if $\langle e_m | e_n \rangle = \delta_{mn}$ (orthogonal basis)\\
if $\langle e_z | e_{z'} \rangle = \delta(z-z') \implies \int | e_z \rangle \langle e_z| dz = 1\\$


\hdashrule[0.5ex][c]{\linewidth}{0.5pt}{1.5mm}


\underline{Note:} operator functions such as $e^{\hat{Q}}$ are defined in terms of their Maclaurin series\\


\hdashrule[0.5ex][c]{\linewidth}{0.5pt}{1.5mm}


\underline{Note:} $ \hat{P}^2 = \hat{P} $ (Idempotent)\\


\hdashrule[0.5ex][c]{\linewidth}{0.5pt}{1.5mm}


$f(x) = \frac{1}{\sqrt{2 \pi}} \int_{-\infty}^{\infty} F(k) e^{ikx} dk$\\
$F(k) = \frac{1}{\sqrt{2 \pi}} \int_{- \infty}^{\infty} f(x) e^{-ikx} dx\\$


\hdashrule[0.5ex][c]{\linewidth}{0.5pt}{1.5mm}


\item \underline{$\delta(x) = \frac{1}{2 \pi } \int_{- \infty}^{\infty} e^{ikx} dk$}\\
$f(x)=\delta(x) = \frac{1}{\sqrt{2 \pi}} \int_{-\infty}^\infty F(k) e^{ikx} dk\\
F(k) = \frac{1}{\sqrt{2 \pi}} \int_{-\infty}^{\infty} \delta(x) e^{-ikx} dx = \frac{1}{\sqrt{2 \pi}} \implies \delta (x) = \frac{1}{2 \pi} \int_{-\infty}^{\infty} e^{ikx} dk
$

\hdashrule[0.5ex][c]{\linewidth}{0.5pt}{1.5mm}



\item \underline{$\langle x | p \rangle = f_p(z) = \frac{1}{\sqrt{2 \pi \hbar}} e^{ipx/\hbar}$} (momentum eigenfunction in position basis)\\
$\hat{p} | p \rangle = p |p \rangle\\
\implies \langle x | \hat{p} | p \rangle = p \langle x | p \rangle$ (The operator comes out after $\langle x |$ acts on it, this changes $\hat{p}$ into the position basis) $\implies -i \hbar \frac{df_p}{dx}=p f_p \\$
$\implies f_p(x) = A e^{ipx/\hbar}\\ 
\int_{-\infty}^{\infty} f_{p'}^* f_p dx =|A|^2 \int_{-\infty}^{\infty} e^{(p-p')ix/\hbar}dx = |A|^2 2 \pi \hbar \delta(\frac{p-p'}{\hbar})\\
|A|^2 2 \pi \hbar \delta(p-p')$ choose $|A | =\frac{1}{\sqrt{2 \pi \hbar}}$ so that $\langle f_{p'}|f_p \rangle = \delta (p-p;)$\\
$\therefore \langle x | p \rangle = \frac{1}{2 \pi \hbar} e^{ipx/\hbar}$\\
\\
Need derivation of position eigenfunction


\hdashrule[0.5ex][c]{\linewidth}{0.5pt}{1.5mm}


\underline{Note:} $\Phi(p,t) = \langle p | S(t) \rangle;\,\, \Psi(x,t) = \langle x | S(t) \rangle$


\hdashrule[0.5ex][c]{\linewidth}{0.5pt}{1.5mm}


\item \underline{$\Phi (p,t)= \int \frac{1}{\sqrt{2 \pi \hbar}} e^{-ipx/\hbar} \Psi (x,t) dx$}\\
$\Phi(p,t)=\langle p | S(t) \rangle\\
= \langle p | ( \int dx | x \rangle \langle x | ) | S(t) \rangle \\
=\int \langle p | x \rangle \langle x | S(t) \rangle dx\\
=\int \langle p | x \rangle \Psi(x,t) dx\\
\langle x | p \rangle =f_p(x)\\
\implies \langle p | x \rangle = \langle x | p \rangle^* = [f_p(x)]^* = \frac{1}{\sqrt{2 \pi \hbar}} e^{-ipx/\hbar} \\
\therefore \Phi (p,t) = \int \frac{1}{\sqrt{2 \pi \hbar}} e^{-ipx/\hbar} \Psi(x,t) dx$


\hdashrule[0.5ex][c]{\linewidth}{0.5pt}{1.5mm}


$\langle x | \hat{x} | S(t) \rangle =$ action of position operator in $x$ basis $= x \Psi(x,t)\\
\langle p | \hat{x} | S(t) \rangle =$ action of position operator in $p$ basis = $i \hbar \frac{\partial \Phi}{\partial p}$


\hdashrule[0.5ex][c]{\linewidth}{0.5pt}{1.5mm}


\item \underline{$\langle p | \hat{x} | S(t) \rangle = i \hbar \frac{\partial}{\partial p} \Phi(p,t)$}\\
$\langle p | \hat{x} | S(t) \rangle = \langle p | \hat{x} (\int dx | x \rangle \langle x | ) | S(t) \rangle \\
=\int \langle p | \hat{x} x \rangle \langle x | S(t) \rangle dx;\,\, \hat{x} | x \rangle = x | x \rangle\\
=\int x \langle p | x \rangle \langle x | S(t) \rangle dx\\$
\underline{recall:} $ \langle x | S(t) \rangle = \Psi(x,t);\,\, \langle p | x \rangle = \langle x | p \rangle = f_p^*\\
=\frac{1}{\sqrt{2 \pi \hbar}} e^{-ipx/\hbar} \\
\therefore \langle p | \hat{x} | S(t) \rangle = \int x \frac{e^{-ipx/\hbar}}{\sqrt{2 \pi \hbar}} \Psi(x,t) dx\\
=i \hbar \frac{\partial}{\partial p} \int \frac{e^{-ipx/\hbar}}{\sqrt{2 \pi \hbar}} \Psi(x,t) dx\\$


\hdashrule[0.5ex][c]{\linewidth}{0.5pt}{1.5mm}

\section*{Quantum Mechanics: Chapter 4}


\item \underline{$i \hbar \frac{\partial \Psi}{\partial t} = - \frac{\hbar^2}{2m} \nabla^2 \Psi + V \Psi$}\\
\underline{recall:} $i \hbar \frac{\partial \Psi}{\partial t}=\hat{H} \Psi$\\
$\hat{H} = \frac{1}{2} m v^2 + V = \frac{1}{2m}(p_x^2 + p_y^2 + p_z^2) + V\\
p_x=-i \hbar \frac{\partial}{\partial x};\,\, p_y = - i \hbar \frac{\partial}{\partial y};\,\, p_z = -i \hbar \frac{\partial}{\partial z}\\
\implies \vec{p} \rightarrow -i \hbar \nabla\\
\therefore i \hbar \frac{\partial \Psi}{\partial t} = - \frac{\hbar^2}{2m} \nabla^2 \Psi + V \Psi$


\hdashrule[0.5ex][c]{\linewidth}{0.5pt}{1.5mm}


$V(\vec{r},t)=V(\vec{r}) \implies \Psi_n(\vec{r},t) = \psi_n (\vec{r}) e^{-i E_nt/\hbar}\\
\implies -\frac{\hbar^2}{2m} \nabla^2 \psi + V \psi = E \psi\\
\implies$ Gen solution $= \Psi(\vec{r},t) = \sum_n c_n \psi_n(\vec{r}) e^{-i E_n t/\hbar}\\$


\hdashrule[0.5ex][c]{\linewidth}{0.5pt}{1.5mm}


\item \underline{$\frac{1}{R} \frac{d}{dr}(r^2 \frac{dR}{dr}) - \frac{2 m r^2}{\hbar^2}[V(r) - E] = \ell (\ell + 1);\,\, \frac{1}{Y} \{ \frac{1}{\sin \theta} \frac{\partial}{\partial \theta}(\sin \theta \frac{\partial Y}{\partial \theta}) + \frac{1}{\sin^2 \theta} \frac{\partial^2 Y}{\partial \phi^2} \} = - \ell(\ell+1)$}\\
\underline{recall:} $ - \frac{\hbar^2}{2m} \nabla^2 \psi + V \psi = E \psi\\
\nabla^2 = \frac{1}{r^2} \frac{\partial}{\partial r} (r^2 \frac{\partial}{\partial r}) + \frac{1}{r^2 \sin \theta} \frac{\partial}{\partial \theta}(\sin \theta \frac{\partial}{\partial \theta}) + \frac{1}{r^2 \sin^2 \theta}(\frac{\partial^2}{\partial \phi^2})$
$\implies - \frac{\hbar^2}{2m} [ \frac{1}{r^2} \frac{\partial}{\partial r} (r^2 \frac{\partial \psi}{\partial r}) + \frac{1}{r^2 \sin \theta} \frac{\partial}{\partial \theta}(\sin \theta \frac{\partial \psi}{\partial \theta}) + \frac{1}{r^2 \sin^2 \theta} (\frac{\partial^2 \psi}{\partial \phi^2})] + B \psi = E \psi\\
\psi(r, \theta, \phi) = R(r) Y(\theta, \phi)\\
-\frac{\hbar^2}{2m} [ \frac{Y}{r^2} \frac{d}{dr} (r^2 \frac{d R}{dr}) + \frac{R}{r^2 \sin^2 \theta} \frac{\partial}{\partial \theta}(\sin \theta \frac{\partial Y}{\partial \theta}) + \frac{R}{r^2 \sin^2 \theta} \frac{\partial^2 Y}{\partial \phi^2}] + V R Y = E R Y\\$
divide by YR mult. $-\frac{2m r^2}{\hbar^2}\\
\implies \{ \frac{1}{R} \frac{d}{dr} (r^2 \frac{d R}{dr}) - \frac{2 m r^2}{\hbar^2} [ V(r) - E ] \} + \frac{1}{Y} \{ \frac{1}{\sin \theta} \frac{\partial}{\partial \theta}(\sin \theta \frac{\partial Y}{\partial \theta}) + \frac{1}{\sin^2 \theta} \frac{\partial^2 Y}{\partial \phi^2} \} = 0\\$


\hdashrule[0.5ex][c]{\linewidth}{0.5pt}{1.5mm}


\item \underline{$\Phi(\phi) = e^{im \phi}$}\\
\underline{recall:} $\frac{1}{Y} \{ \frac{1}{\sin \theta} \frac{\partial}{\partial \theta} ( \sin \theta \frac{\partial Y}{\partial \theta}) + \frac{1}{\sin^2 \theta} \frac{\partial^2 Y}{\partial \phi^2} = - \ell ( \ell + 1)\\
\implies \sin \theta \frac{\partial}{\partial \theta} ( \sin \theta \frac{\partial Y}{\partial \theta}) + \frac{\partial^2 Y}{\partial \phi^2} = - \ell ( \ell + 1) \sin^2 \theta Y\\
Y(\theta, \phi) = \Theta (\theta) \Phi(\phi)\\
\implies \sin \theta \frac{\partial}{\partial \theta}(\sin \theta \Phi (\phi) \frac{\partial \Theta}{\partial \theta}) + \Theta \frac{\partial \Phi}{\partial^2} = - \ell (\ell + 1) \sin ^2 \theta \Phi \Theta\\
\implies \frac{1}{\Theta} [ \sin \theta \frac{d}{d \theta} ( \sin \theta \frac{d \Theta}{d \theta})] + \ell ( \ell +1) \sin^2 \theta + \frac{1}{\Phi} \frac{d^2 \Phi}{d \phi^2} = 0\\
\implies 
\begin{cases}
	\frac{1}{\Theta} [ \sin \theta \frac{d}{d \theta} ( \sin \theta \frac{d \Theta}{d \theta}) ] + \ell ( \ell + 1) \sin^2 \theta = m^2\\
	\frac{1}{\Phi} \frac{d^2 \Phi}{d \phi^2} = - m^2
\end{cases}\\
\therefore \Phi(\phi) = e^{im \phi} m$ can be pos or neg\\


\hdashrule[0.5ex][c]{\linewidth}{0.5pt}{1.5mm}


\item \underline{$m= 0, \pm 1, \pm 2, \dots$}\\
Natural to require $\Phi(\phi + 2 \pi) = \Phi(\phi)\\
\implies e^{im (\phi + 2 \pi)} = e^{im \phi} \implies e^{2 \pi m i} = 1\\
\implies \cos 2 \pi m = 1,\,\, \sin 2 \pi m = 0\\
\implies m \in \mathbb{Z} \implies m = \frac{1}{2}, 1 , \frac{3}{2}\\
\implies m \in \mathbb{Z}\\
\therefore m= 0 , \pm 1, \pm 2 , \dots\\$


\hdashrule[0.5ex][c]{\linewidth}{0.5pt}{1.5mm}


\item \underline{$\Theta (\theta) = A P^m_{\ell} (\cos \theta)$}\\
\underline{recall:} $\sin \theta \frac{d}{d \theta} ( \sin \theta \frac{d \Theta}{d \theta}) + [ \ell (\ell + 1 ) \sin^2 \theta - m^2] \Theta = 0\\
\implies \Theta (\theta) = A P^m_{\ell} (\cos \theta)\\
P^m_{\ell} (\cos \theta) \equiv (-1)^m (1- x^2)^{|m|/2} (\frac{d}{dx})^{|m|} P_{\ell}(x) \sim$ associated Legendre polynomial\\
$P_{\ell} (x) = \frac{1}{2^{\ell} \ell!} (\frac{d}{dx})^{\ell} (x^2 -1 )^{\ell}$ (Legendre polynomial)\\


\hdashrule[0.5ex][c]{\linewidth}{0.5pt}{1.5mm}


\underline{Note:} $\ell >0$ and $\ell \in \mathbb{N}\cup \{0\}$ (because of derivative) from above formula also note that $P_{\ell}(x)$ is an $\ell$ th degree polynomial so $|m|<=\ell \implies -\ell<|m|<\ell$, otherwise above formula will yield $P_{\ell}^m(x)=0$\\


\hdashrule[0.5ex][c]{\linewidth}{0.5pt}{1.5mm}


$\int |\psi|^2 r^2 \sin \theta dr d \theta d \phi = \int |R|^2 r^2 dr \int |Y|^2 d \Omega = 1\\
\int_0^R |R|^2 r^2 dr = 1\,\, \int_0^\pi \int_0^{2 \pi} |Y|^2 \sin \theta d \theta d \phi = 1\\
Y_{\ell}^m (\theta, \phi) = \sqrt{\frac{(2 \ell + 1)}{4 \pi} \frac{(\ell-m)!}{(\ell + m)!}} e^{i m \phi} P_{\ell}^m (\cos \theta)\\
$ (spherical Harmonics)\\


\hdashrule[0.5ex][c]{\linewidth}{0.5pt}{1.5mm}


\item \underline{$- \frac{\hbar^2}{2m} \frac{d^2 u}{dr^2} + [V + \frac{\hbar^2}{2m} \frac{\ell(\ell +1)}{r^2}] u = E u$} (radial equation\\
\underline{recall:} $ \frac{1}{2} m \dot{r}^2 + \left[ \frac{1}{2} \frac{\ell^2}{m r^2} + V(r) \right ] = E$\\
\underline{recall:} $\frac{d}{dr} (r^2 \frac{dR}{dr}) - \frac{2 m r^2}{\hbar^2} [ V(r) - E ] R = \ell (\ell+1) R\\
u(r) = r R(r) \implies R = \frac{u}{r},\,\, \frac{dR}{dr} = [r \frac{du}{dr} - u ] \frac{1}{r^2}\\
\frac{d}{dr} [ r^2 \frac{dR}{dr}] = \frac{d}{dr} [ r^2 (\frac{1}{r^2} ( r \frac{du}{dr} - u))] = \frac{du}{dr} + r \frac{d^2 u}{dr^2} - \frac{du}{dr} = r \frac{d^2 u}{dr^2}\\
\implies r \frac{d^2 u}{dr^2} - \frac{2m r^2}{\hbar^2} [ V(r) - E] \frac{u}{r} = \ell (\ell + 1) \frac{u}{r}\\
\therefore - \frac{\hbar^2}{2m} \frac{d^2 u}{dr^2} + [ V + \frac{\hbar^2}{2m} \frac{\ell (\ell+1)}{r^2}]u = E u\\$


\hdashrule[0.5ex][c]{\linewidth}{0.5pt}{1.5mm}


\underline{Note:} $V_{eff} = V + \frac{\hbar^2}{2 m} \frac{\ell ( \ell + 1)}{r^2}$ (effective potential)\\


\hdashrule[0.5ex][c]{\linewidth}{0.5pt}{1.5mm}

$\star$
\item \underline{$E_n = - [ \frac{m_e}{2 \hbar^2} ( \frac{e^2}{4 \pi \epsilon_0})^2] \frac{1}{n^2} \frac{E_1}{n^2},\,\, n= 1,2,3, \dots$}\\
(Hydrogen)\\
\item \underline{$\frac{d^2 u}{d \rho^2} = [ 1- \frac{\rho_0}{\rho} + \frac{\ell (\ell + 1)}{\rho^2}]u$}\\
\underline{recall:} $- \frac{\hbar^2}{2m} \frac{d^2 u}{dr^2} + [ V + \frac{\hbar^2}{2m} \frac{\ell (\ell+1)}{r^2}] u = E u\\
V(r) = - \frac{e^2}{4 \pi \epsilon_0} \frac{1}{r}$ ( Potential energy/ not potential)\\
$\implies - \frac{\hbar^2}{2m} \frac{d^2 u}{dr^2} + [ - \frac{e^2}{4 \pi \epsilon_0} \frac{1}{r} + \frac{\hbar^2}{2 m_e} \frac{\ell(\ell+1)}{r^2}] u = Eu\\
\kappa \equiv  \frac{\sqrt{-2 m_e E}}{\hbar}\\
\implies \frac{1}{ \kappa^2} \frac{d^2 u}{dr^2} = [ 1- \frac{m_e e^2}{2 \pi \epsilon_0 \hbar \kappa} \frac{1}{(\kappa r)} + \frac{\ell (\ell +1)}{(\kappa r)^2}]u\\
\rho \equiv \kappa r\,\, \rho_0 \equiv \frac{m_e e^2}{2 \pi \epsilon_0 \hbar^2 \kappa}\\
\implies \frac{d^2 u}{d \rho^2} = [ 1- \frac{\rho_0}{\rho} + \frac{\ell (\ell + 1)}{\rho^2}]u$\\


\hdashrule[0.5ex][c]{\linewidth}{0.5pt}{1.5mm}


\item need $E>V_{\rm{min}}$\\


\hdashrule[0.5ex][c]{\linewidth}{0.5pt}{1.5mm}


\item \underline{$u(\rho) \sim C \rho^{\ell+1}$}\\
$\rho \rightarrow \infty \implies \frac{d^2 u}{d \rho^2} \rightarrow u\\
\implies \frac{d^2 u}{d \rho^2} =u \implies r^2 = 1 \implies r = \pm 1 \implies u (\rho) = A e^{- \rho} + B e^{\rho}\\
\rho \rightarrow \infty \implies e^{\rho} \rightarrow \infty \implies B=0\\
\implies u(\rho) \sim A e^{-\rho},\,\, \rho \rightarrow 0\\
\implies \frac{d^2 u}{d \rho^2} \approx \frac{\ell(\ell + 1)}{\rho^2} u \implies u(\rho) = \rho^m \implies u' = m \rho^{m-1},\,\, u'' = m(m-1) \rho^{m-2}\\
\implies m(m-1) \rho^{m-2} = \ell(\ell + 1) \rho^{m-2}\\
\implies m^2 - m - \ell(\ell+1) = 0 \implies m = \frac{1 \pm \sqrt{1 + 4 \ell(\ell + 1)}}{2}\\
= \frac{1 \pm \sqrt{4 \ell^2 + 4 \ell + 1}}{2} = \frac{1 \pm 2 \ell + 1}{2} = \ell + 1,\,\, - \ell\\
\implies u(\rho) = C \rho^{\ell + 1} + D \rho^{-\ell}\\$
but $\rho \rightarrow 0 \implies D \rho^{- \ell} \rightarrow \infty \implies D=0\\
\implies u(\rho) \sim C \rho^{\ell+1}\\$
our solution new looks like $u(\rho) \sim \rho^{\ell + 1} e^{- \rho}$\\
but this is only accurate for large and small $\rho$, so lets tack on a new function to force it to become accurate in the middle


\hdashrule[0.5ex][c]{\linewidth}{0.5pt}{1.5mm}\\


\item \underline{$\rho \frac{d^2 v}{d \rho^2} + 2 ( \ell + 1 - \rho) \frac{dv}{d \rho} + [ \rho_0 - 2 ( \ell+1)] v = 0$}\\
$u(\rho) = \rho^{\ell + 1} e^{- \rho} v (\rho)\\
\frac{d u}{d \rho} = ( \ell + 1) \rho^{\ell} e^{- \rho} v(\rho) - \rho^{\ell + 1} e^{- \rho} v(\rho) + \rho^{\ell+1} e^{- \rho} v'(\rho)\\
= \rho^{\ell} e^{- \rho} [ ( \ell + 1 - \rho) v + \rho \frac{dv}{d \rho}]\\
\frac{d^2 u}{d \rho^2} = \rho^{\ell} e^{- \rho} \{ [ - 2 \ell - 2 + \rho + \frac{\ell(\ell+1)}{\rho} ] v + 2( \ell + 1 - \rho) \frac{dv}{d\rho} + \rho \frac{d^2 v}{d \rho^2}\}\\
\frac{d^2 u}{d \rho^2} = [ 1- \frac{\rho_0}{\rho} + \frac{\ell(\ell + 1)}{\rho^2}]u\\
\implies \rho^{\ell} e^{- \rho} \{ [ - 2 \ell - 2 + \rho + \frac{\ell(\ell+1)}{\rho}] v + 2(\ell + 1 - \rho) \frac{dv}{d\rho} + \rho \frac{d^2 v}{d \rho^2}\}\\
= [ 1- \frac{\rho_0}{\rho} + \frac{\ell(\ell + 1)}{\rho^2}] \rho^{\ell + 1} e^{- \rho} v(\rho)\\
\implies [ -2 \ell - 2 + \rho + \frac{\ell(\ell+1)}{\rho} ] v + 2 ( \ell + 1 - \rho) v' + \rho v''\\
= [ \rho - \rho_0 + \frac{\ell(\ell + 1)}{\rho} ] v(\rho)\\
\implies [ -2 \ell -2 + \rho_0] v + 2(\ell + 1 - \rho) v' + \rho v''=0\\
\implies \rho \frac{d^2 v}{d \rho^2} + 2 ( \ell + 1 - \rho) \frac{dv}{d \rho} + [ \rho_0 - 2 ( \ell+1)] v = 0\\$


\hdashrule[0.5ex][c]{\linewidth}{0.5pt}{1.5mm}\\

\item \underline{$ c_{j+1} = \frac{2 j}{j(j+1)} = \frac{2}{j+1} c_j$}\\
assume $v(\rho) = \sum_{j=0}^{\infty} c_j \rho^j\\
\implies \frac{dv}{d \rho} = \sum_{j=0}^{\infty} c_j j \rho^{j-1} = \sum_{j=1}^{\infty} c_j j \rho^{j-1} = \sum_{j=0}^{\infty} c_{j+1} (j+1) \rho^j\\
\implies \frac{d^2 v}{d \rho^2} = \sum_{j=0}^{\infty} c_{j+1}(j+1) j \rho^{j-1}\\
\implies \sum_{j=0}^{\infty} c_{j+1}(j+1) j \rho^j + 2 ( \ell +1 - \rho)(\sum_{j=0}^{\infty} c_{j+1}(j+1) \rho^j) + [ \rho_0 - 2 ( \ell + 1)] \sum_{j=0}^{\infty} c_j \rho^j = 0\\
\implies \sum_{j=0}^{\infty} [ c_{j+1} (j+1) j + 2 (\ell + 1) c_{j+1} (j+1)\\
-2 c_j j + c_j \rho_0 - 2 (\ell + 1) c_j ] \rho^j =0\\
\implies j( j+1) c_{j+1} + 2 (\ell +1)(j+1) c_{j+1} - 2 j c_j + [ \rho_0 - 2(\ell + 1)] c_j = 0\\
\implies c_{j+1} = \frac{2 (j + \ell +1) - \rho_0}{(j+1)(j+2 \ell + 2)} c_j\\$
large $j\\
\implies c_{j+1} = \frac{2 j}{j(j+1)} = \frac{2}{j+1} c_j\\$


\hdashrule[0.5ex][c]{\linewidth}{0.5pt}{1.5mm}\\

\item \underline{$2 n = \rho_0$}
not dropping j+1 makes it cleaner.\\
$c_1 = 2 c_0\\
c_2 = \frac{2}{2} c_1 = 2 c_0\\
c_33 = \frac{2}{3} c_2 = \frac{2}{3} 2 c_0\\
c_4 = \frac{2}{4} c_3 = \frac{2 \cdot 2 \cdot 2 \cdot 2}{ 4 \cdot 3 \cdot 2 \cdot 1} c_0\\
\implies c_j \approx \frac{2^j}{j!} c_0\\
\implies v(\rho) = \sum_{j=0}^{\infty} c_j \rho^j = c_0 \sum_j \frac{2^j}{j!} \rho^j = c_0 \sum_{j=0}^{\infty} \frac{(2 \rho)^j}{j!}\\
= c_0 e^{2 \rho}\\
\implies u(\rho) = \rho^{\ell + 1} e^{- \rho} v(\rho) = c_0 \rho^{\ell + 1} e^{\rho} \implies$ blows up for large $\rho\\
\implies$ series must terminate $\implies c_{N-1} \neq 0$ but $c_N = 0\\
c_{(N-1) + 1} = c_N = \frac{2(N-1 + \ell + 1) - \rho_0}{(N-1 + 1)(N-1 + 2 \ell + 2)} c_{N-1}\\
= \frac{2 (N + \ell) - \rho_0}{N(N + 2 \ell + 1)} c_{N-1} = 0\\
\implies 2(N + \ell) - \rho_0 = 0 ;\,\, n \equiv N + \ell\\
\implies 2 n = \rho_0\\$


\hdashrule[0.5ex][c]{\linewidth}{0.5pt}{1.5mm}\\


\item \underline{$\rho \equiv \kappa r = \frac{r}{an}$}\\
\underline{recall:} $\rho \equiv \frac{m_e e^2}{2 \pi \epsilon_0 \hbar^2 \kappa} ;\\,\, \kappa \equiv \frac{\sqrt{-2 m_e E}}{\hbar}\\
\implies \rho_0 = \frac{m_e e^2 \hbar}{2 \pi \epsilon_0 \hbar^2 \sqrt{-2 m_e E}} = 2 n\\
\therefore E_n=\implies ( \frac{m_e e^2}{4 \pi \epsilon_0 \hbar n})^2 \frac{1}{-2 m_e} \\
\implies - [ \frac{m_e}{2 \hbar^2} ( \frac{e^2}{4 \pi \epsilon_0})^2] \frac{1}{n^2} = \frac{E_1}{n^2},\,\, n=1,2,3 \dots$\\
$\kappa = \frac{\sqrt{-2 m_e E_n}}{\hbar} =\frac{1}{\hbar} \sqrt{ \frac{2 m_e^2}{2 \hbar^2}( \frac{e^2}{4 \pi \epsilon_0})^2} \frac{1}{n}\\
= \frac{m_e}{\hbar^2} \frac{e^2}{4 \pi \epsilon_0} \frac{1}{n} = \frac{1}{an};\,\, a = \frac{4 \pi \epsilon_0 \hbar^2}{m_e e^2} = 0.529 E -10 m\\$
(Bohr radius)\\
$\implies \rho \equiv \kappa r = \frac{r}{an}\\$


\hdashrule[0.5ex][c]{\linewidth}{0.5pt}{1.5mm}


$\psi_{n \ell m} (r, \theta, \phi) = R_{n \ell} (r) Y_{\ell}^m (\theta, \phi)\\$
where $R_{n \ell} (r) = \frac{1}{r} \rho^{\ell + 1} e^{- \rho} v(\rho)\\
E_1 = - 13.6 eV$ (ground state)\\
\underline{recall:} $c_{j+1} = \frac{2(j + \ell + 1 - n)}{(j+1)(j+2 \ell + 2)} c_j\\
R_{10} (r) = \frac{c_0}{a} e^{-r/a}\\
\implies \int_0^{\infty} | R_{10}|^2 r^2 dr = 1 \implies c_0 = \frac{2}{\sqrt{a}}$


\hdashrule[0.5ex][c]{\linewidth}{0.5pt}{1.5mm}


\item \underline{$\ell = 0,1,2, \dots, n-1$}\\
\underline{recall:} $\rho_0 = 2n = \frac{m_e e^2}{2 \pi \epsilon_0 \hbar^2 \kappa} >0\\
n \equiv N + \ell \implies \\$
\underline{recall:} $\Theta(\theta) = A P^m_{\ell} (\cos \theta),\,\, P_{\ell} \sim ( \frac{d}{dx})^{\ell} \implies \ell \geq 0\\$\\
and $\ell \in \mathbb{Z}$\\
\underline{Note:} $\ell = n - N$, so the largest this could possibly be occurs when $N=1 \implies \ell<= n-1\\
\therefore \ell = 0,1,2, \dots, n-1\\$


\hdashrule[0.5ex][c]{\linewidth}{0.5pt}{1.5mm}


\item \underline{$\sum_{j=1}^n j = \frac{n(n+1)}{2}$}\\
$S_n = 1 + 2 + 3 + \cdots + n\\
S_n = n + ( n-1) + ( n-2) + \cdots + 1\\
\implies 2 S_n = ( n+1) + (n+1) + \cdots + ( n+1) = n ( n+1)\\
\implies S_n = \frac{n(n+1)}{2}$


\hdashrule[0.5ex][c]{\linewidth}{0.5pt}{1.5mm}


\item \underline{$d(n) = n^2$} (degeneracy of $E_n$)\\
$d(n) = \sum_{\ell= 0}^{n-1} ( 2 \ell + 1)$ (for each value of $\ell$ there are $2 \ell + 1$ possible values of $m$)\\
$d(n) = 2 \sum_{\ell = 0}^{n-1} \ell + n\\
\underline{recall:} \sum_{\ell=0}^{n-1} \ell= \frac{(n-1)n}{2}\\
\implies d(n) = 2 \frac{(n-1)n}{2} + n = n^2\\$


\hdashrule[0.5ex][c]{\linewidth}{0.5pt}{1.5mm}


\item \underline{$\psi_{n \ell m } = \sqrt{( \frac{2}{na})^3 \frac{(n- \ell - 1)!}{2 n ( n + \ell)!}} e^{-r/na} ( \frac{2 r}{na})^{\ell} [ L_{n-\ell -1}^{2 \ell + 1} ( 2r/na)] Y^m_{\ell} ( \theta, \phi)$}\\
\underline{recall:} $\psi ( r, \theta, \phi ) = R(r) Y^m_{\ell} ( \theta, \phi)\\$
also $v( \rho) = \sum_{j=0}^{\infty} c_j \rho^j,\,\, w/ c_{j+1} = \frac{2( j + \ell + 1 - n)}{(j + 1) ( j + 2 \ell + 2)} c_j\\
\implies v(\rho) = L_{n- \ell - 1}^{2 \ell + 1} (2 \rho )$ where $L_q^p \equiv ( -1)^p ( \frac{d}{dx})^p L_{p+q} ( x)\\$
and $L_q(x) \equiv \frac{e^x}{q!} ( \frac{d}{dx})^q ( e^{-x} x^q)$ (Laguerre polynomial)\\
$\implies R(\rho) = \frac{u(\rho)}{\rho} = \frac{\rho^{\ell + 1} e^{- \rho} v(\rho)}{\rho} = \rho^{\ell} e^{- \rho} v( \rho)\\
\rho = \frac{r}{an}\\
\implies \psi_{n \ell m} = ( \frac{r}{an})^{\ell} e^{- r/na} L_{n- \ell - 1}^{2 \ell + 1} ( \frac{2r}{na}) Y_{\ell}^n ( \theta, \phi) N$
If normalization is calculated we get above\\


\hdashrule[0.5ex][c]{\linewidth}{0.5pt}{1.5mm}


\underline{Note:} $\int \psi_{n \ell m}^* \psi_{n' \ell' m'} r^2 dr d \Omega = \delta n n' \delta_{\ell \ell'} \delta_{m m'}$


\hdashrule[0.5ex][c]{\linewidth}{0.5pt}{1.5mm}


$E_{\gamma} = E_i - E_f = - 13.6 e V ( \frac{1}{n_i^2} - \frac{1}{n_f^2})$


\hdashrule[0.5ex][c]{\linewidth}{0.5pt}{1.5mm}


$\hat{\vec{L}} = \vec{r} \times \vec{p} = L_x \hat{x} + L_y \hat{y} + L_z \hat{z}= ( y p_z - z p_y) \hat{x} + ( z p_x - x p_z) \hat{y} + (xp_y - y p_x) \hat{z}\\$


\hdashrule[0.5ex][c]{\linewidth}{0.5pt}{1.5mm}


\item \underline{$[L_x, L_y] = i \hbar L_z;\,\, [ L_y, L_z] = i \hbar L_x;\,\, [L_z, L_x] = i \hbar L_y$}\\
$[L_x, L_y ] = [ y p_z - z p_y, z p_x - x p_z ]\\
= [ y p_z, z p_x - x p_z] - [ z p_y, zp_x - xp_z]\\
= [y p_z, z p_x] - [ y p_z, x p_z] - [ zp_y, z p_x] + [ z p_y, x p_z]\\
\underline{2nd term:} y p_z x p_z - x p_z y p_z = y x p_z^2 - xy p_z^2 = 0\\
\underline{3rd term:} z p_y z p_x - z p_x z p_y = z^2 ( p_y p_x - p_x p_y) = 0\\
\implies [ L_x, L_y ] = [ y p_z, z p_x] + [ z p_y, x p_z]\\
\underline{recall:} [A B, C] = A [ B, C] + [ A, C] B\\
\implies [ y p_z, z p_x] = y [ p_z, z p_x] + [ y , zp_x] p_z\\
= - y [ z p_x, p_z ] - [z p_x, y ] p_z\\
{[z p_x, p_z ]} = z [ p_x, p_z] + [ z, p_z] p_x = [ z, p_z] p_x\\
{[z p_x, y ]} = z [ p_x, y ] + [ z, y] p_x = z[ p_x, y]\\
\implies [ y p_z, z p_x] = - y [ z, p_z ] p_x - z [ p_x, y ] p_z\\$
$[z p_y, x p_z ] = z [ p_y, x p_z ] + [ z, x p_z] p_y\\
= - z[ x p_z, p_y] - [ x p_z, z] p_y\\
{[xp_z, p_y]} = x [p_z, p_y] + [ x , p_y] p_z = [ x, p_y ] p_z\\
{[x p_z, z]} = x [ p_z, z] + [ x, z] p_z = x [ p_z, z]\\
\implies [ z p_y , x p_z] = -z [ x, p_y] p_z - x [ p_z, z] p_y\\
\implies [ L_x, L_y] = - y p_x [ z, p_z] - z p_z [p_x, y]\\
- z p_z [ x, p_y] - x p_y [p_z, z]\\
= - y p_x i \hbar + z p_z i \hbar - z p_z i \hbar + x p_y i \hbar\\
= ( x p_y - y p_x ) i \hbar = i \hbar L_z\\
\implies [ L_x, L_y ] = i \hbar L_z\\$
permute! $x \rightarrow y, y \rightarrow z, z \rightarrow x\\
\implies [ L_y, L_z] = i \hbar L_x\\
y \rightarrow x,\,\, x \rightarrow z\,\, z \rightarrow y\\
\implies [ L_z, L_x] = i \hbar L_y$


\hdashrule[0.5ex][c]{\linewidth}{0.5pt}{1.5mm}


\item \underline{$[r_i, p_j] = i \hbar \delta_{ij};\,\, [r_i, r_j ] = [p_i, p_j ] = 0$}\\
$(r_i p_j - p_j r_i) \psi = (- i \hbar r_i \frac{\partial}{\partial x^j} + i \hbar \frac{\partial}{\partial x^j} (r_i) \psi\\
= i \hbar \frac{\partial r_i}{\partial x^j} \psi + i \hbar r_i \frac{\partial \psi}{\partial x^j} - i \hbar r_i \frac{\partial}{\partial x^j} \psi = i \hbar \delta_{ij} \psi\\
{[ r_i, r_j]} = 0$ follows since $xy = yx\\
{[ p_i, p_j]} = 0$ follows since  $\frac{\partial}{\partial x} \frac{\partial }{\partial y} = \frac{\partial}{\partial y} \frac{\partial}{\partial x}\\
$


\hdashrule[0.5ex][c]{\linewidth}{0.5pt}{1.5mm}


\item \underline{$[ L^2, \hat{\vec{L}}] = 0$}\\
$[L^2, L_x] = [ L_x^2, L_x] + [L_y^2, L_x] + [L_z^2, L_x]\\
=[L_y^2, L_x] + [L_z^2,L_x]\\
=L_y [L_y, L_x] + [L_y, L_x] L_y + L_z [L_z, L_x] + [L_z, L_x] L_z\\
= L_y (- i \hbar L_z) + ( - i \hbar L_z ) L_y + L_z ( i \hbar L_y) + ( i \hbar L_y) L_z\\
=0\\$


\hdashrule[0.5ex][c]{\linewidth}{0.5pt}{1.5mm}


\underline{Note:} $\sigma_{L_x}^2 \sigma_{L_y}^2 \geq ( \frac{1}{2 i} \langle i \hbar L_z \rangle )^2 = \frac{\hbar^2}{4} \langle L_z \rangle^2\\
\implies \sigma_{L_x} \sigma_{L_y} \geq \frac{\hbar}{2} | \langle L_z \rangle |\\
\implies L_x, L_y, L_z$ in compatible observables\\


\hdashrule[0.5ex][c]{\linewidth}{0.5pt}{1.5mm}


$[L^2, \hat{\vec{L}} ] = 0 \implies L^2 f = \lambda f;\,\, L_z f = \mu f\\$


\hdashrule[0.5ex][c]{\linewidth}{0.5pt}{1.5mm}


\item \underline{$L^2 ( L_{\pm} f) = \lambda( L_{\pm} f);\,\, L_z( L_{\pm f}) = ( \mu \pm \hbar)(L_{\pm} f)$}\\
$L_{\pm} = L_x \pm i L_y\\$
\underline{aside:} $[ L_z, L_{\pm}] = [ L_z, L_x] \pm i [ L_z, L_y]\\
=i \hbar L_y \pm i (-i \hbar L_x) = \pm \hbar (L_x \pm i L_y)\\
= \pm \hbar L_{\pm}\\
\implies L^2 ( L_{\pm} f) = ( L_{\pm} L^2 f)\\
= L_{\pm} ( \lambda f) = \lambda( L_{\pm} f)\\
L_z ( L_{\pm f}) = ( L_z L_{\pm} - L_{\pm} L_z) f + L_{\pm} L_z f\\
= [ L_z , L_{\pm} ] f + \mu ( L_{\pm} f)\\
= \pm \hbar( L_{\pm} f) + \mu ( L_{\pm} f) = ( \mu \pm \hbar) (L_{\pm} f)$


\hdashrule[0.5ex][c]{\linewidth}{0.5pt}{1.5mm}


\item \underline{$\lambda= \hbar^2 \ell(\ell +1)$}\\
eventually $L_z (L_+ f) = \beta f\\$
where $\beta > L$ but $\beta = L_z < L\\
\implies \exists f_t s.t. L_+ f_t = 0$ ( top rung)\\
$L_z f_t = \hbar \ell f_t,\,\, L^2 f_t = \lambda f_t\\
L_{\pm} L_{\mp} = ( L_x \pm i L_y )(L_x \mp i L_y)\\
= L_x^2 \mp i L_x L_y \pm i L_y L_x + L_y^2\\
L_x^2 + L_y^2 \mp i ( L_x L_y - L_y L_x)\\
= L^2 - L_z^2 \mp i ( i \hbar L_z)\\
\implies L^2 f_t = ( L_- L_+ + L_z^2 + \hbar L_z) f_t\\
= 0+\hbar^2 \ell^2 + \hbar^2 \ell = \hbar^2 \ell( \ell+ 1)\\$


\hdashrule[0.5ex][c]{\linewidth}{0.5pt}{1.5mm}


\underline{ $\lambda = \hbar^2 \bar{\ell} ( \bar{\ell} - 1)$}\\
there is also a bottom rung for the same reason\\\
$L_- f_b = 0\\
\implies L_z f_b = \hbar \bar{\ell} f_b;\,\, L^2 f_b = \lambda f_b\\
L^2 f_b = ( L_+ L_- + L_z^2 - \hbar L_z ) f_b = \hbar^2 \bar{\ell}^2 - \hbar^2 \bar{\ell} = \hbar \bar{\ell} ( \bar{\ell} - 1) f_b\\
\therefore \lambda = \hbar^2 \bar{\ell}(\bar{\ell} - 1)\\$


\hdashrule[0.5ex][c]{\linewidth}{0.5pt}{1.5mm}


\item \underline{$\bar{\ell} = - \ell$}\\
$\lambda = \ell( \ell + 1) \hbar^2 = \hbar^2 \bar{\ell} ( \bar{\ell} (\bar{\ell} -1)\\
\implies \bar{\ell}^2 - \bar{\ell} - \ell( \ell + 1)\\
\implies \frac{1 \pm \sqrt{1 + 4 \ell( \ell + 1)}}{2} = \frac{1 \pm \sqrt{4 \ell^2 + 4 \ell + 1}}{2}\\
\frac{1 \pm 2 \ell + 1}{2} = \ell + 1,\,\, \ell \\
\bar{\ell} = \ell + 1$ ( bottom rung cant be higher than top rung)\\
$\therefore \bar{\ell} = - \ell\\$


\hdashrule[0.5ex][c]{\linewidth}{0.5pt}{1.5mm}


\item \underline{$L^2 f_{\ell}^m = \hbar^2 \ell( \ell + 1) f^m_{\ell};\,\, L_z f^m_{\ell} = \hbar m f^m_{\ell}$}\\
$L^2 f^m_{\ell} = \lambda f^m_{\ell} = \hbar^2 \ell( \ell+1);\,\,\\$
bottom rung $-\hbar \ell$, top rung $\hbar \ell$ increases in units of $\hbar \implies L_z f^m_{\ell} = \hbar m f^m_{\ell}\\
m= - \ell, - \ell + 1, \dots, \ell -1 , \ell\\$
$m$ goes from $- \ell$ to $\ell$ in $N$ integer steps $\implies \ell = - \ell + N \implies \ell = \frac{N}{2},\,\, N \geq 0\\
\implies \ell = 0, \frac{1}{2}, 1, \frac{3}{2}, \dots\\$

\hdashrule[0.5ex][c]{\linewidth}{0.5pt}{1.5mm}


\hdashrule[0.5ex][c]{\linewidth}{0.5pt}{1.5mm}


\item \underline{$L_z = - i \hbar \frac{\partial}{\partial \phi}$}\\
\underline{recall:} $\vec{L} = \vec{r} \times \hat{p} = - i \hbar ( \vec{r} \times \nabla)\\
\nabla = \frac{\partial}{\partial r} \hat{r} + \frac{1}{r} \frac{\partial}{\partial \theta} \hat{\theta} + \hat{\phi} \frac{1}{r \sin \theta} \frac{\partial}{\partial \phi}\\
\vec{r} \times \nabla = r( \hat{r} \times ( \hat{r} \frac{\partial}{\partial r} + \hat{\theta} \frac{1}{r} \frac{\partial}{\partial \theta} + \hat{\phi} \frac{1}{r \sin \theta} \frac{\partial \phi}))\\
= r( \hat{r} \times \hat{r}) \frac{\partial}{\partial r} + \hat{r} \times \hat{\theta} \frac{\partial}{\partial \theta} + \frac{\hat{r} \times \hat{\phi}}{\sin \theta} \frac{\partial}{\partial \phi}\\
\hat{r} \times \hat{\phi} = - \hat{\theta},\,\, \hat{r} \times \hat{\theta} = \hat{\phi}\\
\implies \hat{\vec{L}} = - i \hbar ( \hat{\phi} \frac{\partial}{\partial \theta} - \hat{\theta} \frac{1}{\sin \theta} \frac{\partial}{\partial \phi})\\
\vec{e}_{\alpha'} = \Lambda^{\alpha}_{\alpha'} \vec{e}_{\alpha},\,\, \hat{e}_{\alpha' } = \frac{\vec{e}_{\alpha'}}{|\vec{e}_{\alpha '} |}\\
\implies \hat{\theta} = ( \cos \theta \cos \phi) \hat{i} + ( \cos \theta \sin \phi) \hat{j} - ( \sin \theta) \hat{k}\\
\hat{\phi} = - ( \sin \phi) \hat{i} + ( \cos \phi) \hat{j}\\
\implies \begin{cases} 
L_x = - i \hbar( - \sin \phi \frac{\partial}{\partial \theta} - \cos \phi \cot \theta \frac{\partial}{\partial \phi})\\
L_y = - i \hbar( \cos \phi \frac{\partial}{\partial \theta} - \sin \phi \cot \theta \frac{\partial}{\partial \phi})\\
L_z = - i \hbar \frac{\partial}{\partial \phi}
\end{cases}\\$











$L_{\pm} = L_X \pm i L_y = - i \hbar [(- \sin \phi \pm i \cos \phi) \frac{\partial}{\partial \theta} - ( \cos \phi \pm i \sin \phi)\cot \theta \frac{\partial}{\partial}]\\
= \pm \hbar e^{\pm i \phi} ( \frac{\partial}{\partial \theta} \pm i \cot \theta \frac{\partial}{\partial \phi})\\
L_+ L_- = - \hbar^2(\frac{\partial^2}{\partial \theta^2} + \cot \theta \frac{\partial}{\partial \theta} + \cot^2 \theta \frac{\partial^2}{\partial \phi^2} + i \frac{\partial}{\partial \phi})\\$


\hdashrule[0.5ex][c]{\linewidth}{0.5pt}{1.5mm}


\item \underline{$L^2 = - \hbar^2[ \frac{1}{\sin \theta} \frac{\partial}{\partial \theta} ( \sin \theta \frac{\partial}{\partial \theta}) + \frac{1}{\sin^2 \theta} \frac{\partial^2}{\partial \phi^2} ]$}\\
\underline{recall:} $L_+ L_- = L^2 - L_z^2 - i ( i \hbar L_z)\\
\implies L^2 = L_+ L_- + L_z^2 + i ( i \hbar L_z)\\$
plug in to get result\\


\hdashrule[0.5ex][c]{\linewidth}{0.5pt}{1.5mm}


\item \underline{$f^m_{\ell} = Y^m_{\ell}$}\\
\underline{recall:} $L^2 f^m_{\ell} = - \hbar^2 [ \frac{1}{\sin \theta} \frac{\partial}{\partial \theta} ( \sin \theta \frac{\partial}{\partial \theta}) + \frac{1}{\sin^2 \theta} \frac{\partial^2}{\partial \phi^2}]f^m_{\ell}\\\
= \hbar^2 \ell( \ell + 1) f^m_{\ell} ;\,\, L_z f^m_{\ell} = - i \hbar \frac{\partial}{\partial \-phi} f^m_{\ell} = \hbar m f^{m}_{\ell}\\$
the first equation is the angular equationf or $Y^m_{\ell}\\
\implies f^m_{\ell} = Y^m_{\ell}\\$
\underline{Note:} $f^m_{\ell} = \Phi \Theta \implies - i \hbar \Theta \frac{\partial}{\partial \phi} \Phi = \hbar m \Phi \Theta\\
\implies - i \hbar \frac{\partial \Phi}{\partial \phi} = \hbar m \Phi,\,\,$ i.e. we must solve the first equation to obtain full solution.\\


\hdashrule[0.5ex][c]{\linewidth}{0.5pt}{1.5mm}


$\implies H$ has simultaneous eigen functions with $L^2$ and $L_z$\\
$\implies H \psi = E \psi,\,\, L^2 \psi = \hbar^2 \ell( \ell + 1) \psi,\,\, L_z \psi = \hbar m \psi\\$


\hdashrule[0.5ex][c]{\linewidth}{0.5pt}{1.5mm}


\item \underline{$\frac{1}{2 m r^2} [ - \hbar^2 \frac{\partial}{\partial r} (r^2 \frac{\partial}{\partial r}) + L^2] \psi + V \psi = E \psi$}\\
\underline{recall:} $H \psi = - \frac{\hbar^2}{2 m} [ \frac{1}{r^2} \frac{\partial}{\partial r} ( r^2 \frac{\partial \psi}{\partial r}) + \frac{1}{r^2 \sin \theta} \frac{\partial}{\partial \theta} ( \sin \theta \frac{\partial \psi}{\partial \theta}) + \frac{1}{r^2 \sin^2 \theta}( \frac{\partial^2 \psi}{\partial \phi^2})] + V \psi = E \psi;\,\, L^2 = - \hbar^2 [ \frac{1}{ \sin \theta} \frac{\partial}{\partial \theta} ( \sin \theta \frac{\partial}{\partial \theta}) + \frac{1}{\sin^2 \theta} \frac{\partial^2}{\partial \phi^2}]\\
\implies H \psi = \frac{1}{2 m r^2} [ - \hbar^2 \frac{\partial}{\partial r} ( r^2 \frac{\partial \psi}{|partial r}) - \hbar^2( \frac{1}{\sin \theta} \frac{\partial}{|partial \theta} ( \sin \theta \frac{\partial \psi}{\partial \theta})\\
+ \frac{1}{ \sin^2 \theta} ( \frac{\partial^2 \psi}{\partial \phi^2}) ] + V \psi = E \psi\\
\therefore H \psi = \frac{1}{2 m r^2} [ - \hbar^2 \frac{\partial}{\partial r} ( r^2 \frac{\partial \psi}{\partial r}) + L^2 \psi ] + V \psi = E \psi\\$


\hdashrule[0.5ex][c]{\linewidth}{0.5pt}{1.5mm}


\underline{Note:} algebraic theory of angular momentum permits $\ell$, m to be half integer while separation of variables method only allows integer values ( strange) gthese half integers are also important.\\


\hdashrule[0.5ex][c]{\linewidth}{0.5pt}{1.5mm}


(Spin) $\sim \vec{S} = I \vec{\omega};\,\,$ ( orbital) $\sim \vec{L} = \vec{r} \times \vec{p}$


\hdashrule[0.5ex][c]{\linewidth}{0.5pt}{1.5mm}


Since spin in QM is not a classical concept ( i.e. an electron can have spin even though it is not rotatiing). We take the algebraic theory of spin to be identical to the theory of L (except eigenfunctions are now eigenvectors)\\
$\implies [S_x, S_y]= i \hbar S_z,\,\, [S_y, S_z] = i \hbar S_x,\,\, [ S_z, S_x] = i \hbar S_y\\
S^2 |s m \rangle = \hbar^2 s( s+1) |s m \rangle,\,\, S_z | s m \rangle = \hbar m | s m \rangle\\
S_{\pm} | s m \rangle = \hbar \sqrt{ s(s + 1) - m_s ( m_s \pm 1) } | s ( m \pm 1 ) \rangle\\
s= 0, \frac{1}{2}, 1, \frac{3}{2}, \dots ;\,\, m_s = - s, -s + 1, \dots , s-1, s\\$
each elementary particle has a specific value of s but can take any value of $\ell$, allowed.


\hdashrule[0.5ex][c]{\linewidth}{0.5pt}{1.5mm}


\item \underline{$| \frac{1}{2} \frac{1}{2} \rangle ( spin up),\,\, | \frac{1}{2} ( - \frac{1}{2} )\rangle$ (spin down)}\\
if $s= \frac{1}{2}$ then $m= - \frac{1}{2}, \frac{1}{2}$ and there are two possible eigenvectors: $| s m \rangle = | \frac{1}{2} \rangle, | \frac{1}{2} ( - \frac{1}{2} )\rangle\\$
$| s m \rangle$ is a $2 s+ 1$ dimensional vector\\
$\chi_+ = | \frac{1}{2} \frac{1}{2} \rangle = \begin{pmatrix} 1 \\ 0 \end{pmatrix};\,\, \chi_- = | \frac{1}{2} ( - \frac{1}{2} \rangle = \begin{pmatrix} 0 \\ 1 \end{pmatrix}\\
\chi = \begin{pmatrix} a \\ b \end{pmatrix} = a \chi_+ + b \chi_-$ (spin state $\sim$ different from $\psi$)\\
the full state looks like $\psi( \vec{r}) \chi\\$

\hdashrule[0.5ex][c]{\linewidth}{0.5pt}{1.5mm}


\item \underline{$S^2 = \frac{3}{4} \hbar^2 \begin{pmatrix} 1 & 0 \\ 0 & 1 \end{pmatrix}$}\\
$S^2 \chi_+ = S^2 | \frac{1}{2} \frac{1}{2} \rangle = \hbar^2 \frac{1}{2} ( \frac{3}{2}) | \frac{1}{2} \frac{1}{2} \rangle = \hbar^2 \frac{3}{4} \chi_+\\
S^2 \chi_- = \hbar^2 \frac{3}{4} \chi_-\\
S^2 = \begin{pmatrix} c & d \\ e & f \end{pmatrix} \implies S^2 \chi_+ = \begin{pmatrix} c \\ e \end{pmatrix} = \hbar^2 \frac{3}{4} \begin{pmatrix} 1 \\ 0 \end{pmatrix}\\
\implies c = \frac{3}{4} \hbar^2,\,\, e = 0\\
S^2 \chi_- = \begin{pmatrix} d \\ f \end{pmatrix} = \hbar^2 \frac{3}{4} \begin{pmatrix} 0 \\ 1 \end{pmatrix} \implies d = 0 ,\,\, f = \frac{3}{4} \hbar^2\\
\implies S^2 = \frac{3}{4} \hbar^2 \begin{pmatrix} 1 & 0 \\ 0 & 1 \end{pmatrix}\\$


\hdashrule[0.5ex][c]{\linewidth}{0.5pt}{1.5mm}


\item \underline{$S_z = \frac{\hbar}{2} \begin{pmatrix} 1 & 0 \\ 0 & -1 \end{pmatrix}$}\\
$S_z \chi_+ = S_z | \frac{1}{2} \frac{1}{2} \rangle = \frac{\hbar}{2} | \frac{1}{2} \frac{1}{2} \rangle = \frac{\hbar}{2} \chi_+\\
S_z \chi_- = - \frac{\hbar}{2} \chi_-\\
S_z = \begin{pmatrix} a & b \\ c & d \end{pmatrix}\\
S_z \chi_+ = \begin{pmatrix} a \\ c \end{pmatrix} = \frac{\hbar}{2} \begin{pmatrix} 1 \\ 0 \end{pmatrix} \implies a = \frac{\hbar}{2},\,\, c = 0\\
S_z \chi_- = \begin{pmatrix} b \\ d \end{pmatrix} = - \frac{\hbar}{2} \begin{pmatrix} 0 \\ 1 \end{pmatrix} \implies b = 0,\,\, d = - \frac{\hbar}{2}\\
\implies S_z = \frac{\hbar}{2} \begin{pmatrix} 1 & 0 \\ 0 & -1 \end{pmatrix}\\
$

\hdashrule[0.5ex][c]{\linewidth}{0.5pt}{1.5mm}


\item \underline{$S_+ = \hbar \begin{pmatrix} 0 &1 \\ 0 & 0 \end{pmatrix},\,\, S_- = \hbar \begin{pmatrix} 0& 0 \\ 1 & 0 \end{pmatrix}$}\\
$S_+ \chi_+ = S_+ | \frac{1}{2} \frac{1}{2} \rangle = \hbar \sqrt{ \frac{3}{4} - \frac{3}{4}} = 0 = S_- \chi_-\\
S_+ \chi_- = S_+ | \frac{1}{2} ( - \frac{1}{2}) \rangle = \hbar \sqrt{ \frac{3}{4} + \frac{1}{4}} | \frac{1}{2} \frac{1}{2} \rangle = \hbar \chi_+\\
S_- \chi_+ = S_- | \frac{1}{2} \frac{1}{2} \rangle = \hbar \sqrt{ \frac{3}{4} - \frac{1}{2} ( - \frac{1}{2})} | \frac{1}{2} ( - \frac{1}{2}) \rangle = \hbar \chi_-\\$
plug in\\
$\implies S_+ = \hbar \begin{pmatrix} 0 & 1 \\ 0 & 0 \end{pmatrix},\,\, S_- = \hbar \begin{pmatrix} 0 & 0 \\ 1 & 0 \end{pmatrix}$


\hdashrule[0.5ex][c]{\linewidth}{0.5pt}{1.5mm}


\item \underline{$S_x = \frac{\hbar}{2} \begin{pmatrix} 0 & 1 \\ 1 & 0 \end{pmatrix},\,\, S_y = \frac{\hbar}{2} \begin{pmatrix} 0 & - i \\ i & 0 \end{pmatrix}$}\\
$S_{\pm} = S_x \pm i S_y \implies \begin{cases} S_+ = S_x + i S_y \\ S_- = S_x - i S_y \end{cases}\\
\implies S_x = \frac{1}{2} ( S_+ + S_- ),\,\, S_y = \frac{1}{2i} ( S_+ - S_-)\\
\implies S_x = \frac{\hbar}{2} \begin{pmatrix} 0 &1 \\ 1 & 0 \end{pmatrix},\,\, S_y = \frac{\hbar}{2} \begin{pmatrix} 0 & - i \\ i & 0 \end{pmatrix}\\$


\hdashrule[0.5ex][c]{\linewidth}{0.5pt}{1.5mm}


$\sigma_x \equiv \begin{pmatrix} 0 & 1 \\ 1 & 0 \end{pmatrix},\,\, \sigma_y \equiv \begin{pmatrix} 0 & - i \\ i & 0 \end{pmatrix},\,\, \sigma_z \equiv \begin{pmatrix} 1 & 0 \\ 0 & -1 \end{pmatrix}\\$
( Pauli spin matrices)\\


\hdashrule[0.5ex][c]{\linewidth}{0.5pt}{1.5mm}


$\chi_+ = \begin{pmatrix} 1 \\ 0 \end{pmatrix}$ ( eigenvalue $\frac{\hbar}{2};\,\, \chi_- = \begin{pmatrix} 0 \\ 1 \end{pmatrix}$ ( eigenvalue $- \frac{\hbar}{2})\\$
eigenspinors of $S_z\\$


\hdashrule[0.5ex][c]{\linewidth}{0.5pt}{1.5mm}


spin $\frac{\hbar}{2}$ has probability $|a|^2;\,\, \chi^{\dagger} \chi = 1 = |a|^2 + |b|^2$


\hdashrule[0.5ex][c]{\linewidth}{0.5pt}{1.5mm}


$\chi_+^{(x)} = \begin{pmatrix} 1/ \sqrt{2} \\ 1/ \sqrt{2} \end{pmatrix} ( eigenvalue \frac{\hbar}{2});\,\, \chi_-^{(x)} = \begin{pmatrix} 1/ \sqrt{2} \\ - 1/ \sqrt{2} \end{pmatrix} (eigenvalue - \frac{\hbar}{2})\\$


\hdashrule[0.5ex][c]{\linewidth}{0.5pt}{1.5mm}


\item \underline{$\chi = a \chi_+ + b \chi_- = ( \frac{a + b}{\sqrt{2}}) \chi_+^{(x)} + \frac{a-b}{\sqrt{2}} \chi_{-}^{(x)}$}\\
$S_x \chi^{(x)} = \lambda \chi^{(x)}\\$
\underline{recall:} $S_x = \frac{\hbar}{2} \begin{pmatrix} 0 & 1 \\ 1 & 0 \end{pmatrix}\\
\implies \begin{vmatrix} - \lambda & \hbar/2 \\ \hbar/2 & - \lambda \end{vmatrix} = \lambda^2 - ( \frac{\hbar}{2})^2 \implies \lambda = \pm \frac{\hbar}{2}\\
\frac{\hbar}{2} \begin{pmatrix} 0 & 1 \\ 1 & 0 \end{pmatrix} \begin{pmatrix} \alpha \\ \beta \end{pmatrix} = \pm \frac{\hbar}{2} \begin{pmatrix} \alpha \\ \beta \end{pmatrix} \implies \begin{pmatrix} \beta \\ \alpha \end{pmatrix} = \pm \begin{pmatrix} \alpha \\ \beta \end{pmatrix}\\
\implies \chi^{(x)} = \begin{pmatrix} 1 \\ 1 \end{pmatrix}, \begin{pmatrix} 1 \\ -1 \end{pmatrix},\,\, \chi^{\dagger} \chi = 1\\
\implies \chi_+^{(x)} = \begin{pmatrix} 1/ \sqrt{2} \\ 1/ \sqrt{2} \end{pmatrix};\,\, \chi_-^{(x)} = \begin{pmatrix} 1/ \sqrt{2} \\ - 1/\sqrt{2} \end{pmatrix}$ (normalized)\\
invert $\chi_{\pm}^{(x)}$ and plug into $\chi = a \chi_+ + b \chi_-\\
\implies \chi = ( \frac{1 + b}{ \sqrt{2}}) \chi_+^{(x)} + ( \frac{a-b}{\sqrt{2}}) \chi_-^{(x)}\\$


\hdashrule[0.5ex][c]{\linewidth}{0.5pt}{1.5mm}


\item \underline{$H = - \gamma \vec{B} \cdot \vec{S}$} (Hamiltonian (matrix) for spinning charged particles in $\vec{B}$)\\
$\vec{\mu} = \gamma \vec{S}\\$
\underline{recall:} $H =$ energy $= - \vec{\mu} \cdot \vec{B}\\
\therefore H = - \gamma \vec{B} \cdot \vec{S}\\$


\hdashrule[0.5ex][c]{\linewidth}{0.5pt}{1.5mm}


particle $1 \sim | s_1 m_1 \rangle;\,\,$ particle $2 \sim | s_2 m_2 \rangle\\
{S^{(1)}}^2 | s_1 s_2 m_1 m_2 \rangle = s_1 (s_1 + 1) \hbar^2 | s_1 s_2 m_1 m_2 \rangle\\
{S^{(2)}}^2 | s_1 s_2 m_1 m_2 \rangle = s_2 (s_1 +1) \hbar^2 | s_1 s_2 m_1 m_2 \rangle\\
S_z^{(1)} | s_1 s_2 m_1 m_2 \rangle = m_1 \hbar | s_1 s_2 m_1 m_2 \rangle\\
S_z^{(2)} | s_1 s_2 m_1 m_2 \rangle = m_2 \hbar | s_1 s_2 m_1 m_2 \rangle\\$
whats the total $z$ component of angular momentum for s (i.e. $m_s^{tot}$)?


\hdashrule[0.5ex][c]{\linewidth}{0.5pt}{1.5mm}


\item \underline{$m = m_1 + m_2$ }\\
$S_z | s_1 s_2 m_1 m_2 \rangle = S_z ^{(1)} |s_1 s_2 m_1 m_2 \rangle + S_z ^{(2)} |s_1 s_2 m_1 m_2 \rangle\\
= \hbar ( m_1 + m_2) |s_1 s_2 m_1 m_2 \rangle = \hbar m | s_1 s_2 m_1 m_2 \rangle\\
\therefore m= m_1 + m_2 \\$


\hdashrule[0.5ex][c]{\linewidth}{0.5pt}{1.5mm}


what is the total angular momentum?
$\hat{S} = \hat{S}^{(1)} + \hat{S}^{(2)}$


\hdashrule[0.5ex][c]{\linewidth}{0.5pt}{1.5mm}


\item \underline{$s= (s_1 + s_2),\,\, (s_1 + s_2 - 1),\,\, (s_1 + s_2 - 2), \dots , | s_1 - s_2 |$}\\
consider a simple example of an electron and a proton $s_1 = s_2 = \frac{1}{2},\,\,$ there are 4 possible states\\
$| \uparrow \uparrow \rangle = | \frac{1}{2} \frac{1}{2} \frac{1}{2} \frac{1}{2} \rangle , m = \frac{1}{2} + \frac{1}{2} = 1\\
| \uparrow \downarrow \rangle = | \frac{1}{2} \frac{1}{2} \frac{1}{2} - \frac{1}{2} \rangle, m=0\\
| \downarrow \uparrow \rangle = | \frac{1}{2} \frac{1}{2} - \frac{1}{2} \frac{1}{2} \rangle, m=0\\
| \downarrow \downarrow \rangle = | \frac{1}{2} \frac{1}{2} - \frac{1}{2} - \frac{1}{2} \rangle, m=-1\\$
m increases from $-s$ to $s$ in integer steps $\implies s=1,\,\,$ but we cannnot have 2 $m=0$ states, to fix this apply lowering operator\\
$S_- | \uparrow \uparrow \rangle = \hbar ( | \downarrow \uparrow \rangle + | \uparrow \downarrow \rangle )\\
\implies ( m=0\rm{ state}) = \frac{1}{\sqrt{2}} ( | \downarrow \uparrow \rangle + | \uparrow \downarrow \rangle)\\
\implies \begin{cases} |s m \rangle = | 1 1 \rangle = | \uparrow \uparrow \rangle\\
| 1 0 \rangle = \frac{1}{\sqrt{2}} ( | \downarrow \uparrow \rangle + | \uparrow \downarrow \rangle )$ (triplet) ($s=1)\\
| 1 -1 \rangle = | \downarrow \downarrow \rangle\\
\end{cases}$\\
since $s_1 = s_2 = \frac{1}{2}$ and $s= 0 , \frac{1}{2}, 1, \dots\\$
then it makes sense there could be configurations with $s=0 \implies m=0\\$
$\{ | 00 \rangle = \frac{1}{\sqrt{2}} ( | \uparrow \downarrow \rangle - | \downarrow \uparrow \rangle ) \} s=0$ (singlet)\\
applying raising or lowering operator yields zero.\\
\underline{claim:} $s=0$ or 1 for combined state\\
$S^2 = ( S^{(1)} + S^{(2)}) \cdot ( S^{(1)} + S^{(2)}) = (S^{(1)})^2 + (S^{(2)})^2 + 2 S^{(1)} \cdot S^{(2)}\\
\implies S^{(1)} \cdot S^{(2)} | \uparrow \downarrow \rangle = ( S_x^{(1)} | \uparrow \rangle)( S_x^{(2)} |\downarrow \rangle) + ( S_y^{(1)} | \uparrow \rangle)( S_y^{(2)} | \downarrow \rangle ) + ( S_z^{(1)} | \uparrow \rangle) ( S_z^{(2)} |\downarrow \rangle )\\$
\underline{recall:} $S_x | \uparrow \rangle = \frac{\hbar}{2} | \downarrow \rangle,$ etc.
$= ( \frac{\hbar}{2} | \downarrow \rangle)( \frac{\hbar}{2} | \uparrow \rangle) + ( \frac{i \hbar}{2} | \downarrow \rangle) ( - \frac{i \hbar}{2} | \uparrow \rangle)\\
+ ( \frac{\hbar}{2} | \uparrow \rangle )(- \frac{\hbar}{2} | \downarrow \rangle) = \frac{\hbar^2}{4} (2 | \downarrow \uparrow \rangle - | \uparrow \downarrow \rangle)\\$
Similarly $S^{(1)} \cdot S^{(2)} ( | \downarrow \uparrow \rangle) = \frac{\hbar^2}{4} (2 | \uparrow \downarrow \rangle - | \downarrow \uparrow \rangle )\\
\implies S^{(2)} \cdot S^{(2)} | 1 0 \rangle = \frac{\hbar^2}{4} | 1 0 \rangle\\
S^{(1)} \cdot S^{(2)} | 0 0 \rangle = - \frac{3 \hbar^2}{4} | 00 \rangle\\
\implies \begin{cases} S^2 | 1 0 \rangle = \hbar^2 s(s+1)=2 \hbar^2 | 1 0 \rangle\\
S^2 | 00 \rangle =\hbar^2 s(s+1)= 0\\
\end{cases}\\
\implies \begin{cases} 2 = s(s+1) \\ 0 = s(s+1) \end{cases}\\
\implies \begin{cases} s^2 + s - 2 = (s+2)(s-1)\\ s=0,-1 \end{cases}$ ????? (dont understand)


\hdashrule[0.5ex][c]{\linewidth}{0.5pt}{1.5mm}



\section*{Chapter 5}


\item \underline{$- \frac{\hbar^2}{2 m_1} \nabla_1^2 \psi - \frac{\hbar^2}{2 m_2} \nabla_2^2 \psi + V \psi = E \psi$}\\
insert $\hat{H} = - \frac{\hbar^2}{2m_1} \nabla_1^2 \psi - \frac{\hbar^2}{2m_2} \nabla_2^2 \psi + V \psi = E \psi$ into $\hat{H} \psi = E \psi$


\hdashrule[0.5ex][c]{\linewidth}{0.5pt}{1.5mm}


\item \underline{$\begin{cases} - \frac{\hbar^2}{2 m_1} \nabla_1^2 \psi_a( \vec{r}_1) + V_1 ( \vec{r}_1) \psi_a(\vec{r}_1) = E_a \psi_a ( \vec{r}_1) \\ - \frac{\hbar^2}{2 m_2} \nabla_2^2 \psi_b( \vec{r}_2) + V( \vec{r}_2) \psi_b ( \vec{r}_2) = E_b \psi_b( \vec{r}_2) \end{cases}$}\\
Non-interacting particles $\implies V( \vec{r}_1, \vec{r}_2) = V_1( \vec{r}_1) + V_2( \vec{r}_2)\\$
plug in $\psi( \vec{r}_1, \vec{r}_2) = \psi_a( \vec{r}_1) \psi_b( \vec{r}_2)$ and separate w/ $E = E_1 + E_2$\\


\hdashrule[0.5ex][c]{\linewidth}{0.5pt}{1.5mm}


\underline{Note:} $\Psi( \vec{r}_1, \vec{r}_2, t) = \Psi_a( \vec{r}_1, t) \Psi_b(\vec{r}_2,t)\\$


\hdashrule[0.5ex][c]{\linewidth}{0.5pt}{1.5mm}


$\psi( \vec{r}_1, \vec{r}_2) = \psi_a( \vec{r}_1) \psi_b ( \vec{r}_2) (distinguishable, i.e., \psi( \vec{r}_1, \vec{r}_2) \neq \psi( \vec{r}_2, \vec{r}_1))\\$
two ways to make an indistinguishable state\\
$\psi_{\pm} ( \vec{r}_1, \vec{r}_2) = A [ \psi_a ( \vec{r}_1) \psi_b ( \vec{r}_2) \pm \psi_b ( \vec{r}_1) \psi_a( \vec{r}_2)]\\$
bosons: $\psi_{+} ( \vec{r}_1, \vec{r}_2) = \psi_+ ( \vec{r}_2, \vec{r}_1)$ ( symmetric)\\
fermions: $\psi_- ( \vec{r}_1, \vec{r}_2) = - \psi_- ( \vec{r}_2, \vec{r}_1)$ ( antisymmetric)\\
but both satisfy $|\psi_{\pm} ( \vec{r}_1, \vec{r}_2)|^2 = | \psi_{\pm} ( \vec{r}_2, \vec{r}_1)|^2$


\hdashrule[0.5ex][c]{\linewidth}{0.5pt}{1.5mm}


two identical fermions cannot occupy the same state\\
$\psi_- ( \vec{r}_1, \vec{r}_2) = A ( \psi_a( \vec{r}_1) \psi_a ( \vec{r}_2) - \psi_a( \vec{r}_1) \psi_a( \vec{r}_2)) = 0$\\


\hdashrule[0.5ex][c]{\linewidth}{0.5pt}{1.5mm}


\item \underline{$\langle ( x_1 - x_2 )^2 \rangle_d = \langle x^2 \rangle_a + \langle x^2 \rangle_b -2 \langle x \rangle_a \langle x \rangle_b$} (distinguishable)\\
$\psi( x_1, x_2) = \psi_a (x_1) \psi_b ( x_2)\\
\langle (x_1 - x_2)^2 \rangle = \langle x_1^2 \rangle + \langle x_2^2 \rangle - 2 \langle x_1 x_2 \rangle\\
\langle x_1^2 \rangle = \int x_1^2 | \psi_a(x_1) |^2 dx_1 \int | \psi_b (x_2)|^2 dx_2 = \langle x^2 \rangle_a\\
\langle x_2^2 \rangle = \int | \psi_a(x_1) |^2 dx_1 \int x_2^2 | \psi_b(x_2)|^2 dx_2 = \langle x^2 \rangle_b\\
\langle x_1 x_2 \rangle = \int x_1 | \psi_a (x_1)|^2 dx_1 \int x_2 | \psi_b(x_2) |^2 dx_2 = \langle x \rangle_a \langle x \rangle_b\\
\therefore \langle (x_1- x_2)^2 \rangle_d = \langle x^2 \rangle_a + \langle x^2 \rangle_b - 2 \langle x \rangle_a \langle x \rangle_b$


\hdashrule[0.5ex][c]{\linewidth}{0.5pt}{1.5mm}


\item \underline{$\langle (x_1-x_2)^2 \rangle_{\pm} = \langle x^2 \rangle_a + \langle x^2 \rangle_b -2 \langle x \rangle_a \langle x \rangle_b \mp 2 | \langle x \rangle_{ab} |^2$}\\
(indistinguishable)\\
$\langle x_1^2 \rangle = \iint x_1^2 | \psi(x_1, x_2)|^2 dx_1 dx_2\\
\psi(x_1,x_2) = A [ \psi_a(x_1) \psi_b(x_2) \pm \psi_b(x_1) \psi_a(x_2)];\,\, A = \frac{1}{\sqrt{2}}\\
\implies \langle x_1^2 \rangle = \frac{1}{2} [ \int x_1^2 | \psi_a(x_1)|^2 dx_1 \int | \psi_b(x_2)|^2 dx_2\\
+ \int x_1^2 | \psi_b(x_1) |^2 dx_1 \int | \psi_a(x_2)|^2 dx_2\\
\pm \int x_1^2 \psi_a(x_1)^* \psi_b(x_1) dx_1 \int \psi_b(x_2)^* \psi_a(x_2) dx_2\\
\pm \int x_1^2 \psi_b(x_1)^* \psi_a(x_1) dx_1 \int \psi_a(x_2)^* \psi_b(x_2) d x_2]\\
= \frac{1}{2} [ \langle x^2 \rangle_a + \langle x^2 \rangle_b \pm 0 \pm 0 ] = \frac{1}{2} ( \langle x^2 \rangle_a + \langle x^2 \rangle_b)\\$
also $\langle x_2^2 \rangle = \frac{1}{2} ( \langle x^2 \rangle_b + \langle x^2 \rangle_a)\\$
and $\langle x_1 x_2 \rangle = \langle x \rangle_a \langle x \rangle_b \pm | \langle x \rangle_{ab}|^2\\
\langle x \rangle_{ab} \equiv \int x \psi_a(x)^* \psi_b(x) dx\\
\therefore \langle ( x_1 - x_2)^2 \rangle_{\pm} = \langle x^2 \rangle_a + \langle x^2 \rangle_b - 2 \langle x \rangle_a \langle x \rangle_b \mp 2 | \langle x \rangle_{ab} |^2$


\hdashrule[0.5ex][c]{\linewidth}{0.5pt}{1.5mm}


\underline{Note:} $\langle ( \Delta x)^2 \rangle_{\pm} = \langle ( \Delta x)^2 \rangle_d \mp 2 | \langle x \rangle_{ab}|^2\\$
$\implies$ bosons tend to be closer while fermions tend to be closer than distinguishable particles\\


\hdashrule[0.5ex][c]{\linewidth}{0.5pt}{1.5mm}


$\psi(\vec{r}) \chi$ (wave function for a particle with spin, if we have two particles\\
$\implies \psi( \vec{r}_1, \vec{r}_2) \chi(1,2)\\$
w/ $\psi( \vec{r}_1, \vec{r}_2) \chi(1,2) = - \psi( \vec{r}_2, \vec{r}_2) \chi(2,1)$\\
If spin and position are coupled (spin depends on position)\\
$\implies \psi_+ ( \vec{r}) \chi_+ + \psi_- ( \vec{r}) \chi_-$\\


\hdashrule[0.5ex][c]{\linewidth}{0.5pt}{1.5mm}


$\hat{P} |(1,2) \rangle = | (2,1) \rangle$ (exchange operator)\\


\hdashrule[0.5ex][c]{\linewidth}{0.5pt}{1.5mm}


$\hat{P}^2 = 1 \implies$ eigenvalues are $= \pm 1\\$


\hdashrule[0.5ex][c]{\linewidth}{0.5pt}{1.5mm}


\item \underline{$\frac{d \langle \hat{P} \rangle}{dt} = 0$} (identical particles)\\
$\hat{H} = \hat{K}_1 + \hat{K}_2 + V(\vec{r}_1, \vec{r}_2,t)\\
\implies [\hat{P}, \hat{H}] = 0\\
\implies \frac{d \langle \hat{P} \rangle}{dt} = 0\\$


\hdashrule[0.5ex][c]{\linewidth}{0.5pt}{1.5mm}


$|(1,2,\dots, i, \dots, j, \dots, n) \rangle = \pm | ( 1,2, \dots, j, \dots, i, \dots, n ) \rangle\\$
(symmetrization axiom)\\


\hdashrule[0.5ex][c]{\linewidth}{0.5pt}{1.5mm}


\item \underline{$\hat{H} = \sum_{j=1}^Z \{ - \frac{\hbar^2}{2m} \nabla_j^2 - ( \frac{1}{4 \pi \epsilon_0}) \frac{Z e^2}{r_j} \} + \frac{1}{2} ( \frac{1}{4 \pi \epsilon_0}) \sum_{j \neq k} \frac{e^2}{|\vec{r}_j - \vec{r}_k |}$}\\
consider atom, atomic number $Z$,\\
heavy nucleus (electric charge Ze) surrounded by Z electrons first term is kinetic term for j electron $2^{nd}$ term is potential energy of $j^{th}$ electron (at radius $r_j$) caused by nucleus (assumed to be concentrated at center)\\
$3^{rd}$ term is potential energy between $j^{th}$ and $k^{th}$ electron ( $\frac{1}{2}$ occurs since we would be double counting otherwise, i.e., the same if $j \rightarrow \leftarrow k$)\\


\hdashrule[0.5ex][c]{\linewidth}{0.5pt}{1.5mm}


\item \underline{$\psi_{n_x, n_y, n_z} = \sqrt{\frac{8}{\ell_x \ell_y \ell_z}} \sin( \frac{n_x \pi}{\ell_x} x) \sin( \frac{n_y \pi}{\ell_y} y) \sin( \frac{n_z \pi}{\ell_z} z);\,\, E_{n_x n_y n_z} = \frac{\hbar^2 \pi^2}{2m} ( \frac{n_x^2}{\ell_x^21} + \frac{n_y^2}{\ell_y^2} + \frac{n_z^2}{\ell_z^2}) = \frac{\hbar k^2}{2m}$}\\
electron in a box\\
$\implies V(x,y,z) = \begin{cases} 0,\,\, 0< x< \ell_x,\,\, 0< y< \ell_y,\,\, 0 < z < \ell_xz\\ \infty,\,\, o.w. \end{cases}\\
- \frac{\hbar^2}{2m} \nabla^2 \psi = E \psi\\
\psi(x,y,z) = X(x) Y(y) Z(z)\\
\implies - \frac{\hbar^2}{2m} \frac{d^2 X}{dx^2} = E_x X,\,\, - \frac{\hbar^2}{2m} \frac{d^2Y}{dy^2} = E_y Y,\,\, - \frac{\hbar^2}{2m} \frac{d^2 Z}{dz^2} = E_z Z\\
E= E_x + E_y + E_z\\
k_x = \frac{\sqrt{2m E_x}}{\hbar};\,\, k_y = \frac{\sqrt{2 m E_y}}{\hbar};\,\, k_z = \frac{\sqrt{2 m E_z}}{\hbar}\\
\implies X(x) = A_x \sin ( k_X x) + B_x \cos ( k_x x),\,\, Y(y) = A_y \sin(k_y y) + B_y \cos(k_y y)\\
Z(z) = A_z \sin(k_z z) + B_z \cos(k_z z)\\$
boundary conditions:\\
$X(0)=Y(0) = Z(0) = 0 \implies B_x = B_y = B_z = 0\\
X(\ell_x) = Y(\ell_y) = Z(\ell_z) = 0\\
\implies k_x \ell_x = n_x \pi;\,\, k_y \ell_y = n_y \pi;\,\, k_z \ell_z = n_z \pi\\
n_x,\,\, n_y,\,\, n_z \in \mathbb{N}\\
\therefore \psi_{n_x, n_y, n_z} = \sqrt{\frac{8}{\ell_x \ell_y \ell_z}} \sin( \frac{n_x \pi}{\ell_x} x) \sin( \frac{n_y \pi}{\ell_y} y) \sin( \frac{n_z \pi}{\ell_z} z)\\
\therefore E_{n_x n_y n_z} = \frac{\hbar^2 \pi^2}{2m} ( \frac{n_x^2}{\ell_x^21} + \frac{n_y^2}{\ell_y^2} + \frac{n_z^2}{\ell_z^2}) = \frac{\hbar k^2}{2m}$


\hdashrule[0.5ex][c]{\linewidth}{0.5pt}{1.5mm}


each point in $k$ space represents a particle stationary state and occupy volume $\frac{\pi^3}{\ell_x \ell_y \ell_z} = \frac{\pi^3}{V}\\$


\hdashrule[0.5ex][c]{\linewidth}{0.5pt}{1.5mm}


\item \underline{$E_F = \frac{\hbar^2}{2m} ( 3 \rho \pi^2)^{2/3}$}\\
If electrons were bosons they would settle to ground state $\psi_{111} (k=0?)$\\
they are fermions so each state can have $2$ electrons, so it fills up an octant in $k$ space\\
$N$ atoms each has $d$ free electrons\\
$\implies \#$ of states filled $= \frac{1}{8} ( \frac{4}{3} \pi k_f^3) = \frac{Nd}{2} ( \frac{\pi^3}{V})$ (factor of 2 accounts for the fact they are fermions)\\
$= \frac{1}{8} \frac{4}{3} \pi ( \frac{8 \cdot 3}{8} \frac{\pi^2}{V} N d) \implies k_f = (3 \rho \pi^2)^{1/3};\,\, \rho = \frac{Nd}{V}\\
\therefore E_f = \frac{\hbar^2 k_f^2}{2m} = \frac{\hbar^2}{2m} ( 3 \rho \pi^2)^{2/3}\\$


\hdashrule[0.5ex][c]{\linewidth}{0.5pt}{1.5mm}


\item \underline{$P= \frac{(3 \pi^2)^{2/3} \hbar^2}{5 m} \rho^{5/3}$} (degeneracy pressure)\\
Volume of shell in $k$-space= $\frac{1}{8} (4 \pi k^2) dk$\\
Volume of a single block $= \frac{\pi^3}{V}$\\
$\implies \#$ of electron states $= \frac{2 ( \frac{1}{2} \pi k^2)dk}{\frac{\pi^3}{V}} = \frac{V}{\pi^2} k^2 dk\\
E = \frac{\hbar^2 k^2}{2m}\\$
Energy of shell $\implies ( \#$ electron states) $\cdot$ ( Energy of $k$)\\
$\implies dE = ( \frac{V}{\pi^2} k^2 dk) ( \frac{\hbar^2 k^2}{2m})\\
E_{tot} = \frac{\hbar^2 V}{2 \pi^2 m} \int_0^{k_F} k^4 dk = \frac{\hbar^2 k_F^5 V}{10 \pi^2 m} = \frac{\hbar^2(3 \pi^2 N d)^{5/3}}{10 \pi^2 m} V^{-2/3}\\
\implies d E_{tot} = \frac{\partial E_{tot}}{\partial V} d V = - \frac{2}{3} \frac{\hbar^2 ( 3 \pi^2 N d)^{5/3}}{10 \pi^2 m} V^{-5/3} d V = - \frac{2}{3} E_{tot} \frac{dV}{V}\\$
\underline{recall:} $dW=PdV$
$P = \frac{2}{3} \frac{E_{tot}}{V} = \frac{2}{3} \frac{\hbar^2 k_F^5 V}{10 \pi^2 m V} = \frac{(3 \pi^2)^{2/3} \hbar^2}{5 m} \rho^{5/3}\\$


\hdashrule[0.5ex][c]{\linewidth}{0.5pt}{1.5mm}


$V(x+a) = V(x);\,\, - \frac{\hbar^2}{2m} \frac{d^2 \psi}{d x^2} + V(x) \psi = E \psi\\$
$\implies \psi(x+a) = e^{iqa} \psi(x)$ (Block's theorem)\\
$\implies | \psi(x+a)|^2 = | \psi(x)|^2$


\hdashrule[0.5ex][c]{\linewidth}{0.5pt}{1.5mm}


solids are large, lets make periodic boundary conditions $\implies \psi(x + Na) = \psi(x);\,\, N \approx 10^{23}\\
\implies \psi(x + N a) = \psi(x) = e^{iq N a} \psi(x) \implies e^{iqNa} = 1 \implies N q a = 2 \pi n\\
\implies q = \frac{2 \pi n}{Na}\\$
spose $V(x) = \alpha \sum_{j=0}^{N-1} \delta(x- j a)\\$
Blocks theorem allows us to solve schrodingers equation in one cell, say $0 \leq x < a\\$
$\implies - \frac{\hbar^2}{2m} \frac{d^2 \psi}{dx^2} = E \psi \implies \frac{d^2 \psi}{dx^2} = - k^2 \psi;\,\, k \equiv \frac{ \sqrt{2m E}}{\hbar}\\
\implies \psi(x) = A \sin ( k x) + B \cos ( k x);\,\, ( 0 < x < a)\\
\implies \psi(x + a) = ( A \sin ( k (x + a)) + B \cos ( k ( x+a)))\\
= e^{i q a} ( A \sin ( k x) + B \cos ( kx)) (- a < x <0)\\
\implies \psi(x) = e^{- i q a} ( A \sin ( k (x +a)) + B \cos ( k(x+a)))\,\, (-a<x<0)\\
\psi(0_-) = \psi( 0_+)\\
\implies B = e^{-i q 6+} - \frac{d \psi}{da} [ A \sin k a + B \cos k a]\\
\frac{d \psi}{dx}|_{0^+} - \frac{d \psi}{dx} |_{0^-} = ( A k- B k \cdot 0) - e^{-i q a} ( A k \cos ka - B k \sin ka)\\$
Now,\\
$B = e^{-iqa} [ A \sin ka + B \cos ka]\\
\implies (e^{iqa} - \cos ka) B = A \sin ka\\$
substitute\\
$\implies k \frac{(e^{iqa} - \cos ka)}{\sin ka} B - e^{-i q a} k [ \frac{(e^{iqa} - \cos ka) B}{\sin ka} \cos ka - B \sin ka] = \frac{2m \alpha}{\hbar^2} B\\
\frac{(e^{iqa} - \cos ka) B}{\sin ka}\\
\implies ( e^{iqa} - \cos ka) - e^{-iqa} [ (e^{iqa} - \cos ka) \cos ka - \sin^2 ka] = \frac{2m \alpha}{\hbar^2 k} \sin k a\\
\implies (e^{iqa} - \cos ka) - e^{- i qa} (e^{iqa} - \cos ka) \cos ka + e^{-iqa} \sin^2 ka = \frac{2m \alpha}{\hbar^2 k} \sin ka\\
\implies (e^{iqa} - \cos ka)(1- e^{- iqa} \cos ka) + e^{-iqa} \sin^2 ka = \frac{2 m \alpha}{\hbar^2 k} \sin ka\\
\implies \cos qa = \cos ka + \frac{m \alpha}{\hbar^2 k} \sin ka\\$
this determines possible k\\
$z \equiv ka,\,\, \beta = \frac{m \alpha a}{\hbar^2}\\
\implies f(z) \equiv \cos z + \beta \frac{\sin z}{z}\\$
notice $f(z)$ goes outside of $[-1,1]$ but $-1 < \cos qa < 1$ so this gives rise to bands and forbidden regions (regions outside of $[-1,1]$)\\
Notice that not all energies are allowed for a given band since $\cos (qa) = \cos \frac{2 \pi n}{N}\\$
can only take on discrete values so $f(z)$ would only have certain $k$ hence certain E that satisfy, it is almost continuous though\\


\hdashrule[0.5ex][c]{\linewidth}{0.5pt}{1.5mm}





\section*{QM: Chapter 6}
\item \underline{$\hat{T}(a) = \exp[-\frac{ia}{\hbar} \hat{p}]$}\\
$\hat{T}(a) \psi(x) = \psi(x-a)\\$
\underline{recall:} $f(x) = \sum_{n=0}^{\infty} \frac{f^{(n)}(a)}{n!} (x-a)^n,\,\,$ let $x-a$ be the variable and $x$ be the center\\
$\implies \psi(x-a) = \sum_{n=0}^{\infty} \frac{\frac{d^n}{d(x-a)^n}}{n!} \psi(x) (x-a-x)^n\\$
\underline{Note:} $ \frac{d \psi(x)}{d(x-a)} = (\frac{d(x-a)}{d\psi(x)})^{-1} = (\frac{dx}{d \psi} \frac{d(x-a)}{dx})^{-1}\\
=\frac{d \psi}{dx}\\
\implies \psi(x-a) = \sum_{n=0}^{\infty} \frac{(-a)^n}{n!} \frac{d^n}{dx^n} \psi(x)\\$
\underline{recall:} $ \hat{p} = - i \hbar \frac{d}{dx}\\
\implies \hat{T}(a) \psi(a) = \sum_{n=0}^{\infty} \frac{1}{n!} ((-\frac{a}{i} \frac{i\hbar}{\hbar} \frac{d}{dx})^n \psi(x)\\
=\sum_{n=0}^{\infty} \frac{1}{n!} ( \frac{a}{i \hbar} \hat{p}^n \psi(x) = \sum_{n=0}^{\infty} \frac{1}{n!} ( - \frac{i a}{\hbar} \hat{p})\psi(x)\\
=\exp[-\frac{ia}{\hbar} \hat{p}] \psi(x)\\
\therefore \hat{T}(a) = \exp[-\frac{ia}{\hbar} \hat{p}]\\$


\hdashrule[0.5ex][c]{\linewidth}{0.5pt}{1.5mm}


\underline{Note:} $\hat{T}(a) \psi(x) = \psi(x-a)$ but $\hat{T}(a) | x \rangle = |x+a \rangle,$ this is because $\hat{T}(a) \psi(x) = \hat{T} (a) \langle x | \alpha \rangle = \langle x | \hat{T}(a) \alpha \rangle = \psi(x-a) = \langle x-a | \alpha \rangle\\
\implies \hat{T}(a) | x \rangle = | x + a \rangle$

\hdashrule[0.5ex][c]{\linewidth}{0.5pt}{1.5mm}


$\langle \psi' | \hat{Q} | \psi ' \rangle = \langle \psi | \hat{Q}' | \psi \rangle$ ( how operators transform (dont understand))\\


\hdashrule[0.5ex][c]{\linewidth}{0.5pt}{1.5mm}


\item \underline{$\hat{Q}'=\hat{T}^{\dagger} \hat{Q} \hat{T}$} $Q$ transformed by shifting\\
$\langle \psi' | \hat{Q} | \psi' \rangle = ( \hat{T} | \psi \rangle )^{\dagger} \hat{Q} ( \hat{T} | \psi \rangle ) = \langle \psi | \hat{T}^{\dagger} \hat{Q} \hat{T} | \psi \rangle\\
\therefore \hat{Q}' = \hat{T}^{\dagger} \hat{Q} \hat{T}$


\hdashrule[0.5ex][c]{\linewidth}{0.5pt}{1.5mm}


\item \underline{$\hat{p}'=\hat{p}$}\\
$\hat{p}' \psi(x) = \hat{T}^{\dagger} \hat{p} \hat{T} \psi(x) = \hat{T}^{\dagger} \hat{p} \sum \frac{1}{n!} ( - \frac{ia}{\hbar})^n \hat{p}^n \psi(x)\\
= \hat{T}^{\dagger} \hat{T} \hat{p} \psi = \hat{p} \psi\\
\therefore \hat{p}'= \hat{p}\\$


\hdashrule[0.5ex][c]{\linewidth}{0.5pt}{1.5mm}


\underline{Note:} $\hat{p}' = \hat{T}^{\dagger} \hat{p} \hat{T} = \hat{p} \implies \hat{p} \hat{T} = \hat{T} \hat{p} \implies [\hat{p},\hat{T}]=0$


\hdashrule[0.5ex][c]{\linewidth}{0.5pt}{1.5mm}



\item \underline{$\hat{T}^{\dagger} \hat{x} \hat{T} = \hat{x} + a$} see page 76 in sakuri\\
$\hat{T}^{\dagger} \hat{x} \hat{T} \psi(x) = \hat{T}^{\dagger} \hat{x} \psi(x-a) = \hat{T}^{\dagger} x \psi(x-a) = (x+a) \psi(x)\\
\therefore \hat{x}' = \hat{x}+a\\$


\hdashrule[0.5ex][c]{\linewidth}{0.5pt}{1.5mm}


\item \underline{$\hat{Q}'(\hat{x},\hat{p}) = \hat{Q}(\hat{x} + a, \hat{p})$}\\
\underline{recall:} $\hat{x}' = \hat{T}^{\dagger} \hat{x} \hat{T} = \hat{x} + a;\,\, \hat{p}' = \hat{p}\\$
assume $\hat{Q}(\hat{x},\hat{p}) = \sum_{m=0}^{\infty} \sum_{n=0}^{\infty} a_{mn} \hat{x}^m \hat{p}^n\\
\implies \hat{Q}' = \sum_{m=0}^{\infty} \sum_{n=0}^{\infty} a_{mn} \hat{T}^{\dagger} \hat{x}^m \hat{p}^n \hat{T}\\$
\underline{recall:} $[\hat{p},\hat{T}]=0\\
\implies \hat{Q}'=\sum_{m=0}^{\infty} \sum_{n=0}^{\infty} a_{mn} \hat{T}^{\dagger} \hat{x}^m \hat{T} \hat{p}^n\\$
\underline{Note:} Let $\psi'=\hat{p}^n \psi,\,\,\hat{T}^{\dagger} \hat{x}^m \hat{T} \psi'(x) = \hat{T}^{\dagger} (x^m \psi'(x-a))= (x+a)^m \psi'(x)=(x+a)^m \hat{p}^n \psi(x)\\
\therefore \hat{Q}'(\hat{x},\hat{p}) = \sum_{m=0}^{\infty} \sum_{n=0}^{\infty} a_{mn} (\hat{x} + a)^m \hat{p}^n = \hat{Q}(\hat{x} + a, \hat{p})$\\


\hdashrule[0.5ex][c]{\linewidth}{0.5pt}{1.5mm}


$\hat{H}'=\hat{T}^{\dagger} \hat{H} \hat{T} = \hat{H}$ (Translationally invariant/ translational symmetry (not always the case))\\


\hdashrule[0.5ex][c]{\linewidth}{0.5pt}{1.5mm}


\item \underline{$\hat{H}'=\hat{H} \implies [\hat{H}, \hat{T} ] = 0$}\\
$\hat{H}' = \hat{T}^{\dagger} \hat{H} \hat{T} = \hat{H},\,\, \hat{T}^{\dagger} \hat{T} = 1 \implies \hat{T} \hat{T}^{\dagger} \hat{H} \hat{T} = \hat{T} \hat{H} = \hat{H} \hat{T}\\
\therefore \hat{H} \hat{T}- \hat{T} \hat{H} = [ \hat{H}, \hat{T}]=0\\$


\hdashrule[0.5ex][c]{\linewidth}{0.5pt}{1.5mm}


\item \underline{$\hat{H}'= \hat{H} \implies V(x+a) = V(x)$}( particle in 1D)\\
$\hat{H} = \frac{\hat{p}^2}{2m} + V(x) \implies \hat{H}'= \hat{T}^{\dagger} \hat{H} \hat{T} = \frac{1}{2m} \hat{T}^{\dagger} \hat{p}^2 \hat{T} + \hat{T}^{\dagger} V(x) \hat{T}\\$
\underline{Note:} $\hat{T}^{\dagger} V(x) \hat{T} \psi(x) = \hat{T}^{\dagger} V(x) \psi(x-a) = V(x+a) \psi(x)\\
\implies \hat{T}^{\dagger} V(x) \hat{T} = V(x+a);\,\, \hat{T}^{\dagger} \hat{p}^2 \hat{T} = \hat{T}^{\dagger} \hat{p} \hat{T} \hat{p} = \hat{T}^{\dagger} \hat{T} \hat{p}^2\\$
since $[ \hat{p}, \hat{T}]=0\\\
\implies \hat{H}'=\frac{\hat{p}^2}{2m}  + V(x+a) = \hat{H} = \frac{\hat{p}^2}{2m} + V(x)\\
\therefore V(x+a) = V(x)\\$


\hdashrule[0.5ex][c]{\linewidth}{0.5pt}{1.5mm}


\underline{Note:} If it holds for every $a$ it is a continuous symmetry and if it holds for discrete a$ \implies$ discrete symmetry.\\


\hdashrule[0.5ex][c]{\linewidth}{0.5pt}{1.5mm}


\underline{Note:} $[ \hat{H}, \hat{T} ] = 0 \implies$ complete set f simultaneous eigenstates.\\


\hdashrule[0.5ex][c]{\linewidth}{0.5pt}{1.5mm}

\item \underline{$\hat{T}$ unitary $\implies \lambda = e^{i \phi}$}\\
$\hat{T} | \psi \rangle = \lambda | \psi \rangle \implies \langle \psi | \hat{T}^{\dagger} = \lambda^* \langle \psi |\\
\implies\langle \psi | \hat{T}^{\dagger} \hat{T} | \psi \rangle = \langle \psi | |\lambda|^2 | \psi \rangle = | \lambda|^2 \langle \psi | \psi \rangle\\
\implies | \lambda |^2 = 1 \implies \lambda = e^{i \phi}\\$


\hdashrule[0.5ex][c]{\linewidth}{0.5pt}{1.5mm}

$\star$
\item \underline{$\psi(x) = e^{iq x} u(x);\,\, u(x+ a) = u(x)$} (Blocks Theorem)\\
We can write $\psi(x) = e^{iqx} u(x)$ for some $u(x)$\\
for example $u$ could be $ e^{-iqx} \psi(x)$\\
must prove $u(x+a) = u(x)$.\\
\underline{Note:} $\hat{T}(a) \psi(x)=e^{i \phi} \psi = \psi(x-a) = e^{-iqa} \psi(x)$ (Let $\phi=qa$)\\
$ \implies \hat{T}^{\dagger} \psi(x-a) = \psi(x) = \psi((x+a) - a) = e^{-iqa} \psi(x+a)\\
\implies e^{iqx} u(x) = \psi(x) = e^{-iqa} \psi(x+a) = e^{-iqa} e^{iq(x+a)} u(x+a)\\
\implies \psi(x) = e^{-iqa} e^{iq(x+a)} u(x+a)\\
\implies e^{iqx} u(x) = e^{iqx} u(x+a) \implies u(x) = u(x+a)$\\


\hdashrule[0.5ex][c]{\linewidth}{0.5pt}{1.5mm}

\item \underline{$\frac{d}{dt} \langle p \rangle = 0$} (continuous translational symmetry)\\
continuous $\hat{T}$ symmetry $\implies [ \hat{T}(a), \hat{H} ] = 0$\\
working with exponentials can be hard so let's approximate\\
$\hat{T}(\delta) = e^{- i \delta \hat{p}/\hbar} \approx 1- i \frac{\delta}{\hbar} \hat{p}\\
\implies [ \hat{H}, \hat{T}(\delta) ] = [ \hat{H}, 1- i \frac{\delta}{\hbar} \hat{p}] = 0 \implies [ \hat{H}, \hat{p}] = 0\\
\underline{recall:} \frac{d}{dt} \langle Q \rangle = \frac{i}{\hbar} \langle [ \hat{H}, \hat{Q}] \rangle + \langle \frac{\partial \hat{Q}}{\partial t} \rangle;\,\, \langle \frac{\partial \hat{T}}{\partial t} \rangle = 0\\
\implies \frac{d \langle p \rangle }{d t} = \frac{i}{\hbar} \langle [ \hat{H}, \hat{p}] \rangle = 0\\$
symmetries $\implies$ conservation laws.\\


\hdashrule[0.5ex][c]{\linewidth}{0.5pt}{1.5mm}

What does Q conserved mean?\\
2 possibilities:\\
\underline{1$^{st}$ definition:} expectation value $\langle Q \rangle$ is independent of time\\
\underline{2$^{nd}$ definition:} probability of getting a particular value is independent of time\\


\hdashrule[0.5ex][c]{\linewidth}{0.5pt}{1.5mm}

We show $1^{st} \implies 2^{nd}$\\ (make this more coherent)
\underline{recall:} $\frac{d}{dt} \langle Q \rangle = \frac{i}{\hbar} \langle [ \hat{H}, \hat{Q} ] \rangle$ ( assume $\langle \frac{\partial Q}{\partial t} \rangle = 0)\\$
so $1^{st} \implies \frac{d}{dt} \langle Q \rangle = 0 \implies [ \hat{H}, \hat{Q} ] = 0\\
P(q_n) = |c_n|^2 = | \langle f_n | \Psi(t) \rangle |^2$ where $\hat{Q} | f_n \rangle = q_n | f_n \rangle\\$
\underline{Note:} $P(q_n) = \sum_i | \langle f_n^{(i)} | \Psi(t) \rangle |^2$ if degenerate\\
$| \Psi (t) \rangle = \sum_m e^{-i E_m t}{\hbar} c_m | \psi_m \rangle\\
\implies P(q_n) = | \langle f_n | \Psi(t) \rangle |^2 = | \sum_m e^{-i E_m t}{\hbar} c_m \langle f_n | \psi_m \rangle |^2\\$
since $[\hat{Q}, \hat{H} ] = 0 \implies \hat{Q} f_n = \lambda_n f_n$ and $\hat{H} f_n = E_n f_n\\
\implies | f_n \rangle = | \psi_n \rangle\\
\therefore P(q_n) = | \sum_m e^{-i E_m t}{\hbar} c_m \langle \psi_n | \psi_m \rangle |^2 = |c_n|^2\\
|c_n|^2$ is independent of time.\\


\hdashrule[0.5ex][c]{\linewidth}{0.5pt}{1.5mm}


$\hat{\Pi} \psi(x) = \psi'(x)  = \psi(-x);\,\, \hat{\Pi}^{\dagger} = \hat{\Pi};\,\, \hat{\Pi}^{-1} = \hat{\Pi}\\
\implies \hat{\Pi}^{-1} = \hat{\Pi}= \hat{\Pi}^{\dagger};\,\, \hat{Q}'= \hat{\Pi}^{\dagger} \hat{Q} \hat{\Pi}\\$


\hdashrule[0.5ex][c]{\linewidth}{0.5pt}{1.5mm}


\item \underline{$\hat{x}'= \hat{\Pi}^{\dagger} \hat{x} \hat{\Pi} = - \hat{x}$}\\
\item \underline{$\hat{p}' = \hat{\Pi}^{\dagger} \hat{p} \hat{\Pi} = - \hat{p}$}\\
$\hat{x}' \psi(x) = \hat{\Pi}^{\dagger} \hat{x} \hat{\Pi} \psi(x) = \hat{\Pi}^{\dagger} \hat{x} \psi(-x) = \hat{\Pi}^{\dagger} x \psi(-x) = x \psi(x)\\
\therefore \hat{x}'=\hat{x}\\
\hat{p}' \psi(x) = \hat{\Pi}^{\dagger} \hat{p} \hat{\Pi} \psi(x) = \hat{\Pi}^{\dagger} (-i \hbar \frac{d}{dx} \psi(-x) = \hat{\Pi}^{\dagger} i \hbar \frac{d}{d(-x)} \psi(-x)\\
= i \hbar \frac{d}{dx} \psi(x) = - \hat{p} \psi(x)\\
\therefore \hat{p}' = - \hat{p}\\$


\hdashrule[0.5ex][c]{\linewidth}{0.5pt}{1.5mm}


\item \underline{$\hat{Q}'(\hat{x},\hat{p}) = \hat{\Pi}^{\dagger} \hat{Q}(\hat{x},\hat{p}) \hat{\Pi} = \hat{Q}(- \hat{x},-\hat{p})$}\\
\underline{recall:} $\hat{Q}(\hat{x},\hat{p}) = \sum_{m=0}^{\infty} \sum_{n=0}^{\infty} a_{mn} \hat{x}^m \hat{p}^n\\
\implies \hat{Q}'(\hat{x},\hat{p}) \psi(x) = \sum_{m,n} a_{mn} \hat{\Pi}^{\dagger} \hat{x}^m \hat{p}^n \hat{\Pi} \psi(x)\\
= \sum_{m,n} a_{mn} \hat{\Pi}^{\dagger} \hat{x}^m \hat{p}^n \psi(-x)\\
= \sum_{m,n} a_{mn} \hat{\Pi}^{\dagger} \hat{x}^m (- i \hbar)^n \frac{d^n}{dx^n} \psi(-x)\\
\sum_{m,n} a_{mn} \hat{\Pi}^{\dagger} x^m (i \hbar)^n \frac{d^n}{d(-x)^n} \psi(-x)\\
\sum_{m,n} a_{mn} (-x)^m (i \hbar \frac{d}{dx})^n \psi(x)\\
= \sum_{m,n} a_{mn} (- \hat{x})^m (- \hat{p})^n \psi(x) = \hat{Q}(- \hat{x},- \hat{p})\\
\therefore \hat{Q}'(\hat{x}, \hat{p}) = \hat{Q}(- \hat{x},- \hat{p})\\
$

\hdashrule[0.5ex][c]{\linewidth}{0.5pt}{1.5mm}


$\hat{H}' = \hat{\Pi}^{\dagger} \hat{H} \hat{\Pi} = \hat{H}$ ( inversion symmetry)\\
$\implies [ \hat{H}, \hat{\Pi}] = 0\\$


\hdashrule[0.5ex][c]{\linewidth}{0.5pt}{1.5mm}


\item \underline{$\hat{H}' = \hat{H}$ for 1D particle $\implies V(x) = V(-x)$}\\
$\hat{H} = \frac{\hat{p}^2}{2m} + V(x)\\
\hat{H}' \psi(x) = \hat{\Pi}^{\dagger} \hat{H} \hat{\Pi} \psi(x) = \frac{1}{2m} \hat{\Pi}^{\dagger} \hat{p}^2 \hat{\Pi} \psi(x) + \hat{\Pi}^{\dagger} V(x) \hat{\Pi} \psi(x)\\
= \frac{1}{2m} \hat{\Pi}^{\dagger} \hat{p}^2 \psi(-x) + \hat{\Pi}^{\dagger} (V(x) \psi(-x))\\
= \frac{1}{2m} \hat{p}^2 \psi(x) + V(-x) \psi(x) = \frac{\hat{p}^2}{wm} \psi(x) + V(x) \psi(x)\\
\implies V(-x) = V(x)$\\


\hdashrule[0.5ex][c]{\linewidth}{0.5pt}{1.5mm}


\underline{Implications of inversion symmetry}\\
(i) $\hat{H}' = \hat{H} \implies [ \hat{\Pi}, \hat{H}] = 0 \implies \hat{\Pi} \psi_n = \lambda \psi_n;\,\, \hat{H} \psi_n = E \psi_n\\
\hat{\Pi}^2 \psi(x) = \hat{\Pi} \psi(-x) = \psi(x) \implies \hat{\Pi} = \pm 1 \implies \hat{\Pi} \psi_n = \pm \psi_n(x) = \psi_n(-x)\\
\therefore$ since $V(x) = V(-x) \implies \psi$ is also even or odd\\
(ii) $\frac{d}{dt} \langle \hat{\Pi} \rangle = \frac{i}{\hbar} \langle [ \hat{H}, \hat{\Pi} ] \rangle = 0\\
\implies$ Parity conserved for particle moving in symmetric potential.\\


\hdashrule[0.5ex][c]{\linewidth}{0.5pt}{1.5mm}


\underline{Note:} $\hat{\Pi} \psi(\vec{r}) = \psi'(\vec{r}) = \psi(- \vec{r})\\
\hat{\vec{r}}' = \hat{\Pi}^{\dagger} \hat{\vec{r}} \,\,\hat{\Pi} = - \hat{\vec{r}};\,\, \hat{\vec{p}}' = \hat{\Pi}^{\dagger} \hat{\vec{p}} \hat{\Pi} = - \hat{\vec{p}}\\
\hat{Q}(\hat{\vec{r}}, \hat{\vec{p}}) = \hat{\Pi}^{\dagger} \hat{Q}(\hat{\vec{r}}, \hat{\vec{p}}) \hat{\Pi} = \hat{Q}(- \hat{\vec{r}},- \hat{\vec{p}})$


\hdashrule[0.5ex][c]{\linewidth}{0.5pt}{1.5mm}


\item \underline{$\hat{p}_e' = - \hat{p}_e$};$\,\, \hat{p}_e = q \hat{r}\\
\hat{p}_e' = \hat{\Pi} \hat{p}_e \hat{\Pi} = q \hat{\Pi} \hat{r} \hat{\Pi} = q \hat{r}'=- q \hat{r} = - \hat{p}_e\\$


\hdashrule[0.5ex][c]{\linewidth}{0.5pt}{1.5mm}


\underline{Note:} $\hat{\Pi} \psi_{n \ell m} (r, \theta, \phi) = (-1)^{\ell} \psi_{n \ell m} (r, \theta, \phi)\\$


\hdashrule[0.5ex][c]{\linewidth}{0.5pt}{1.5mm}


\item \underline{$\langle n' \ell' m' | \hat{p}_e | n \ell m \rangle = 0 if \ell + \ell'$ is even} $\psi_{n \ell m} \implies | n \ell m \rangle\\$
$\langle n' \ell' m' | \hat{p}_e | n \ell m \rangle = - \langle n' \ell' m' | \hat{\Pi}^{\dagger} \hat{p}_e \hat{\Pi} | n \ell m \rangle\\
= - \langle n' \ell' m' | (-1)^{\ell'} \hat{p}_e (-1)^{\ell} | n \ell m \rangle\\
= (-1)^{\ell + \ell' + 1} \langle n' \ell' m' | \hat{p}_e | n \ell m \rangle$\\
(dont understand)
if $\ell' + \ell = 2k \implies (-1)^{2k+1} = -1\\
\implies \langle n' \ell' m' | \hat{p}_e | n \ell m \rangle = - \langle n' \ell' m' | \hat{p}_e | n \ell m \rangle\\
\implies \langle n' \ell' m' | \hat{p}_e | n \ell m \rangle = 0$ (Laporte's rule)\\


\hdashrule[0.5ex][c]{\linewidth}{0.5pt}{1.5mm}


\underline{Notes} $\hat{R}_z (\varphi) \psi(r,\theta,\varphi) = \psi'(r, \theta, \varphi)= \psi(r,\theta, \phi - \varphi)\\
\hat{R}_z(\varphi) = \exp[- \frac{i \varphi}{\hbar} \hat{L}_z]\\$
taylor expand $\psi(r, \theta, \phi-\varphi)$ with $\phi-\varphi$ as variable and $\phi$ as center and also use $\hat{L}_z = - i \hbar \frac{\partial}{\partial \varphi}
\implies \hat{R}_z(\delta) \approx 1- \frac{i \delta}{\hbar} \hat{L}_z\\$


\hdashrule[0.5ex][c]{\linewidth}{0.5pt}{1.5mm}


\item \underline{$\hat{x}' = \hat{R}^{\dagger} \hat{x} \hat{R} = \hat{x} - \delta \hat{y};\,\, \hat{y}' = \hat{y} + \delta \hat{x};\,\, \hat{z}' = \hat{z}$} (infinitesimal rotations)\\
$\hat{x}' = \hat{R}^{\dagger} \hat{x} \hat{R} = ( 1 + \frac{i \delta}{\hbar} \hat{L}_z) \hat{x}(1- \frac{i \delta}{\hbar} \hat{L}_z)\\
=(1+ \frac{i \delta}{\hbar} \hat{L}_z)(\hat{x} - \frac{i \delta}{\hbar} \hat{x} \hat{L}_z) = \hat{x} - \frac{i \delta}{\hbar} \hat{x} \hat{L}_z + \frac{i \delta}{\hbar} \hat{L}_z \hat{x} + \frac{\delta^2}{\hbar^2} \hat{L}_z \hat{x} \hat{L}_z\\
\frac{\delta^2}{\hbar^2} \hat{L}_z \hat{L}_z \approx 0\\
\implies \hat{x}' \approx \hat{x} + \frac{i \delta}{\hbar}[\hat{L}_z, \hat{x}]\\$
\underline{recall:} $[ \hat{L}_z, \hat{x}] = i \hbar \hat{y}$ (derive)\\
$\therefore \hat{x}' = \hat{x} + i \delta i \hat{y} = \hat{x} - \delta \hat{y}\\$


\hdashrule[0.5ex][c]{\linewidth}{0.5pt}{1.5mm}


$\begin{pmatrix} \hat{x}' \\ \hat{y}' \\ \hat{z}' \end{pmatrix} = \begin{pmatrix} \cos \delta & - \sin \delta & 0 \\ \sin \delta & \cos \delta & 0 \\ 0 & 0 & 1 \end{pmatrix} \begin{pmatrix} \hat{x} \\ \hat{y} \\ \hat{z} \end{pmatrix} \approx \begin{pmatrix} 1 & - \delta & 0 \\ \delta &1 & 0 \\ 0 & 0 & 1 \end{pmatrix} \begin{pmatrix} \hat{x} \\ \hat{y} \\ \hat{z} \end{pmatrix}\\
or \hat{x}' = \hat{x} - \delta \hat{y},\,\, \hat{y}' = \hat{y} + \delta \hat{x}\\
\hat{z}' = \hat{z} \implies \begin{pmatrix} \hat{x}' \\ \hat{y}' \\ \hat{z}' \end{pmatrix} = \begin{pmatrix} \hat{x} - \delta \hat{y} + 0 \hat{z} \\ \delta \hat{x} + \hat{y} + 0 \hat{z} \\ 0 \hat{x} + 0 \hat{y} + \hat{z} \end{pmatrix}\\
= \begin{pmatrix} 1 & - \delta & 0 \\ \delta & 1 & 0 \\ 0 & 0 & 1 \end{pmatrix} \begin{pmatrix} \hat{x} \\ \hat{y} \\ \hat{z} \end{pmatrix}$\\


\hdashrule[0.5ex][c]{\linewidth}{0.5pt}{1.5mm}


\item \underline{$\hat{R}_{\hat{n}}(\varphi) = \exp[- \frac{i \varphi}{\hbar} \hat{n} \cdot \hat{\vec{L}}]$}\\


\hdashrule[0.5ex][c]{\linewidth}{0.5pt}{1.5mm}


If $\hat{\vec{r}}' = D \hat{\vec{r}} \implies \hat{\vec{V}}' = D \hat{\vec{V}}$ then $\hat{\vec{V}}$ is called a vector operator\\

\hdashrule[0.5ex][c]{\linewidth}{0.5pt}{1.5mm}


\underline{Note:} $[L_z , x] = i \hbar y,\,\, [ L_z , y ] = - i \hbar x,\,\, [L_z , z] = 0\\
\implies [ \hat{L}_i, \hat{x}_i] = i \hbar \epsilon_{ijk} \hat{x}_k$ (summation on k)\\
$\hat{\vec{V}}' = \hat{R}_z^{\dagger}(\varphi) \hat{\vec{V}} \hat{R}_z (\varphi)$ (unfinished)\\


\hdashrule[0.5ex][c]{\linewidth}{0.5pt}{1.5mm}


$[\hat{L}_i, \hat{r}_j] = i \hbar \epsilon_{ijk} \hat{r}_k;\,\, [ \hat{L}_i, \hat{p}_j] = i \hbar \epsilon_{ijk} \hat{p}_k;\,\, [ \hat{L}_i , \hat{L}_j] = i \hbar \epsilon_{ijk} \hat{L}_k$ (all vector operators )\\
we can take $[ \hat{L}_i, \hat{B}_j] = i \hbar \epsilon_{ijk} \hat{V}_k$ as the definition for a vector operator\\


\hdashrule[0.5ex][c]{\linewidth}{0.5pt}{1.5mm}


$[\hat{L}_i, \hat{f}] = 0$ (scalar operator)\\
or $\hat{f}' = \hat{R} \hat{f} \hat{R} = \hat{f}$\\


\hdashrule[0.5ex][c]{\linewidth}{0.5pt}{1.5mm}
\section*{Chapter 6}
\item \underline{$V(\vec{r}) = V(r) \implies \hat{H}' = \hat{H} \implies [ \hat{H}, \hat{R}_{\hat{n}}(\varphi)] = 0$}\\
$\hat{H}' = \hat{R}_{\hat{n}}^{\dagger} (\varphi) \hat{H} \hat{R}_{\hat{n}} (\varphi) = \frac{1}{2m} \hat{R}^{\dagger} V(r, \theta, \phi) \hat{R} = \frac{1}{2m} \hat{p}^n + \hat{R}^{\dagger} V \hat{R}\\
\hat{R}^{\dagger} V(r,\theta, \phi) \hat{R} f(r,\theta, \phi) = \hat{R}^{\dagger} (V(r, \theta, \phi) f(r, \theta, \phi- \varphi)\\
= V(r, \theta, \phi + \varphi) f(r, \theta, \phi)\\$
but $V(\vec{r}) = V(r,\theta, \phi) = V(r)\\
\implies B(r, \theta, \varphi + \phi) = V(r)\\
\therefore \hat{H}' = \hat{H} \implies \hat{R}^{\dagger} \hat{H} \hat{R} = \hat{H} \implies \hat{H} \hat{R} - \hat{R} \hat{H} = [ \hat{H}, \hat{R}] = 0\\$


\hdashrule[0.5ex][c]{\linewidth}{0.5pt}{1.5mm}


\underline{Theorem:} Symmetry $\implies$ degeneracy (sometimes)\\
\underline{Proof:} assume $[ \hat{H}, \hat{Q}] = 0\\$
Spose $\hat{H} | \psi_n \rangle = E_n | \psi_n \rangle,\,\,$ Let $| \psi'_n \rangle = \hat{Q} | \psi_n \rangle\\
\implies \hat{H} | \psi_n' \rangle = \hat{H} \hat{Q} | \psi_n \rangle = \hat{Q} \hat{H} | \psi_n \rangle = \hat{Q} E_n |\psi_n \rangle = E_n | \psi_n' \rangle\\$
however it could happen that $| \\psi_n' \rangle = | \psi_n \rangle\\$
case 1 one symmetry operator $\hat{Q}$ or more than one and they all commute $\implies$ no degeneracy.\\


\hdashrule[0.5ex][c]{\linewidth}{0.5pt}{1.5mm}


\item \underline{$\frac{d}{dt} \langle \hat{\vec{L}} \rangle = \frac{i}{\hbar} \langle [ \hat{H}, \hat{\vec{L}}] \rangle = 0$}( rotational invariance $\implies$ conservation of angular momentum)\\
\underline{recall:} $\frac{d}{dt} \langle \hat{Q} \rangle = \frac{i}{\hbar} \langle [ \hat{H}, \hat{Q}] \rangle + \langle \frac{\partial \hat{Q}}{\partial t} \rangle\\$
we usually assume $\langle \frac{\partial \hat{Q}}{\partial t} \rangle = 0 \\
\implies \frac{d}{dt} \langle \hat{\vec{L}} \rangle = \frac{i}{\hbar} \langle [ \hat{H}, \hat{ \vec{L}} \rangle\\$
\underline{recall:} $[ \hat{H}, \hat{R}_{\hat{n}}(\phi) ] = 0\\
\implies [ \hat{H}, \hat{R}_{\hat{n}}(\delta)] = 0,\,\, \hat{R}_{\hat{n}}( \delta) \approx 1 - i \frac{\delta}{\hbar} \hat{n} \cdot \hat{\vec{L}}\\
\implies [ \hat{H}, \hat{R}_{\hat{n}}(\delta) ] = [ \hat{H}, 1 ] - i \frac{\delta}{\hbar}[\hat{H}, \hat{N} \cdot \hat{\vec{L}}] = 0\\
$

$[\hat{H},1 ] = 0 \implies [\hat{H}, \hat{n} \cdot \hat{\vec{L}}] = \hat{H}( \hat{n} \cdot \hat{\vec{L}}) - ( \hat{n} \cdot \hat{\vec{L}})\hat{H} = 0\\
= \hat{n} \cdot ( \hat{H} \hat{\vec{L}} - \hat{\vec{L}} \hat{H}) = 0\\
\implies [ \hat{H}, \hat{\vec{L}}] = 0\\
\therefore \frac{d}{dt} \langle \hat{\vec{L}} \rangle = 0\\$

\hdashrule[0.5ex][c]{\linewidth}{0.5pt}{1.5mm}
$
\begin{cases}
	\hat{H} \psi_{n \ell m} = E_n \psi_{n \ell m}\\
	\hat{L}_z \psi_{n \ell m } = m \hbar \psi_{n \ell m}\\
	\hat{L}^2 \psi_{n \ell m} = \ell(\ell + 1) \bar{\psi}^2 \psi_{n \ell m}
\end{cases}$



$[\hat{H}, \hat{\vec{L}}] = 0 \implies [ \hat{H}, \hat{L}^2] = 0 and [ \hat{H}, \hat{L}_z] = 0\\
{[L_z , \hat{L}^2]} = 
L_z \hat{L}^2 - \hat{L}^2 L_z\\$
\underline{two operators commute with $\hat{H}$ and not with eachother $\implies$ degeneracy}\\
Consider $\hat{Q}$, $\hat{Lambda}$, $\hat{H} [ \hat{Q}, \hat{H} ] = [ \hat{\Lambda}, \hat{H}] =0\\
{[ \hat{Q}, \hat{\Lambda}] }\neq 0\\$
skip $\dots$ go back to 6.6

\hdashrule[0.5ex][c]{\linewidth}{0.5pt}{1.5mm}


\item \underline{$
\begin{cases}
	{[\hat{L}^2, \hat{f}]} = 0\\
	{[\hat{L}_z, \hat{f}]} = 0\\
	{[\hat{L}_z, \hat{f}]} = 0
\end{cases}$}\\
\underline{recall:} $[\hat{L}_i, \hat{f}] = 0, \hat{L}_{\pm} = \hat{L}_x \pm i \hat{L}_y\\
{[ \hat{L}^2, \hat{f}]} = \sum_i[\hat{L}_i^2, \hat{f}]\\$
\underline{recall:} $[ \hat{A}^2, \hat{B}] = \hat{A}[\hat{A}, \hat{B}] - [\hat{B}, \hat{A}] \hat{A}\\
\implies [\hat{L}^2, \hat{f}] = \sum_i [ \hat{L}_i^2, \hat{f}] = 0\\
\implies 
\begin{cases}
	[\hat{L}^2, \hat{f}] = 0\\
	[\hat{L}_z, \hat{f}] = 0\\
	[\hat{L}_z, \hat{f}] = 0
\end{cases}$


\hdashrule[0.5ex][c]{\linewidth}{0.5pt}{1.5mm}


\item \underline{$\langle n' \ell' m' | \hat{f} | n \ell m \rangle = \delta_{\ell \ell
} \delta_{m m'} \langle n' \ell || f || n \ell \rangle$}\\
\underline{recall:} $[ \hat{L}_z, \hat{f}] = 0
\implies \langle n' \ell m' | \hat{L}_z \hat{f}|n \ell m \rangle - \langle n' \ell' m' | \hat{f} \hat{L}_z | n \ell m \rangle = 0\\$
\underline{Note:} $| n \ell m \rangle$ satisfy $L^2 f^m_{\ell} = \hbar^2 \ell(\ell + 1) f^m_{\ell};\,\, L_z f^m_{\ell} = \hbar m f^m_{\ell} and L_+ f^m_{\ell} = ( A^m_{\ell}) f^{m+1}_{\ell};\,\, L_- f^m_{\ell} = ( B^m_{\ell}) f^{m-1}_{\ell}\\$
\underline{Note:} $\langle n' \ell' m' | \hat{L}_z = \langle n' \ell' m' | \hbar m'\\
\implies \langle n' \ell' m' | [ \hat{L}_z, \hat{f}] | n \ell m \rangle = \hbar m' \langle n' \ell' m' | \hat{f}| n \ell m \rangle - \hbar m \langle n' \ell' m' | \hat{f} | n \ell m \rangle\\
\implies (m' - m ) \langle n' \ell' m' | \hat{f} | n \ell m \rangle = 0\\$
"matrix elements of a scalar operation vanish unless $m' - m \equiv \Delta m = 0\\
\langle n' \ell' m' | [ \hat{L}^2, \hat{f}] | n \ell m \rangle = 0\\$
\underline{Note:} $L^2 | n \ell m \rangle = \hbar^2 \ell(\ell + 1) | n \ell m \rangle \\
\implies \langle n' \ell' m' | \hat{L}^2 \hat{f} | n \ell m \rangle - \langle n' \ell' m' | \hat{f} \hat{L}^2 | n \ell m \rangle\\
= \hbar^2 \ell' (\ell' + 1) \langle n' \ell' m ' | \hat{f} | n \ell m \rangle - \hbar^2 \ell(\ell + 1) \langle n' \ell' m' | \hat{f} | n \ell m \rangle\\
\implies [\ell' (\ell' + 1) - \ell (\ell + 1)] \langle n' \ell' m' | \hat{f} | n \ell m \rangle = 0\\$
This tells us matrix elements vanish unless\\
$\ell'(\ell' +1) - \ell (\ell + 1) = 0\\
\implies \ell'^2 + \ell' - \ell^2 - \ell = \ell'^2 + \ell' - \ell - \ell^2\\
= \ell'^2 + 2 \ell \ell' - 2 \ell \ell' + \ell' - \ell - \ell^2\\
= \ell'^2 - 2 \ell \ell' + \ell^2 - 2 \ell^2 + 2 \ell \ell' + \ell' - \ell\\
= \ell'^2 - \ell \ell' - \ell \ell' + \ell^2 - 2 \ell^2 + 2 \ell \ell' + \ell' - \ell\\
= \ell' (\ell' - \ell) - \ell (\ell' - \ell) - 2 \ell^2 + 2 \ell \ell' + \ell' - \ell\\
== \ell'(\ell'-\ell) - \ell(\ell' - \ell) + 2 \ell(\ell' - \ell) + ( \ell' - \ell)\\
= ( \ell' - \ell + 2 \ell +1)(\ell' - \ell) = ( \ell' + \ell + 1)(\ell' - \ell)\\
\implies \ell' - \ell = 0\\$
\underline{Note:} We don't care about $\ell' = - \ell - 1$ since $\ell'$ cant be negative\\
$\therefore$ selection rule for scalar operators $\Delta \ell = \Delta m = 0\\$
$\langle n' \ell' m' | [ \hat{L}_+, \hat{f}] | n \ell m \rangle = \langle n' \ell' m' | \hat{L}_+ \hat{f}| n \ell m \rangle - \langle n' \ell' m' | \hat{f} \hat{L}_+ | n \ell m \rangle\\$
\underline{recall:} $\hat{L}_+ |n \ell m \rangle = A^m_{\ell} | n \ell (m+1) \rangle,\,\, \langle n' \ell' m' | \hat{L}_- = \langle n' \ell' (m' + 1)| B_{\ell'}^{m'}\\
\implies B^{m'}_{\ell'} \langle n' \ell' (m'-1)|\hat{f} | n \ell m \rangle - A^m_{\ell} \langle n' \ell' m' | \hat{f} | n \ell(m+1) \rangle = 0\\$
\underline{recall:} if $m' - m \neq 0 \implies \langle n' \ell' m ' | \hat{f} | n \ell m \rangle = 0\\
j\implies m' = m+1 ,\,\, \ell' = \ell\\
\implies B^{m+1}_{\ell} \langle n' \ell m | \hat{f} | n \ell m \rangle - A^m_{\ell} \langle n' \ell(m+1)| \hat{f}|n \ell (m+1) \rangle = 0\\$
\underline{Note:} $A^m_{\ell} = \hbar \sqrt{\ell(\ell+1) - m ( m+1)};\,\, B^m_{\ell} = \hbar \sqrt{\ell(\ell + 1) - m( m-1)}\\
\implies A^m _{\ell} = B^{m+1}_{\ell}\\
\implies \langle n' \ell m | \hat{f} | n \ell m \rangle = \langle n' \ell (m+1) | \hat{f} | n \ell(m+1) \rangle\\$
Notice this equation doesn't depend on $m$\\
$\implies \langle n'\ell m | \hat{f} | n \ell m \rangle = \langle n' \ell || \hat{f} || n \ell \rangle$ (reduced matrix)\\
If $m \neq m' or \ell \neq \ell'$ then matrix elements are zero.\\
Summarizing,\\
$\therefore \langle n' \ell' m' | \hat{f} | n \ell m \rangle = \delta_{\ell \ell'} \delta_{m m'} \rangle n' \ell || \hat{f} || n \ell \rangle$\\


\hdashrule[0.5ex][c]{\linewidth}{0.5pt}{1.5mm}


Skip 6.7 come back to during a weekend


\hdashrule[0.5ex][c]{\linewidth}{0.5pt}{1.5mm}


\item \underline{$\hat{U}(t) = \exp[ - \frac{i t}{\hbar} \hat{H}]$}\\
$\hat{H} \Psi(x,t) = i \hbar \frac{\partial}{\partial t} \Psi(x,t)\\
\hat{U}(t) \Psi(x,t) = \Psi(x,t)$ definition of $\hat{U}(t)\\$
assume $\hat{H}(t) = \hat{H}\\
\implies \hat{U}(t) \Psi(x,0) = \Psi(x,t) = \sum_{n=0}^{\infty} \frac{1}{n!} \frac{\partial^n}{\partial t^n} \Psi(x,t) |_{t=0} t^n\\$
but $\frac{\partial^n}{\partial t^n} \Psi(x,t) |_{t=0} = ( \frac{1}{i \hbar} \hat{H})^n \Psi(x,t)|_{t=0} = ( \frac{1}{i\hbar} \hat{H})^n \Psi(x,0)\\
\implies \hat{U}(t) \Psi(x,0) = \sum_{n=0}^{\infty} \frac{1}{n!}(- \frac{i}{\hbar} \hat{H} t)^n \Psi(x,0)\\
= \exp[- \frac{i}{\hbar} \hat{H} t] \Psi(x,0)$


\hdashrule[0.5ex][c]{\linewidth}{0.5pt}{1.5mm}


\item \underline{$\Psi(x,t) = \sum_n c_n e^{-i E_n t/\hbar} \psi_n(x)$}\\
$\Psi(x,0) = \sum_n c_n \psi_n(x)\\
\Psi(x,t) = \hat{U}(t) \Psi(x,0) = \sum_n c_n \hat{U}(t) \psi_n(x)\\
\underline{recall:} \hat{H} \psi_n = E_n \psi_n\\
\implies \Psi(x,t) = \sum_n c_n \exp[- \frac{i t}{\hbar} \hat{H}] \psi_n(x)\\
= \sum_n c_n e^{-itE_n/\hbar} \psi_n(x)\\$


\hdashrule[0.5ex][c]{\linewidth}{0.5pt}{1.5mm}


$\hat{Q}_H(t) = \hat{U}^{\dagger}(t) \hat{Q} \hat{U}(t)$ (Heisenberg-picture operators)


\hdashrule[0.5ex][c]{\linewidth}{0.5pt}{1.5mm}


Schrodinger picture wave functions depend on time, operators dont\\
Heisenberg picture is the opposite.\\
\underline{Note:} $\Psi_H(x) = \Psi(x,0)\\
\langle \Psi(t) | \hat{Q} |\Psi(t) \rangle = \langle \Psi(0) | \hat{U}^{\dagger} \hat{Q} \hat{U} | \Psi(0) \rangle = \langle \Psi_H | \hat{Q}_H(t) | \Psi_H \rangle\\$
i.e. their pictures are identical.\\


\hdashrule[0.5ex][c]{\linewidth}{0.5pt}{1.5mm}


\item \underline{$\Psi(x,t) =\hat{U}(t,t_0) \Psi(x,t_0)$}\\
$\Psi(x,t) = \sum_n \frac{(t-t_0)^n}{n!} \frac{\partial^n}{\partial t^n} \Psi(x,t_0)\\$
\underline{recall:} $\hat{H} \Psi = i \hbar \frac{\partial}{\partial t} \Psi\\
\implies \sum_n \frac{(t-t_0)^n}{n!} (\frac{1}{i \hbar} \hat{H})^n \Psi(x,t_0)\\
= \sum_ \frac{1}{n!}((t-t_0)(- \frac{i}{\hbar} \hat{H}))^n \Psi(x,t_0)\\$
\underline{recall:} $\hat{U}(t) = \exp[- \frac{it}{\hbar} \hat{H} ]\\
\therefore \Psi(x,t) = \exp[- \frac{i(t-t_0)}{\hbar}] \Psi(x,t_0) = \hat{U}(t,t_0) \Psi(x,t_0)\\$


\hdashrule[0.5ex][c]{\linewidth}{0.5pt}{1.5mm}


\underline{Note:} $\hat{U}(t_) + \delta, t_0) = \sum_n \frac{1}{n!}(- \frac{i \delta}{\hbar} \hat{H})^n\\
\approx 1- \frac{i}{\hbar} \hat{H}(t_0) \delta\\$


\hdashrule[0.5ex][c]{\linewidth}{0.5pt}{1.5mm}


$\hat{U}(t_1 + \delta, t_1) = \hat{U}(t_2 + \delta, t_2)$ (time-translation invariance)\\


\hdashrule[0.5ex][c]{\linewidth}{0.5pt}{1.5mm}


$\implies 1- \frac{i}{\hbar} \hat{H}(t_1) \delta = 1- \frac{i}{\hbar} \hat{H}(t_2) \delta \implies \hat{H}(t_1) = \hat{H}(t_2)\\\
\frac{d}{dt} \langle \hat{H} \rangle = \frac{i}{\partial} \langle [ \hat{H}, \hat{H} ] \rangle + \langle \frac{\partial \hat{H}}{\partial t} \rangle = 0$ ( time invariance)\\
since $\hat{H}(t_1) = \hat{H}(t_2) \implies \frac{\partial \hat{H}}{\partial t} = 0\\$


\hdashrule[0.5ex][c]{\linewidth}{0.5pt}{1.5mm}


\section*{Chapter 7}


\item \underline{$E^1_n = \langle \psi_n^0 | H^1 | \psi^0_n \rangle$} (first order)\\
$H^0 \psi_n^0 = E_n^0 \psi_n^0$ ( unperturbed)\\
$\langle \psi_n^0 | \psi_m^0 \rangle = \delta_{nm}\\
H \psi_n = E_n \psi_n$ (perturbed)\\
$H= H^0 + \lambda H^1\\
\psi_n = \psi_n^0 + \lambda \psi_n^1 + \lambda^2 \psi_n^2 + \cdots$ ( perturbed)\\
$E_n = E^0_n + \lambda E_n^1 + \lambda^2 E_n^2 + \cdots\\
H \psi_n = E_n \psi_n\\
\implies ( H^0 + \lambda H^1) ( \psi_n^0 + \lambda \psi_n^1 + \lambda^2 \psi_n^2 + \cdots)\\
= ( E_n^0 + \lambda E_n^1 + \lambda^2 E_n^2 + \cdots) ( \psi_n^0 + \lambda \psi^1_n + \lambda^2 \psi^2_n + \cdots)\\
= H^0 \psi^0_n + ( H^0 \psi_n^1 + H^1 \psi_n^0) \lambda\\
+ ( H^0 \psi^2_n + H^1 \psi^1_n) \lambda^2 + \cdots\\
= E_n^0 \psi_n^0 + ( E_n^0 \psi^1_n + E^1_n \psi_n^0 (first order)\\
H^0 \psi^2_n + H^1 \psi^1_n = E^0_n \psi^2_n + E_n^1 \psi^1_n + E^2_n \psi^0_n\\$
($2^{nd}$ order)\\
$1^{\rm{st}}$ order\\
$\implies \langle \psi_n^0 | H^0 \psi^1_n \rangle + \langle \psi^0_n | H^1 \psi^0_n \rangle = E_n^0 \langle \psi_n^0 | \psi^1_n \rangle + E_n^1 \langle \psi_n^0 | \psi_n^0 \rangle\\$
$H^0$ hermitian\\
$\implies \langle H^0 \psi_n^0 | \psi_n^1 \rangle = E_n^0 \langle \psi_n^0 | \psi_n^1 \rangle\\
\therefore \langle \psi^0_n | H^1 \psi^0_n \rangle = E_n^1 ( \langle \psi^0_n | \psi_n^0 \rangle =1 )\\$
recall: $H^0 \psi^1_n + H^1 \psi^0_n = E_n^0 \psi^1_n + E_n^1 \psi_n^0\\
\implies ( H^1 - E^1_n) \psi_n^0 = - (H^0 - E_n^0) \psi_n^1\\$
inhomogeneous ODE for $\psi^1_n$


\hdashrule[0.5ex][c]{\linewidth}{0.5pt}{1.5mm}


\item \underline{$\psi_n^1 = \sum_{m \neq n} \frac{\langle \psi_m^0 |H^1 | \psi_n^0 \rangle}{(E_n^0 - E_m^0)} \psi_m^0$}\\
$(H^0 - E_n^0) \psi_n^1 = - ( H^1 - E_n^1 ) \psi_n^0\\
\psi_n^0$ complete\\
$\implies \psi_n^1 = \sum_{m \neq n} c_m^{(n)} \psi_m^0$ why $m \neq n?\\$
We want $\langle \psi_n | \psi_n \rangle = 1\\
\implies \langle \psi_n^0 + \lambda \psi_n^1 | \psi_n^0 + \lambda \psi_n^1 \rangle\\
= \langle \psi_n^0 | \psi_n^0 + \lambda \psi_n^1 \rangle + \lambda \langle \psi_n^1 | \psi_n^0 + \lambda \psi_n^1 \rangle\\
= \langle \psi_n^0 | \psi_n^0 \rangle + \lambda \langle \psi_n^0 | \psi_n^1 \rangle + \langle \psi_n^1 | \psi_n^0 \rangle + \lambda^2 \langle \psi_n^1 | \psi_n^1 \rangle\\
\approx \langle \psi_n^0 | \psi_n^0 \rangle + \lambda ( \langle \psi_n^0 | \psi_n^1 \rangle + \langle \psi_n^1 | \psi_n^0 \rangle )\\
\langle \psi_n^0 | \psi_n^1 \rangle = \langle \psi_n^1 | \psi_n^0 \rangle = 0$ iff $\psi_n^1$ does not contain $\psi_n^0\\
\implies \psi_n^1 = \sum_{m \neq n } c_m^{(n)} \psi_m^0\\
\implies (H^0 - E_n^0) \sum_{m \neq n} c_m^{(n)} \psi_m^0 = - ( H^1 - E_n^1) \psi_n^0\\
\implies \sum_{m \neq n} ( E_m^0 - E_n^0) c_m^{(n)} \psi_m^0 = - ( H^1 - E^1_n) \psi_n^0\\
\implies \sum_{m \neq n} ( E_m^0 - E_n^0) c_m^{(n)} \langle \psi_{\ell}^0 | \psi_m^0 \rangle\\
= - \langle \psi_{\ell}0 | H^1 | \psi_n^0 \rangle + E_m^1 \langle \psi_{\ell}^0 | \psi_n^0 \rangle\\
\ell = n\\
\implies E_n^1 = \langle \psi_n^0 | H^1 | \psi_n^0 \rangle\\
\ell \neq n\\
\implies (E_{\ell}^0 - E_n^0) c_{\ell}^{(n)} = - \langle \psi_{\ell}^0 | H^1 | \psi_n^0 \rangle\\
\implies c_m^{(n)} = \frac{\langle \psi_m^{(0)} | H^1 | \psi_n^0 \rangle}{E_n^0 - E_m^0}\\
\therefore \psi_n^1 = \sum_{m \neq n} \frac{\langle \psi_m^0 | H^1 | \psi_n^0 \rangle}{(E_n^0 - E_m^0)} \psi_m^0 $(non-degenerate)


\hdashrule[0.5ex][c]{\linewidth}{0.5pt}{1.5mm}


\item \underline{$P_{a \rightarrow b}(t) = | c_b(t) |^2 \approx \frac{|V_{ab}|^2}{\hbar^2} \frac{\sin^2[(\omega_0 - \omega)t/2]}{(\omega_0 - \omega)^2}$ (sinusoidal perturbations)}\\
(transition probability: the probability that a particle in state $\psi_a$ will be found in state $\psi_b$)\\
$\hat{H}'(\vec{r},t) = V(\vec{r}) \cos \omega t (sinusoidal perturbations\\
\implies \langle \psi_a | \hat{H}' | \psi_b \rangle \equiv H_{ab} = \langle \psi_a | V | \psi_b \rangle \cos \omega t = V_{ab} \cos \omega t\\$
\underline{recall:} $c_b^{(1)} = - \frac{i}{\hbar} \int_0^t H'_{ba} (t') e^{i \omega_0 t'} d t'\\
\implies c_b(t) \approx - \frac{i}{\hbar} V_{ba} \int_0^t \cos(\omega t') e^{i \omega_0 t'} d t'\\
= - \frac{i}{\hbar} \frac{V_{ba}}{2} \int_0^t (e^{i \omega t'} + e^{- i \omega t'}) e^{i \omega_0 t'} dt'\\
= - \frac{i V_{ba}}{2 \hbar} \int_0^t e^{i( \omega + \omega_0) t'} + e^{i (\omega_0 - \omega)t'} d t'\\
=- \frac{i V_{ba}}{2 \hbar} ( \frac{e^{i( \omega + \omega_0)t'}}{i( \omega + \omega_0)} + \frac{e^{i( \omega_0 - \omega) t'}}{i( \omega_0 - \omega)})\\
= - \frac{V_{ba}}{2 \hbar} ( \frac{e^{i( \omega + \omega_0)t'}}{\omega + \omega_0} + \frac{e^{i( \omega_0 - \omega) t'}}{\omega_0 - \omega} |_0^t\\
= - \frac{V_{ba}}{2 \hbar} ( \frac{e^{i (\omega + \omega_0)t} - 1}{\omega + \omega_0} + \frac{e^{i( \omega_0 - \omega)t} - 1}{\omega_0 - \omega})\\$
assume $\omega_0$ is close to $\omega$ (resonant frequency)\\
$\implies \omega_0 + \omega >> | \omega_0 - \omega|
\implies c_b(t) \approx - \frac{V_{ba}}{2 \hbar} \frac{e^{i( \omega_0 - \omega)t} - 1}{\omega_0 - \omega}\\
= - \frac{V_{ba}}{2 \hbar} \frac{e^{i( \omega_0 - \omega) t/2}}{\omega_0 - \omega} ( e^{i( \omega_0 - \omega)t/2} - e^{i( \omega_0 - \omega)t/2})\\
= - \frac{V_{ba} }{\hbar} i \frac{e^{i( \omega_0 - \omega)t/2}}{ \omega_0 - \omega} \sin(( \omega_0 - \omega)t/2)\\
\therefore P_{a \rightarrow b}(t) = | c_b(t) |^2 \approx \frac{|V_{ba}|^2}{\hbar^2} \frac{\sin^2[(\omega_0 - \omega)t/2]}{( \omega_0 - \omega)^2}$



\hdashrule[0.5ex][c]{\linewidth}{0.5pt}{1.5mm}












 










































\end{enumerate}






























\end{document}